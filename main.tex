% !TEX program = lualatex
\documentclass[11pt, a4paper, oneside]{book}

% ===== Fonts and Language =====
\usepackage{fontspec}
\usepackage{polyglossia}
\setdefaultlanguage{english}

% ===== Math =====
\usepackage{amsmath, amssymb, amsthm}

% ===== Bibliography =====
\usepackage[backend=biber, style=numeric, sorting=nyt, maxbibnames=10]{biblatex}
\addbibresource{references.bib}
\setcounter{MaxMatrixCols}{12} 
% ===== Graphics and Layout =====
\usepackage{graphicx}
\usepackage{xcolor}
\usepackage{geometry}
\geometry{margin=1in}

% ===== TikZ and PGFPlots =====
\usepackage{tikz}
\usepackage{tikz-3dplot}
\usetikzlibrary{arrows.meta,calc,positioning}

\usepackage{pgfplots}
\pgfplotsset{compat=1.18}
\usetikzlibrary{
    3d,
    arrows.meta,
    calc,
    patterns,
    positioning,
    shapes.geometric,
    shapes.misc,
    shapes.arrows,
    decorations.pathreplacing,
    decorations.markings,
    angles,
    quotes,
    backgrounds,
    fit
}



% Define colors for consistent figure styling
\definecolor{xaxis}{RGB}{220,50,50}      % Red for x-axis
\definecolor{yaxis}{RGB}{50,170,50}      % Green for y-axis
\definecolor{zaxis}{RGB}{50,50,220}      % Blue for z-axis
\definecolor{frameA}{RGB}{0,0,0}         % Black for frame A
\definecolor{frameB}{RGB}{128,128,128}   % Gray for frame B
\definecolor{vectorcolor}{RGB}{180,50,180} % Magenta for vectors

% TikZ styles for consistent appearance
\tikzset{
    % Coordinate frame axes
    axis/.style={-{Stealth[length=3mm]}, thick},
    xaxisstyle/.style={axis, xaxis},
    yaxisstyle/.style={axis, yaxis},
    zaxisstyle/.style={axis, zaxis},
    % Vectors
    vec/.style={-{Stealth[length=2.5mm]}, thick, vectorcolor},
    % Dashed axes for secondary frames
    dashedaxis/.style={axis, dashed, opacity=0.7},
    % Blocks for diagrams
    block/.style={draw, rectangle, minimum height=2em, minimum width=3em, align=center},
    sumnode/.style={draw, circle, minimum size=1.5em, inner sep=0pt},
    % Quadrotor motor
    motor/.style={draw, circle, fill=gray!30, minimum size=8mm},
    % Annotations
    annot/.style={font=\small\itshape},
}

% ===== Hyperlinks =====
\usepackage{hyperref}
\hypersetup{
    colorlinks=true,
    linkcolor=blue,
    urlcolor=cyan,
    citecolor=green
}

% ===== Chapter and Section Formatting =====
\usepackage{titlesec}
\titleformat{\chapter}[display]
  {\normalfont\huge\bfseries}{\chaptertitlename\ \thechapter}{20pt}{\Huge}
\titlespacing*{\chapter}{0pt}{-30pt}{40pt}

% ===== Headers and Footers =====
\usepackage{fancyhdr}
\pagestyle{fancy}
\fancyhf{}
\fancyhead[L]{\leftmark}
\fancyhead[R]{\thepage}
\renewcommand{\headrulewidth}{0.4pt}

% ===== Theorem Environments =====
\theoremstyle{definition}
\newtheorem{definition}{Definition}[chapter]
% Example environment defined below using tcolorbox
\theoremstyle{plain}
\newtheorem{theorem}{Theorem}[chapter]
\newtheorem{lemma}[theorem]{Lemma}
\newtheorem{proposition}[theorem]{Proposition}
\theoremstyle{remark}
\newtheorem{remark}{Remark}[chapter]

% ===== Boxes =====
\usepackage{tcolorbox}
\tcbuselibrary{skins, breakable, theorems}
\newtcolorbox{keyidea}[1][]{
  colback=blue!5!white,
  colframe=blue!75!black,
  fonttitle=\bfseries,
  title=Key Idea,
  #1
}
\newtcolorbox{notebox}[1][]{
  colback=yellow!10!white,
  colframe=yellow!50!black,
  fonttitle=\bfseries,
  title=Note,
  #1
}
\newtcolorbox{warningbox}[1][]{
  colback=red!5!white,
  colframe=red!75!black,
  fonttitle=\bfseries,
  title=Warning,
  #1
}

% Example environment with automatic numbering (styled box)
\newcounter{example}[chapter]
\renewcommand{\theexample}{\thechapter.\arabic{example}}
\newtcolorbox{example}[1][]{
  colback=green!5!white,
  colframe=green!50!black,
  fonttitle=\bfseries,
  breakable,
  enhanced,
  before upper={\refstepcounter{example}},
  title={Example~\theexample\ifx&#1&\else: #1\fi},
}

% Exercise environment with automatic numbering
\newcounter{exercise}[chapter]
\renewcommand{\theexercise}{\thechapter.\arabic{exercise}}
\newtcolorbox{exercise}[1][]{
  colback=orange!5!white,
  colframe=orange!60!black,
  fonttitle=\bfseries,
  breakable,
  enhanced,
  before upper={\refstepcounter{exercise}},
  title={Exercise~\theexercise\ifx&#1&\else: #1\fi},
}

% ===== Code Listings =====
\usepackage{listings}
\lstset{
  basicstyle=\ttfamily\small,
  keywordstyle=\color{blue},
  commentstyle=\color{green!60!black},
  stringstyle=\color{red},
  showstringspaces=false,
  breaklines=true,
  frame=single,
  numbers=left,
  numberstyle=\tiny\color{gray}
}

\lstdefinelanguage{yaml}{
  keywords={true,false,null,y,n},
  keywordstyle=\color{blue}\bfseries,
  sensitive=false,
  comment=[l]{\#},
  commentstyle=\color{gray}\ttfamily,
  stringstyle=\color{red}\ttfamily,
  morestring=[b]',
  morestring=[b]"
}
\lstdefinelanguage{cmake}{
  keywords={
    add_executable, add_library, add_subdirectory,
    target_link_libraries, target_include_directories,
    target_compile_definitions, target_compile_options,
    set, unset, option, message,
    if, elseif, else, endif,
    foreach, endforeach,
    while, endwhile,
    function, endfunction,
    macro, endmacro,
    include, find_package, find_library, find_path,
    project, cmake_minimum_required
  },
  keywordstyle=\color{blue}\bfseries,
  sensitive=true,
  comment=[l]{\#},
  commentstyle=\color{gray}\ttfamily,
  stringstyle=\color{red}\ttfamily,
  morestring=[b]",
  morestring=[b]'
}

\lstdefinelanguage{arm}{%
  sensitive=true,%
  backgroundcolor=\color[rgb]{0.98,0.98,0.98},%
  basicstyle=\ttfamily\small,%
  keywordstyle=\color{blue}\bfseries,%
  commentstyle=\color{green!50!black}\itshape,%
  stringstyle=\color{red!70!black},%
  numberstyle=\scriptsize\color{gray},%
  frame=single,%
  rulecolor=\color{gray!40},%
  numbers=left,%
  numbersep=8pt,%
  tabsize=2,%
  keepspaces=true,%
  columns=fullflexible,%
  breaklines=true,%
  showstringspaces=false,%
  morecomment=[l]{;},%
  morecomment=[l]{@},%
  morecomment=[l]{//},%
  morekeywords={%
    adc,adcs,add,adds,and,ands,asr,asrs,b,bal,beq,bne,bcs,bhs,bcc,blo,%
    bmi,bpl,bvs,bvc,bhi,bls,bge,blt,bgt,ble,bx,bl,blx,cbz,cbnz,%
    cmn,cmp,cpsid,cpsie,eor,eors,isb,ldr,ldrb,ldrh,ldrsb,ldrsh,%
    ldm,ldmia,ldmfd,str,strb,strh,stm,stmia,stmfd,%
    lsl,lsls,lsr,lsrs,mov,movs,mrs,msr,mul,muls,mvn,mvns,%
    nop,orr,orrs,pop,push,rev,rev16,revsh,ror,rrx,%
    rsb,rsbs,sbc,sbcs,sev,sub,subs,tst,teq,%
    sxtb,sxth,uxtb,uxth,wfe,wfi,yield,%
    it,ite,itt,itte,svc,udiv,sdiv,%
    vadd,vmul,vmla,vmls,vdiv,vsqrt,vcvt,vldr,vstr,vpush,vpop%
  },%
  morekeywords=[2]{%
    r0,r1,r2,r3,r4,r5,r6,r7,r8,r9,r10,r11,r12,r13,r14,r15,%
    sp,lr,pc,ip,fp,%
    cpsr,spsr,apsr,ipsr,epsr,%
    primask,basepri,basepri_max,faultmask,control%
  },%
  keywordstyle=[2]\color{violet}\bfseries,%
  morekeywords=[3]{%
    .text,.data,.bss,.section,.global,.globl,.extern,%
    .type,.size,.align,.balign,.equ,.set,%
    .word,.hword,.byte,.ascii,.asciz,.space,.fill,.org,.end%
  },%
  keywordstyle=[3]\color{teal},%
}

% ===== Tables =====
\usepackage{booktabs}
\usepackage{multirow}

% ===== Index =====
\usepackage{makeidx}
\makeindex

% ===== Custom Commands =====
\newcommand{\lecture}[3]{%
  \chapter*{Lecture #1: #2}%
  \addcontentsline{toc}{chapter}{Lecture #1: #2}%
  \markboth{Lecture #1: #2}{}%
  \textbf{Date:} #3\par
  \vspace{0.5em}\hrule\vspace{1em}
}

% ===== Title =====
\title{Model-based Development of Cyber-Physical Systems\\[0.5em]\large Lecture Notes -- SSY191}
\author{Knut \AA kesson\\
Department of Electrical Engineering\\
Chalmers University of Technology}
\date{Spring 2024}

\begin{document}

\maketitle

\chapter*{Foreword}
\addcontentsline{toc}{chapter}{Foreword}
\markboth{Foreword}{}

This book grew out of the lecture notes for the course \emph{Model-based Development of Cyber-Physical Systems} (SSY191) at Chalmers University of Technology. The course takes a project-centered approach where students progressively work through all the steps required to control a quadrotor---from building mathematical models and designing controllers, through simulation and verification, to implementing the control software on actual hardware.

Each week, students make tangible progress on their quadrotor project: deriving the equations of motion, understanding sensor characteristics, designing attitude and position controllers, simulating the closed-loop system, and ultimately seeing their code fly on a real drone. This hands-on progression---from mathematical abstraction to physical reality---embodies the essence of cyber-physical systems engineering.

Beyond the core modeling and control topics, the course covers essential aspects that are often overlooked in traditional control textbooks: real-time operating systems, task scheduling and timing analysis, and systematic testing methods for safety-critical embedded systems. These topics are crucial for anyone who wants to move beyond simulation and deploy control systems in the real world, where timing constraints must be met and software correctness must be verified.

We developed these lecture notes because no single textbook adequately covers this breadth of material---the intersection of control theory, embedded systems, and verification---at the depth required for a project-based course. For years, students relied on a patchwork of slides, handouts, and supplementary readings. Recent advances in generative techniques have made it possible to transform these scattered notes into a coherent, comprehensive text with detailed explanations and worked examples, all within a reasonable effort.

The result is this book: a self-contained guide to the theory and practice of building cyber-physical systems, using the quadrotor as a concrete and motivating example throughout. We hope it serves not only students in the course but also practitioners and researchers who seek a unified treatment of modeling, control, real-time systems, and testing.

\section*{Suggested Course Structure}

The material in this book is designed for a 7-week course with two 90-minute lectures per week. We recommend the following structure, which prioritizes conceptual understanding and mathematical derivations during lectures while assigning implementation details and tool-specific content as independent reading.

\textbf{Module 1: Quadrotor Modelling and Sensor Fusion} (6 lectures)
\begin{enumerate}
\item \textbf{Introduction \& Coordinate Frames.} CPS motivation, the sense-estimate-compute-actuate cycle, coordinate conventions (ENU/NED), and 2D rotation intuition.
\item \textbf{3D Rotations \& Euler Angles.} Rotation matrix composition, demonstrating non-commutativity, Euler angles (ZYX convention), and the gimbal lock problem.
\item \textbf{Quaternions \& Transformations.} Quaternion algebra, the rotation formula, how quaternions avoid gimbal lock, and homogeneous transformations.
\item \textbf{Sensors \& Attitude Estimation.} Gyroscope and accelerometer models, the drift problem, complementary filtering (frequency domain insight), and the Mahony filter.
\item \textbf{Kalman Filtering.} Probability review, the linear Kalman filter derivation, Kalman gain interpretation, and Extended Kalman Filter for attitude.
\item \textbf{Quadrotor Dynamics \& Control.} Force and torque balance, motor mixing, state-space formulation, linearization at hover, and LQR design.
\end{enumerate}

\textbf{Module 2: Simulation and Hybrid Systems} (2 lectures)
\begin{enumerate}
\setcounter{enumi}{6}
\item \textbf{Equation-Based Modeling \& Hybrid Systems.} Acausal versus causal modeling, Modelica concepts, hybrid automata formalism, and flight mode state machines.
\item \textbf{Numerical Simulation.} Euler and Runge-Kutta methods, fixed versus variable step solvers, zero-crossing detection, and multi-rate system timing.
\end{enumerate}

\textbf{Module 3: Real-Time Embedded Systems} (3 lectures)
\begin{enumerate}
\setcounter{enumi}{8}
\item \textbf{RTOS \& Scheduling.} Why real-time matters, the task model, Rate-Monotonic Scheduling and its optimality, the Liu-Layland bound, and priority inversion.
\item \textbf{Concurrency \& Synchronization.} Race conditions, mutexes versus semaphores, deadlock conditions and prevention, and producer-consumer patterns.
\item \textbf{Timing Analysis \& Fault Tolerance.} Response-time analysis, WCET estimation approaches, watchdog timers, and failsafe design principles.
\end{enumerate}

\textbf{Module 4: Testing and Verification} (3 lectures)
\begin{enumerate}
\setcounter{enumi}{11}
\item \textbf{Testing Levels \& Requirements.} The V-model, MIL/SIL/PIL/HIL testing, formalizing requirements, and Signal Temporal Logic basics.
\item \textbf{STL Robustness \& Falsification.} Quantitative robustness semantics, falsification as optimization, and using tools like Breach.
\item \textbf{Coverage, Testing \& Integration.} Code coverage metrics (including MC/DC), unit testing embedded code, CI/CD pipelines, and a complete case study.
\end{enumerate}

The book contains considerably more detail than can be covered in lectures. We encourage instructors to focus lecture time on geometric intuition, mathematical derivations, and worked examples, while directing students to the text for calibration procedures, API details, and tool-specific syntax. The appendices on C programming, number representation, ARM assembly, and Simulink code generation serve as reference material for students with varying backgrounds.

\vspace{2em}
\noindent\textit{Gothenburg, 2026}

\vspace{0.5em}
\noindent Knut \AA kesson

% Only show chapters and sections in table of contents (not subsections)
\setcounter{tocdepth}{1}

\tableofcontents

% Introduction: Cyber-Physical Systems and Model-Based Development
\chapter{Introduction to Cyber-Physical Systems}
\label{ch:introduction}
\index{cyber-physical systems|(}

% Learning objectives:
% - Understand what cyber-physical systems are and their key characteristics
% - Appreciate the role of model-based development in CPS engineering
% - Recognize why control engineers need real-time systems knowledge
% - Understand why software engineers need control systems knowledge
% - See how the four modules of this course connect

\section{The Challenge of Flying Robots}

Consider a quadrotor\index{quadrotor} hovering in your living room. To the casual observer, it appears to float effortlessly, making tiny adjustments to maintain its position. But beneath this apparent simplicity lies an extraordinary engineering challenge that spans multiple disciplines.

Every two milliseconds---five hundred times per second---the following sequence must complete:

\begin{enumerate}
    \item \textbf{Sense}: Read accelerometer, gyroscope, and barometer data from the IMU
    \item \textbf{Estimate}: Fuse noisy sensor readings into a coherent estimate of orientation and velocity
    \item \textbf{Compute}: Calculate the control commands needed to track the desired trajectory
    \item \textbf{Actuate}: Send PWM signals to the four motors
\end{enumerate}

If any step takes too long, arrives late, or produces incorrect results, the quadrotor falls. There is no graceful degradation---gravity is unforgiving.

\begin{center}
\begin{tikzpicture}[every node/.style={font=\small}]
    % Timeline
    \draw[thick, -{Stealth}] (0,0) -- (12,0) node[right] {time};

    % 2ms markers
    \foreach \x in {0, 4, 8} {
        \draw[thick] (\x, -0.15) -- (\x, 0.15);
    }
    \node[below] at (0, -0.2) {$0$};
    \node[below] at (4, -0.2) {$2$ ms};
    \node[below] at (8, -0.2) {$4$ ms};

    % Control cycle
    \fill[blue!30] (0.1, 0.3) rectangle (0.6, 0.8);
    \fill[green!30] (0.6, 0.3) rectangle (1.4, 0.8);
    \fill[orange!30] (1.4, 0.3) rectangle (2.2, 0.8);
    \fill[red!30] (2.2, 0.3) rectangle (2.6, 0.8);

    \fill[blue!30] (4.1, 0.3) rectangle (4.6, 0.8);
    \fill[green!30] (4.6, 0.3) rectangle (5.4, 0.8);
    \fill[orange!30] (5.4, 0.3) rectangle (6.2, 0.8);
    \fill[red!30] (6.2, 0.3) rectangle (6.6, 0.8);

    % Legend
    \fill[blue!30] (0, -1) rectangle (0.4, -0.7);
    \node[right] at (0.5, -0.85) {Sense};
    \fill[green!30] (2, -1) rectangle (2.4, -0.7);
    \node[right] at (2.5, -0.85) {Estimate};
    \fill[orange!30] (4.5, -1) rectangle (4.9, -0.7);
    \node[right] at (5, -0.85) {Compute};
    \fill[red!30] (7.5, -1) rectangle (7.9, -0.7);
    \node[right] at (8, -0.85) {Actuate};

    % Deadline arrow
    \draw[dashed, red!60] (4, 0) -- (4, 1.2);
    \node[above, red!60] at (4, 1.2) {Deadline};
\end{tikzpicture}
\end{center}

What makes this problem so challenging is that it requires expertise from multiple engineering disciplines simultaneously:

\begin{itemize}
    \item \textbf{Control theory}: Designing controllers that stabilize an inherently unstable system
    \item \textbf{Signal processing}: Filtering noisy sensor data and fusing multiple measurements
    \item \textbf{Real-time systems}: Ensuring computations complete within strict deadlines
    \item \textbf{Embedded software}: Implementing algorithms efficiently on resource-constrained hardware
    \item \textbf{Verification}: Testing that the system behaves correctly under all conditions
\end{itemize}

No single discipline provides all the answers. A control engineer who designs a beautiful controller on paper may find it unstable when implemented due to computational delays. A software engineer who writes elegant code may inadvertently break the timing assumptions that the control design depends on. A verification engineer who tests only the software may miss failures that emerge from the interaction between code and physics.

\begin{keyidea}
Cyber-physical systems require engineers who can work across the boundaries of control theory, computer science, and electrical engineering. This course teaches you to be that kind of engineer.
\end{keyidea}

\section{What Are Cyber-Physical Systems?}

The term \emph{cyber-physical systems}\index{cyber-physical systems|textbf} (CPS) was coined by Helen Gill at the National Science Foundation in 2006 to describe systems where computational and physical processes are deeply intertwined. Unlike traditional embedded systems that merely monitor or control physical processes, CPS are characterized by tight, bidirectional coupling between computation and physics.

\begin{definition}[Cyber-Physical System]
A \textbf{cyber-physical system} is an engineered system where computational algorithms and physical processes are integrated such that each affects the other through feedback loops, and correctness depends on both the logical behavior of software and the dynamic behavior of physical components.
\end{definition}

\subsection{Key Characteristics of CPS}

Several characteristics distinguish cyber-physical systems from traditional software or control systems:

\paragraph{Tight Coupling of Computation and Physics}
In CPS, the computational elements do not merely observe the physical world---they actively shape it, and the physical world shapes the computation in return. The quadrotor's motors change its orientation, which changes what the sensors measure, which changes what the controller computes, which changes the motor commands. This closed loop operates continuously, with each domain affecting the other.

\paragraph{Real-Time Constraints}
\index{real-time constraints}
CPS must respond to physical events within bounded time. A late response is often as bad as a wrong response. If the quadrotor's attitude controller runs 10\% slower than designed, the control loop becomes unstable---not because the algorithm is wrong, but because timing is part of correctness.

\paragraph{Concurrency and Distribution}
\index{concurrency}
CPS typically involve multiple computational tasks running concurrently, often on distributed processors. The quadrotor runs sensor fusion, attitude control, position control, and communication tasks simultaneously, and these tasks must coordinate without interfering with each other.

\paragraph{Safety Criticality}
\index{safety criticality}
Many CPS operate in contexts where failures have serious consequences. Autonomous vehicles, medical devices, and industrial robots can cause physical harm if they malfunction. This demands rigorous approaches to design, implementation, and verification that go beyond typical software engineering practices.

\paragraph{Heterogeneity}
CPS integrate diverse components: analog sensors, digital processors, mechanical actuators, communication networks, and human operators. Each component has its own characteristics, failure modes, and constraints that must be understood and managed.

\subsection{CPS Examples Beyond Quadrotors}

While this course uses the quadrotor as a running example, cyber-physical systems are ubiquitous in modern technology:

\begin{itemize}
    \item \textbf{Automotive}: Engine control, anti-lock braking, adaptive cruise control, autonomous driving. Modern vehicles contain 50--100 electronic control units (ECUs) coordinating thousands of functions.

    \item \textbf{Aerospace}: Flight control systems, autopilots, engine management. Aircraft like the Boeing 787 are ``fly-by-wire''---the pilot's inputs are interpreted by computers that send commands to control surfaces.

    \item \textbf{Medical devices}: Pacemakers, insulin pumps, surgical robots. These systems directly affect human health and must meet stringent safety requirements.

    \item \textbf{Power systems}: Smart grids balance electricity generation and consumption across millions of nodes in real-time, integrating renewable sources with varying output.

    \item \textbf{Manufacturing}: Industrial robots, CNC machines, and automated production lines require precise coordination of multiple axes and processes.

    \item \textbf{Building systems}: HVAC control, lighting, security systems increasingly operate as integrated CPS, optimizing energy use while maintaining comfort.
\end{itemize}

\subsection{The Interdisciplinary Challenge}

CPS exist at the intersection of multiple engineering disciplines, as illustrated in Figure~\ref{fig:cps-intersection}.

\begin{figure}[htbp]
\centering
\begin{tikzpicture}[scale=0.9]
    % Three overlapping circles
    \begin{scope}[blend group=soft light]
        \fill[red!30] (0,0) circle (2.5cm);
        \fill[blue!30] (2.2,0) circle (2.5cm);
        \fill[green!30] (1.1,1.9) circle (2.5cm);
    \end{scope}

    % Labels outside
    \node[font=\bfseries] at (-1.5, -1.5) {Control};
    \node[font=\bfseries] at (3.7, -1.5) {Computer};
    \node[font=\bfseries] at (-1.5, -1.9) {Engineering};
    \node[font=\bfseries] at (3.7, -1.9) {Science};
    \node[font=\bfseries] at (1.1, 3.8) {Electrical};
    \node[font=\bfseries] at (1.1, 3.4) {Engineering};

    % Topics in each region
    \node[font=\footnotesize, align=center] at (-0.8, 0.3) {Feedback\\Stability\\Observers};
    \node[font=\footnotesize, align=center] at (3, 0.3) {Algorithms\\Scheduling\\Software};
    \node[font=\footnotesize, align=center] at (1.1, 2.5) {Sensors\\Circuits\\Signals};

    % CPS in the center
    \node[font=\bfseries\large, fill=white, rounded corners, inner sep=3pt] at (1.1, 0.6) {CPS};

    % Pairwise intersections
    \node[font=\scriptsize, align=center] at (0.3, 1.5) {Embedded\\Control};
    \node[font=\scriptsize, align=center] at (1.9, 1.5) {Real-Time\\Systems};
    \node[font=\scriptsize, align=center] at (1.1, -0.6) {Software\\Control};
\end{tikzpicture}
\caption{Cyber-physical systems emerge at the intersection of control engineering, computer science, and electrical engineering. Effective CPS development requires fluency across these boundaries.}
\label{fig:cps-intersection}
\end{figure}

Historically, these disciplines developed largely independently, each with its own assumptions, tools, and culture:

\begin{itemize}
    \item \textbf{Control engineering} assumes continuous-time dynamics, ideal sensors, and instantaneous computation. Analysis focuses on stability, performance, and robustness using differential equations and transfer functions.

    \item \textbf{Computer science} abstracts away physical timing, focusing on logical correctness, computational complexity, and software architecture. Real-time is often treated as a specialized concern.

    \item \textbf{Electrical engineering} deals with analog signals, noise, power constraints, and hardware interfaces. The focus is on circuits, signal integrity, and electromagnetic compatibility.
\end{itemize}

CPS engineering requires integrating insights from all three disciplines. This is more than superficial knowledge---it requires understanding how decisions in one domain affect the others. Choosing a faster sampling rate (control decision) increases computational load (software impact) and power consumption (hardware impact). Using a cheaper sensor (hardware decision) increases noise (signal processing impact) and may require a more conservative controller (control impact).

\begin{notebox}
This course is designed to develop fluency across these disciplines. You will learn not just the techniques from each field, but how to reason about their interactions when building real systems.
\end{notebox}

\section{Model-Based Development}
\index{model-based development|(}

\subsection{What is Model-Based Development?}

Traditionally, engineering development follows a document-centric approach: requirements are written in documents, designs are described in documents, and implementation is done by programmers reading those documents. This creates several problems:

\begin{itemize}
    \item Documents are ambiguous and incomplete
    \item Translation from documents to code introduces errors
    \item There is no automated way to verify consistency
    \item Changes require updating multiple documents and code manually
\end{itemize}

\textbf{Model-based development}\index{model-based development|textbf} (MBD) addresses these problems by making executable models---not documents---the central artifacts throughout the development process.

\begin{definition}[Model-Based Development]
\textbf{Model-based development} is a software and systems engineering approach where mathematical and computational models serve as the primary artifacts for specifying, analyzing, designing, and implementing systems. Models are not just documentation---they are executable specifications that can be simulated, analyzed, and automatically translated into implementation code.
\end{definition}

The key insight is that a model is both a specification and an implementation draft. When you create a Simulink\index{Simulink} model of an attitude controller, you are simultaneously:

\begin{enumerate}
    \item Specifying what the controller should do (the mathematical relationship between inputs and outputs)
    \item Creating something that can be simulated to verify behavior
    \item Providing a source from which implementation code can be generated
\end{enumerate}

\subsection{The Model-Based Development Workflow}

Figure~\ref{fig:mbd-workflow} shows the typical workflow in model-based development.

\begin{figure}[htbp]
\centering
\begin{tikzpicture}[
    node distance=2cm,
    stage/.style={draw, rectangle, rounded corners, minimum width=2.8cm, minimum height=1.2cm, align=center, font=\small},
    arrow/.style={-{Stealth}, thick}
]
    % Stages
    \node[stage, fill=blue!15] (req) {Requirements\\Capture};
    \node[stage, fill=green!15, right of=req, xshift=1.5cm] (model) {System\\Modeling};
    \node[stage, fill=yellow!15, right of=model, xshift=1.5cm] (sim) {Simulation \&\\Analysis};
    \node[stage, fill=orange!15, right of=sim, xshift=1.5cm] (codegen) {Code\\Generation};
    \node[stage, fill=red!15, right of=codegen, xshift=1.5cm] (test) {Testing \&\\Verification};

    % Forward arrows
    \draw[arrow] (req) -- (model);
    \draw[arrow] (model) -- (sim);
    \draw[arrow] (sim) -- (codegen);
    \draw[arrow] (codegen) -- (test);

\end{tikzpicture}
\caption{The model-based development workflow. Models created early in development are refined and eventually used for code generation.}
\label{fig:mbd-workflow}
\end{figure}

\begin{enumerate}
    \item \textbf{Requirements Capture}: System requirements are captured in a structured form, often linked directly to model elements. Requirements traceability is maintained throughout development.

    \item \textbf{System Modeling}: Engineers create mathematical models of the plant (physical system) and the controller. These models capture the essential dynamics and control logic.

    \item \textbf{Simulation and Analysis}: Models are simulated to verify behavior before any hardware exists. Analysis tools check stability, performance, and safety properties.

    \item \textbf{Code Generation}: Production code is automatically generated from verified models. The generated code is structurally similar to the model, maintaining traceability.

    \item \textbf{Testing and Verification}: Generated code is tested at multiple levels---from software-in-the-loop (same code, simulated plant) to hardware-in-the-loop (real hardware, simulated environment) to full system testing.
\end{enumerate}

Crucially, the model remains the single source of truth throughout this process. When requirements change, engineers update the model, re-simulate, re-generate code, and re-test. The model ensures consistency across all stages.

\subsection{The V-Model and Testing Levels}

Model-based development is often organized using the \textbf{V-model}\index{V-model}, a development framework that pairs each design phase with a corresponding verification phase. Understanding the V-model is essential for planning verification activities and for understanding the testing levels used throughout this course.

\paragraph{The V-Model Framework}
The V-model gets its name from its shape: design activities descend on the left side, implementation occurs at the bottom, and verification activities ascend on the right side. Each design phase on the left has a corresponding test phase on the right:

\begin{center}
\begin{tikzpicture}[
    every node/.style={font=\small},
    phase/.style={draw, rectangle, rounded corners, minimum width=2.8cm, minimum height=0.7cm, align=center},
    arrow/.style={-{Stealth}, thick}
]
    % Left side - Design
    \node[phase, fill=blue!20] (req) at (0, 3) {Requirements};
    \node[phase, fill=blue!15] (arch) at (-1.5, 2) {Architecture};
    \node[phase, fill=blue!10] (design) at (-3, 1) {Detailed Design};
    \node[phase, fill=yellow!20] (impl) at (-3, 0) {Implementation};

    % Right side - Verification
    \node[phase, fill=green!20] (accept) at (6, 3) {Acceptance Test};
    \node[phase, fill=green!15] (sys) at (4.5, 2) {System Test};
    \node[phase, fill=green!10] (int) at (3, 1) {Integration Test};
    \node[phase, fill=orange!20] (unit) at (3, 0) {Unit Test};

    % Design flow
    \draw[arrow] (req) -- (arch);
    \draw[arrow] (arch) -- (design);
    \draw[arrow] (design) -- (impl);

    % Implementation to test
    \draw[arrow] (impl) -- (unit);

    % Test flow
    \draw[arrow] (unit) -- (int);
    \draw[arrow] (int) -- (sys);
    \draw[arrow] (sys) -- (accept);

    % Horizontal traceability
    \draw[dashed, gray] (req) -- (accept);
    \draw[dashed, gray] (arch) -- (sys);
    \draw[dashed, gray] (design) -- (int);

    % Labels
    \node[left of=design, xshift=-1cm, font=\footnotesize\itshape] {Design};
    \node[right of=int, xshift=1cm, font=\footnotesize\itshape] {Verify};
\end{tikzpicture}
\end{center}

The dashed horizontal lines show \emph{traceability}: requirements are verified by acceptance tests, architecture by system tests, and detailed design by integration tests. This structure ensures that every design decision has a corresponding verification activity.

\paragraph{Why the V-Model for CPS?}
The V-model is particularly valuable for cyber-physical systems:

\begin{itemize}
    \item \textbf{Traceability}: Each test verifies a specific design artifact. When a test fails, you know which design element is problematic.
    \item \textbf{Early planning}: Test strategies are defined during design, not as an afterthought. This prevents discovering untestable requirements late in development.
    \item \textbf{Certification}: Safety standards like DO-178C\index{DO-178C} (aerospace) and ISO~26262\index{ISO 26262} (automotive) require evidence of systematic verification at each level.
\end{itemize}

\paragraph{Testing Levels for Model-Based Development}
For model-based CPS development, the V-model's verification side is implemented through a progression of testing levels. Each level increases the amount of real hardware involved, trading speed for fidelity:

\begin{center}
\begin{tikzpicture}[
    every node/.style={font=\small},
    level/.style={draw, rectangle, rounded corners, minimum width=1.6cm, minimum height=1.4cm, align=center},
    arrow/.style={-{Stealth}, thick}
]
    % Testing levels
    \node[level, fill=blue!20] (mil) at (0, 0) {\textbf{MIL}\\Model-in-\\the-Loop};
    \node[level, fill=green!20] (sil) at (2.8, 0) {\textbf{SIL}\\Software-in-\\the-Loop};
    \node[level, fill=yellow!20] (pil) at (5.6, 0) {\textbf{PIL}\\Processor-in-\\the-Loop};
    \node[level, fill=orange!20] (hil) at (8.4, 0) {\textbf{HIL}\\Hardware-in-\\the-Loop};
    \node[level, fill=red!20] (vil) at (11.2, 0) {\textbf{VIL}\\Vehicle-in-\\the-Loop};

    % Arrows
    \draw[arrow] (mil) -- (sil);
    \draw[arrow] (sil) -- (pil);
    \draw[arrow] (pil) -- (hil);
    \draw[arrow] (hil) -- (vil);

    % Labels below - positioned clearly below the boxes
    \node[below of=mil, yshift=-0.3cm, font=\scriptsize, align=center] {All\\simulated};
    \node[below of=sil, yshift=-0.3cm, font=\scriptsize, align=center] {Code on\\host};
    \node[below of=pil, yshift=-0.3cm, font=\scriptsize, align=center] {Code on\\target};
    \node[below of=hil, yshift=-0.3cm, font=\scriptsize, align=center] {Full\\hardware};
    \node[below of=vil, yshift=-0.3cm, font=\scriptsize, align=center] {Real\\environment};

    % "More hardware" arrow spanning above all boxes
    \draw[arrow] (0, 1.5) -- (11.2, 1.5) node[midway, above, font=\small] {More hardware};
\end{tikzpicture}
\end{center}

\textbf{MIL (Model-in-the-Loop)}\index{MIL (Model-in-the-Loop)}: Both the controller and the plant are Simulink models running on the development computer. MIL tests verify that the control algorithm meets requirements before any code is generated.

\emph{Quadrotor example}: Test that the attitude controller Simulink model achieves less than 5\% overshoot in response to a step command.

\textbf{SIL (Software-in-the-Loop)}\index{SIL (Software-in-the-Loop)}: The controller is now generated C code, compiled for the host computer (e.g., x86). The plant remains a simulation. SIL tests verify that code generation preserved the model's behavior.

\emph{Quadrotor example}: Compare the output of \texttt{attitude\_ctrl.c} running on your laptop against the Simulink model for the same inputs.

\textbf{PIL (Processor-in-the-Loop)}\index{PIL (Processor-in-the-Loop)}: The controller code runs on the actual target processor (e.g., STM32), but the plant is still simulated on the host. PIL tests catch target-specific issues like numerical precision, stack overflow, and timing.

\emph{Quadrotor example}: Run the attitude controller on the Crazyflie's microcontroller while the quadrotor dynamics are simulated on a PC connected via serial link.

\textbf{HIL (Hardware-in-the-Loop)}\index{HIL (Hardware-in-the-Loop)}: The complete embedded system (controller plus I/O hardware) runs in real-time, but the physical plant is replaced by a real-time simulator. HIL tests verify hardware interfaces, interrupt handling, and real-time behavior.

\emph{Quadrotor example}: The Crazyflie firmware runs on actual hardware, but instead of real sensors, a HIL simulator provides simulated IMU readings and captures motor commands.

\textbf{VIL (Vehicle-in-the-Loop)}\index{VIL (Vehicle-in-the-Loop)}: The complete system operates in a controlled real environment. For ground vehicles, this might be a test track; for aircraft, a flight test range.

\emph{Quadrotor example}: The Crazyflie flies in a motion-capture room where position can be precisely measured and safety nets catch failures.

\paragraph{Choosing the Right Testing Level}
Each level offers different tradeoffs:

\begin{center}
\begin{tabular}{llllp{3.5cm}}
\toprule
\textbf{Level} & \textbf{Controller} & \textbf{Plant} & \textbf{Speed} & \textbf{What It Catches} \\
\midrule
MIL & Model & Model & Fastest & Algorithm errors \\
SIL & Code & Model & Fast & Code generation errors \\
PIL & Code & Model & Medium & Target-specific issues \\
HIL & Code + HW & Simulator & Slow & Integration issues \\
VIL & Full system & Real & Slowest & System-level issues \\
\bottomrule
\end{tabular}
\end{center}

The principle is to find bugs at the earliest (cheapest) level possible. An algorithm error should be caught in MIL, not discovered during HIL testing when setup takes hours. But some bugs---like race conditions in interrupt handlers---only appear when real hardware is involved.

\begin{keyidea}
Each testing level adds realism at the cost of speed and convenience. MIL tests run in seconds and require no hardware; HIL tests require physical setup and run in real-time. Use the right level for each verification need, and automate as much as possible.
\end{keyidea}

These testing levels integrate directly into CI/CD pipelines, as described in the next section. Module~4 covers each testing level in detail, including formal requirements specification using temporal logic and falsification-based testing techniques.

\subsection{Benefits of Model-Based Development}

Industry experience and research studies have documented significant benefits from adopting MBD:

\paragraph{Early Error Detection}
Simulation allows testing designs before hardware exists. Errors caught in simulation are orders of magnitude cheaper to fix than errors found during integration or field testing. A study by the National Institute of Standards and Technology estimated that software errors cost the U.S. economy \$59.5 billion annually, with a large fraction attributed to errors caught late in development.

\paragraph{Automatic Code Generation}
\index{code generation}
Code generated from models is correct by construction---if the model is correct and the code generator is qualified, the code will implement the intended behavior. This eliminates the translation errors that occur when programmers manually implement specifications.

\paragraph{Traceability}
\index{traceability}
Every line of generated code traces back to a model element, which traces back to a requirement. This traceability is essential for safety-critical systems and is required by standards such as DO-178C for aerospace and ISO~26262 for automotive.

\paragraph{Reuse}
Models can be reused across projects and platforms. A controller model developed for one quadrotor can be adapted for another with different dynamics. Libraries of verified components accelerate development.

\paragraph{Formal Verification}
Mathematical models enable formal analysis techniques that can prove properties about system behavior. While full formal verification remains challenging for complex systems, partial verification of critical properties is increasingly practical.

\paragraph{Communication}
Models provide a common language for multidisciplinary teams. Control engineers, software developers, and test engineers can all work with the same model, reducing miscommunication.

\begin{keyidea}
Model-based development shifts effort from coding to modeling. Since models are more abstract and can be analyzed before implementation, errors are found earlier when they are cheaper to fix.
\end{keyidea}

\subsection{Challenges and Limitations}

Despite its benefits, model-based development is not without challenges:

\paragraph{Tool Costs and Learning Curve}
Commercial MBD tools (MATLAB/Simulink, dSPACE, ETAS) are expensive, and mastering them requires significant training. Organizations must invest in tools, training, and process changes.

\paragraph{The Model-Reality Gap}
All models are simplifications of reality. A model that works perfectly in simulation may fail when confronted with unmodeled dynamics, sensor noise, or environmental disturbances. The famous aphorism applies: ``All models are wrong, but some are useful.''

\begin{warningbox}
Never trust simulation results blindly. Models always omit some aspects of reality. Verification must include testing on real hardware under realistic conditions.
\end{warningbox}

\paragraph{Over-Reliance on Simulation}
Teams can become overconfident in simulation results, skipping hardware testing that would reveal model limitations. The Mars Climate Orbiter was lost partly because ground testing relied on simulation rather than verifying actual software behavior.

\paragraph{Tool Qualification}
For safety-critical systems, the tools themselves must be qualified. If a code generator has bugs, generated code may be incorrect even if the model is correct. Standards like DO-330 (Tool Qualification) address this, but qualification is expensive and complex.

\paragraph{Legacy Integration}
Most organizations have existing codebases, tools, and processes. Integrating model-based approaches with legacy systems requires careful planning and often hybrid workflows.

\paragraph{Model Maintenance}
Models must be maintained along with code. If models become outdated, they lose their value as specifications. Organizations need processes to ensure models remain synchronized with the deployed system.

\subsection{Industry Adoption and Trends}

Model-based development has moved from research curiosity to industry standard over the past two decades:

\paragraph{Automotive Industry}
The automotive industry has been a pioneer in MBD adoption. Major automakers and suppliers use Simulink, ASCET, and similar tools for engine control, chassis systems, and ADAS (Advanced Driver Assistance Systems). ISO~26262, the functional safety standard for road vehicles, explicitly supports model-based development with requirements for model verification and code generation qualification.

A study by the Aberdeen Group found that automotive companies using MBD achieved 50\% fewer software defects and 35\% shorter development cycles compared to traditional approaches.

\paragraph{Aerospace Industry}
DO-178C, the software certification standard for airborne systems, includes DO-331 as a model-based development supplement. This allows models to serve as design representations, with appropriate verification objectives. Major programs including the Boeing 787 and Airbus A350 have used MBD extensively.

\paragraph{Industrial Automation}
IEC~61131-3, the standard for programmable logic controllers, supports model-based approaches. Industry 4.0 initiatives increasingly rely on digital twins---simulation models that run alongside physical systems for monitoring and optimization.

\paragraph{Medical Devices}
IEC~62304, the software lifecycle standard for medical devices, is compatible with MBD approaches. The FDA has issued guidance on the use of models in medical device development and is increasingly open to model-based evidence for regulatory approval.

\paragraph{Emerging Trends}
Several trends are shaping the future of model-based development:

\begin{itemize}
    \item \textbf{Digital twins}: Simulation models that run in parallel with physical systems, enabling predictive maintenance and what-if analysis.

    \item \textbf{AI integration}: Machine learning components are being integrated into MBD workflows, with challenges around verification and validation of learned components.

    \item \textbf{Cloud-based simulation}: Massive simulation campaigns running in the cloud enable more thorough verification than was previously practical.

    \item \textbf{Open-source tools}: Tools like Modelica, OpenModelica, and FMI (Functional Mock-up Interface) are lowering barriers to MBD adoption.
\end{itemize}

\subsection{Continuous Integration and Deployment}

While model-based development provides the methodology for building CPS, \emph{continuous integration and continuous deployment}\index{continuous integration}\index{continuous deployment}\index{CI/CD|see {continuous integration}} (CI/CD) provides the automation infrastructure that makes MBD practical at scale. CI/CD practices, originally developed for web applications, are increasingly essential for embedded systems and CPS development.

\paragraph{What is CI/CD?}
\textbf{Continuous Integration} (CI) is the practice of automatically building and testing software whenever changes are committed to version control. Rather than integrating changes weekly or monthly, developers integrate continuously---often multiple times per day. Each integration triggers automated checks that catch problems immediately.

\textbf{Continuous Deployment} (CD) extends this by automatically deploying verified changes to production systems. For CPS, ``deployment'' might mean updating firmware on a test fleet or releasing a new software version to manufacturing.

The key insight is automation: tasks that humans perform inconsistently (and slowly) are performed by machines consistently (and quickly). This is particularly valuable for CPS, where the build-test cycle involves multiple tools and long simulation runs.

\paragraph{CI/CD for Model-Based Development}
A CI/CD pipeline for model-based CPS development typically includes the following stages:

\begin{center}
\begin{tikzpicture}[
    node distance=0.3cm,
    stage/.style={draw, rectangle, rounded corners, minimum width=1.8cm, minimum height=0.9cm, align=center, font=\scriptsize},
    arrow/.style={-{Stealth}, thick},
    gate/.style={draw, diamond, minimum width=0.5cm, minimum height=0.5cm, inner sep=1pt, fill=green!30}
]
    % Stages
    \node[stage, fill=blue!20] (commit) {Model\\Commit};
    \node[stage, fill=blue!10, right=of commit] (static) {Static\\Analysis};
    \node[stage, fill=yellow!20, right=of static] (codegen) {Code\\Generation};
    \node[stage, fill=yellow!10, right=of codegen] (compile) {Cross\\Compile};
    \node[stage, fill=orange!20, right=of compile] (mil) {MIL\\Tests};
    \node[stage, fill=orange!30, right=of mil] (sil) {SIL\\Tests};
    \node[stage, fill=red!20, right=of sil] (hil) {HIL\\Tests};
    \node[stage, fill=green!20, right=of hil] (deploy) {Deploy};

    % Arrows with gates
    \draw[arrow] (commit) -- (static);
    \draw[arrow] (static) -- (codegen);
    \draw[arrow] (codegen) -- (compile);
    \draw[arrow] (compile) -- (mil);
    \draw[arrow] (mil) -- (sil);
    \draw[arrow] (sil) -- (hil);
    \draw[arrow] (hil) -- (deploy);

    % Pass/fail indicators
    \node[below=0.4cm of static, font=\tiny, green!50!black] {pass?};
    \node[below=0.4cm of mil, font=\tiny, green!50!black] {pass?};
    \node[below=0.4cm of sil, font=\tiny, green!50!black] {pass?};
    \node[below=0.4cm of hil, font=\tiny, green!50!black] {pass?};
\end{tikzpicture}
\end{center}

Each stage acts as a quality gate:

\begin{enumerate}
    \item \textbf{Model Commit}: Developer commits Simulink model changes to version control (Git)
    \item \textbf{Static Analysis}: Automated checks verify model guidelines, naming conventions, and coding standards (e.g., MAAB guidelines, MISRA C)
    \item \textbf{Code Generation}: Simulink Coder or Embedded Coder generates C code from the model
    \item \textbf{Cross Compilation}: Generated code compiles for the target platform (e.g., ARM Cortex-M4)
    \item \textbf{MIL Tests}: Model-in-the-Loop tests verify model behavior against requirements
    \item \textbf{SIL Tests}: Software-in-the-Loop tests run generated code on the host, comparing results to model
    \item \textbf{HIL Tests}: Hardware-in-the-Loop tests run code on actual target hardware with simulated plant
    \item \textbf{Deploy}: Verified firmware is packaged and released
\end{enumerate}

If any stage fails, the pipeline stops and notifies the developer. This prevents broken changes from propagating downstream.

\paragraph{Benefits for CPS Development}
CI/CD provides several advantages for CPS projects:

\begin{itemize}
    \item \textbf{Immediate feedback}: Developers learn about problems within minutes of committing, while the changes are fresh in mind
    \item \textbf{Reproducibility}: The same tests run in the same environment every time, eliminating ``works on my machine'' problems
    \item \textbf{Regression prevention}: Automated tests catch when new changes break existing functionality
    \item \textbf{Certification evidence}: Pipeline logs provide traceability records showing that all verification steps were performed
    \item \textbf{Faster iterations}: Automated processes that previously took hours of manual work complete in minutes
\end{itemize}

\paragraph{CPS-Specific Challenges}
Applying CI/CD to cyber-physical systems presents unique challenges:

\begin{itemize}
    \item \textbf{Hardware-in-the-loop infrastructure}: HIL testing requires physical hardware. Organizations build ``HIL farms''---racks of target boards connected to CI servers---to run hardware tests automatically.

    \item \textbf{Long simulation times}: Some CPS simulations take hours to complete. Strategies include parallelization, reduced test suites for fast feedback, and nightly full test runs.

    \item \textbf{Target hardware availability}: Embedded targets may be scarce or expensive. CI systems must manage hardware resources and queue jobs appropriately.

    \item \textbf{Tool qualification}: For safety-critical systems, CI tools may need qualification under standards like DO-330. This adds complexity but is increasingly supported by tool vendors.
\end{itemize}

\paragraph{Example: Quadrotor Firmware Pipeline}
Consider a developer modifying the attitude controller in the Crazyflie firmware:

\begin{enumerate}
    \item Developer updates the PID gains in the Simulink model and commits to Git
    \item CI server detects the commit and starts the pipeline
    \item Static analysis checks pass (no guideline violations)
    \item Code generation produces updated \texttt{attitude\_ctrl.c}
    \item Cross-compilation succeeds for STM32F405
    \item MIL tests verify step response meets requirements
    \item SIL tests confirm generated code matches model behavior
    \item HIL tests run the code on an actual Crazyflie, verifying stable hover
    \item All tests pass; firmware binary is tagged and stored
    \item Developer receives ``pipeline passed'' notification within 15 minutes
\end{enumerate}

Without CI/CD, this verification might take a day of manual work and could easily be skipped under schedule pressure. With CI/CD, it happens automatically on every change.

\begin{keyidea}
CI/CD transforms model-based development from a methodology into an automated, repeatable process. Every change is validated against the same tests, ensuring consistent quality and providing evidence for certification.
\end{keyidea}
\index{model-based development|)}

\section{Why Control Engineers Need Real-Time Systems Knowledge}
\index{control engineers!real-time knowledge}

Control theory typically assumes ideal conditions: sensors provide instantaneous, noise-free measurements; control laws compute instantly; actuators respond immediately. These assumptions are mathematically convenient but physically impossible. Every real implementation introduces delays, and those delays affect stability and performance.

\subsection{The Control-Implementation Gap}

Consider a simple discrete-time controller designed for a 500~Hz sampling rate ($T_s = 2$~ms). The control design assumes:

\begin{enumerate}
    \item Sensors are sampled exactly every 2~ms
    \item The control computation completes instantly
    \item Actuator commands are applied immediately after computation
\end{enumerate}

In reality, none of these assumptions hold precisely:

\begin{itemize}
    \item \textbf{Sampling jitter}\index{jitter}: The actual sampling interval varies (1.9~ms, 2.1~ms, 2.0~ms, ...) due to interrupt latency, scheduling decisions, and hardware timing.

    \item \textbf{Computational delay}: The control algorithm takes time to execute---perhaps 100--500~$\mu$s on a typical microcontroller.

    \item \textbf{Actuation delay}: PWM updates may be synchronized to a different timer, introducing additional delay.
\end{itemize}

These timing imperfections have consequences. Jitter effectively adds noise to the control loop, degrading performance and potentially exciting high-frequency dynamics. Computational delay adds phase lag, reducing phase margin and potentially causing instability.

A controller designed with 10$^\circ$ of phase margin may become unstable if computational delay consumes that margin. The control engineer who ignores implementation effects is designing for a system that does not exist.

\subsection{What You Cannot Delegate to Software Engineers}

A natural reaction might be: ``Let the software engineers handle the timing. I'll focus on the control algorithm.'' This division of labor fails for several reasons:

\paragraph{Software Engineers Cannot Assign Priorities Correctly}
In a real-time system, tasks must be assigned priorities to ensure critical computations meet their deadlines. But which tasks are critical? A software engineer sees that attitude control runs at 500~Hz and position control at 100~Hz. Without understanding the control system, they might reason that both are ``control'' and assign equal priorities, or prioritize position control because it seems more important for the mission.

The control engineer knows that a missed attitude control deadline can crash the quadrotor in milliseconds, while a missed position control deadline causes gradual drift. This knowledge---which comes from understanding the dynamics, not from looking at code---is essential for correct priority assignment.

\paragraph{Software Engineers Cannot Evaluate Deadline Importance}
Not all deadlines are equally important. Some missed deadlines cause immediate catastrophe; others cause graceful degradation. The control engineer knows which is which based on understanding the physical system's time constants and stability margins.

\paragraph{Software Engineers Cannot Design for Sensor Fusion Requirements}
\index{sensor fusion}
Sensor fusion algorithms depend on timing. If IMU data is delayed relative to barometer data, the fusion algorithm produces incorrect estimates. The control engineer understands these timing dependencies; they emerge from the physical and mathematical properties of the estimation algorithm, not from the code structure.

\begin{warningbox}
\textbf{The Mars Polar Lander Failure (1999)}: During descent, vibration from the landing legs deploying generated spurious signals on the touchdown sensors. The software interpreted these as actual touchdown and shut off the descent engines while the lander was still 40 meters above the surface. A software engineer might have implemented the touchdown logic correctly; only someone who understood the physical system would have anticipated the spurious signals.
\end{warningbox}

\subsection{The Minimum Knowledge Set}

Control engineers working on cyber-physical systems need not become real-time systems experts, but they must understand certain core concepts:

\begin{itemize}
    \item \textbf{Tasks and scheduling}: What is a task? How does the scheduler decide which task runs? What is preemption?

    \item \textbf{Priorities and rate-monotonic scheduling}: How should priorities be assigned? Why does the highest-rate task typically get highest priority?

    \item \textbf{Worst-case execution time (WCET)}\index{worst-case execution time}: What is the longest time a task can take? How is this measured or analyzed? Why does average-case analysis fail?

    \item \textbf{Schedulability analysis}: How do we know if all tasks will meet their deadlines? What is the utilization bound? What is response-time analysis?

    \item \textbf{Priority inversion}\index{priority inversion}: What happens when a high-priority task is blocked by a low-priority task? What is priority inheritance?

    \item \textbf{Multi-rate systems}: How do tasks at different rates exchange data safely? What are rate transition blocks?
\end{itemize}

Module~3 of this course covers these topics in detail, specifically in the context of control system implementation.

\section{Why Software Engineers Need Control Knowledge}
\index{software engineers!control knowledge}

The converse problem is equally real: software engineers who implement control systems without understanding control theory make systematic errors that undermine system behavior.

\subsection{Control Systems Are Not Ordinary Software}

Control software differs from typical application software in fundamental ways:

\paragraph{Continuous-Time Origins}
Control algorithms are typically designed in continuous time, then discretized for implementation. The discretization introduces approximation errors that depend on the sampling rate. A software engineer who changes the sampling rate without updating the controller gains is changing the controller's behavior, not just its timing.

\paragraph{Stability Depends on Timing}
In ordinary software, if a computation takes twice as long, it just takes twice as long. In control software, if a computation takes twice as long, the system may become unstable. Timing is not a performance optimization---it is a correctness requirement.

\paragraph{Feedback Loops Have Unique Failure Modes}
\index{feedback loops}
Feedback systems can exhibit behaviors that seem paradoxical to someone unfamiliar with control theory. A well-intentioned ``improvement'' can make things worse:

\begin{itemize}
    \item Increasing controller gain to ``respond faster'' can cause oscillation or instability
    \item Adding filtering to ``smooth'' signals can introduce phase lag that destabilizes the loop
    \item Saturating outputs to ``prevent damage'' can cause integrator windup\index{integrator windup}
\end{itemize}

\paragraph{State Estimation Is Not Optional}
Control systems often cannot measure what they need to control directly. The quadrotor cannot measure its absolute position directly; it must estimate it from noisy, biased, delayed sensor data. This estimation is not a preprocessing step that can be separated from control---the estimation and control are coupled, and errors in estimation propagate to control.

\subsection{Common Software Engineering Mistakes in Control Systems}

Without control systems background, software engineers make predictable mistakes:

\paragraph{Incorrect Sampling Rate Selection}
``The IMU can output data at 1000~Hz, so let's sample at 1000~Hz for best accuracy.'' But the controller was designed for 500~Hz sampling, and doubling the rate changes the controller dynamics. The right answer requires understanding what sampling rate the control design assumes.

\paragraph{Breaking Feedback Loops with Buffering}
``I'll add a buffer to smooth out timing variations.'' But buffering adds delay, and delay in a feedback loop reduces phase margin. The buffer that ``improves'' data flow may destabilize the control loop.

\paragraph{Ignoring Numerical Precision}
Filter coefficients must be represented precisely. A coefficient designed as 0.999 that becomes 0.99 due to fixed-point quantization changes the filter's time constant by a factor of 10. Software engineers trained on applications where 1\% error is acceptable may not recognize that control requires much higher precision in certain quantities.

\paragraph{Improper Initialization}
Control algorithms have state (integrators, filters, estimates). If these states are not initialized correctly, transients can be large. A software engineer might initialize all states to zero, not knowing that the estimator's covariance matrix should be initialized to reflect uncertainty.

\paragraph{Not Understanding Sensor Fusion Requirements}
Sensor fusion depends on knowing the characteristics of each sensor: its noise, bias, delay, and failure modes. Software engineers may treat sensors as black boxes providing ``the measurement,'' not understanding that fusion must account for each sensor's peculiarities.

\subsection{The Integration Challenge}

The solution is not to train all software engineers as control engineers or vice versa. Rather, CPS development requires:

\begin{enumerate}
    \item \textbf{T-shaped engineers}: Professionals who have deep expertise in one area but working knowledge of related areas. A control engineer should understand enough about real-time systems to specify timing requirements correctly and communicate effectively with software engineers. A software engineer should understand enough about control to recognize when implementation decisions affect control behavior.

    \item \textbf{Integrated teams}: Multidisciplinary teams where specialists work together throughout development, not sequential handoffs where control engineers ``throw specifications over the wall'' to software engineers.

    \item \textbf{Common representations}: Tools and notations that are meaningful to all team members. Simulink models, for instance, can be understood by control engineers (as block diagrams) and software engineers (as data flow graphs).

    \item \textbf{End-to-end responsibility}: Engineers who follow their designs from concept through implementation and testing, seeing how their decisions play out in the real system.
\end{enumerate}

\begin{keyidea}
Neither control expertise alone nor software expertise alone is sufficient to build safe, reliable cyber-physical systems. This course develops the cross-disciplinary understanding that effective CPS engineering requires.
\end{keyidea}

\section{Course Roadmap}

This course is organized into four modules that progressively build the knowledge and skills needed to develop cyber-physical systems. Each module corresponds to a major phase of CPS development, and together they span from mathematical modeling to verified implementation.

\subsection{The Four Modules}

Figure~\ref{fig:course-roadmap} shows how the four modules connect.

\begin{figure}[htbp]
\centering
\begin{tikzpicture}[
    module/.style={draw, rectangle, rounded corners, minimum width=3.5cm, minimum height=2cm, align=center, font=\small},
    arrow/.style={-{Stealth}, thick}
]
    % Modules - Module 1,2,3 in a row, Module 4 below center
    \node[module, fill=blue!15] (m1) at (0, 0) {\textbf{Module 1}\\Modeling \&\\Sensor Fusion};
    \node[module, fill=green!15] (m2) at (5, 0) {\textbf{Module 2}\\Simulation \&\\Hybrid Systems};
    \node[module, fill=orange!15] (m3) at (10, 0) {\textbf{Module 3}\\Real-Time\\Implementation};
    \node[module, fill=red!15] (m4) at (5, -4) {\textbf{Module 4}\\Testing \&\\Verification};

    % Arrows between Module 1-2-3 (horizontal)
    \draw[arrow] (m1) -- (m2);
    \draw[arrow] (m2) -- (m3);

    % Arrows to Module 4 - entering from LEFT and RIGHT sides
    \draw[arrow] (m1.south) -- (m1.south |- m4.west) -- (m4.west);
    \draw[arrow] (m3.south) -- (m3.south |- m4.east) -- (m4.east);

    % Content lists ABOVE Module 1, 2, 3 boxes
    \node[above of=m1, yshift=1.2cm, font=\scriptsize, align=left] {
        $\bullet$ Coordinate frames\\
        $\bullet$ Rotations, quaternions\\
        $\bullet$ IMU models\\
        $\bullet$ Attitude estimation\\
        $\bullet$ Quadrotor dynamics
    };

    \node[above of=m2, yshift=1.2cm, font=\scriptsize, align=left] {
        $\bullet$ Equation-based modeling\\
        $\bullet$ Hybrid automata\\
        $\bullet$ Numerical methods\\
        $\bullet$ Zero-crossing detection\\
        $\bullet$ Multi-rate systems
    };

    \node[above of=m3, yshift=1.2cm, font=\scriptsize, align=left] {
        $\bullet$ RTOS concepts\\
        $\bullet$ Task scheduling\\
        $\bullet$ Synchronization\\
        $\bullet$ Timing analysis\\
        $\bullet$ WCET
    };

    % Content list BELOW Module 4 box
    \node[below of=m4, yshift=-1.5cm, font=\scriptsize, align=left] {
        $\bullet$ MIL/SIL/PIL/HIL\\
        $\bullet$ Temporal logic\\
        $\bullet$ Falsification\\
        $\bullet$ Code coverage\\
        $\bullet$ Debugging
    };
\end{tikzpicture}
\caption{Course module organization. Module~1 provides mathematical foundations, Module~2 covers simulation, Module~3 addresses implementation, and Module~4 deals with verification. All modules contribute to the ability to test and verify the complete system.}
\label{fig:course-roadmap}
\end{figure}

\paragraph{Module 1: Modeling and Sensor Fusion}
We begin with the mathematical foundations needed to describe and control a quadrotor. You will learn:
\begin{itemize}
    \item How to represent rotations using rotation matrices, Euler angles, and quaternions
    \item How coordinate frames relate positions and orientations in 3D space
    \item How inertial sensors (accelerometers, gyroscopes, magnetometers) work and what noise they produce
    \item How to fuse noisy sensor data into reliable state estimates
    \item How to derive the equations of motion for a quadrotor
\end{itemize}

By the end of Module~1, you will have a complete mathematical model of the quadrotor system.

\paragraph{Module 2: Simulation and Hybrid Systems}
With models in hand, we turn to simulation. You will learn:
\begin{itemize}
    \item How equation-based (acausal) modeling differs from signal-flow (causal) modeling
    \item How to model systems with both continuous dynamics and discrete events (hybrid systems)
    \item How numerical solvers work and how to choose appropriate methods
    \item How to detect and handle discontinuities in simulation
    \item How to manage multi-rate subsystems
\end{itemize}

By the end of Module~2, you will be able to simulate your quadrotor model accurately.

\paragraph{Module 3: Real-Time Implementation}
Simulation is not enough---we must implement controllers on real hardware. You will learn:
\begin{itemize}
    \item Why real-time operating systems are necessary for control applications
    \item How task scheduling determines when code executes
    \item How to analyze whether tasks will meet their deadlines
    \item How to handle concurrency and synchronization safely
    \item How to measure and bound worst-case execution time
    \item How timing affects control performance and stability
\end{itemize}

By the end of Module~3, you will understand how to implement control algorithms on embedded systems with guaranteed timing.

\paragraph{Module 4: Testing and Verification}
Finally, we must verify that our implementation is correct. You will learn:
\begin{itemize}
    \item Why CPS testing is fundamentally different from traditional software testing
    \item How Model-in-the-Loop, Software-in-the-Loop, and Hardware-in-the-Loop testing work
    \item How to specify requirements formally using temporal logic
    \item How falsification-based testing finds failures efficiently
    \item How code coverage relates to test adequacy
    \item How to debug embedded control systems
\end{itemize}

By the end of Module~4, you will have systematic approaches to verifying CPS correctness.

\subsection{The Quadrotor as Running Example}

Throughout this course, we use the quadrotor as a concrete example. This choice is deliberate:

\begin{itemize}
    \item \textbf{Complexity}: Quadrotors are complex enough to illustrate real CPS challenges---nonlinear dynamics, coupled axes, sensor fusion, multi-rate control---without being overwhelming.

    \item \textbf{Accessibility}: Small quadrotors like the Crazyflie are affordable and safe enough for laboratory work. You can see the consequences of your design decisions flying (or crashing) in front of you.

    \item \textbf{Relevance}: Drones are increasingly important in applications from delivery to inspection to agriculture. The skills you develop transfer directly to industry applications.

    \item \textbf{Motivation}: There is something deeply satisfying about seeing equations translate into a physical object that flies. This tangible feedback reinforces learning.
\end{itemize}

The Crazyflie\index{Crazyflie} platform used in this course is a 27-gram nano quadrotor with an STM32F405\index{STM32F405} microcontroller, IMU, barometer, and various expansion options. It runs open-source firmware, allowing you to see and modify actual flight code.

\subsection{Prerequisites and Approach}

This course assumes background in:
\begin{itemize}
    \item Linear algebra (vectors, matrices, eigenvalues)
    \item Calculus and differential equations
    \item Basic control theory (transfer functions, feedback, stability)
    \item Programming (preferably C and/or Python)
\end{itemize}

The pedagogical approach emphasizes:
\begin{itemize}
    \item \textbf{Theory with practice}: Every theoretical concept is connected to practical implementation
    \item \textbf{Worked examples}: Detailed examples show how to apply concepts
    \item \textbf{Hands-on projects}: You will implement and test algorithms on real hardware
    \item \textbf{Incremental building}: Each module builds on previous ones toward a complete system
\end{itemize}

\section{Summary}

This chapter has introduced the key themes that run throughout the course:

\begin{enumerate}
    \item \textbf{Cyber-physical systems} are characterized by tight coupling between computation and physics, real-time constraints, and safety criticality. They require expertise spanning control, computer science, and electrical engineering.

    \item \textbf{Model-based development} uses executable models as the central artifacts throughout development. This enables early error detection, automatic code generation, and systematic verification, though it requires investment in tools and processes.

    \item \textbf{Control engineers need real-time systems knowledge} because timing affects stability, and software engineers cannot correctly prioritize tasks or evaluate timing requirements without understanding the control system.

    \item \textbf{Software engineers need control knowledge} because control systems have different correctness requirements than typical software, and well-intentioned implementation decisions can break control behavior.

    \item \textbf{This course bridges disciplines} by developing the cross-functional understanding that effective CPS engineering requires, using the quadrotor as a concrete and motivating example.
\end{enumerate}

With this foundation, we are ready to begin building our quadrotor system. Module~1 starts with the mathematical tools needed to describe motion in three dimensions.
\index{cyber-physical systems|)}


% ===== MODULE 1 =====
\part{Module 1: Quadrotor Modelling and Sensor Fusion}
%   - Coordinate Frames and 3D Transformations
%   - Inertial Sensors and Measurement Models
%   - Orientation Estimation
%   - Quadrotor Dynamics
%======================================================================
% MODULE 1: Modelling and Sensor Fusion
%
% Learning Objectives:
% After completing this module, students will be able to:
% - Represent 3D orientations using rotation matrices, Euler angles, and quaternions
% - Explain why quaternions are preferred in flight control systems
% - Model IMU sensor behavior including noise and bias
% - Design and implement complementary filters for orientation estimation
% - Derive and linearize quadrotor dynamics for control design
%======================================================================

\chapter{Coordinate Frames and 3D Transformations}

%----------------------------------------------------------------------
% FIGURE: Quadrotor with World and Body Frames
%----------------------------------------------------------------------
\begin{figure}[htbp]
\centering
\begin{tikzpicture}[scale=1.0,
    % Define 3D view angles
    x={(0.866cm,-0.25cm)},
    y={(0.866cm,0.25cm)},
    z={(0cm,0.8cm)}
]
    % Ground grid
    \fill[gray!10] (-2,-2,0) -- (4,-2,0) -- (4,4,0) -- (-2,4,0) -- cycle;
    \foreach \i in {-2,-1,...,4} {
        \draw[gray!30, thin] (\i,-2,0) -- (\i,4,0);
        \draw[gray!30, thin] (-2,\i,0) -- (4,\i,0);
    }

    % World frame {W} at origin
    \draw[xaxisstyle] (0,0,0) -- (2.5,0,0) node[below right] {$x_W$ (North)};
    \draw[yaxisstyle] (0,0,0) -- (0,2.5,0) node[below right] {$y_W$ (East)};
    \draw[zaxisstyle] (0,0,0) -- (0,0,-1.8) node[left] {$z_W$ (Down)};
    \node[below left] at (0,0,0) {$\{W\}$};

    % Quadrotor position (elevated and tilted)
    \begin{scope}[shift={(2,2,3)}, rotate around z=15, rotate around x=-12]
        % Quadrotor body (simplified X shape)
        \fill[black!20] (-0.15,-0.15,0) -- (0.15,-0.15,0) -- (0.15,0.15,0) -- (-0.15,0.15,0) -- cycle;

        % Arms
        \draw[thick, black!60] (-1.2,0,0) -- (1.2,0,0);
        \draw[thick, black!60] (0,-1.2,0) -- (0,1.2,0);

        % Motors (circles at arm ends)
        \fill[gray!50] (1.2,0,0) circle (0.2);
        \fill[gray!50] (-1.2,0,0) circle (0.2);
        \fill[gray!50] (0,1.2,0) circle (0.2);
        \fill[gray!50] (0,-1.2,0) circle (0.2);

        % Nose indicator
        \fill[red!70] (1.2,0,0.1) -- (1.4,0,0) -- (1.2,0,-0.1) -- cycle;

        % Body frame {B} (dashed)
        \draw[xaxisstyle, dashed] (0,0,0) -- (1.8,0,0) node[right] {$x_B$};
        \draw[yaxisstyle, dashed] (0,0,0) -- (0,1.8,0) node[above] {$y_B$};
        \draw[zaxisstyle, dashed] (0,0,0) -- (0,0,-1.2) node[below left] {$z_B$};
        \node[above] at (0.,0.,0) {$\{B\}$};
    \end{scope}

    % Connecting line (position vector, optional)
%    \draw[dotted, gray] (0,0,0) -- (2,2,3);

    % Annotations
    \node[annot, text width=3.5cm, align=center] at (6,1,3) {Body frame moves\\with the quadrotor};
    \node[annot, text width=3cm, align=center] at (-4,0,0) {World frame\\stays fixed};
\end{tikzpicture}
\caption{A quadrotor with two coordinate frames: the world frame $\{W\}$ fixed to the ground (NED convention), and the body frame $\{B\}$ attached to the vehicle. The body frame is tilted relative to the world frame, illustrating that transformations between frames are needed for control.}
\label{fig:world-body-frames}
\end{figure}


\section{Introduction: Why Study 3D Transformations?}

Imagine you want to fly a quadrotor from point A to point B. This seemingly simple task requires answering several fundamental questions:

\begin{itemize}
    \item \textbf{Where is the quadrotor?} We need to describe its position in 3D space.
    \item \textbf{Which way is it pointing?} We need to describe its orientation.
    \item \textbf{How do we measure these?} Sensors like accelerometers and gyroscopes give us data, but in what reference frame?
    \item \textbf{How do we command motors?} The control system must translate desired motion into motor commands, which requires understanding how the quadrotor's orientation affects its motion.
\end{itemize}

All of these questions require a precise mathematical language for describing \emph{position} and \emph{orientation} in 3D space. This chapter develops that language.

\begin{keyidea}[title=The Central Problem]
A quadrotor's sensors measure quantities in the \textbf{body frame} (attached to the quadrotor), but we want to control its motion in the \textbf{world frame} (fixed to the ground). Transforming between these frames is the mathematical heart of quadrotor control.
\end{keyidea}

\section{Coordinate Frames}

\subsection{Coordinate Conventions: ENU and NED}

Two Earth-fixed coordinate conventions are commonly used in robotics and aerospace:
\emph{ENU (East--North--Up)}\index{ENU (East-North-Up)} and \emph{NED (North--East--Down)}\index{NED (North-East-Down)}. 
Both define Cartesian coordinate frames attached to the local tangent plane of the Earth, and both are \emph{right-handed}.

In the \textbf{ENU} convention, the coordinate axes are defined as
\[
x \text{ (East)}, \quad y \text{ (North)}, \quad z \text{ (Up)} .
\]
ENU is widely used in robotics, computer vision, and simulation environments, as it aligns with standard Cartesian geometry and is intuitive for visualization, mapping, and perception tasks.

In the \textbf{NED} convention, the coordinate axes are defined as
\[
x \text{ (North)}, \quad y \text{ (East)}, \quad z \text{ (Down)} .
\]
NED is the dominant convention in aerospace and navigation and is commonly used for aircraft and quadrotor modeling, inertial navigation systems, and flight-control software.

\paragraph{Right-handed coordinate frames}
Let \(\{\mathbf{e}_x, \mathbf{e}_y, \mathbf{e}_z\}\) denote the unit basis vectors of a coordinate frame.  
The frame is said to be \emph{right-handed} if
\[
\mathbf{e}_x \times \mathbf{e}_y = \mathbf{e}_z ,
\]
where \(\times\) denotes the vector cross product. Equivalently, the ordered basis has a positive orientation, i.e.,
\[
\det\!\begin{bmatrix}
\mathbf{e}_x & \mathbf{e}_y & \mathbf{e}_z
\end{bmatrix} = +1 .
\]

This property defines the \textbf{right-hand rule}: if the index finger of the right hand points along the positive \(x\)-axis and the middle finger along the positive \(y\)-axis, the thumb points in the positive \(z\)-direction. This convention ensures consistent definitions of cross products, angular velocities, torques, and rotation directions.

Both ENU and NED satisfy this relation, even though their axes point in different physical directions (e.g., up versus down). What matters is the ordering and orientation of the axes, not their semantic labels.

\paragraph{Motivation for NED in this course}
This course adopts the NED convention because the focus is on quadrotors. NED is standard in aerospace literature and flight-control implementations, and it allows gravity to be represented as a positive acceleration along the \(z\)-axis. This simplifies the equations of motion and control design and ensures direct consistency with GNSS/IMU models and widely used autopilot frameworks.


\subsection{What is a Coordinate Frame?}

\begin{definition}[Coordinate Frame]
\index{coordinate frame}
A \textbf{coordinate frame}\index{coordinate frame|textbf} is a set of three mutually perpendicular unit vectors (axes) with a defined origin. We denote frames with curly braces: $\{W\}$, $\{B\}$, etc.
\end{definition}

\textbf{Intuition}: A coordinate frame is like a ruler system attached to an object. When you describe a point as being ``2 meters to the right,'' you're implicitly using a coordinate frame---your ``right'' defines one axis, and you're measuring distance along it.

\subsection{Why Do We Need Multiple Frames?}

Consider measuring the position of your coffee cup:
\begin{itemize}
    \item Relative to your desk: ``30 cm from the left edge, 20 cm from the front''
    \item Relative to the room: ``2 m from the north wall, 3 m from the east wall''
    \item Relative to the Earth: latitude, longitude, altitude
\end{itemize}

These are all valid descriptions of the \emph{same} position, just in different coordinate frames. For quadrotors, we primarily use two frames:

\subsection{World Frame (Inertial Frame) $\{W\}$}
\index{world frame}\index{inertial frame|see{world frame}}

The world frame\index{world frame|textbf} is fixed to the Earth and serves as our absolute reference for describing where we want the quadrotor to go.

\begin{notebox}[title=Assumption: Earth as Inertial Frame]
We assume the Earth is an \textbf{inertial frame}---not rotating or accelerating. This is valid for:
\begin{itemize}
    \item Flight durations of minutes (Earth's rotation is negligible)
    \item Altitudes below ~10 km (Earth's curvature is negligible)
    \item Non-polar regions (Coriolis effects are small)
\end{itemize}
For intercontinental flights or spacecraft, this assumption breaks down.
\end{notebox}

Two conventions are commonly used:

%----------------------------------------------------------------------
% FIGURE: NED vs ENU Coordinate Conventions
%----------------------------------------------------------------------
\begin{figure}[htbp]
\centering
\begin{tikzpicture}[scale=1.0,
    % Same 3D view as quadrotor figure for consistency
    x={(0.866cm,-0.25cm)},
    y={(0.866cm,0.25cm)},
    z={(0cm,0.8cm)}
]
    % === NED Frame (Left) ===
    \begin{scope}[shift={(-4,0)}]
        % Axes matching quadrotor figure convention
        \draw[-{Stealth}, thick, red] (0,0,0) -- (2,0,0) node[right, font=\small] {$x$ (N)};
        \draw[-{Stealth}, thick, green!70!black] (0,0,0) -- (0,2,0) node[right, font=\small] {$y$ (E)};
        \draw[-{Stealth}, thick, blue] (0,0,0) -- (0,0,-2) node[left, font=\small] {$z$ (D)};

        % Origin
        \fill (0,0,0) circle (2pt);

        % Title below axes
        \node[font=\bfseries\large] at (1,-2.5) {NED};
        \node[font=\small] at (1,-3.2) {North-East-Down};
    \end{scope}

    % === Gravity vector (centered between frames) ===
    \draw[-{Stealth}, very thick, orange] (0,0,1) -- (0,0,-1.5)
        node[left, font=\small, pos=0.5] {$\mathbf{g}$};

    % === ENU Frame (Right) ===
    \begin{scope}[shift={(4,0)}]
        % Axes
        \draw[-{Stealth}, thick, red] (0,0,0) -- (2,0,0) node[right, font=\small] {$x$ (E)};
        \draw[-{Stealth}, thick, green!70!black] (0,0,0) -- (0,2,0) node[right, font=\small] {$y$ (N)};
        \draw[-{Stealth}, thick, blue] (0,0,0) -- (0,0,2) node[left, font=\small] {$z$ (U)};

        % Origin
        \fill (0,0,0) circle (2pt);

        % Title below axes
        \node[font=\bfseries\large] at (1,-2.5) {ENU};
        \node[font=\small] at (1,-3.2) {East-North-Up};
    \end{scope}
\end{tikzpicture}
\caption{Comparison of NED (North-East-Down) and ENU (East-North-Up) coordinate conventions. In NED, gravity aligns with $+z$; in ENU, gravity aligns with $-z$. The choice affects sign conventions throughout control algorithms.}
\label{fig:ned-enu-conventions}
\end{figure}
%----------------------------------------------------------------------

\begin{center}
\begin{tabular}{lccl}
\toprule
\textbf{Convention} & \textbf{Axes} & \textbf{$z$-axis} & \textbf{Common Use} \\
\midrule
NED & North, East, Down & Points into Earth & Aviation\\
ENU & East, North, Up & Points toward sky & Robotics\\
\bottomrule
\end{tabular}
\end{center}

\begin{warningbox}[title=Convention Matters!]
The choice between NED and ENU affects:
\begin{itemize}
    \item Sign of gravity ($+g$ in NED, $-g$ in ENU)
    \item Sign of altitude (positive down in NED, positive up in ENU)
    \item Direction of positive yaw rotation
\end{itemize}
\textbf{This course uses NED unless otherwise stated.} When using external libraries or data, always verify the convention first!
\end{warningbox}

\subsection{Body Frame $\{B\}$}
\index{body frame}

The body frame\index{body frame|textbf} is rigidly attached to the quadrotor and moves with it. This is the natural frame for:
\begin{itemize}
    \item Sensor measurements (IMU measures acceleration/rotation in body frame)
    \item Thrust direction (motors push ``down'' in body frame)
    \item Aerodynamic forces (drag depends on velocity in body frame)
\end{itemize}

%----------------------------------------------------------------------
% FIGURE: Quadrotor Body Frame Definition
%----------------------------------------------------------------------
\begin{figure}[htbp]
\centering
\begin{tikzpicture}[scale=0.85]
    % === Top-down view (Left panel) ===
    \begin{scope}[shift={(-5,0)}]
        % Arms (+ configuration)
        \draw[very thick, black!60] (-2,0) -- (2,0);
        \draw[very thick, black!60] (0,-2) -- (0,2);

        % Central body
        \fill[black!30] (-0.3,-0.3) rectangle (0.3,0.3);

        % Motors
        \node[motor] (M1) at (2,0) {};
        \node[motor] (M2) at (0,-2) {};
        \node[motor] (M3) at (-2,0) {};
        \node[motor] (M4) at (0,2) {};

        % Motor labels
        \node[font=\footnotesize, above right] at (M1.north east) {M1};
        \node[font=\footnotesize, below right] at (M2.south east) {M2};
        \node[font=\footnotesize, above left] at (M3.north west) {M3};
        \node[font=\footnotesize, above left] at (M4.north west) {M4};

        % Nose indicator
        \fill[red!70] (2.3,0.15) -- (2.6,0) -- (2.3,-0.15) -- cycle;

        % Body frame axes
        \draw[xaxisstyle] (0,0) -- (3,0) node[right] {$x_B$ (forward)};
        \draw[yaxisstyle] (0,0) -- (0,-3) node[below] {$y_B$ (right)};
        \fill[black] (0,0) circle (0.08) node[above left, font=\footnotesize] {COG};

        % View label
        \node[font=\bfseries\small] at (0,3) {Top View};
    \end{scope}

    % === Side view (Right panel) ===
    \begin{scope}[shift={(4,0)}]
        % Arm (side view - just a line)
        \draw[very thick, black!60] (-2,0) -- (2,0);

        % Central body
        \fill[black!30] (-0.3,-0.15) rectangle (0.3,0.15);

        % Motors (front and back visible)
        \draw[thick, black!60, fill=gray!30] (2,0) ellipse (0.4 and 0.15);
        \draw[thick, black!60, fill=gray!30] (-2,0) ellipse (0.4 and 0.15);

        % Landing gear
        \draw[thick, black!50] (-0.8,0) -- (-0.8,-0.8) -- (-1,-0.8);
        \draw[thick, black!50] (0.8,0) -- (0.8,-0.8) -- (1,-0.8);
        \draw[thick, black!50] (-1,-0.8) -- (1,-0.8);

        % Nose indicator
        \fill[red!70] (2.3,0.1) -- (2.5,0) -- (2.3,-0.1) -- cycle;

        % Body frame axes
        \draw[xaxisstyle] (0,0) -- (3,0) node[right] {$x_B$};
        \draw[zaxisstyle] (0,0) -- (0,-2) node[below] {$z_B$ (down)};
        \fill[black] (0,0) circle (0.08) node[above left, font=\footnotesize] {COG};

        % View label
        \node[font=\bfseries\small] at (0,1.5) {Side View};

        % "Pilot perspective" annotation
        \draw[<-, gray, thin] (0.5,0.5) -- (1.5,1)
            node[right, font=\scriptsize, text width=2cm] {Sitting inside: forward, right, down};
    \end{scope}
\end{tikzpicture}
\caption{Body frame definition for a quadrotor in + configuration. Left: top-down view showing $x_B$ (forward) and $y_B$ (right) axes. Right: side view showing $x_B$ and $z_B$ (down) axes. The origin is at the center of gravity (COG).}
\label{fig:body-frame-definition}
\end{figure}
%----------------------------------------------------------------------

\textbf{Standard definition} (for NED world frame):
\begin{itemize}
    \item \textbf{Origin}: Center of gravity (COG)
    \item \textbf{$x_B$}: Points forward (``nose'' direction)
    \item \textbf{$y_B$}: Points right (starboard)
    \item \textbf{$z_B$}: Points down (toward landing gear)
\end{itemize}

\textbf{Intuition}: Imagine sitting inside the quadrotor. The body frame axes are your ``forward,'' ``right,'' and ``down.''

\section{Describing Orientation: The Rotation Matrix}
\index{rotation matrix}

\subsection{The Problem}

We need to answer: ``How is the body frame oriented relative to the world frame?''

Equivalently: ``If I know a vector's coordinates in the body frame, what are its coordinates in the world frame?''

\subsection{Building Intuition in 2D}

Before tackling 3D, let's understand rotations in 2D, where the math is simpler.

Consider a vector $\mathbf{v}$ that has coordinates $(v_x, v_y)$ in a frame $\{A\}$. If frame $\{B\}$ is rotated by angle $\theta$ counter-clockwise relative to $\{A\}$, what are the coordinates of $\mathbf{v}$ in frame $\{A\}$ if we know them in frame $\{B\}$?

%----------------------------------------------------------------------
% FIGURE: 2D Rotation Between Frames
%----------------------------------------------------------------------
\begin{figure}[htbp]
\centering
\begin{tikzpicture}[scale=1.1]
    % Define angle
    \pgfmathsetmacro{\angle}{35}

    % Frame {A} axes (solid, black)
    \draw[axis, frameA] (0,0) -- (3.5,0) node[right] {$x_A$};
    \draw[axis, frameA] (0,0) -- (0,3.5) node[above] {$y_A$};
    \node[below left] at (0,0) {$O$};

    % Frame {B} axes (dashed, gray) rotated by theta
    \draw[dashedaxis, frameB] (0,0) -- ({3.2*cos(\angle)},{3.2*sin(\angle)}) node[right] {$x_B$};
    \draw[dashedaxis, frameB] (0,0) -- ({-3.2*sin(\angle)},{3.2*cos(\angle)}) node[above] {$y_B$};

    % Angle arc and label
    \draw[thick, blue!70] (1.2,0) arc (0:\angle:1.2);
    \node[blue!70] at ({1.5*cos(\angle/2)},{1.5*sin(\angle/2)}) {$\theta$};

    % Vector v
    \pgfmathsetmacro{\vx}{2.5}
    \pgfmathsetmacro{\vy}{2.0}
    \draw[vec, line width=1.5pt] (0,0) -- (\vx,\vy) node[above right] {$\mathbf{v}$};
    \fill[vectorcolor] (\vx,\vy) circle (0.06);

    % Projections onto frame A (dotted)
    \draw[dotted, thick, xaxis] (\vx,\vy) -- (\vx,0) node[below, font=\small] {$v_x^A$};
    \draw[dotted, thick, yaxis] (\vx,\vy) -- (0,\vy) node[left, font=\small] {$v_y^A$};

    % Projections onto frame B (dotted, lighter)
    \pgfmathsetmacro{\vxB}{\vx*cos(\angle)+\vy*sin(\angle)} % v dot x_B_hat
    \pgfmathsetmacro{\vyB}{-\vx*sin(\angle)+\vy*cos(\angle)} % v dot y_B_hat
    \coordinate (vBx) at ({(\vxB)*cos(\angle)},{(\vxB)*sin(\angle)});
    \coordinate (vBy) at ({-(\vyB)*sin(\angle)},{(\vyB)*cos(\angle)});
    \draw[dotted, thick, gray!60] (\vx,\vy) -- (vBx);
    \draw[dotted, thick, gray!60] (\vx,\vy) -- (vBy);
    \node[font=\small, gray!70] at ($(vBx)+(0.3,-0.3)$) {$v_x^B$};
    \node[font=\small, gray!70] at ($(vBy)+(-0.4,0.2)$) {$v_y^B$};

    % Annotation: column interpretation
    \node[text width=4.5cm, font=\small, align=left] at (5.5,2.5) {
        \textbf{Rotation matrix columns:}\\[2pt]
        1st column: $\hat{x}_B$ in $\{A\}$\\
        $= [\cos\theta, \sin\theta]^T$\\[4pt]
        2nd column: $\hat{y}_B$ in $\{A\}$\\
        $= [-\sin\theta, \cos\theta]^T$
    };

    % Annotation box
    \node[draw, rounded corners, fill=blue!5, text width=3.5cm, font=\small, align=center]
        at (5.5,-0.3) {Same vector $\mathbf{v}$,\\different coordinates\\in each frame};
\end{tikzpicture}
\caption{A vector $\mathbf{v}$ expressed in two coordinate frames. Frame $\{B\}$ is rotated by angle $\theta$ relative to frame $\{A\}$. The rotation matrix columns tell us where the rotated frame's unit vectors point.}
\label{fig:2d-rotation-frames}
\end{figure}
%----------------------------------------------------------------------

\begin{equation}
\begin{bmatrix} v_x^A \\ v_y^A \end{bmatrix} =
\underbrace{\begin{bmatrix} \cos\theta & -\sin\theta \\ \sin\theta & \cos\theta \end{bmatrix}}_{R(\theta)}
\begin{bmatrix} v_x^B \\ v_y^B \end{bmatrix}
\end{equation}

\textbf{Intuition}: The rotation matrix $R(\theta)$ tells us where the $x$ and $y$ axes of frame $\{B\}$ point when expressed in frame $\{A\}$:
\begin{itemize}
    \item First column: where $\hat{x}_B$ points in $\{A\}$
    \item Second column: where $\hat{y}_B$ points in $\{A\}$
\end{itemize}

\begin{keyidea}[title=Properties of Rotation Matrices]
Rotation matrices have special structure:
\begin{itemize}
    \item \textbf{Orthogonal}: $R^T R = I$, so $R^{-1} = R^T$ (inverse is just transpose)
    \item \textbf{Unit determinant}: $\det(R) = +1$ (preserves volume and handedness)
    \item \textbf{Columns are orthonormal}: Each column is a unit vector, and columns are perpendicular
\end{itemize}
These properties reflect the physical fact that rotation preserves lengths, angles and orientation.
\end{keyidea}

\begin{definition}[Rotation]
A \emph{rotation} in $\mathbb{R}^n$, with $n \ge 2$, is a linear transformation
$R:\mathbb{R}^n \rightarrow \mathbb{R}^n$ that preserves lengths, angles, and
orientation. Equivalently, for all vectors $x,y \in \mathbb{R}^n$,
\begin{equation}
(Rx)\cdot(Ry) = x \cdot y ,
\end{equation}
and positively oriented coordinate frames remain positively oriented after the
transformation. 
\end{definition}


\paragraph{Property 1: Orthogonality.}
From inner-product preservation, remember that $x \cdot y = x^\top y$ and $(Ax)^\top=x^\top A^\top$, we have
\[
(Rx)\cdot(Ry) = (Rx)^\top R y = x^\top R^\top R y = x^\top y
\quad \forall x,y .
\]
This implies
\begin{equation}
R^\top R = I .
\end{equation}
Hence, the columns (and rows) of $R$ form an orthonormal basis, and $R$ is an orthogonal matrix.

\paragraph{Property 2: Inverse equals transpose.}
Since $R^\top R = I$, the inverse of $R$ exists and is given by
\begin{equation}
R^{-1} = R^\top .
\end{equation}
Geometrically, this corresponds to the fact that applying a rotation and then its reverse restores the original vectors.

\paragraph{Property 3: Determinant equals $+1$.}
Orthogonality implies $\lvert \det(R) \rvert = 1$, which expresses volume preservation.
Because a rotation preserves orientation (right-handedness), reflections are excluded, and therefore
\begin{equation}
\det(R) = +1.
\end{equation}

\begin{definition}[Special Orthogonal group]
A rotation matrix is therefore an orthogonal matrix with determinant $+1$. In $n$ dimensions, the set of all such matrices forms the group $SO(n)$, known as the \emph{Special Orthogonal group} in $n \geq 2$ dimensions, i.e.,
\[
SO(n) = \{ R \in \mathbb{R}^{n\times n} \mid R^\top R = I,\ \det(R)=+1 \}.
\]
\end{definition}

In our context, we are primarily concerned with rotations in two and three
dimensions. The difference between $SO(2)$ and $SO(3)$ is primarily one of dimensionality and
degrees of freedom. A general $3\times 3$ matrix has $9$ independent elements, but a
3D rotation has only $3$ degrees of freedom. This reduction follows from the
orthogonality condition $R^\top R = I$, which requires the three column vectors of
$R$ to form an orthonormal basis. Concretely, this imposes three constraints that each
column has unit length and three additional constraints that distinct columns are
mutually orthogonal, yielding $6$ independent scalar constraints in total. Subtracting
these from the original $9$ free parameters leaves $9-6=3$ degrees of freedom. The
determinant condition $\det(R)=+1$ does not further reduce this number; it merely
excludes reflections.

Despite this difference, rotations in 2D and 3D share the same fundamental properties:
they preserve lengths, angles, and orientation, satisfy $R^\top R = I$,
$R^{-1}=R^\top$, and compose by matrix multiplication. The key distinction is that a
2D rotation acts in a plane and is fully specified by a single angle (one degree of
freedom), whereas a 3D rotation specifies an orientation in space and requires three
degrees of freedom. This also affects rotation order: in 2D, all rotations are about
the same axis and therefore commute, while in 3D rotations about different axes
generally do not commute, making the order of rotations significant.
\subsection{3D Elementary Rotations}

In 3D, we can rotate about any of the three coordinate axes. The rotation matrices are:

%----------------------------------------------------------------------
% FIGURE: Elementary 3D Rotations (Roll, Pitch, Yaw)
%----------------------------------------------------------------------
\begin{figure}[htbp]
\centering
\begin{tikzpicture}[scale=0.75]
    % === Panel 1: Roll (about x-axis) ===
    \begin{scope}[shift={(-5.5,0)},
        x={(0.7cm,-0.4cm)}, y={(0.7cm,0.4cm)}, z={(0cm,0.9cm)}]

        % Axes
        \draw[xaxisstyle, line width=1.5pt] (0,0,0) -- (2.5,0,0) node[right] {$x$};
        \draw[yaxisstyle] (0,0,0) -- (0,2,0) node[above right] {$y$};
        \draw[zaxisstyle] (0,0,0) -- (0,0,2) node[above] {$z$};

        % Rotation arc around x-axis
        \draw[thick, xaxis, -{Stealth}] (0,0,1.3) arc (90:0:1.3) node[pos=0.5, right] {$\phi$};

        % Quadrotor (tilted - roll)
        \begin{scope}[rotate around x=25]
            \draw[thick, gray] (-0.8,0,1.8) -- (0.8,0,1.8);
            \draw[thick, gray] (0,-0.8,1.8) -- (0,0.8,1.8);
            \fill[gray!40] (0.8,0,1.8) circle (0.15);
            \fill[gray!40] (-0.8,0,1.8) circle (0.15);
            \fill[gray!40] (0,0.8,1.8) circle (0.15);
            \fill[gray!40] (0,-0.8,1.8) circle (0.15);
        \end{scope}

        % Label
        \node[font=\bfseries\small, text width=2.5cm, align=center] at (0,-1.5,0)
            {Roll ($\phi$)\\about $x$-axis};
    \end{scope}

    % === Panel 2: Pitch (about y-axis) ===
    \begin{scope}[shift={(0,0)},
        x={(0.7cm,-0.4cm)}, y={(0.7cm,0.4cm)}, z={(0cm,0.9cm)}]

        % Axes
        \draw[xaxisstyle] (0,0,0) -- (2,0,0) node[right] {$x$};
        \draw[yaxisstyle, line width=1.5pt] (0,0,0) -- (0,2.5,0) node[above right] {$y$};
        \draw[zaxisstyle] (0,0,0) -- (0,0,2) node[above] {$z$};

        % Rotation arc around y-axis
        \draw[thick, yaxis, -{Stealth}] (1.3,0,0) arc (0:90:1.3) node[pos=0.5, above left] {$\theta$};

        % Quadrotor (tilted - pitch)
        \begin{scope}[rotate around y=-20]
            \draw[thick, gray] (-0.8,0,1.8) -- (0.8,0,1.8);
            \draw[thick, gray] (0,-0.8,1.8) -- (0,0.8,1.8);
            \fill[gray!40] (0.8,0,1.8) circle (0.15);
            \fill[gray!40] (-0.8,0,1.8) circle (0.15);
            \fill[gray!40] (0,0.8,1.8) circle (0.15);
            \fill[gray!40] (0,-0.8,1.8) circle (0.15);
        \end{scope}

        % Label
        \node[font=\bfseries\small, text width=2.5cm, align=center] at (0,-1.5,0)
            {Pitch ($\theta$)\\about $y$-axis};
    \end{scope}

    % === Panel 3: Yaw (about z-axis) ===
    \begin{scope}[shift={(5.5,0)},
        x={(0.7cm,-0.4cm)}, y={(0.7cm,0.4cm)}, z={(0cm,0.9cm)}]

        % Axes
        \draw[xaxisstyle] (0,0,0) -- (2,0,0) node[right] {$x$};
        \draw[yaxisstyle] (0,0,0) -- (0,2,0) node[above right] {$y$};
        \draw[zaxisstyle, line width=1.5pt] (0,0,0) -- (0,0,2.5) node[above] {$z$};

        % Rotation arc around z-axis (in xy plane)
        \draw[thick, zaxis, -{Stealth}] (1.3,0,0) arc (0:70:1.3);
        \node[zaxis] at (0.6,1.1,0) {$\psi$};

        % Quadrotor (rotated - yaw)
        \begin{scope}[rotate around z=25]
            \draw[thick, gray] (-0.8,0,1.8) -- (0.8,0,1.8);
            \draw[thick, gray] (0,-0.8,1.8) -- (0,0.8,1.8);
            \fill[gray!40] (0.8,0,1.8) circle (0.15);
            \fill[gray!40] (-0.8,0,1.8) circle (0.15);
            \fill[gray!40] (0,0.8,1.8) circle (0.15);
            \fill[gray!40] (0,-0.8,1.8) circle (0.15);
            % Nose marker
            \fill[red!70] (0.9,0,1.8) -- (1.05,0,1.8) -- (0.9,0.1,1.8) -- cycle;
        \end{scope}

        % Label
        \node[font=\bfseries\small, text width=2.5cm, align=center] at (0,-1.5,0)
            {Yaw ($\psi$)\\about $z$-axis};
    \end{scope}

    % Right-hand rule reminder at bottom
    \node[draw, rounded corners, fill=yellow!10, font=\small, text width=10cm, align=center]
        at (0,-3.5) {
        \textbf{Right-hand rule:} Point thumb along positive axis direction;
        fingers curl in positive rotation direction.
    };
\end{tikzpicture}
\caption{Elementary rotations about the three principal axes. Roll ($\phi$) rotates about the $x$-axis, pitch ($\theta$) about the $y$-axis, and yaw ($\psi$) about the $z$-axis. The highlighted axis in each panel is the rotation axis.}
\label{fig:elementary-rotations}
\end{figure}
%----------------------------------------------------------------------

\textbf{Rotation about $x$-axis by angle $\phi$} (roll):
\begin{equation}
R_x(\phi) = \begin{bmatrix}
1 & 0 & 0 \\
0 & \cos\phi & -\sin\phi \\
0 & \sin\phi & \cos\phi
\end{bmatrix}
\end{equation}

\textbf{Intuition}: The $x$-axis stays fixed; the $y$ and $z$ axes rotate in the $yz$-plane.

\textbf{Rotation about $y$-axis by angle $\theta$} (pitch):
\begin{equation}
R_y(\theta) = \begin{bmatrix}
\cos\theta & 0 & \sin\theta \\
0 & 1 & 0 \\
-\sin\theta & 0 & \cos\theta
\end{bmatrix}
\end{equation}

\textbf{Rotation about $z$-axis by angle $\psi$} (yaw):
\begin{equation}
R_z(\psi) = \begin{bmatrix}
\cos\psi & -\sin\psi & 0 \\
\sin\psi & \cos\psi & 0 \\
0 & 0 & 1
\end{bmatrix}
\end{equation}

\begin{notebox}[title=Notation: Euler Angles]
The angles $(\phi, \theta, \psi)$ have standard names in aerospace:
\begin{itemize}
    \item $\phi$ (phi): \textbf{Roll} --- rotation about the forward ($x$) axis
    \item $\theta$ (theta): \textbf{Pitch} --- rotation about the lateral ($y$) axis
    \item $\psi$ (psi): \textbf{Yaw} --- rotation about the vertical ($z$) axis
\end{itemize}
Positive directions follow the \textbf{right-hand rule}: thumb along axis, fingers curl in positive rotation direction.
\end{notebox}

\subsection{The Critical Difference from 2D: Order Matters!}

\begin{warningbox}[title=Non-Commutativity of 3D Rotations]
In 2D, the order of rotations doesn't matter: rotating 30° then 45° gives the same result as 45° then 30°.

\textbf{In 3D, order matters!}
\[
R_x(\alpha) R_y(\beta) \neq R_y(\beta) R_x(\alpha)
\]
This is perhaps the most important conceptual difference between 2D and 3D geometry.
\end{warningbox}

%----------------------------------------------------------------------
% FIGURE: Rotation Order Matters (Non-Commutativity)
%----------------------------------------------------------------------
\begin{figure}[htbp]
\centering
\begin{tikzpicture}[scale=0.65,
    book/.style={thick},
    bookcover/.style={fill=blue!30, draw=blue!60},
    bookspine/.style={fill=blue!50, draw=blue!70},
    bookpages/.style={fill=gray!10, draw=gray!50}
]
    % Helper for drawing a 3D book with orientation
    % Row 1: Pitch then Yaw
    \node[font=\bfseries] at (-6, 2.5) {Sequence A:};

    % A1: Start - book flat
    \begin{scope}[shift={(-5,0)}]
        \fill[bookcover] (-0.8,-0.5) rectangle (0.8,0.5);
        \fill[bookspine] (-0.8,-0.5) -- (-0.8,0.5) -- (-0.9,0.6) -- (-0.9,-0.4) -- cycle;
        \fill[bookpages] (0.8,-0.5) -- (0.8,0.5) -- (0.9,0.6) -- (0.9,-0.4) -- cycle;
        \node[font=\scriptsize] at (0,0) {Cover};
        \node[below, font=\small] at (0,-1) {Start};
    \end{scope}

    \draw[-{Stealth}, thick] (-3.5,0) -- (-2.5,0) node[midway, above, font=\scriptsize] {Pitch 90°};

    % A2: After pitch - cover faces forward
    \begin{scope}[shift={(-1,0)}]
        \fill[bookpages] (-0.8,-0.5) rectangle (0.8,0.5);
        \fill[bookspine] (-0.8,-0.5) -- (-0.8,0.5) -- (-0.9,0.4) -- (-0.9,-0.6) -- cycle;
        \fill[bookcover] (-0.8,-0.5) -- (0.8,-0.5) -- (0.9,-0.6) -- (-0.9,-0.6) -- cycle;
        \node[font=\scriptsize, rotate=-10] at (0,-0.7) {Cover};
    \end{scope}

    \draw[-{Stealth}, thick] (0.5,0) -- (1.5,0) node[midway, above, font=\scriptsize] {Yaw 90°};

    % A3: Final A - after yaw
    \begin{scope}[shift={(3,0)}]
        \fill[bookpages] (-0.5,-0.8) rectangle (0.5,0.8);
        \fill[bookcover] (0.5,-0.8) -- (0.5,0.8) -- (0.6,0.7) -- (0.6,-0.9) -- cycle;
        \fill[bookspine] (-0.5,0.8) -- (0.5,0.8) -- (0.6,0.7) -- (-0.4,0.7) -- cycle;
        \node[font=\scriptsize, rotate=90] at (0.7,0) {Cover};
        \node[below, font=\small\bfseries, red!70!black] at (0,-1.2) {Final A};
    \end{scope}

    % Row 2: Yaw then Pitch
    \node[font=\bfseries] at (-6, -2.5) {Sequence B:};

    % B1: Start - book flat (same as A1)
    \begin{scope}[shift={(-5,-4)}]
        \fill[bookcover] (-0.8,-0.5) rectangle (0.8,0.5);
        \fill[bookspine] (-0.8,-0.5) -- (-0.8,0.5) -- (-0.9,0.6) -- (-0.9,-0.4) -- cycle;
        \fill[bookpages] (0.8,-0.5) -- (0.8,0.5) -- (0.9,0.6) -- (0.9,-0.4) -- cycle;
        \node[font=\scriptsize] at (0,0) {Cover};
        \node[below, font=\small] at (0,-1) {Start};
    \end{scope}

    \draw[-{Stealth}, thick] (-3.5,-4) -- (-2.5,-4) node[midway, above, font=\scriptsize] {Yaw 90°};

    % B2: After yaw - rotated in plane
    \begin{scope}[shift={(-1,-4)}]
        \fill[bookcover] (-0.5,-0.8) rectangle (0.5,0.8);
        \fill[bookspine] (-0.5,-0.8) -- (-0.5,0.8) -- (-0.6,0.7) -- (-0.6,-0.9) -- cycle;
        \fill[bookpages] (-0.5,0.8) -- (0.5,0.8) -- (0.6,0.7) -- (-0.4,0.7) -- cycle;
        \node[font=\scriptsize, rotate=90] at (0,0) {Cover};
    \end{scope}

    \draw[-{Stealth}, thick] (0.5,-4) -- (1.5,-4) node[midway, above, font=\scriptsize] {Pitch 90°};

    % B3: Final B - different from A3!
    \begin{scope}[shift={(3,-4)}]
        \fill[bookpages] (-0.5,-0.8) rectangle (0.5,0.8);
        \fill[bookcover] (-0.5,-0.8) -- (-0.5,0.8) -- (-0.6,0.7) -- (-0.6,-0.9) -- cycle;
        \fill[bookspine] (-0.5,-0.8) -- (0.5,-0.8) -- (0.6,-0.9) -- (-0.6,-0.9) -- cycle;
        \node[font=\scriptsize, rotate=90] at (-0.7,0) {Cover};
        \node[below, font=\small\bfseries, red!70!black] at (0,-1.2) {Final B};
    \end{scope}

    % Not equal sign
    \draw[very thick, red!70!black] (5,-2) -- (6,-2);
    \draw[very thick, red!70!black] (5.2,-1.7) -- (5.8,-2.3);
    \draw[very thick, red!70!black] (5.2,-2.3) -- (5.8,-1.7);
    \node[font=\bfseries, red!70!black] at (5.5,-3) {$\neq$};

    % Conclusion box
    \node[draw, rounded corners, fill=red!10, text width=7cm, align=center, font=\small]
        at (0,-7.5) {
        \textbf{Final A $\neq$ Final B}\\
        Rotation order matters in 3D!\\
        $R_x R_y \neq R_y R_x$
    };
\end{tikzpicture}
\caption{Demonstration that 3D rotation order matters. Starting from the same position, applying pitch then yaw (Sequence A) yields a different result than yaw then pitch (Sequence B). This non-commutativity is fundamental to 3D rotations.}
\label{fig:rotation-order-matters}
\end{figure}
%----------------------------------------------------------------------

\begin{example}[Rotation Order Demonstration]
Try this with your phone:
\begin{enumerate}
    \item Hold your phone flat, screen up, top edge pointing away from you
    \item Rotate 90° about the axis pointing toward you (pitch up), then 90° about the vertical axis (yaw right)
    \item Return to start. Now do the opposite: yaw right 90°, then pitch up 90°
\end{enumerate}
The final orientations are completely different! This is why we must always specify the \emph{order} of rotations.
\end{example}



\section{Euler Angles: A Human-Friendly Representation}
\index{Euler angles}

\subsection{Motivation}

Rotation matrices have 9 numbers but only 3 degrees of freedom. Can we represent orientations with just 3 numbers?

Yes: Euler angles\index{Euler angles|textbf} use three successive rotations about coordinate axes.

\begin{theorem}[Euler's Rotation Theorem]
Any orientation in 3D can be achieved by a sequence of not more than three rotations about coordinate axes.
\end{theorem}

\subsection{The 12 Euler Angle Conventions}

There are 12 valid ways to combine three axis rotations:

\textbf{Tait-Bryan angles} (all three axes different):
\[
xyz, \quad xzy, \quad yxz, \quad yzx, \quad zxy, \quad zyx
\]

\textbf{Proper Euler angles} (first and third axis the same):
\[
xyx, \quad xzx, \quad yxy, \quad yzy, \quad zxz, \quad zyz
\]

%----------------------------------------------------------------------
% FIGURE: ZYX Euler Angle Sequence
%----------------------------------------------------------------------
\begin{figure}[htbp]
\centering
\begin{tikzpicture}[scale=0.6,
    x={(0.8cm,-0.3cm)}, y={(0.8cm,0.3cm)}, z={(0cm,0.9cm)}
]
    % === Step 0: Identity ===
    \begin{scope}[shift={(-8,0,0)}]
        % Reference axes
        \draw[xaxisstyle, thin] (0,0,0) -- (1.5,0,0);
        \draw[yaxisstyle, thin] (0,0,0) -- (0,1.5,0);
        \draw[zaxisstyle, thin] (0,0,0) -- (0,0,1.5);

        % Quadrotor (level)
        \draw[thick, gray] (-0.8,0,2) -- (0.8,0,2);
        \draw[thick, gray] (0,-0.8,2) -- (0,0.8,2);
        \fill[gray!40] (0.8,0,2) circle (0.12);
        \fill[gray!40] (-0.8,0,2) circle (0.12);
        \fill[gray!40] (0,0.8,2) circle (0.12);
        \fill[gray!40] (0,-0.8,2) circle (0.12);
        \fill[red!60] (0.9,0,2) -- (1.05,0,2) -- (0.9,0.08,2) -- cycle;

        \node[font=\small\bfseries, text width=2cm, align=center] at (0,-2,0) {Step 0\\$\phi{=}\theta{=}\psi{=}0$};
    \end{scope}

    \draw[-{Stealth}, thick, blue!70] (-5.5,0,2) -- (-4,0,2) node[midway, above, font=\scriptsize] {Yaw $\psi$};

    % === Step 1: After Yaw ===
    \begin{scope}[shift={(-2,0,0)}]
        % Reference axes
        \draw[gray!40, thin, ->] (0,0,0) -- (1.5,0,0);
        \draw[gray!40, thin, ->] (0,0,0) -- (0,1.5,0);
        \draw[zaxisstyle, thin] (0,0,0) -- (0,0,1.5);

        % Yaw arc
        \draw[thick, zaxis] (0.8,0,0) arc (0:30:0.8);
        \node[zaxis, font=\scriptsize] at (0.9,0.4,0) {$\psi$};

        % Quadrotor (yawed)
        \begin{scope}[rotate around z=30]
            \draw[thick, gray] (-0.8,0,2) -- (0.8,0,2);
            \draw[thick, gray] (0,-0.8,2) -- (0,0.8,2);
            \fill[gray!40] (0.8,0,2) circle (0.12);
            \fill[gray!40] (-0.8,0,2) circle (0.12);
            \fill[gray!40] (0,0.8,2) circle (0.12);
            \fill[gray!40] (0,-0.8,2) circle (0.12);
            \fill[red!60] (0.9,0,2) -- (1.05,0,2) -- (0.9,0.08,2) -- cycle;
        \end{scope}

        \node[font=\small\bfseries, text width=2.5cm, align=center] at (0,-2,0) {Step 1\\Yaw about $z$};
    \end{scope}

    \draw[-{Stealth}, thick, blue!70] (0.5,0,2) -- (2,0,2) node[midway, above, font=\scriptsize] {Pitch $\theta$};

    % === Step 2: After Yaw then Pitch ===
    \begin{scope}[shift={(4,0,0)}]
        % Reference axes (faded)
        \draw[gray!30, thin, ->] (0,0,0) -- (1.2,0,0);
        \draw[gray!30, thin, ->] (0,0,0) -- (0,1.2,0);
        \draw[gray!30, thin, ->] (0,0,0) -- (0,0,1.2);

        % Quadrotor (yawed and pitched)
        \begin{scope}[rotate around z=30, rotate around y=-20]
            \draw[thick, gray] (-0.8,0,2) -- (0.8,0,2);
            \draw[thick, gray] (0,-0.8,2) -- (0,0.8,2);
            \fill[gray!40] (0.8,0,2) circle (0.12);
            \fill[gray!40] (-0.8,0,2) circle (0.12);
            \fill[gray!40] (0,0.8,2) circle (0.12);
            \fill[gray!40] (0,-0.8,2) circle (0.12);
            \fill[red!60] (0.9,0,2) -- (1.05,0,2) -- (0.9,0.08,2) -- cycle;
        \end{scope}

        \node[font=\small\bfseries, text width=2.5cm, align=center] at (0,-2,0) {Step 2\\Pitch about $y'$};
    \end{scope}

    \draw[-{Stealth}, thick, blue!70] (6.5,0,2) -- (8,0,2) node[midway, above, font=\scriptsize] {Roll $\phi$};

    % === Step 3: Final (all three rotations) ===
    \begin{scope}[shift={(10,0,0)}]
        % Final body axes (dashed)
        \begin{scope}[rotate around z=30, rotate around y=-20, rotate around x=15]
            \draw[xaxisstyle, dashed, thin] (0,0,2) -- (1.2,0,2);
            \draw[yaxisstyle, dashed, thin] (0,0,2) -- (0,1.2,2);
            \draw[zaxisstyle, dashed, thin] (0,0,2) -- (0,0,0.8);

            \draw[thick, gray] (-0.8,0,2) -- (0.8,0,2);
            \draw[thick, gray] (0,-0.8,2) -- (0,0.8,2);
            \fill[gray!40] (0.8,0,2) circle (0.12);
            \fill[gray!40] (-0.8,0,2) circle (0.12);
            \fill[gray!40] (0,0.8,2) circle (0.12);
            \fill[gray!40] (0,-0.8,2) circle (0.12);
            \fill[red!60] (0.9,0,2) -- (1.05,0,2) -- (0.9,0.08,2) -- cycle;
        \end{scope}

        \node[font=\small\bfseries, text width=2.5cm, align=center] at (0,-2,0) {Step 3\\Roll about $x''$};
        \node[font=\scriptsize, blue!60] at (1.5,1,2) {$\{B\}$};
    \end{scope}

    % Key insight
    \node[draw, rounded corners, fill=blue!5, font=\small, text width=9cm, align=center]
        at (1,-5.5,0) {
        \textbf{ZYX (Intrinsic):} Each rotation is about the \emph{current} body axis,\\
        not the original world axis. Order: $R = R_z(\psi) \cdot R_y(\theta) \cdot R_x(\phi)$
    };
\end{tikzpicture}
\caption{The ZYX Euler angle sequence (yaw-pitch-roll). Starting from alignment with the world frame, we first yaw about $z$, then pitch about the new $y'$-axis, then roll about the newest $x''$-axis. This is an intrinsic rotation sequence.}
\label{fig:zyx-euler-sequence}
\end{figure}
%----------------------------------------------------------------------

\begin{notebox}[title=Convention Used in This Course]
We use the \textbf{ZYX Tait-Bryan} convention (aerospace standard):
\begin{enumerate}
    \item First: Yaw ($\psi$) about $z$-axis
    \item Second: Pitch ($\theta$) about (new) $y$-axis
    \item Third: Roll ($\phi$) about (new) $x$-axis
\end{enumerate}
This is also called ``yaw-pitch-roll'' or ``3-2-1'' convention.
\end{notebox}

\subsection{Intrinsic vs. Extrinsic: A Source of Endless Confusion}

When we say ``rotate about $y$, then about $x$,'' there's an ambiguity:

\begin{definition}[Intrinsic Rotation]
Each rotation is about the axes of the \textbf{current} (already-rotated) frame. The axes ``move with'' the object.
\end{definition}

\begin{definition}[Extrinsic Rotation]
Each rotation is about the axes of the \textbf{fixed} world frame. The axes stay fixed in space.
\end{definition}

\begin{keyidea}[title=The Reversal Rule]
Intrinsic rotations with sequence $ABC$ produce the same result as extrinsic rotations with sequence $CBA$ (reversed).

Mathematically:
\begin{itemize}
    \item Intrinsic $ZYX$: $R = R_z(\psi) \cdot R_y(\theta) \cdot R_x(\phi)$ (post-multiply, right-to-left)
    \item Extrinsic $XYZ$: Same matrix, but conceptually rotations are in opposite order
\end{itemize}
\end{keyidea}

\textbf{Intuition}: Think of applying rotations from the ``inside out'' (intrinsic) vs. ``outside in'' (extrinsic).

\subsection{Rotation Matrix Notation}
\label{sec:rotation-notation}

When working with multiple coordinate frames, we need a notation that clearly indicates \emph{which} rotation matrix we are referring to. This section introduces the notation convention used throughout this course.

\begin{definition}[Rotation Matrix Notation]
\index{rotation matrix!notation}
The rotation matrix $R^j_i$ describes the orientation of Frame $i$ with respect to Frame $j$.
\begin{itemize}
    \item The \textbf{superscript} $j$ indicates the \emph{reference frame} (the frame in which the result is expressed)
    \item The \textbf{subscript} $i$ indicates the \emph{source frame} (the frame being described or transformed from)
\end{itemize}
\end{definition}

\textbf{Reading the notation}: $R^W_B$ (read as ``R-W-B'' or ``rotation from B to W'') is the rotation matrix that:
\begin{enumerate}
    \item Describes the orientation of the body frame $\{B\}$ relative to the world frame $\{W\}$
    \item Transforms vectors from body-frame coordinates to world-frame coordinates
\end{enumerate}

The columns of $R^W_B$ are the unit vectors of the body frame axes expressed in world coordinates:
\[
R^W_B = \begin{bmatrix} | & | & | \\ \mathbf{x}_B^W & \mathbf{y}_B^W & \mathbf{z}_B^W \\ | & | & | \end{bmatrix}
\]
where $\mathbf{x}_B^W$ is the body $x$-axis direction expressed in world coordinates, and similarly for the other axes.

\begin{keyidea}[title=Coordinate Transformation Rule]
To transform a vector $\mathbf{p}$ from frame $\{B\}$ coordinates to frame $\{W\}$ coordinates:
\[
\mathbf{p}^W = R^W_B \, \mathbf{p}^B
\]
The superscript on the vector indicates the frame in which its coordinates are expressed.
\end{keyidea}

\begin{example}[Using the Rotation Matrix]
Suppose a sensor measures an acceleration vector $\mathbf{a}^B = [1, 0, 0]^T$ in body-frame coordinates (i.e., acceleration along the body $x$-axis). To express this in world coordinates:
\[
\mathbf{a}^W = R^W_B \, \mathbf{a}^B
\]
The result $\mathbf{a}^W$ gives the same physical acceleration, but expressed in world-frame coordinates.

\textbf{Detailed calculation}: Consider a quadrotor with orientation $\phi = 0°$ (roll), $\theta = 30°$ (pitch), $\psi = 0°$ (yaw). This means the quadrotor is pitched nose-down by $30°$.

First, we compute the rotation matrix $R^W_B$ using the ZYX formula (derived in Section~\ref{sec:zyx-rotation}), where $c_\alpha = \cos\alpha$ and $s_\alpha = \sin\alpha$:
\[
R^W_B = \begin{bmatrix}
c_\psi c_\theta & c_\psi s_\theta s_\phi - s_\psi c_\phi & c_\psi s_\theta c_\phi + s_\psi s_\phi \\
s_\psi c_\theta & s_\psi s_\theta s_\phi + c_\psi c_\phi & s_\psi s_\theta c_\phi - c_\psi s_\phi \\
-s_\theta & c_\theta s_\phi & c_\theta c_\phi
\end{bmatrix}
\]

Substituting $\phi = 0°$, $\theta = 30°$, $\psi = 0°$, we have $c_\phi = c_\psi = 1$, $s_\phi = s_\psi = 0$, $c_\theta = \frac{\sqrt{3}}{2}$, and $s_\theta = \frac{1}{2}$:
\[
R^W_B = \begin{bmatrix}
1 \cdot \frac{\sqrt{3}}{2} & 1 \cdot \frac{1}{2} \cdot 0 - 0 \cdot 1 & 1 \cdot \frac{1}{2} \cdot 1 + 0 \cdot 0 \\[6pt]
0 \cdot \frac{\sqrt{3}}{2} & 0 \cdot \frac{1}{2} \cdot 0 + 1 \cdot 1 & 0 \cdot \frac{1}{2} \cdot 1 - 1 \cdot 0 \\[6pt]
-\frac{1}{2} & \frac{\sqrt{3}}{2} \cdot 0 & \frac{\sqrt{3}}{2} \cdot 1
\end{bmatrix}
= \begin{bmatrix}
\frac{\sqrt{3}}{2} & 0 & \frac{1}{2} \\[6pt]
0 & 1 & 0 \\[6pt]
-\frac{1}{2} & 0 & \frac{\sqrt{3}}{2}
\end{bmatrix}
\]

Now we transform the body-frame acceleration $\mathbf{a}^B = [1, 0, 0]^T$:
\[
\mathbf{a}^W = R^W_B \, \mathbf{a}^B =
\begin{bmatrix}
\frac{\sqrt{3}}{2} & 0 & \frac{1}{2} \\[6pt]
0 & 1 & 0 \\[6pt]
-\frac{1}{2} & 0 & \frac{\sqrt{3}}{2}
\end{bmatrix}
\begin{bmatrix} 1 \\ 0 \\ 0 \end{bmatrix}
= \begin{bmatrix}
\frac{\sqrt{3}}{2} \cdot 1 + 0 \cdot 0 + \frac{1}{2} \cdot 0 \\[6pt]
0 \cdot 1 + 1 \cdot 0 + 0 \cdot 0 \\[6pt]
-\frac{1}{2} \cdot 1 + 0 \cdot 0 + \frac{\sqrt{3}}{2} \cdot 0
\end{bmatrix}
= \begin{bmatrix} \frac{\sqrt{3}}{2} \\[6pt] 0 \\[6pt] -\frac{1}{2} \end{bmatrix}
\]

\textbf{Physical interpretation}: The acceleration that was purely along the body $x$-axis (forward) now has components in the world frame:
\begin{itemize}
    \item $a^W_x = \frac{\sqrt{3}}{2} \approx 0.866$: Component pointing North (forward, but reduced due to pitch)
    \item $a^W_y = 0$: No East-West component (no yaw or roll)
    \item $a^W_z = -\frac{1}{2}$: Component pointing \emph{up} (negative in NED), because the nose is pitched down
\end{itemize}

Note that the magnitude is preserved: $\|\mathbf{a}^W\| = \sqrt{\left(\frac{\sqrt{3}}{2}\right)^2 + 0^2 + \left(-\frac{1}{2}\right)^2} = \sqrt{\frac{3}{4} + \frac{1}{4}} = 1 = \|\mathbf{a}^B\|$. Rotation matrices preserve vector lengths---they only change the direction, not the magnitude.
\end{example}

\textbf{Important properties}:
\begin{itemize}
    \item \textbf{Inverse}: $(R^W_B)^{-1} = (R^W_B)^T = R^B_W$ (the inverse transforms from world to body)
    \item \textbf{Composition}: $R^A_C = R^A_B \cdot R^B_C$ (transformations chain by matrix multiplication)
    \item \textbf{Orthogonality}: $R^T R = I$ and $\det(R) = +1$ for all rotation matrices
\end{itemize}

\subsection{The ZYX Rotation Matrix}
\label{sec:zyx-rotation}

Using intrinsic ZYX (or equivalently, extrinsic XYZ):

\begin{equation}
R^W_B = R_z(\psi) \cdot R_y(\theta) \cdot R_x(\phi)
\end{equation}

Multiplying out:
\begin{equation}
R^W_B = \begin{bmatrix}
c_\psi c_\theta & c_\psi s_\theta s_\phi - s_\psi c_\phi & c_\psi s_\theta c_\phi + s_\psi s_\phi \\
s_\psi c_\theta & s_\psi s_\theta s_\phi + c_\psi c_\phi & s_\psi s_\theta c_\phi - c_\psi s_\phi \\
-s_\theta & c_\theta s_\phi & c_\theta c_\phi
\end{bmatrix}
\end{equation}
where $c_\alpha = \cos\alpha$, $s_\alpha = \sin\alpha$.

\textbf{Reading the matrix}:
\begin{itemize}
    \item The columns tell us where the body axes point in world coordinates
    \item First column: body $x$-axis (forward) direction in world frame
    \item Third column: body $z$-axis (down) direction in world frame
\end{itemize}

\subsection{Extracting Euler Angles from a Rotation Matrix}

Given a rotation matrix $R = [r_{ij}]$, we can recover $(\phi, \theta, \psi)$:

\begin{align}
\theta &= \text{atan2}\left(-r_{31}, \sqrt{r_{11}^2 + r_{21}^2}\right) \\
\psi &= \text{atan2}(r_{21}, r_{11}) \\
\phi &= \text{atan2}(r_{32}, r_{33})
\end{align}

\begin{notebox}[title=Why atan2 Instead of atan?]
The function $\text{atan2}(y, x)$ computes $\arctan(y/x)$ but correctly handles all four quadrants:
\begin{itemize}
    \item $\arctan(1/1) = \arctan(-1/-1) = 45°$ --- ambiguous!
    \item $\text{atan2}(1, 1) = 45°$, but $\text{atan2}(-1, -1) = -135°$ --- unambiguous
\end{itemize}
Always use atan2 for angle recovery.
\end{notebox}

\subsection{Gimbal Lock: The Fatal Flaw of Euler Angles}
\index{gimbal lock}

%----------------------------------------------------------------------
% FIGURE: Gimbal Lock Visualization
%----------------------------------------------------------------------
\begin{figure}[htbp]
\centering
\begin{tikzpicture}[scale=0.8]
    % === Left panel: Normal gimbal ===
    \begin{scope}[shift={(-5,0)}]
        \node[font=\bfseries\small] at (0,3.5) {Normal Operation};

        % Outer ring (yaw) - tilted for 3D effect
        \draw[very thick, blue!60] (0,0) ellipse (2.5 and 0.8);
        \draw[very thick, blue!60, dashed] (-2.5,0) arc (180:360:2.5 and 0.8);
        \node[blue!60, font=\scriptsize] at (3.2,0.3) {Yaw};

        % Middle ring (pitch) - rotated
        \draw[very thick, yaxis] (0,0) ellipse (1.8 and 2);
        \node[yaxis, font=\scriptsize] at (0,2.5) {Pitch};

        % Inner ring (roll)
        \draw[very thick, xaxis] (0,0) ellipse (1.2 and 0.4);
        \node[xaxis, font=\scriptsize] at (1.6,-0.7) {Roll};

        % Center object
        \fill[gray!40] (0,0) circle (0.3);

        % Rotation axes shown
        \draw[-{Stealth}, blue!60, thick] (0,0) -- (0,-1.5) node[below, font=\tiny] {$z$ axis};
        \draw[-{Stealth}, yaxis, thick] (0,0) -- (1.5,0) node[below right, font=\tiny] {$y$ axis};
        \draw[-{Stealth}, xaxis, thick] (0,0) -- (0,1.2) node[above, font=\tiny] {$x$ axis};

        \node[font=\scriptsize, text width=3cm, align=center] at (0,-3) {All three axes\\independent};
    \end{scope}

    % === Right panel: Gimbal lock ===
    \begin{scope}[shift={(4,0)}]
        \node[font=\bfseries\small] at (0,3.5) {Gimbal Lock ($\theta = 90°$)};

        % Outer ring (yaw) - horizontal
        \draw[very thick, blue!60] (0,0) ellipse (2.5 and 0.8);
        \draw[very thick, blue!60, dashed] (-2.5,0) arc (180:360:2.5 and 0.8);

        % Middle ring (pitch) - rotated 90 degrees, now horizontal too!
        \draw[very thick, yaxis] (0,0) ellipse (1.8 and 0.6);
        \draw[very thick, yaxis, dashed] (-1.8,0) arc (180:360:1.8 and 0.6);

        % Inner ring (roll) - also horizontal!
        \draw[very thick, xaxis] (0,0) ellipse (1.2 and 0.4);
        \draw[very thick, xaxis, dashed] (-1.2,0) arc (180:360:1.2 and 0.4);

        % Center object
        \fill[gray!40] (0,0) circle (0.3);

        % Both yaw and roll axes now point in the same direction!
        \draw[-{Stealth}, red!70!black, very thick] (0,0) -- (0,-2)
            node[below, font=\scriptsize, text width=2.5cm, align=center] {Roll \& Yaw axes\\now \textbf{aligned}!};

        \node[font=\scriptsize, text width=3cm, align=center] at (0,-4) {Lost 1 DOF:\\only control $\phi + \psi$};

        % Warning highlight
        \draw[red!50, very thick, dashed] (-3,-1.5) rectangle (3,1.5);
    \end{scope}

    % Bottom annotation
    \node[draw, rounded corners, fill=red!10, font=\small, text width=10cm, align=center]
        at (0,-5.5) {
        At $\theta = \pm 90°$: The roll and yaw axes align, making them indistinguishable.\\
        This causes numerical instability: $\tan\theta$ and $\sec\theta$ become infinite.
    };
\end{tikzpicture}
\caption{Gimbal lock visualization. Left: Normal operation with three independent rotation axes. Right: At pitch $\theta = 90°$, the roll and yaw axes align, reducing the system to only two effective degrees of freedom.}
\label{fig:gimbal-lock}
\end{figure}
%----------------------------------------------------------------------

\begin{warningbox}[title=Gimbal Lock]
When pitch $\theta = \pm 90°$, the Euler angle representation breaks down.

\textbf{What happens mathematically}: When $\theta = 90°$, we have $\cos\theta = 0$, so:
\[
r_{11} = r_{21} = 0 \quad \text{and} \quad r_{32} = r_{33} = 0
\]
The formulas for $\psi$ and $\phi$ become $\text{atan2}(0, 0)$---undefined!

\textbf{Physical interpretation}: At $\theta = 90°$, the roll and yaw axes align (both vertical). We can only determine their sum or difference, not both individually. One degree of freedom is ``lost.''

\textbf{Practical consequence}: A quadrotor pitching straight up or down will cause numerical instability in Euler-angle-based controllers.
\end{warningbox}

\textbf{Intuition}: Imagine a gimbal (three nested rotating rings). When two rings align, they rotate together---you've lost independent control of one axis. The term ``gimbal lock'' comes from this physical phenomenon in mechanical gyroscopes.

\section{Quaternions: The Robust Alternative}
\index{quaternion}

Quaternions provide a singularity-free representation for 3D rotations. For a comprehensive survey of attitude representations including quaternions, see Shuster~\cite{shuster1993survey}.

\subsection{Motivation: Why Do We Need Another Representation?}

We've seen that:
\begin{itemize}
    \item Rotation matrices work but have 9 numbers for 3 DOF (redundant)
    \item Euler angles have 3 numbers but suffer gimbal lock (singular)
\end{itemize}

Quaternions offer the best of both worlds:
\begin{itemize}
    \item Only 4 numbers (minimal redundancy: 1 constraint)
    \item No singularities (work for all orientations)
    \item Efficient composition (quaternion multiplication is cheaper than matrix multiplication)
    \item Easy to normalize (cheap to fix numerical drift)
\end{itemize}

\subsection{What is a Quaternion?}

\begin{definition}[Unit Quaternion]
A \textbf{unit quaternion}\index{quaternion!unit|textbf} is a 4-dimensional vector with unit norm:
\[
q = \begin{bmatrix} q_0 \\ q_1 \\ q_2 \\ q_3 \end{bmatrix} = \begin{bmatrix} q_0 \\ \mathbf{q}_v \end{bmatrix}, \quad \|q\| = \sqrt{q_0^2 + q_1^2 + q_2^2 + q_3^2} = 1
\]
where $q_0$ is the \textbf{scalar part} and $\mathbf{q}_v = [q_1, q_2, q_3]^T$ is the \textbf{vector part}.
\end{definition}

\subsection{The Geometric Meaning: Axis-Angle}

A quaternion encodes rotation by angle $\theta$ about a unit axis $\hat{n} = [n_x, n_y, n_z]^T$:

%----------------------------------------------------------------------
% FIGURE: Quaternion Axis-Angle Representation
%----------------------------------------------------------------------
\begin{figure}[htbp]
\centering
\begin{tikzpicture}[scale=1.0,
    x={(0.75cm,-0.35cm)}, y={(0.75cm,0.35cm)}, z={(0cm,0.9cm)}
]
    % Reference axes (light gray)
    \draw[gray!40, ->] (0,0,0) -- (3,0,0) node[right, font=\small] {$x$};
    \draw[gray!40, ->] (0,0,0) -- (0,3,0) node[above, font=\small] {$y$};
    \draw[gray!40, ->] (0,0,0) -- (0,0,3) node[above, font=\small] {$z$};

    % Rotation axis n (arbitrary direction)
    \pgfmathsetmacro{\nx}{0.5}
    \pgfmathsetmacro{\ny}{0.5}
    \pgfmathsetmacro{\nz}{0.707}
    \draw[-{Stealth}, very thick, black] (0,0,0) -- (2.5*\nx, 2.5*\ny, 2.5*\nz)
        node[above right] {$\hat{\mathbf{n}}$};
    \node[font=\small] at (2.2*\nx, 2.2*\ny, 2.2*\nz+0.5) {Rotation axis};

    % Rotation arc around n (simplified as partial ellipse)
    \draw[thick, blue!70, -{Stealth}] (1.5,0,0.5) arc (-30:60:0.8);
    \node[blue!70, font=\small] at (1.8,0.8,1) {$\theta$};

    % Object before rotation (small arrow)
    \draw[very thick, orange!80] (1.3,-0.2,0.3) -- (2,-0.2,0.3);
    \fill[orange!80] (2,-0.2,0.3) -- (2.2,-0.2,0.3) -- (2,0,0.3) -- cycle;
    \node[font=\tiny, orange!80] at (1.5,-0.6,0.3) {Initial};

    % Object after rotation (small arrow, rotated position)
    \draw[very thick, orange!80, dashed] (0.8,1.2,1.5) -- (1.3,1.6,1.8);
    \fill[orange!80, opacity=0.6] (1.3,1.6,1.8) -- (1.5,1.7,1.9) -- (1.3,1.8,1.85) -- cycle;
    \node[font=\tiny, orange!80, left] at (0.5,1.0,1.5) {Final};

    % Annotation box
    \node[draw, rounded corners, fill=blue!5, font=\small, text width=5.5cm, align=left]
        at (5.5,1,1.5) {
        \textbf{Quaternion encoding:}\\[3pt]
        $q = \begin{bmatrix} \cos(\theta/2) \\ \sin(\theta/2) \cdot \hat{\mathbf{n}} \end{bmatrix}$\\[6pt]
        $\theta = 0°$: $q = [1,0,0,0]^T$\\
        $\theta = 180°$: $q = [0, n_x, n_y, n_z]^T$\\[3pt]
        Note: $q$ and $-q$ represent\\the same rotation!
    };

    % Right-hand rule reminder
    \draw[gray, thick, ->] (-0.5,-0.5,0) -- (-0.5,-0.5,1);
    \draw[gray, thick] (-0.5,-0.5,0.5) arc (90:0:0.3);
    \node[font=\tiny, gray] at (-1,-0.5,0.5) {RH rule};
\end{tikzpicture}
\caption{Quaternion axis-angle representation. A rotation by angle $\theta$ about unit axis $\hat{\mathbf{n}}$ is encoded as $q = [\cos(\theta/2), \sin(\theta/2)\hat{\mathbf{n}}]^T$. The vector part $\mathbf{q}_v$ points along the rotation axis, and the scalar part $q_0$ encodes the rotation amount.}
\label{fig:quaternion-axis-angle}
\end{figure}
%----------------------------------------------------------------------

\begin{equation}
q = \begin{bmatrix} \cos(\theta/2) \\ \sin(\theta/2) \cdot n_x \\ \sin(\theta/2) \cdot n_y \\ \sin(\theta/2) \cdot n_z \end{bmatrix}
\end{equation}

\textbf{Intuition}:
\begin{itemize}
    \item The vector part $\mathbf{q}_v$ points along the rotation axis
    \item The scalar part $q_0$ encodes ``how much'' rotation
    \item A larger $|q_0|$ means less rotation; $q_0 = 1$ means no rotation (identity)
\end{itemize}

\begin{keyidea}[title=The Half-Angle]
Note that a rotation of $\theta$ uses $\theta/2$ in the quaternion. This has an important consequence: both $q$ and $-q$ represent the \textbf{same rotation}!
\begin{itemize}
    \item $q$ corresponds to rotating $\theta$ about $\hat{n}$
    \item $-q$ corresponds to rotating $\theta + 360°$ about $\hat{n}$, which is the same orientation
\end{itemize}
This ``double cover'' is occasionally a source of confusion but rarely causes practical problems.
\end{keyidea}

\subsection{Quaternion Operations}

\textbf{Quaternion multiplication} (Hamilton product)---composing rotations:
\begin{equation}
p \otimes q = \begin{bmatrix}
p_0 q_0 - \mathbf{p}_v \cdot \mathbf{q}_v \\
p_0 \mathbf{q}_v + q_0 \mathbf{p}_v + \mathbf{p}_v \times \mathbf{q}_v
\end{bmatrix}
\end{equation}

\textbf{Quaternion conjugate} (represents inverse rotation):
\begin{equation}
q^* = \begin{bmatrix} q_0 \\ -\mathbf{q}_v \end{bmatrix}
\end{equation}

For unit quaternions, $q^{-1} = q^*$ (conjugate equals inverse).

\textbf{Rotating a vector} $\mathbf{v}$ by quaternion $q$:
\begin{equation}
\mathbf{v}' = q \otimes \begin{bmatrix} 0 \\ \mathbf{v} \end{bmatrix} \otimes q^*
\end{equation}

\textbf{Composing rotations}---first $q_1$, then $q_2$:
\begin{equation}
q_{total} = q_2 \otimes q_1
\end{equation}

(Note the order: second rotation appears on the left, like matrix multiplication.)

\subsection{Converting Between Representations}

\textbf{Quaternion to Rotation Matrix}:
\begin{equation}
R = \begin{bmatrix}
1 - 2(q_2^2 + q_3^2) & 2(q_1 q_2 - q_0 q_3) & 2(q_1 q_3 + q_0 q_2) \\
2(q_1 q_2 + q_0 q_3) & 1 - 2(q_1^2 + q_3^2) & 2(q_2 q_3 - q_0 q_1) \\
2(q_1 q_3 - q_0 q_2) & 2(q_2 q_3 + q_0 q_1) & 1 - 2(q_1^2 + q_2^2)
\end{bmatrix}
\end{equation}

\textbf{Rotation Matrix to Quaternion}:
\begin{align}
q_0 &= \frac{1}{2}\sqrt{1 + r_{11} + r_{22} + r_{33}} \\
q_1 &= \frac{r_{32} - r_{23}}{4q_0}, \quad
q_2 = \frac{r_{13} - r_{31}}{4q_0}, \quad
q_3 = \frac{r_{21} - r_{12}}{4q_0}
\end{align}

(Alternative formulas exist for numerical stability when $q_0 \approx 0$.)

\subsection{Summary: Choosing a Representation}

\begin{center}
\begin{tabular}{lccc}
\toprule
\textbf{Property} & \textbf{Euler Angles} & \textbf{Rotation Matrix} & \textbf{Quaternion} \\
\midrule
Parameters & 3 & 9 & 4 \\
Constraints & 0 & 6 & 1 \\
Singularities & Yes (gimbal lock) & No & No \\
Human-readable & Yes & No & No \\
Composition & Complex & $O(27)$ multiplies & $O(16)$ multiplies \\
Interpolation & Poor & Difficult & Excellent (SLERP) \\
Fixing numerical drift & N/A & Re-orthogonalize (expensive) & Normalize (cheap) \\
\bottomrule
\end{tabular}
\end{center}

\begin{keyidea}[title=Practical Recommendation]
\begin{itemize}
    \item Use \textbf{quaternions internally} for computation and state estimation
    \item Convert to \textbf{Euler angles} for human display and interpretation
    \item Use \textbf{rotation matrices} for transforming vectors
\end{itemize}
\end{keyidea}

\subsection{Worked Example: Euler Angles vs. Quaternions}
\label{sec:euler-quaternion-example}
\index{Euler angles!worked example}
\index{quaternion!worked example}

This example demonstrates two scenarios: one where both Euler angles and quaternions work equally well, and one where Euler angles fail due to gimbal lock while quaternions succeed. These numerical calculations reinforce the theoretical concepts from the previous sections.

\subsubsection{Part 1: A Rotation Both Approaches Handle Well}

\textbf{Scenario}: A quadrotor performs a typical banking turn maneuver with roll $\phi = 30°$, pitch $\theta = 15°$, and yaw $\psi = 45°$.

\paragraph{Euler Angle Approach (ZYX Convention)}

Using the ZYX Tait-Bryan convention from Section~\ref{sec:euler-angles}, we compute the rotation matrix $R = R_z(\psi) \cdot R_y(\theta) \cdot R_x(\phi)$.

First, compute the elementary rotation matrices:
\begin{align}
R_x(30°) &= \begin{bmatrix} 1 & 0 & 0 \\ 0 & 0.866 & -0.5 \\ 0 & 0.5 & 0.866 \end{bmatrix} \\[1em]
R_y(15°) &= \begin{bmatrix} 0.966 & 0 & 0.259 \\ 0 & 1 & 0 \\ -0.259 & 0 & 0.966 \end{bmatrix} \\[1em]
R_z(45°) &= \begin{bmatrix} 0.707 & -0.707 & 0 \\ 0.707 & 0.707 & 0 \\ 0 & 0 & 1 \end{bmatrix}
\end{align}

The combined rotation matrix is:
\begin{equation}
R = R_z(45°) \cdot R_y(15°) \cdot R_x(30°) = \begin{bmatrix} 0.683 & -0.683 & 0.259 \\ 0.707 & 0.612 & -0.354 \\ 0.183 & 0.399 & 0.899 \end{bmatrix}
\end{equation}

Applying this to the body x-axis vector $\mathbf{v} = [1, 0, 0]^T$:
\begin{equation}
\mathbf{v}' = R \cdot \mathbf{v} = \begin{bmatrix} 0.683 \\ 0.707 \\ 0.183 \end{bmatrix}
\end{equation}

\paragraph{Quaternion Approach}

Convert each rotation to a unit quaternion using the axis-angle formula $\mathbf{q} = [\cos(\theta/2), \sin(\theta/2) \cdot \hat{\mathbf{n}}]^T$:
\begin{align}
\mathbf{q}_{\text{roll}} &= [\cos(15°), \sin(15°), 0, 0]^T = [0.966, 0.259, 0, 0]^T \\
\mathbf{q}_{\text{pitch}} &= [\cos(7.5°), 0, \sin(7.5°), 0]^T = [0.991, 0, 0.131, 0]^T \\
\mathbf{q}_{\text{yaw}} &= [\cos(22.5°), 0, 0, \sin(22.5°)]^T = [0.924, 0, 0, 0.383]^T
\end{align}

Compose the rotations using the Hamilton product (note: apply roll first, then pitch, then yaw):
\begin{equation}
\mathbf{q}_{\text{total}} = \mathbf{q}_{\text{yaw}} \otimes \mathbf{q}_{\text{pitch}} \otimes \mathbf{q}_{\text{roll}} = [0.872, 0.215, 0.189, 0.392]^T
\end{equation}

To rotate the vector $\mathbf{v} = [1, 0, 0]^T$, we use $\mathbf{v}' = \mathbf{q} \otimes [0, \mathbf{v}]^T \otimes \mathbf{q}^*$:
\begin{equation}
\mathbf{v}' = \begin{bmatrix} 0.683 \\ 0.707 \\ 0.183 \end{bmatrix}
\end{equation}

\begin{keyidea}
Both Euler angles and quaternions produce \textbf{identical results} for this typical flight maneuver. When operating away from gimbal lock ($|\theta| \ll 90°$), either representation works well.
\end{keyidea}

%----------------------------------------------------------------------
\subsubsection{Part 2: Gimbal Lock---Where Euler Angles Fail}

\textbf{Scenario}: The quadrotor pitches straight up ($\theta = 90°$), and we need to track subsequent orientation changes.

\paragraph{The Problem with Euler Angles}

When pitch $\theta = 90°$, the ZYX rotation matrix becomes:
\begin{equation}
R = R_z(\psi) \cdot R_y(90°) \cdot R_x(\phi) = \begin{bmatrix} 0 & \sin(\psi - \phi) & \cos(\psi - \phi) \\ 0 & \cos(\psi - \phi) & -\sin(\psi - \phi) \\ -1 & 0 & 0 \end{bmatrix}
\end{equation}

Notice that roll ($\phi$) and yaw ($\psi$) only appear as the \emph{difference} $(\psi - \phi)$. This means:

\begin{center}
\begin{tikzpicture}[every node/.style={font=\small}]
    % Two orientations
    \node[draw, rectangle, rounded corners, fill=blue!15, minimum width=4cm, minimum height=1.5cm, align=center] (A) at (0, 0) {Orientation A:\\$\phi = 30°$, $\theta = 90°$, $\psi = 0°$};
    \node[draw, rectangle, rounded corners, fill=red!15, minimum width=4cm, minimum height=1.5cm, align=center] (B) at (7, 0) {Orientation B:\\$\phi = 0°$, $\theta = 90°$, $\psi = -30°$};

    % Arrow showing equivalence
    \draw[thick, <->] (A) -- node[above] {Same $R$ matrix!} (B);

    % Note below
    \node[below of=A, yshift=-0.5cm, xshift=3.5cm, font=\itshape] {Both have $\psi - \phi = -30°$};
\end{tikzpicture}
\end{center}

\textbf{Numerical demonstration}: Both orientations produce the \emph{identical} rotation matrix:
\begin{equation}
R = \begin{bmatrix} 0 & -0.5 & 0.866 \\ 0 & 0.866 & 0.5 \\ -1 & 0 & 0 \end{bmatrix}
\end{equation}

If we try to extract Euler angles from this matrix using the standard formulas:
\begin{align}
\theta &= -\arcsin(R_{31}) = -\arcsin(-1) = 90° \quad \checkmark \\
\phi &= \text{atan2}(R_{32}, R_{33}) = \text{atan2}(0, 0) \quad \text{\textbf{undefined!}} \\
\psi &= \text{atan2}(R_{21}, R_{11}) = \text{atan2}(0, 0) \quad \text{\textbf{undefined!}}
\end{align}

\begin{warningbox}
At $\theta = \pm 90°$, one degree of freedom is lost. We cannot distinguish between ``rolled then pitched up'' and ``pitched up then yawed.'' The atan2$(0, 0)$ calculation is undefined, and any numerical implementation will produce arbitrary or unstable results.
\end{warningbox}

\paragraph{The Quaternion Solution}

The same maneuver using quaternions has no such problem. Starting from the identity orientation:
\begin{enumerate}
    \item Initial orientation: $\mathbf{q}_0 = [1, 0, 0, 0]^T$
    \item Pitch up 90°: $\mathbf{q}_{\text{pitch}} = [\cos(45°), 0, \sin(45°), 0]^T = [0.707, 0, 0.707, 0]^T$
    \item After pitch: $\mathbf{q}_1 = \mathbf{q}_{\text{pitch}} \otimes \mathbf{q}_0 = [0.707, 0, 0.707, 0]^T$
    \item Roll 30° about body x-axis: $\mathbf{q}_{\text{roll}} = [0.966, 0.259, 0, 0]^T$
    \item Final: $\mathbf{q}_2 = \mathbf{q}_1 \otimes \mathbf{q}_{\text{roll}} = [0.683, 0.183, 0.683, 0.183]^T$
\end{enumerate}

The quaternion $\mathbf{q}_2 = [0.683, 0.183, 0.683, 0.183]^T$ is:
\begin{itemize}
    \item \textbf{Unique}: No other unit quaternion represents this orientation (except $-\mathbf{q}_2$, which represents the same rotation)
    \item \textbf{Well-defined}: No division by zero, no undefined operations
    \item \textbf{Composable}: We can continue applying rotations without any singularity
\end{itemize}

Converting this quaternion to a rotation matrix gives a valid, well-defined result:
\begin{equation}
R = \begin{bmatrix} 0.067 & -0.5 & 0.863 \\ 0.25 & 0.866 & 0.433 \\ -0.966 & 0.0 & 0.259 \end{bmatrix}
\end{equation}

This matrix can be used to transform vectors correctly, even though we passed through $\theta = 90°$.

%----------------------------------------------------------------------
\subsubsection{Part 3: Practical Implications for Quadrotor Control}

\paragraph{When Gimbal Lock Matters}

\begin{itemize}
    \item \textbf{Aerobatic maneuvers}: Flips, loops, and rolls pass through $\theta = \pm 90°$
    \item \textbf{Aggressive recovery}: A tumbling quadrotor may have any orientation
    \item \textbf{Continuous tracking}: Numerical integration of angular velocity can drift through singularities
\end{itemize}

\paragraph{When Euler Angles Are Fine}

\begin{itemize}
    \item \textbf{Normal flight}: Hover and forward flight typically have $|\theta| < 30°$
    \item \textbf{Display and logging}: Humans understand ``roll = 10°, pitch = 5°'' more easily than quaternion components
    \item \textbf{Small-angle control}: Near hover, Euler angle controllers work well
\end{itemize}

\begin{keyidea}[title=Best Practice for Quadrotor Software]
\begin{enumerate}
    \item \textbf{Store orientation internally as a quaternion}---no singularities, efficient composition
    \item \textbf{Perform sensor fusion using quaternions}---avoid gimbal lock in Kalman filters
    \item \textbf{Convert to Euler angles only for display}---let pilots see familiar roll/pitch/yaw values
    \item \textbf{Use rotation matrices for vector transformations}---efficient for sensor frame conversions
\end{enumerate}
\end{keyidea}

\begin{center}
\begin{tikzpicture}[every node/.style={font=\small},
    block/.style={draw, rectangle, rounded corners, minimum width=3cm, minimum height=1cm, align=center}]

    \node[block, fill=green!20] (sensors) at (0, 0) {IMU Data\\(angular rates)};
    \node[block, fill=blue!20] (quat) at (5, 0) {Quaternion\\(internal state)};
    \node[block, fill=yellow!20] (euler) at (10, 0) {Euler Angles\\(display only)};

    \draw[-{Stealth}, thick] (sensors) -- node[above] {integrate} (quat);
    \draw[-{Stealth}, thick] (quat) -- node[above] {convert} (euler);

    \node[below of=quat, yshift=0.3cm, font=\footnotesize\itshape] {No singularities here};
    \node[below of=euler, yshift=0.3cm, font=\footnotesize\itshape] {Human-readable};
\end{tikzpicture}
\end{center}

%======================================================================

\section{Homogeneous Transformations}
\label{sec:homogeneous-transformations}

\subsection{Motivation: Combining Rotation and Translation}

So far, we have treated rotation and translation separately. To transform a point $\mathbf{p}^B$ from the body frame to the world frame, we must:
\begin{enumerate}
    \item Rotate the point: $R^W_B \, \mathbf{p}^B$
    \item Add the translation: $R^W_B \, \mathbf{p}^B + \mathbf{t}^W_B$
\end{enumerate}

This two-step process becomes cumbersome when chaining multiple transformations. For example, transforming through three frames requires:
\[
\mathbf{p}^A = R^A_B \left( R^B_C \, \mathbf{p}^C + \mathbf{t}^B_C \right) + \mathbf{t}^A_B
\]

\textbf{Homogeneous transformations} provide an elegant solution: by embedding the 3D transformation in a $4 \times 4$ matrix, both rotation and translation can be performed with a single matrix multiplication.

\subsection{The Homogeneous Transformation Matrix}

\begin{definition}[Homogeneous Transformation Matrix]
\index{homogeneous transformation}
The homogeneous transformation matrix $T^j_i \in \mathbb{R}^{4 \times 4}$ combines the rotation $R^j_i$ and translation $\mathbf{t}^j_i$ into a single matrix:
\begin{equation}
T^j_i = \begin{bmatrix} R^j_i & \mathbf{t}^j_i \\ \mathbf{0}^T & 1 \end{bmatrix}
= \begin{bmatrix}
r_{11} & r_{12} & r_{13} & t_x \\
r_{21} & r_{22} & r_{23} & t_y \\
r_{31} & r_{32} & r_{33} & t_z \\
0 & 0 & 0 & 1
\end{bmatrix}
\end{equation}
where:
\begin{itemize}
    \item $R^j_i \in \mathbb{R}^{3 \times 3}$: rotation matrix (upper-left $3 \times 3$ block)
    \item $\mathbf{t}^j_i \in \mathbb{R}^{3}$: translation vector (upper-right column)
    \item $\mathbf{0}^T = [0, 0, 0]$: bottom row padding
    \item The subscript $i$ is the source frame, superscript $j$ is the reference frame
\end{itemize}
\end{definition}

For the quadrotor, the transformation from body frame $\{B\}$ to world frame $\{W\}$ is:
\begin{equation}
T^W_B = \begin{bmatrix} R^W_B & \mathbf{t}^W_B \\ \mathbf{0}^T & 1 \end{bmatrix}
\end{equation}
where $\mathbf{t}^W_B = [x, y, z]^T$ is the position of the quadrotor (body frame origin) in world coordinates.

\subsection{Homogeneous Coordinates}

To use the $4 \times 4$ matrix, we extend 3D points to \textbf{homogeneous coordinates} by appending a 1:
\[
\mathbf{p} = \begin{bmatrix} p_x \\ p_y \\ p_z \end{bmatrix}
\quad \rightarrow \quad
\tilde{\mathbf{p}} = \begin{bmatrix} p_x \\ p_y \\ p_z \\ 1 \end{bmatrix}
\]

The transformation then becomes a single matrix multiplication:
\begin{equation}
\begin{bmatrix} \mathbf{p}^W \\ 1 \end{bmatrix} = T^W_B \begin{bmatrix} \mathbf{p}^B \\ 1 \end{bmatrix}
\end{equation}

\begin{example}[Transforming a Point on the Quadrotor]
Consider a sensor mounted on the quadrotor at position $\mathbf{p}^B = [0.1, 0, -0.05]^T$ in body coordinates (10 cm forward, 5 cm up from the center of mass). The quadrotor is at position $\mathbf{t}^W_B = [2, 3, -10]^T$ in world coordinates (2m North, 3m East, 10m altitude in NED) with orientation $\phi = 0°$, $\theta = 15°$, $\psi = 45°$.

\textbf{Step 1}: Compute the rotation matrix $R^W_B$ for the given Euler angles.

With $c_\alpha = \cos\alpha$, $s_\alpha = \sin\alpha$, and using $\cos 15° = \frac{\sqrt{6}+\sqrt{2}}{4}$, $\sin 15° = \frac{\sqrt{6}-\sqrt{2}}{4}$, $\cos 45° = \sin 45° = \frac{\sqrt{2}}{2}$:

For simplicity, we use decimal approximations: $c_{15} \approx 0.966$, $s_{15} \approx 0.259$, $c_{45} = s_{45} \approx 0.707$.

Since $\phi = 0$: $c_\phi = 1$, $s_\phi = 0$.

\[
R^W_B = \begin{bmatrix}
c_\psi c_\theta & -s_\psi & c_\psi s_\theta \\
s_\psi c_\theta & c_\psi & s_\psi s_\theta \\
-s_\theta & 0 & c_\theta
\end{bmatrix}
= \begin{bmatrix}
0.683 & -0.707 & 0.183 \\
0.683 & 0.707 & 0.183 \\
-0.259 & 0 & 0.966
\end{bmatrix}
\]

\textbf{Step 2}: Construct the homogeneous transformation matrix:
\[
T^W_B = \begin{bmatrix}
0.683 & -0.707 & 0.183 & 2 \\
0.683 & 0.707 & 0.183 & 3 \\
-0.259 & 0 & 0.966 & -10 \\
0 & 0 & 0 & 1
\end{bmatrix}
\]

\textbf{Step 3}: Transform the sensor position:
\[
\begin{bmatrix} \mathbf{p}^W \\ 1 \end{bmatrix}
= T^W_B \begin{bmatrix} 0.1 \\ 0 \\ -0.05 \\ 1 \end{bmatrix}
= \begin{bmatrix}
0.683 \cdot 0.1 + (-0.707) \cdot 0 + 0.183 \cdot (-0.05) + 2 \\
0.683 \cdot 0.1 + 0.707 \cdot 0 + 0.183 \cdot (-0.05) + 3 \\
-0.259 \cdot 0.1 + 0 \cdot 0 + 0.966 \cdot (-0.05) + (-10) \\
1
\end{bmatrix}
\]
\[
= \begin{bmatrix}
0.0683 - 0.0092 + 2 \\
0.0683 - 0.0092 + 3 \\
-0.0259 - 0.0483 - 10 \\
1
\end{bmatrix}
= \begin{bmatrix}
2.059 \\
3.059 \\
-10.074 \\
1
\end{bmatrix}
\]

The sensor is at world position $\mathbf{p}^W \approx [2.06, 3.06, -10.07]^T$ meters---slightly ahead of and above the quadrotor center.
\end{example}

\subsection{Composing Transformations}

A key advantage of homogeneous transformations is that \textbf{chaining transformations} reduces to matrix multiplication:

\begin{equation}
T^A_C = T^A_B \cdot T^B_C
\end{equation}

This follows from the associativity of matrix multiplication. Transforming a point through multiple frames:
\[
\tilde{\mathbf{p}}^A = T^A_B \, \tilde{\mathbf{p}}^B = T^A_B \left( T^B_C \, \tilde{\mathbf{p}}^C \right) = \left( T^A_B \cdot T^B_C \right) \tilde{\mathbf{p}}^C = T^A_C \, \tilde{\mathbf{p}}^C
\]

\begin{example}[Chaining Transformations: Camera on a Gimbal]
Consider a camera mounted on a 2-axis gimbal attached to the quadrotor. We have three frames:
\begin{itemize}
    \item $\{W\}$: World frame (NED)
    \item $\{B\}$: Body frame (quadrotor center of mass)
    \item $\{C\}$: Camera frame (at the gimbal)
\end{itemize}

The gimbal is mounted 15 cm below the quadrotor center: $\mathbf{t}^B_C = [0, 0, 0.15]^T$ (in body coordinates, positive $z$ is down).

The gimbal tilts the camera $20°$ nose-down (pitch only, no roll or yaw relative to body):
\[
R^B_C = R_y(20°) = \begin{bmatrix}
\cos 20° & 0 & \sin 20° \\
0 & 1 & 0 \\
-\sin 20° & 0 & \cos 20°
\end{bmatrix}
\approx \begin{bmatrix}
0.940 & 0 & 0.342 \\
0 & 1 & 0 \\
-0.342 & 0 & 0.940
\end{bmatrix}
\]

Thus:
\[
T^B_C = \begin{bmatrix}
0.940 & 0 & 0.342 & 0 \\
0 & 1 & 0 & 0 \\
-0.342 & 0 & 0.940 & 0.15 \\
0 & 0 & 0 & 1
\end{bmatrix}
\]

If the quadrotor is hovering level at $\mathbf{t}^W_B = [0, 0, -5]^T$ (5m altitude) with $\psi = 90°$ (facing East):
\[
T^W_B = \begin{bmatrix}
0 & -1 & 0 & 0 \\
1 & 0 & 0 & 0 \\
0 & 0 & 1 & -5 \\
0 & 0 & 0 & 1
\end{bmatrix}
\]

The camera-to-world transformation is:
\[
T^W_C = T^W_B \cdot T^B_C =
\begin{bmatrix}
0 & -1 & 0 & 0 \\
1 & 0 & 0 & 0 \\
0 & 0 & 1 & -5 \\
0 & 0 & 0 & 1
\end{bmatrix}
\begin{bmatrix}
0.940 & 0 & 0.342 & 0 \\
0 & 1 & 0 & 0 \\
-0.342 & 0 & 0.940 & 0.15 \\
0 & 0 & 0 & 1
\end{bmatrix}
\]
\[
= \begin{bmatrix}
0 & -1 & 0 & 0 \\
0.940 & 0 & 0.342 & 0 \\
-0.342 & 0 & 0.940 & -4.85 \\
0 & 0 & 0 & 1
\end{bmatrix}
\]

A point directly in front of the camera at distance 1m ($\mathbf{p}^C = [0, 0, 1]^T$ in camera coordinates) is at world position:
\[
\tilde{\mathbf{p}}^W = T^W_C \begin{bmatrix} 0 \\ 0 \\ 1 \\ 1 \end{bmatrix}
= \begin{bmatrix} 0 \\ 0.342 \\ -4.85 + 0.940 \\ 1 \end{bmatrix}
= \begin{bmatrix} 0 \\ 0.342 \\ -3.91 \\ 1 \end{bmatrix}
\]

The point is at $[0, 0.34, -3.91]^T$: slightly East and at 3.91m altitude (the camera is looking down and forward).
\end{example}

\subsection{Inverse of a Homogeneous Transformation}

The inverse transformation $T^B_W = (T^W_B)^{-1}$ transforms points from world frame to body frame. Rather than computing a general $4 \times 4$ matrix inverse, we can use the structure:

\begin{equation}
T^B_W = (T^W_B)^{-1} = \begin{bmatrix} (R^W_B)^T & -(R^W_B)^T \mathbf{t}^W_B \\ \mathbf{0}^T & 1 \end{bmatrix}
= \begin{bmatrix} R^B_W & -R^B_W \mathbf{t}^W_B \\ \mathbf{0}^T & 1 \end{bmatrix}
\end{equation}

\begin{keyidea}[title=Why This Formula Works]
To invert the transformation ``rotate then translate,'' we must ``translate back then rotate back'':
\begin{enumerate}
    \item The inverse rotation is $R^B_W = (R^W_B)^T$
    \item The translation $\mathbf{t}^W_B$ was applied \emph{after} rotation, so to undo it we must first subtract it, then rotate back: $-R^B_W \mathbf{t}^W_B$
\end{enumerate}
\end{keyidea}

\begin{example}[Computing the Inverse Transformation]
Given the quadrotor at position $\mathbf{t}^W_B = [2, 3, -10]^T$ with a simple $90°$ yaw rotation:
\[
R^W_B = R_z(90°) = \begin{bmatrix} 0 & -1 & 0 \\ 1 & 0 & 0 \\ 0 & 0 & 1 \end{bmatrix}, \quad
T^W_B = \begin{bmatrix} 0 & -1 & 0 & 2 \\ 1 & 0 & 0 & 3 \\ 0 & 0 & 1 & -10 \\ 0 & 0 & 0 & 1 \end{bmatrix}
\]

The inverse is:
\[
R^B_W = (R^W_B)^T = \begin{bmatrix} 0 & 1 & 0 \\ -1 & 0 & 0 \\ 0 & 0 & 1 \end{bmatrix}
\]
\[
-R^B_W \mathbf{t}^W_B = -\begin{bmatrix} 0 & 1 & 0 \\ -1 & 0 & 0 \\ 0 & 0 & 1 \end{bmatrix} \begin{bmatrix} 2 \\ 3 \\ -10 \end{bmatrix}
= -\begin{bmatrix} 3 \\ -2 \\ -10 \end{bmatrix}
= \begin{bmatrix} -3 \\ 2 \\ 10 \end{bmatrix}
\]
\[
T^B_W = \begin{bmatrix} 0 & 1 & 0 & -3 \\ -1 & 0 & 0 & 2 \\ 0 & 0 & 1 & 10 \\ 0 & 0 & 0 & 1 \end{bmatrix}
\]

\textbf{Verification}: A point at the world origin $\mathbf{p}^W = [0, 0, 0]^T$ should transform to the quadrotor's view of the origin:
\[
\tilde{\mathbf{p}}^B = T^B_W \begin{bmatrix} 0 \\ 0 \\ 0 \\ 1 \end{bmatrix} = \begin{bmatrix} -3 \\ 2 \\ 10 \\ 1 \end{bmatrix}
\]
From the quadrotor's perspective (facing East after $90°$ yaw), the world origin is 3m to the left, 2m forward, and 10m below---which matches our setup.
\end{example}

\begin{notebox}[title=When to Use Homogeneous Transformations]
Homogeneous transformations are essential when:
\begin{itemize}
    \item Chaining multiple coordinate frame transformations
    \item Working with robotic manipulators (each joint adds a transformation)
    \item Computer graphics and visualization
    \item Sensor fusion involving multiple sensor mounting positions
\end{itemize}
For pure rotation problems (like attitude estimation), the $3 \times 3$ rotation matrix or quaternion representation is more efficient.
\end{notebox}

\chapter{Inertial Sensors and Measurement Models}

%----------------------------------------------------------------------
% FIGURE: IMU Sensor Package Overview
%----------------------------------------------------------------------
\begin{figure}[htbp]
\centering
\begin{tikzpicture}[scale=0.9]
    % Main IMU chip (3D box)
    \begin{scope}[shift={(-3,0)}]
        % Chip body
        \fill[gray!20] (0,0) rectangle (4,3);
        \draw[thick] (0,0) rectangle (4,3);

        % 3D effect
        \fill[gray!30] (4,0) -- (4.5,0.5) -- (4.5,3.5) -- (4,3) -- cycle;
        \fill[gray!10] (0,3) -- (0.5,3.5) -- (4.5,3.5) -- (4,3) -- cycle;
        \draw[thick] (4,0) -- (4.5,0.5) -- (4.5,3.5) -- (4,3);
        \draw[thick] (0,3) -- (0.5,3.5) -- (4.5,3.5);

        % Internal blocks
        % Accelerometer
        \fill[xaxis!20] (0.3,1.8) rectangle (1.8,2.7);
        \draw[xaxis] (0.3,1.8) rectangle (1.8,2.7);
        \node[font=\tiny, text width=1.3cm, align=center] at (1.05,2.25) {3-axis\\Accel};

        % Gyroscope
        \fill[yaxis!20] (2.2,1.8) rectangle (3.7,2.7);
        \draw[yaxis] (2.2,1.8) rectangle (3.7,2.7);
        \node[font=\tiny, text width=1.3cm, align=center] at (2.95,2.25) {3-axis\\Gyro};

        % Magnetometer (optional)
        \fill[zaxis!20] (1.25,0.3) rectangle (2.75,1.2);
        \draw[zaxis] (1.25,0.3) rectangle (2.75,1.2);
        \node[font=\tiny, text width=1.3cm, align=center] at (2,0.75) {3-axis\\Mag};

        % Body frame axes
        \draw[xaxisstyle, thin] (0.2,0.2) -- (1,0.2) node[right, font=\tiny] {$x_B$};
        \draw[yaxisstyle, thin] (0.2,0.2) -- (0.2,1) node[above, font=\tiny] {$y_B$};

        % Chip label
        \node[font=\scriptsize\bfseries] at (2,3.8) {9-DOF IMU};

        % Output signals
        \draw[->, thick] (4.5,2.5) -- (5.5,2.5) node[right, font=\tiny] {$\omega_x, \omega_y, \omega_z$};
        \draw[->, thick] (4.5,1.5) -- (5.5,1.5) node[right, font=\tiny] {$a_x, a_y, a_z$};
        \draw[->, thick] (4.5,0.5) -- (5.5,0.5) node[right, font=\tiny] {$m_x, m_y, m_z$};
    \end{scope}

    % MEMS principle inset
    \begin{scope}[shift={(5,-0.5)}]
        \draw[thick, rounded corners] (-0.3,-0.3) rectangle (3.3,3.3);
        \node[font=\scriptsize\bfseries] at (1.5,3) {MEMS Principle};

        % Housing
        \draw[thick, fill=gray!20] (0,0.5) -- (0,2) -- (0.3,2) -- (0.3,0.5) -- cycle;
        \draw[thick, fill=gray!20] (2.7,0.5) -- (2.7,2) -- (3,2) -- (3,0.5) -- cycle;

        % Springs
        \draw[thick, decorate, decoration={zigzag, segment length=3mm, amplitude=1mm}]
            (0.3,1.25) -- (1,1.25);
        \draw[thick, decorate, decoration={zigzag, segment length=3mm, amplitude=1mm}]
            (2,1.25) -- (2.7,1.25);

        % Proof mass
        \fill[orange!40] (1,0.8) rectangle (2,1.7);
        \draw[thick] (1,0.8) rectangle (2,1.7);
        \node[font=\tiny] at (1.5,1.25) {mass};

        % Capacitor plates
        \draw[thick, blue!60] (0.5,0.5) -- (0.5,2);
        \draw[thick, blue!60] (2.5,0.5) -- (2.5,2);
        \node[font=\tiny, blue!60] at (0.5,0.2) {C1};
        \node[font=\tiny, blue!60] at (2.5,0.2) {C2};

        % Acceleration arrow
        \draw[-{Stealth}, very thick, red!70] (3.5,1.25) -- (4.2,1.25)
            node[right, font=\tiny] {$a$};

        % Annotation
        \node[font=\tiny, text width=2.5cm, align=center] at (1.5,-0.6) {Mass displacement\\$\rightarrow$ capacitance change};
    \end{scope}

    % Size comparison
    \node[font=\scriptsize, text width=4cm, align=center] at (1,-1.5) {
        Typical size: $3\times3\times1$ mm\\
        (smaller than a fingernail!)
    };
\end{tikzpicture}
\caption{Inertial Measurement Unit (IMU) overview. Left: A 9-DOF IMU contains a 3-axis accelerometer, gyroscope, and magnetometer. Right: MEMS accelerometers measure displacement of a proof mass suspended by springs.}
\label{fig:imu-overview}
\end{figure}
%----------------------------------------------------------------------

\section{Introduction: How Do We Measure Orientation?}

In the previous chapter, we developed the mathematics for describing orientation. Now we ask: how do we \emph{measure} orientation in practice?

The answer involves \textbf{inertial sensors}---devices that measure motion without external references (like GPS satellites or visual landmarks). These are crucial because they work everywhere: indoors, underground, in space, or when GPS is jammed.

\begin{keyidea}[title=The Fundamental Challenge]
No single sensor directly measures orientation. Instead, we measure \emph{related quantities} (angular velocity, acceleration, magnetic field) and must \emph{compute} orientation from these. Each sensor has strengths and weaknesses; combining them intelligently is the art of sensor fusion.
\end{keyidea}

For a comprehensive tutorial on using inertial sensors for position and orientation estimation, see Kok et al.~\cite{kok2017using}.

\section{The Inertial Measurement Unit (IMU)}
\index{IMU (Inertial Measurement Unit)}

An IMU\index{IMU (Inertial Measurement Unit)|textbf} is a sensor package typically containing:
\begin{itemize}
    \item \textbf{3-axis accelerometer}\index{accelerometer}: Measures specific force (acceleration + gravity)
    \item \textbf{3-axis gyroscope}\index{gyroscope}: Measures angular velocity
    \item \textbf{3-axis magnetometer}\index{magnetometer} (in 9-DOF IMUs): Measures magnetic field
\end{itemize}

Modern MEMS (Micro-Electro-Mechanical Systems) IMUs are remarkably small and cheap---your smartphone contains one. The Crazyflie uses the BMI088 (accel+gyro) and similar sensors.

\begin{notebox}[title=Assumption: Sensor Location]
We assume the IMU is located at the quadrotor's center of gravity. If not, centripetal accelerations during rotation would affect the accelerometer readings. For small quadrotors like the Crazyflie, this error is typically negligible.
\end{notebox}

\section{Gyroscope}

\subsection{What Does a Gyroscope Measure?}

\begin{definition}[Gyroscope Measurement]
An ideal gyroscope measures the \textbf{angular velocity} of the body frame relative to an inertial frame, expressed in body coordinates:
\[
\boldsymbol{\omega} = \begin{bmatrix} p \\ q \\ r \end{bmatrix}
\]
where $p$, $q$, $r$ are rotation rates about the body $x$, $y$, $z$ axes (roll rate, pitch rate, yaw rate).
\end{definition}

\textbf{Intuition}: The gyroscope tells you ``how fast am I rotating right now?''---not ``what is my orientation?''

\subsection{Realistic Measurement Model}

Real gyroscopes don't measure perfect angular velocity. The measurement is corrupted:

\begin{equation}
\boldsymbol{\omega}_{meas} = \boldsymbol{\omega}_{true} + \mathbf{b}_g(t) + \mathbf{n}_g
\end{equation}

\begin{notebox}[title=Gyroscope Error Sources]
\begin{itemize}
    \item $\mathbf{b}_g(t)$: \textbf{Bias}---a slowly-varying offset. Even when stationary, the gyro reports a small non-zero reading. Bias can drift due to temperature changes or aging.
    \item $\mathbf{n}_g$: \textbf{White noise}---rapid random fluctuations, typically modeled as zero-mean Gaussian
\end{itemize}
\end{notebox}

\begin{definition}[Gyroscope Noise Parameters]
\begin{itemize}
    \item \textbf{Angular Random Walk (ARW)}: Noise density, in units of $°/\sqrt{h}$ or $°/s/\sqrt{Hz}$
    \item \textbf{Bias Instability}: How much the bias drifts, in $°/h$
\end{itemize}
\end{definition}

\subsection{The Drift Problem}

To get orientation from angular velocity, we integrate:
\begin{equation}
\theta(t) = \theta_0 + \int_0^t \omega(\tau) \, d\tau
\end{equation}

\begin{warningbox}[title=Integration Amplifies Errors]
Integration is mathematically beautiful but practically treacherous:
\begin{itemize}
    \item \textbf{Bias}: A constant bias $b$ causes error that grows \textbf{linearly}: $\epsilon(t) = b \cdot t$
    \item \textbf{White noise}: Random noise causes a \textbf{random walk}: $\epsilon(t) \propto \sqrt{t}$
\end{itemize}
After enough time, the estimated orientation diverges arbitrarily far from truth.
\end{warningbox}

%----------------------------------------------------------------------
% FIGURE: Gyroscope Drift Over Time
%----------------------------------------------------------------------
\begin{figure}[htbp]
\centering
\begin{tikzpicture}
    \begin{axis}[
        width=11cm,
        height=6cm,
        xlabel={Time [min]},
        ylabel={Orientation Error [deg]},
        xmin=0, xmax=65,
        ymin=0, ymax=420,
        grid=both,
        grid style={gray!20},
        legend pos=north west,
        legend style={font=\small},
        every axis label/.style={font=\small},
        tick label style={font=\small}
    ]
        % Bias error (linear): 0.1 deg/s = 6 deg/min
        \addplot[thick, blue, domain=0:60, samples=100] {6*x};
        \addlegendentry{Bias error (linear)}

        % Noise error (random walk): sqrt growth, scaled
        \addplot[thick, red, dashed, domain=0:60, samples=100] {8*sqrt(x)};
        \addlegendentry{Noise error ($\propto\sqrt{t}$)}

        % Mark key points
        \node[circle, fill=blue, inner sep=1.5pt] at (axis cs:1,6) {};
        \node[above right, font=\tiny] at (axis cs:1,6) {6°};

        \node[circle, fill=blue, inner sep=1.5pt] at (axis cs:10,60) {};
        \node[above right, font=\tiny] at (axis cs:10,60) {60°};

        \node[circle, fill=blue, inner sep=1.5pt] at (axis cs:60,360) {};
        \node[left, font=\tiny] at (axis cs:60,360) {360°!};

        % Horizontal reference line at 360 deg
        \draw[gray, dashed] (axis cs:0,360) -- (axis cs:65,360);
        \node[right, font=\tiny, gray] at (axis cs:62,370) {Full rotation};

    \end{axis}

    % Inset: Gyro signal
    \begin{scope}[shift={(8.5,1.5)}]
        \draw[thick] (0,0) rectangle (3.5,2.5);
        \node[font=\scriptsize\bfseries] at (1.75,2.3) {Gyro Signal (stationary)};

        % True value (zero)
        \draw[dashed, gray] (0.2,0.8) -- (3.3,0.8);
        \node[font=\tiny, gray, left] at (0.2,0.8) {True: 0};

        % Measured signal with bias and noise
        \draw[thick, blue] plot[smooth, tension=0.5] coordinates {
            (0.2,1.4) (0.5,1.5) (0.8,1.35) (1.1,1.45) (1.4,1.38)
            (1.7,1.52) (2.0,1.42) (2.3,1.48) (2.6,1.4) (2.9,1.45) (3.2,1.38)
        };

        % Bias indicator
        \draw[<->, thick, red!70] (3.4,0.8) -- (3.4,1.45);
        \node[font=\tiny, red!70, right] at (3.4,1.1) {bias};

        % Noise band
        \draw[<->, gray] (0.3,1.3) -- (0.3,1.55);
        \node[font=\tiny, gray, left] at (0.3,1.42) {noise};
    \end{scope}

    % Annotation
    \node[draw, rounded corners, fill=red!10, font=\small, text width=5.5cm, align=center]
        at (5.5,-1.2) {
        After 1 hour with $0.1°/s$ bias:\\
        error $\approx 360°$ (completely wrong!)
    };
\end{tikzpicture}
\caption{Gyroscope drift over time. A bias of $0.1°/s$ (typical for consumer MEMS) causes orientation error to grow linearly, reaching $360°$ after one hour. Noise causes additional random walk error that grows as $\sqrt{t}$.}
\label{fig:gyro-drift}
\end{figure}
%----------------------------------------------------------------------

\begin{example}[Drift Calculation]
Consider a MEMS gyroscope with bias $b = 0.1°/s$ (typical for consumer-grade sensors).

After 1 minute: orientation error $\approx 0.1 \times 60 = 6°$

After 10 minutes: orientation error $\approx 60°$

After 1 hour: orientation error $\approx 360°$ (completely wrong!)
\end{example}

\begin{keyidea}[title=Gyroscope Summary]
The gyroscope is \textbf{excellent for short-term, high-frequency orientation changes} but \textbf{useless for long-term absolute orientation} due to drift. We need another sensor to provide an absolute reference.
\end{keyidea}

\section{Accelerometer}

\subsection{What Does an Accelerometer Measure?}

This is surprisingly subtle. An accelerometer does \emph{not} directly measure acceleration!

\begin{definition}[Specific Force]
An accelerometer measures \textbf{specific force}---the non-gravitational force per unit mass, or equivalently, the acceleration you would ``feel'' if you were a test mass inside the sensor.
\end{definition}

In the body frame:
\begin{equation}
\mathbf{f}_{meas} = R^B_W (\mathbf{a}_{body} - \mathbf{g}) + \mathbf{b}_a + \mathbf{n}_a
\end{equation}

where:
\begin{itemize}
    \item $\mathbf{a}_{body}$: True acceleration of the body in the world frame
    \item $\mathbf{g} = [0, 0, g]^T$: Gravity vector in world frame (NED: points down)
    \item $R^B_W$: Rotation from world to body frame
    \item $\mathbf{b}_a$, $\mathbf{n}_a$: Bias and noise (similar to gyroscope)
\end{itemize}

%----------------------------------------------------------------------
% FIGURE: Accelerometer Specific Force (Spring-Mass Analogy)
%----------------------------------------------------------------------
\begin{figure}[htbp]
\centering
\begin{tikzpicture}[scale=0.85]
    % === Panel 1: Stationary on ground ===
    \begin{scope}[shift={(-5,0)}]
        \node[font=\bfseries\small] at (0,3.5) {At Rest};

        % Ground
        \fill[brown!20] (-1.5,-1.5) rectangle (1.5,-1.2);
        \draw[thick] (-1.5,-1.2) -- (1.5,-1.2);

        % Accelerometer case
        \draw[very thick, fill=gray!10] (-1,-1.2) rectangle (1,1.8);

        % Spring (stretched)
        \draw[thick, decorate, decoration={zigzag, segment length=2.5mm, amplitude=1.5mm}]
            (0,1.5) -- (0,0.3);

        % Proof mass (displaced down)
        \fill[orange!60] (-0.4,-0.3) rectangle (0.4,0.3);
        \draw[thick] (-0.4,-0.3) rectangle (0.4,0.3);
        \node[font=\tiny] at (0,0) {m};

        % Gravity arrow on mass
        \draw[-{Stealth}, very thick, blue!70] (0.6,0) -- (0.6,-0.8)
            node[right, font=\scriptsize] {$mg$};

        % Output reading
        \node[draw, rounded corners, fill=green!20, font=\scriptsize] at (0,-2.2)
            {Output: $f = +g$};

        % Annotation
        \node[font=\tiny, text width=2.2cm, align=center] at (0,-3) {No motion, but\\reads $g$!};
    \end{scope}

    % === Panel 2: Free fall ===
    \begin{scope}[shift={(0,0)}]
        \node[font=\bfseries\small] at (0,3.5) {Free Fall};

        % Motion blur / falling indicator
        \draw[-{Stealth}, very thick, red!70] (1.3,1.5) -- (1.3,-0.5)
            node[right, font=\scriptsize] {falling};

        % Accelerometer case
        \draw[very thick, fill=gray!10] (-1,-1.2) rectangle (1,1.8);

        % Spring (relaxed - not stretched)
        \draw[thick, decorate, decoration={zigzag, segment length=4mm, amplitude=1mm}]
            (0,1.5) -- (0,0.8);

        % Proof mass (centered - no displacement!)
        \fill[orange!60] (-0.4,0.2) rectangle (0.4,0.8);
        \draw[thick] (-0.4,0.2) rectangle (0.4,0.8);
        \node[font=\tiny] at (0,0.5) {m};

        % Both falling together
        \node[font=\tiny, gray] at (0,-0.5) {(both fall at $g$)};

        % Output reading
        \node[draw, rounded corners, fill=yellow!30, font=\scriptsize] at (0,-2.2)
            {Output: $f = 0$};

        % Annotation
        \node[font=\tiny, text width=2.2cm, align=center] at (0,-3) {Accelerating at $g$,\\but reads 0!};
    \end{scope}

    % === Panel 3: Accelerating upward ===
    \begin{scope}[shift={(5,0)}]
        \node[font=\bfseries\small] at (0,3.5) {Accelerating Up};

        % Motion indicator
        \draw[-{Stealth}, very thick, green!60!black] (1.3,-0.5) -- (1.3,1.5)
            node[right, font=\scriptsize] {$a$};

        % Accelerometer case
        \draw[very thick, fill=gray!10] (-1,-1.2) rectangle (1,1.8);

        % Spring (more stretched)
        \draw[thick, decorate, decoration={zigzag, segment length=2mm, amplitude=2mm}]
            (0,1.5) -- (0,-0.1);

        % Proof mass (displaced further down)
        \fill[orange!60] (-0.4,-0.7) rectangle (0.4,-0.1);
        \draw[thick] (-0.4,-0.7) rectangle (0.4,-0.1);
        \node[font=\tiny] at (0,-0.4) {m};

        % Output reading
        \node[draw, rounded corners, fill=blue!20, font=\scriptsize] at (0,-2.2)
            {Output: $f = a + g$};

        % Annotation
        \node[font=\tiny, text width=2.2cm, align=center] at (0,-3) {More than $1g$\\reading};
    \end{scope}

    % Key insight box at bottom
    \node[draw, rounded corners, fill=blue!5, font=\small, text width=12cm, align=center]
        at (0,-4.3) {
        \textbf{Key insight:} Accelerometer measures \emph{specific force} $\mathbf{f} = \mathbf{a} - \mathbf{g}$,\\
        not true acceleration $\mathbf{a}$. It senses what a test mass ``feels.''
    };
\end{tikzpicture}
\caption{Accelerometer spring-mass analogy. The sensor measures displacement of a proof mass, which depends on both acceleration and gravity. At rest, it reads $+g$; in free fall, it reads zero.}
\label{fig:accelerometer-specific-force}
\end{figure}
%----------------------------------------------------------------------

\textbf{Intuition}: Imagine a spring with a mass inside the sensor. When you accelerate upward, the mass ``lags behind,'' compressing the spring. But gravity also pulls on the mass. The sensor measures the total spring compression, which corresponds to $\mathbf{a} - \mathbf{g}$, not $\mathbf{a}$ alone.

\subsection{The Static Case: Measuring Gravity}

\begin{notebox}[title=Assumption: Static or Quasi-Static]
If the body is \textbf{stationary} (or moving at constant velocity), then $\mathbf{a}_{body} = 0$ and:
\[
\mathbf{f}_{meas} = -R^B_W \mathbf{g}
\]
The accelerometer measures gravity, rotated into the body frame.
\end{notebox}

This is powerful: \textbf{since we know gravity points straight down in the world frame, the accelerometer tells us which way is ``down'' in the body frame---i.e., the tilt of the body!}

\subsection{Computing Roll and Pitch from Accelerometer}

Let $\mathbf{f} = [f_x, f_y, f_z]^T$ be the accelerometer reading, normalized to units of $g$.

For the ZYX Euler convention (assuming static and no noise):
\begin{equation}
\begin{bmatrix} f_x \\ f_y \\ f_z \end{bmatrix} = \begin{bmatrix} \sin\theta \\ -\cos\theta \sin\phi \\ -\cos\theta \cos\phi \end{bmatrix}
\end{equation}

Solving:
\begin{align}
\phi &= \text{atan2}(-f_y, -f_z) \label{eq:roll_from_accel} \\
\theta &= \text{atan2}\left(f_x, \sqrt{f_y^2 + f_z^2}\right) \label{eq:pitch_from_accel}
\end{align}

\begin{warningbox}[title=Accelerometer Cannot Measure Yaw]
Gravity is a vertical vector. Rotating about the vertical axis (yaw) doesn't change the gravity direction in the body frame. Therefore, the accelerometer provides \textbf{no information about yaw}. This is not a sensor limitation---it's physics!
\end{warningbox}

\subsection{Limitations During Motion}

%----------------------------------------------------------------------
% FIGURE: Accelerometer Error During Horizontal Acceleration
%----------------------------------------------------------------------
\begin{figure}[htbp]
\centering
\begin{tikzpicture}[scale=0.9]
    % === Left: Reality (level quadrotor accelerating) ===
    \begin{scope}[shift={(-5,0)}]
        \node[font=\bfseries\small] at (0,2.5) {Reality};

        % Quadrotor (level)
        \draw[very thick, gray] (-1.2,0) -- (1.2,0);
        \draw[very thick, gray] (0,-0.6) -- (0,0.6);
        \fill[gray!40] (1.2,0) circle (0.15);
        \fill[gray!40] (-1.2,0) circle (0.15);
        \fill[gray!40] (0,0.6) circle (0.15);
        \fill[gray!40] (0,-0.6) circle (0.15);
        \fill[red!60] (1.35,0) -- (1.5,0) -- (1.35,0.1) -- cycle;

        % Gravity
        \draw[-{Stealth}, very thick, blue!70] (0,0) -- (0,-1.5) node[right] {$\mathbf{g}$};

        % Acceleration
        \draw[-{Stealth}, very thick, green!60!black] (0,0) -- (1.5,0)
            node[above, font=\small] {$\mathbf{a} = 0.5g$};

        \node[font=\scriptsize, text width=2.5cm, align=center] at (0,-2.5)
            {Quadrotor is \textbf{level}\\accelerating right};
    \end{scope}

    % === Middle: What accelerometer measures ===
    \begin{scope}[shift={(0,0)}]
        \node[font=\bfseries\small] at (0,2.5) {Accelerometer Sees};

        % Vector diagram
        % g pointing down
        \draw[-{Stealth}, thick, blue!70] (0,0) -- (0,-1.5);
        \node[left, blue!70, font=\small] at (0,-0.75) {$-\mathbf{g}$};

        % a pointing right (but we measure f = a - g, so -a points left as contribution)
        % Actually f = a - g means we see the combination
        % Let's show the triangle properly

        % f = a - g: starts at 0, show components
        \draw[-{Stealth}, thick, green!60!black] (0,0) -- (1,0);
        \node[above, green!60!black, font=\small] at (0.5,0) {$\mathbf{a}$};

        % Resultant f
        \draw[-{Stealth}, very thick, orange] (0,0) -- (1,-1.5)
            node[right, font=\small] {$\mathbf{f}$};

        % Angle indicator
        \draw[thick, red!70] (0,-0.6) arc (-90:-56:0.6);
        \node[red!70, font=\small] at (0.4,-0.9) {$27°$};

        % Dashed line showing vertical
        \draw[dashed, gray] (0,0) -- (0,-1.8);

        \node[font=\scriptsize, text width=2.5cm, align=center] at (0,-2.5)
            {Measured force tilted\\by $\arctan(0.5) \approx 27°$};
    \end{scope}

    % === Right: Naive interpretation ===
    \begin{scope}[shift={(5,0)}]
        \node[font=\bfseries\small] at (0,2.5) {Naive Interpretation};

        % Ghost quadrotor (tilted - what we'd wrongly conclude)
        \begin{scope}[rotate=-27]
            \draw[very thick, gray, dashed, opacity=0.5] (-1.2,0) -- (1.2,0);
            \draw[very thick, gray, dashed, opacity=0.5] (0,-0.6) -- (0,0.6);
            \draw[gray!40, dashed, opacity=0.5] (1.2,0) circle (0.15);
            \draw[gray!40, dashed, opacity=0.5] (-1.2,0) circle (0.15);
            \draw[gray!40, dashed, opacity=0.5] (0,0.6) circle (0.15);
            \draw[gray!40, dashed, opacity=0.5] (0,-0.6) circle (0.15);
        \end{scope}

        % Real quadrotor (level, behind)
        \draw[very thick, gray] (-1.2,0) -- (1.2,0);
        \draw[very thick, gray] (0,-0.6) -- (0,0.6);
        \fill[gray!40] (1.2,0) circle (0.15);
        \fill[gray!40] (-1.2,0) circle (0.15);
        \fill[gray!40] (0,0.6) circle (0.15);
        \fill[gray!40] (0,-0.6) circle (0.15);

        % Error annotation
        \draw[<->, thick, red!70] (1.5,-0.8) arc (-30:0:1);
        \node[red!70, font=\small] at (2.2,-0.4) {27° error!};

        \node[font=\scriptsize, text width=2.5cm, align=center] at (0,-2.5)
            {Would wrongly\\conclude tilted};
    \end{scope}

    % Warning box
    \node[draw, rounded corners, fill=red!10, font=\small, text width=11cm, align=center]
        at (0,-4) {
        \textbf{Warning:} During horizontal acceleration of $0.5g$, naive tilt estimate\\
        has 27° error even though the quadrotor is perfectly level!
    };
\end{tikzpicture}
\caption{Accelerometer error during horizontal acceleration. A level quadrotor accelerating at $0.5g$ produces a tilted specific force vector, leading to a 27° error if we naively interpret this as gravity direction.}
\label{fig:accel-horizontal-error}
\end{figure}
%----------------------------------------------------------------------

\begin{warningbox}[title=Assumption Violation During Acceleration]
When the quadrotor accelerates, the assumption $\mathbf{a}_{body} = 0$ fails. The accelerometer measures:
\[
\mathbf{f} = R^B_W (\mathbf{a}_{body} - \mathbf{g})
\]
If we naively interpret this as $-R^B_W \mathbf{g}$, we get errors proportional to the acceleration.

\textbf{Example}: A horizontal acceleration of $0.5g$ causes an apparent tilt of $\arctan(0.5) \approx 27°$!
\end{warningbox}

\begin{keyidea}[title=Accelerometer Summary]
The accelerometer provides an \textbf{absolute reference for roll and pitch}---no drift over time. But it's \textbf{noisy} and \textbf{corrupted by any non-gravitational acceleration}. It's reliable for slow, gentle motion but unreliable during aggressive maneuvers.
\end{keyidea}

\section{Magnetometer}

\subsection{What Does a Magnetometer Measure?}

A magnetometer measures Earth's magnetic field in the body frame:
\begin{equation}
\mathbf{m}_{meas} = R^B_W \mathbf{m}_{earth} + \mathbf{b}_m + \mathbf{n}_m
\end{equation}

Since Earth's magnetic field has a known direction at each location, this provides heading (yaw) information.

\subsection{Computing Yaw}

After compensating for tilt (using roll and pitch from the accelerometer), project the magnetic field onto the horizontal plane:
\begin{equation}
\psi = \text{atan2}(-m_y', m_x')
\end{equation}
where $(m_x', m_y')$ are the tilt-compensated horizontal components.

\begin{warningbox}[title=Magnetic Disturbances]
Magnetometers are highly susceptible to interference:
\begin{itemize}
    \item \textbf{Hard iron}: Permanent magnetic fields from motors, batteries, metal parts
    \item \textbf{Soft iron}: Distortion from ferromagnetic materials
    \item \textbf{External fields}: Power lines, buildings, reinforced concrete
\end{itemize}
Near motors or indoors, magnetometer readings can be completely unusable. Calibration helps but can't fix time-varying disturbances.
\end{warningbox}

\section{Sensor Fusion Motivation}
\index{sensor fusion}

Let's summarize what each sensor provides:

\begin{center}
\begin{tabular}{lccc}
\toprule
& \textbf{Gyroscope} & \textbf{Accelerometer} & \textbf{Magnetometer} \\
\midrule
Measures & Angular velocity & Specific force & Magnetic field \\
Provides & All 3 orientation rates & Roll, pitch (static) & Yaw \\
Short-term accuracy & Excellent & Moderate & Moderate \\
Long-term accuracy & Poor (drift) & Good (no drift) & Moderate \\
During acceleration & Good & Poor & Good \\
Noise level & Low & Medium & High \\
\bottomrule
\end{tabular}
\end{center}

%----------------------------------------------------------------------
% FIGURE: Sensor Characteristics in Frequency Domain
%----------------------------------------------------------------------
\begin{figure}[htbp]
\centering
\begin{tikzpicture}
    \begin{semilogxaxis}[
        width=12cm,
        height=7cm,
        xlabel={Frequency [Hz]},
        ylabel={Orientation Error},
        xmin=0.001, xmax=100,
        ymin=0, ymax=10,
        grid=both,
        grid style={gray!30},
        minor grid style={gray!10},
        legend pos=north east,
        legend style={font=\small},
        ytick={0,2,4,6,8,10},
        yticklabels={Low,,,,, High},
    ]
        % Gyroscope error: high at low frequencies (drift), low at high frequencies
        \addplot[thick, blue, domain=0.001:100, samples=200]
            {3/(1 + 5*x) + 0.5};
        \addlegendentry{Gyroscope error}

        % Accelerometer error: low at low frequencies, high at high frequencies
        \addplot[thick, red, domain=0.001:100, samples=200]
            {0.5 + 3*x/(1 + x)};
        \addlegendentry{Accelerometer error}

        % Fusion region (shaded)
        \addplot[fill=green!20, draw=none, domain=0.05:2, samples=50]
            {min(3/(1 + 5*x) + 0.5, 0.5 + 3*x/(1 + x))} \closedcycle;

        % Annotations
        \node[blue, font=\small, align=center] at (axis cs:0.005,6.5)
            {Drift\\dominates};
        \node[red, font=\small, align=center] at (axis cs:30,6.5)
            {Noise/motion\\dominates};
        \node[green!50!black, font=\small, align=center, fill=white,
              rounded corners, inner sep=2pt] at (axis cs:0.3,1.5)
            {Fusion\\region};

        % Crossover frequency marker
        \draw[dashed, gray, thick] (axis cs:0.2,0) -- (axis cs:0.2,4);
        \node[font=\footnotesize, gray] at (axis cs:0.2,4.5) {$f_c$};
    \end{semilogxaxis}

    % Key insight box
    \node[draw=blue!50!black, fill=blue!5, rounded corners,
          font=\small, align=left, text width=7cm]
        at (6,-1.8) {
        \textbf{Complementary filter insight:}\\
        High-pass gyro + Low-pass accel = Best of both!
    };
\end{tikzpicture}
\caption{Sensor error characteristics in the frequency domain. The gyroscope (blue) has low error at high frequencies but drifts at low frequencies. The accelerometer (red) provides good DC reference but is noisy at high frequencies. The complementary filter exploits this frequency separation.}
\label{fig:sensor-frequency-characteristics}
\end{figure}

\begin{keyidea}[title=The Sensor Fusion Insight]
Each sensor has complementary strengths:
\begin{itemize}
    \item \textbf{Gyroscope}: Accurate in the short term, drifts long term
    \item \textbf{Accelerometer}: Noisy but provides absolute roll/pitch reference
\end{itemize}
The solution: \textbf{fuse} them! Trust the gyroscope for fast changes, trust the accelerometer for the long-term average. This is the complementary filter, which we develop next.
\end{keyidea}

\section{Angular Velocity vs. Euler Angle Rates}
\index{angular velocity!vs. Euler rates}
\index{Euler angles!rate transformation}

The gyroscope measures body angular velocity $\boldsymbol{\omega} = [p, q, r]^T$, but Euler angle rates $[\dot\phi, \dot\theta, \dot\psi]^T$ are often more intuitive for understanding motion. A critical insight is that these are \emph{not} the same quantities, and understanding their relationship is essential for correctly implementing attitude estimation.

\begin{notebox}[title=Rotation Convention]
This derivation uses the \textbf{ZYX (yaw-pitch-roll) intrinsic rotation sequence}, consistent with the Euler angle convention used throughout this course. The rotation from world frame to body frame is: first yaw ($\psi$) about the world $z$-axis, then pitch ($\theta$) about the resulting $y'$-axis, then roll ($\phi$) about the final $x''$-axis (which becomes the body $x$-axis).
\end{notebox}

\subsection{Why Are They Different?}

Using the ZYX convention, each Euler angle rate rotates about a \emph{different} axis:
\begin{itemize}
    \item $\dot\psi$ (yaw rate) rotates about the \textbf{world} $z$-axis
    \item $\dot\theta$ (pitch rate) rotates about the \textbf{intermediate} $y'$-axis (after yaw)
    \item $\dot\phi$ (roll rate) rotates about the \textbf{body} $x''$-axis (after yaw and pitch)
\end{itemize}

In contrast, the gyroscope measures angular velocity components $[p, q, r]^T$ all resolved in the \textbf{current body frame}. These three body axes are generally different from the three Euler rotation axes.

%----------------------------------------------------------------------
% FIGURE: Euler Rotation Axes vs Body Axes
%----------------------------------------------------------------------
\begin{figure}[htbp]
\centering
\begin{tikzpicture}[scale=1.1,
    x={(0.85cm,-0.35cm)}, y={(0.85cm,0.35cm)}, z={(0cm,0.95cm)},
    >=Stealth
]
    % World frame (faint)
    \draw[gray!50, thick, ->] (0,0,0) -- (2.5,0,0) node[right, font=\small] {$x_W$};
    \draw[gray!50, thick, ->] (0,0,0) -- (0,2.5,0) node[above, font=\small] {$y_W$};
    \draw[gray!50, thick, ->] (0,0,0) -- (0,0,2.5) node[above, font=\small] {$z_W$};

    % Yaw axis (world z) - green
    \draw[green!60!black, ultra thick, ->] (0,0,0) -- (0,0,2)
        node[above right, font=\small\bfseries] {$\dot\psi$ axis};

    % Body frame (pitched and rolled) - show tilted
    \begin{scope}[rotate around z=20, rotate around y=25]
        % Body axes
        \draw[blue!70, thick, ->] (0,0,0) -- (2,0,0) node[right, font=\small] {$x_B$};
        \draw[blue!70, thick, ->] (0,0,0) -- (0,2,0) node[above, font=\small] {$y_B$};
        \draw[blue!70, thick, ->] (0,0,0) -- (0,0,2) node[above, font=\small] {$z_B$};

        % Roll axis (body x) - red
        \draw[red!70, ultra thick, ->] (0,0,0) -- (1.8,0,0)
            node[below right, font=\small\bfseries] {$\dot\phi$ axis};
    \end{scope}

    % Intermediate pitch axis (after yaw, before roll) - orange
    \begin{scope}[rotate around z=20]
        \draw[orange!80!black, ultra thick, ->] (0,0,0) -- (0,1.8,0)
            node[above left, font=\small\bfseries] {$\dot\theta$ axis};
    \end{scope}

    % Labels
    \node[font=\footnotesize, text width=4cm, align=left] at (5,0,2) {
        \textcolor{green!60!black}{$\bullet$ Yaw axis: World $z$}\\
        \textcolor{orange!80!black}{$\bullet$ Pitch axis: Intermediate $y'$}\\
        \textcolor{red!70}{$\bullet$ Roll axis: Body $x$}
    };

    \node[font=\footnotesize, text width=4cm, align=left] at (5,0,0) {
        \textcolor{blue!70}{Body frame axes $x_B, y_B, z_B$}\\
        \textcolor{blue!70}{(where gyro measures $p, q, r$)}
    };
\end{tikzpicture}
\caption{The three Euler rotation axes do not coincide with body frame axes except at zero orientation. The gyroscope measures angular velocity components along the blue body axes, but Euler rates describe rotations about three different axes (green, orange, red).}
\label{fig:euler-vs-body-axes}
\end{figure}

\subsection{Deriving the Transformation}

To find the relationship, we express the total angular velocity in the body frame as the sum of contributions from each Euler angle rate.

\textbf{Step 1: Yaw rate contribution.}
The yaw rate $\dot\psi$ produces angular velocity $[0, 0, \dot\psi]^T$ in the world frame. To express this in the body frame, we must rotate through the full Euler sequence $R_x(\phi) R_y(\theta) R_z(\psi)$:
\begin{equation}
\boldsymbol{\omega}_\psi^B = R_x(\phi) R_y(\theta) \begin{bmatrix} 0 \\ 0 \\ \dot\psi \end{bmatrix}
= \begin{bmatrix} -\sin\theta \\ \sin\phi\cos\theta \\ \cos\phi\cos\theta \end{bmatrix} \dot\psi
\end{equation}

\textbf{Step 2: Pitch rate contribution.}
The pitch rate $\dot\theta$ produces angular velocity $[0, \dot\theta, 0]^T$ in the intermediate frame (after yaw). We only need to rotate through roll:
\begin{equation}
\boldsymbol{\omega}_\theta^B = R_x(\phi) \begin{bmatrix} 0 \\ \dot\theta \\ 0 \end{bmatrix}
= \begin{bmatrix} 0 \\ \cos\phi \\ -\sin\phi \end{bmatrix} \dot\theta
\end{equation}

\textbf{Step 3: Roll rate contribution.}
The roll rate $\dot\phi$ is already about the body $x$-axis:
\begin{equation}
\boldsymbol{\omega}_\phi^B = \begin{bmatrix} \dot\phi \\ 0 \\ 0 \end{bmatrix}
\end{equation}

\textbf{Total body angular velocity:}
\begin{equation}
\boldsymbol{\omega}^B = \boldsymbol{\omega}_\phi^B + \boldsymbol{\omega}_\theta^B + \boldsymbol{\omega}_\psi^B
\end{equation}

Combining terms gives the \textbf{forward transformation}:
\begin{equation}
\boxed{
\begin{bmatrix} p \\ q \\ r \end{bmatrix} =
\underbrace{\begin{bmatrix}
1 & 0 & -\sin\theta \\
0 & \cos\phi & \sin\phi\cos\theta \\
0 & -\sin\phi & \cos\phi\cos\theta
\end{bmatrix}}_{T(\phi, \theta)}
\begin{bmatrix} \dot\phi \\ \dot\theta \\ \dot\psi \end{bmatrix}
}
\label{eq:euler-to-body-rates}
\end{equation}

\subsection{Physical Interpretation}

\begin{keyidea}[title=When Are Body Rates Equal to Euler Rates?]
At level orientation ($\phi = \theta = 0$), the transformation matrix becomes the identity:
\[
T(0, 0) = \begin{bmatrix} 1 & 0 & 0 \\ 0 & 1 & 0 \\ 0 & 0 & 1 \end{bmatrix}
\]
So $p = \dot\phi$, $q = \dot\theta$, $r = \dot\psi$---body rates equal Euler rates! But as the quadrotor tilts, the three Euler rotation axes diverge from the body axes, and the rates differ.
\end{keyidea}

Examining each row of the transformation:
\begin{itemize}
    \item \textbf{Row 1} ($p$): The body roll rate $p$ equals $\dot\phi$ minus a contribution from $\dot\psi$ when pitched ($-\sin\theta \cdot \dot\psi$). When pitched nose-up, yawing produces some body roll.

    \item \textbf{Row 2} ($q$): The body pitch rate $q$ is a mix of $\dot\theta$ and $\dot\psi$, weighted by roll angle. When rolled, yawing produces body pitch.

    \item \textbf{Row 3} ($r$): The body yaw rate $r$ similarly mixes $\dot\theta$ and $\dot\psi$, weighted by roll angle.
\end{itemize}

\subsection{Worked Examples}

\begin{example}[Level Flight]
At $\phi = 0°$, $\theta = 0°$, with Euler rates $[\dot\phi, \dot\theta, \dot\psi] = [10, 5, 20]$ deg/s:
\begin{align}
\begin{bmatrix} p \\ q \\ r \end{bmatrix} &=
\begin{bmatrix} 1 & 0 & 0 \\ 0 & 1 & 0 \\ 0 & 0 & 1 \end{bmatrix}
\begin{bmatrix} 10 \\ 5 \\ 20 \end{bmatrix} =
\begin{bmatrix} 10 \\ 5 \\ 20 \end{bmatrix} \text{ deg/s}
\end{align}
Body rates equal Euler rates when level.
\end{example}

\begin{example}[Pitched Nose-Up 30°]
At $\phi = 0°$, $\theta = 30°$, with the same Euler rates $[\dot\phi, \dot\theta, \dot\psi] = [10, 5, 20]$ deg/s:
\begin{align}
\begin{bmatrix} p \\ q \\ r \end{bmatrix} &=
\begin{bmatrix} 1 & 0 & -0.5 \\ 0 & 1 & 0 \\ 0 & 0 & 0.866 \end{bmatrix}
\begin{bmatrix} 10 \\ 5 \\ 20 \end{bmatrix} =
\begin{bmatrix} 10 - 10 \\ 5 \\ 17.3 \end{bmatrix} =
\begin{bmatrix} 0 \\ 5 \\ 17.3 \end{bmatrix} \text{ deg/s}
\end{align}
The body roll rate $p$ is now zero! The yaw motion ($\dot\psi = 20$ deg/s) about the world vertical axis contributes to roll in the body frame, canceling the explicit roll rate. The body yaw rate is also reduced.
\end{example}

\subsection{The Inverse Transformation}

To convert gyroscope measurements $[p, q, r]^T$ to Euler angle rates, we invert the matrix:
\begin{equation}
\boxed{
\begin{bmatrix} \dot\phi \\ \dot\theta \\ \dot\psi \end{bmatrix} =
\underbrace{\begin{bmatrix}
1 & \sin\phi\tan\theta & \cos\phi\tan\theta \\
0 & \cos\phi & -\sin\phi \\
0 & \sin\phi\sec\theta & \cos\phi\sec\theta
\end{bmatrix}}_{W(\phi, \theta)}
\begin{bmatrix} p \\ q \\ r \end{bmatrix}
}
\label{eq:body-to-euler-rates}
\end{equation}

This matrix $W(\phi, \theta)$ appears in the attitude kinematics: $\dot{\boldsymbol{\Theta}} = W(\boldsymbol{\Theta}) \boldsymbol{\omega}$.

\subsection{The Singularity Problem}

\begin{warningbox}[title=Singularity at Gimbal Lock]
The inverse transformation contains $\tan\theta$ and $\sec\theta = 1/\cos\theta$. At $\theta = \pm 90°$, these terms are undefined:
\begin{center}
\begin{tabular}{ccc}
\toprule
$\theta$ & $\tan\theta$ & $\sec\theta$ \\
\midrule
$80°$ & 5.67 & 5.76 \\
$85°$ & 11.4 & 11.5 \\
$89°$ & 57.3 & 57.3 \\
$89.9°$ & 573 & 573 \\
$90°$ & $\infty$ & $\infty$ \\
\bottomrule
\end{tabular}
\end{center}
This is another manifestation of gimbal lock---the same singularity we encountered in Euler angle representations.
\end{warningbox}

\begin{example}[Singularity Behavior]
Consider a quadrotor at $\theta = 89°$, $\phi = 10°$, with body angular velocity $[p, q, r] = [0, 10, 0]$ deg/s (pure pitch rate in body frame):
\begin{align}
\dot\phi &= 0 + 10 \cdot \sin(10°) \cdot \tan(89°) + 0 = 10 \cdot 0.174 \cdot 57.3 = 99.6 \text{ deg/s} \\
\dot\theta &= 10 \cdot \cos(10°) = 9.85 \text{ deg/s} \\
\dot\psi &= 10 \cdot \sin(10°) \cdot \sec(89°) = 99.6 \text{ deg/s}
\end{align}

A modest body pitch rate of 10 deg/s requires Euler rates of nearly 100 deg/s! As $\theta \to 90°$, the required Euler rates become unbounded. This is why \textbf{you should never integrate Euler rates directly} near gimbal lock.
\end{example}

\subsection{Quaternion Rate Equation}

For quaternions, the relationship between angular velocity and quaternion rate is elegantly simple and singularity-free:
\begin{equation}
\boxed{\dot{\mathbf{q}} = \frac{1}{2} \mathbf{q} \otimes \begin{bmatrix} 0 \\ \boldsymbol{\omega} \end{bmatrix}}
\label{eq:quaternion-rate}
\end{equation}

where $\boldsymbol{\omega} = [p, q, r]^T$ is the body angular velocity (directly from the gyroscope).

\textbf{Advantages of the quaternion formulation}:
\begin{itemize}
    \item \textbf{No singularity}: Works at any orientation, including $\theta = \pm 90°$
    \item \textbf{Direct gyro use}: The gyroscope output $[p, q, r]$ enters directly---no transformation needed
    \item \textbf{No transcendentals}: Only multiplication and addition (faster than $\sin$, $\cos$, $\tan$)
    \item \textbf{Stable integration}: Quaternion normalization is cheap; Euler angle corrections are not
\end{itemize}

\begin{keyidea}[title=Best Practice for Attitude Integration]
\begin{enumerate}
    \item \textbf{Store orientation as a quaternion} internally
    \item \textbf{Integrate using the quaternion rate equation} $\dot{\mathbf{q}} = \frac{1}{2}\mathbf{q} \otimes [0; \boldsymbol{\omega}]$
    \item \textbf{Normalize} the quaternion periodically: $\mathbf{q} \leftarrow \mathbf{q}/\|\mathbf{q}\|$
    \item \textbf{Convert to Euler angles only for display}---never for computation
\end{enumerate}
This approach avoids gimbal lock entirely and is computationally efficient.
\end{keyidea}


\chapter{Sensor Calibration}
\index{calibration}

Sensor fusion algorithms assume calibrated sensor data. Without proper calibration, even the best estimation algorithm will produce poor results. This chapter covers the practical aspects of IMU calibration\index{calibration|textbf} for quadrotor flight controllers.

\section{Why Calibration Matters}

Raw sensor readings from an IMU contain systematic errors that can be corrected through calibration:

\begin{center}
\begin{tabular}{p{3cm}p{4cm}p{5cm}}
\toprule
\textbf{Error Type} & \textbf{Source} & \textbf{Effect if Uncorrected} \\
\midrule
Bias (offset) & Manufacturing tolerances, temperature & Constant drift, attitude offset \\
Scale factor & Sensitivity variation & Incorrect magnitude, gain error \\
Misalignment & Sensor not aligned with PCB & Cross-axis coupling \\
Non-linearity & Sensor physics & Errors at extreme values \\
Noise & Electronics, quantization & Random fluctuations \\
\bottomrule
\end{tabular}
\end{center}

\begin{example}[Impact of Uncalibrated Gyroscope]
A gyroscope with 1°/s bias, integrated over 10 seconds, produces a 10° attitude error. At 500 Hz sampling, this is:
\[
\theta_{error} = b_{gyro} \cdot t = 1°/\text{s} \times 10\text{ s} = 10°
\]

This error accumulates indefinitely. Without calibration (or sensor fusion), the quadrotor would be uncontrollable within seconds.
\end{example}

\section{The Calibration Model}

A calibrated measurement is obtained from the raw reading using:
\[
\mathbf{m}_{calibrated} = \mathbf{S}^{-1} \cdot \mathbf{M}^{-1} \cdot (\mathbf{m}_{raw} - \mathbf{b})
\]

where:
\begin{itemize}
    \item $\mathbf{m}_{raw}$: Raw sensor reading (vector)
    \item $\mathbf{b}$: Bias vector (offset from zero)
    \item $\mathbf{S}$: Scale factor matrix (diagonal for independent axes)
    \item $\mathbf{M}$: Misalignment matrix (accounts for non-orthogonal axes)
\end{itemize}

For most applications, we can simplify to:
\[
m_{calibrated,i} = \frac{m_{raw,i} - b_i}{s_i}
\]

where $b_i$ is the bias and $s_i$ is the scale factor for axis $i$.

\section{Gyroscope Calibration}

\subsection{Bias Estimation (Static Calibration)}

Gyroscope bias is estimated when the sensor is stationary. The true angular velocity is zero, so any reading is pure bias.

\begin{keyidea}[title=Gyroscope Bias Calibration]
\textbf{Procedure}:
\begin{enumerate}
    \item Place the quadrotor on a stable surface
    \item Collect $N$ samples (typically $N = 1000$ at 500 Hz = 2 seconds)
    \item Compute the average: $\hat{b} = \frac{1}{N} \sum_{i=1}^{N} \omega_{raw,i}$
    \item Store $\hat{b}$ for runtime correction
\end{enumerate}

\textbf{Runtime}: Subtract the estimated bias from each reading:
\[
\omega_{calibrated} = \omega_{raw} - \hat{b}
\]
\end{keyidea}

The complete implementation is provided in Listing~\ref{lst:gyro-calib} (Appendix~\ref{app:code-listings}). The key steps are: (1) collect samples while stationary, (2) compute the mean as the bias estimate, and (3) subtract the bias from all subsequent readings.

\subsection{Bias Drift and Temperature}

Gyroscope bias is not constant---it varies with temperature. A typical MEMS gyroscope has:
\[
b(T) = b_0 + \alpha (T - T_0)
\]

where $\alpha$ is the temperature coefficient (typically 0.01--0.05 °/s/°C).

\begin{notebox}[title=Temperature Compensation]
For precise applications:
\begin{enumerate}
    \item Calibrate at multiple temperatures (e.g., 0°C, 25°C, 50°C)
    \item Fit a polynomial: $b(T) = a_0 + a_1 T + a_2 T^2$
    \item At runtime, read temperature and compute $b(T)$
\end{enumerate}

For the Crazyflie, the temperature change during a short flight is small, so constant bias compensation is usually sufficient.
\end{notebox}

\subsection{Online Bias Estimation}

For longer flights, bias can be estimated online when the quadrotor is detected to be stationary (low acceleration, low angular velocity):

See Listing~\ref{lst:gyro-online} (Appendix~\ref{app:code-listings}) for the complete implementation. The algorithm detects stationary periods by checking both angular velocity and acceleration magnitude, then slowly updates the bias estimate using an exponential moving average.

\section{Accelerometer Calibration}

Accelerometer calibration is more involved because we cannot easily create a ``zero acceleration'' reference (gravity is always present).

\subsection{The Six-Position Method}

The standard method uses gravity as a reference. By placing each axis alternately pointing up and down, we measure $\pm g$ on each axis.

\begin{keyidea}[title=Six-Position Accelerometer Calibration]
\textbf{Positions}:
\begin{enumerate}
    \item $+Z$ up: Expected reading $(0, 0, +g)$
    \item $-Z$ up: Expected reading $(0, 0, -g)$
    \item $+X$ up: Expected reading $(+g, 0, 0)$
    \item $-X$ up: Expected reading $(-g, 0, 0)$
    \item $+Y$ up: Expected reading $(0, +g, 0)$
    \item $-Y$ up: Expected reading $(0, -g, 0)$
\end{enumerate}

From each pair of measurements:
\[
b_i = \frac{m_{+} + m_{-}}{2}, \quad s_i = \frac{m_{+} - m_{-}}{2g}
\]
\end{keyidea}

\begin{example}[Accelerometer Calibration Calculation]
For the Z-axis, we measure:
\begin{itemize}
    \item $+Z$ up (quadrotor upside down): raw reading $= -520$
    \item $-Z$ up (quadrotor right-side up): raw reading $= +510$
\end{itemize}

Expected for $g = 9.81$ m/s² with nominal scale 1000 LSB/g:
\[
b_z = \frac{-520 + 510}{2} = -5 \text{ LSB}
\]
\[
s_z = \frac{-520 - 510}{2 \times 9.81} = \frac{-1030}{19.62} = -52.5 \text{ LSB/(m/s²)}
\]

The negative scale indicates the sensor axis is inverted relative to convention.
\end{example}

The complete implementation including data structures and calibration computation is provided in Listing~\ref{lst:accel-sixpos} (Appendix~\ref{app:code-listings}).

\subsection{Simplified Level Calibration}

For applications where only the level orientation matters (typical for quadrotors), a simpler two-position calibration suffices:

See Listing~\ref{lst:accel-level} (Appendix~\ref{app:code-listings}) for the implementation.

\section{Magnetometer Calibration}

Magnetometers are particularly challenging to calibrate because they are affected by:
\begin{itemize}
    \item \textbf{Hard iron}: Constant magnetic fields from nearby ferromagnetic materials
    \item \textbf{Soft iron}: Distortion of the Earth's field by nearby materials
\end{itemize}

\subsection{Hard Iron Calibration}

Hard iron effects create a constant offset. Calibration involves rotating the sensor through all orientations and finding the center of the resulting ellipsoid.

See Listing~\ref{lst:mag-calib} (Appendix~\ref{app:code-listings}) for the implementation.

\begin{warningbox}[title=Magnetometer Limitations on Quadrotors]
On quadrotors, magnetometers are severely affected by:
\begin{itemize}
    \item Motor magnetic fields (varying with throttle)
    \item Battery current loops
    \item Power wires
\end{itemize}

Many flight controllers ignore magnetometer data during high-throttle flight and use GPS heading instead. For the Crazyflie (indoor, no GPS), yaw estimation often relies on gyroscope integration with occasional magnetometer corrections when motors are at low power.
\end{warningbox}

\section{Calibration Storage and Validation}

\subsection{Storing Calibration Parameters}

Calibration parameters should be stored in non-volatile memory (EEPROM or Flash) so they persist across power cycles:

See Listing~\ref{lst:calib-storage} (Appendix~\ref{app:code-listings}) for a complete implementation including magic number validation and CRC integrity checking.

\subsection{Validating Calibration Quality}

After calibration, verify the quality:

See Listing~\ref{lst:calib-validate} (Appendix~\ref{app:code-listings}) for a complete validation implementation that checks:
\begin{itemize}
    \item Gyro bias within reasonable bounds (typically $< 5$ deg/s)
    \item Accelerometer scale factors close to unity (0.9--1.1)
    \item Calibrated accelerometer magnitude close to $g$ when level
\end{itemize}

\section{Calibration Best Practices}

\begin{keyidea}[title=Calibration Guidelines for Quadrotors]
\begin{enumerate}
    \item \textbf{Calibrate on every cold start}: Gyro bias changes with temperature
    \item \textbf{Wait for thermal equilibrium}: Let the sensors warm up (30--60 seconds) before calibration
    \item \textbf{Use a stable surface}: Any vibration or motion corrupts calibration
    \item \textbf{Recalibrate periodically}: Especially after crashes or sensor replacement
    \item \textbf{Validate after calibration}: Check that calibrated values are reasonable
    \item \textbf{Store calibration data}: Save to EEPROM so it persists across power cycles
    \item \textbf{Log calibration values}: Record for debugging and quality tracking
\end{enumerate}
\end{keyidea}

\begin{notebox}[title=Crazyflie Calibration]
The Crazyflie firmware performs automatic gyroscope bias calibration at startup. When you see the LEDs change pattern after power-on, the gyroscope calibration is complete. The quadrotor must be stationary during this time (typically 1--2 seconds).

Accelerometer calibration parameters are stored in EEPROM and can be updated using the Crazyflie client software.
\end{notebox}


\chapter{Orientation Estimation}

%----------------------------------------------------------------------
% FIGURE: Complementary Filter Concept
%----------------------------------------------------------------------
\begin{figure}[htbp]
\centering
\begin{tikzpicture}[
    block/.style={draw, rectangle, minimum height=2.5em, minimum width=3.5em,
                  align=center, rounded corners=2pt},
    sum/.style={draw, circle, minimum size=1.5em, inner sep=0pt},
    >=Stealth
]
    % === Block Diagram (upper part) ===
    % Gyroscope branch
    \node[block, fill=blue!10] (gyro) at (0,2) {$\omega_g$\\(gyro)};
    \node[block, fill=blue!20] (int) at (2.5,2) {$\displaystyle\frac{1}{s}$};
    \node[block, fill=blue!30] (hp) at (5.5,2) {$\displaystyle\frac{\tau s}{\tau s + 1}$\\High-pass};

    % Accelerometer branch
    \node[block, fill=red!10] (accel) at (0,0) {$\theta_a$\\(accel)};
    \node[block, fill=red!30] (lp) at (5.5,0) {$\displaystyle\frac{1}{\tau s + 1}$\\Low-pass};

    % Summing junction and output
    \node[sum] (sum) at (8,1) {$+$};
    \node[block, fill=green!20] (output) at (10.5,1) {$\hat\theta$\\(estimate)};

    % Connections - gyro branch
    \draw[->] (gyro) -- (int) node[midway, above, font=\small] {rate};
    \draw[->] (int) -- (hp) node[midway, above, font=\small] {$\theta_g$};
    \draw[->] (hp) -- (sum.north west);

    % Connections - accel branch
    \draw[->] (accel) -- (lp) node[midway, above, font=\small] {angle};
    \draw[->] (lp) -- (sum.south west);

    % Output
    \draw[->] (sum) -- (output);

    % Annotations
    \node[blue!70!black, font=\footnotesize, align=center] at (5.5,3.2)
        {Trust gyro at\\high frequencies};
    \node[red!70!black, font=\footnotesize, align=center] at (5.5,-1.2)
        {Trust accel at\\low frequencies};
    \node[green!50!black, font=\footnotesize] at (10.5,2)
        {Best of both!};

    % Key relationship box
    \node[draw=gray, fill=gray!10, rounded corners, font=\small,
          align=center] at (2.5,-2.5)
        {Key property: $G(s) + H(s) = 1$\\(complementary)};

    % === Frequency Response Inset ===
    \begin{scope}[shift={(12.5,1)}]
        \draw[->] (-1.5,-1.5) -- (2,-1.5) node[right, font=\tiny] {$\omega$};
        \draw[->] (-1.5,-1.5) -- (-1.5,1.5) node[above, font=\tiny] {$|H|$};

        % Low-pass response (red)
        \draw[thick, red] (-1.3,1) .. controls (0,0.9) and (0.5,0) .. (1.5,-1.2);
        % High-pass response (blue)
        \draw[thick, blue] (-1.3,-1.2) .. controls (0,-0.5) and (0.5,0.5) .. (1.5,1);

        % Crossover
        \draw[dashed, gray] (0,-1.5) -- (0,1) node[above, font=\tiny] {$\frac{1}{\tau}$};

        % Labels
        \node[red, font=\tiny] at (1.8,-0.8) {LP};
        \node[blue, font=\tiny] at (1.8,0.8) {HP};
    \end{scope}
\end{tikzpicture}
\caption{The complementary filter combines high-pass filtered gyroscope data with low-pass filtered accelerometer data. Since $G(s) + H(s) = 1$, the filters are ``complementary'' and cover all frequencies. The inset shows the frequency responses crossing over at $\omega = 1/\tau$.}
\label{fig:complementary-filter-concept}
\end{figure}

\section{The Sensor Fusion Problem}

We now have all the pieces:
\begin{itemize}
    \item \textbf{Gyroscope}: Gives orientation \emph{rates}, excellent short-term, drifts long-term
    \item \textbf{Accelerometer}: Gives absolute roll/pitch, noisy, corrupted during acceleration
\end{itemize}

The goal: combine these to get accurate orientation estimates that are:
\begin{itemize}
    \item Responsive to fast changes (like the gyroscope)
    \item Accurate in the long term (like the accelerometer)
\end{itemize}

\begin{keyidea}[title=Sensor Fusion Strategy]
\begin{itemize}
    \item For \textbf{fast changes} (high frequency): Trust the gyroscope
    \item For \textbf{slow changes/steady state} (low frequency): Trust the accelerometer
\end{itemize}
This naturally suggests filtering: high-pass filter the gyroscope, low-pass filter the accelerometer, and add.
\end{keyidea}

\section{1D Complementary Filter}
\index{complementary filter}

We derive the filter for a single angle (e.g., pitch). The extension to 3D follows.

\subsection{Setup}

Let:
\begin{itemize}
    \item $\theta$: True angle we want to estimate
    \item $\theta_a$: Angle estimate from accelerometer (noisy but unbiased)
    \item $\omega_g$: Angular velocity from gyroscope (accurate rate but with bias)
\end{itemize}

\subsection{Frequency Domain Derivation}

Define a first-order low-pass filter with time constant $\tau$:
\begin{equation}
G(s) = \frac{1}{\tau s + 1}
\end{equation}

The complementary high-pass filter is:
\begin{equation}
H(s) = 1 - G(s) = \frac{\tau s}{\tau s + 1}
\end{equation}

Key property: $G(s) + H(s) = 1$ for all frequencies. They are ``complementary.''

\textbf{The filter}:
\begin{equation}
\hat\theta(s) = G(s) \cdot \theta_a(s) + H(s) \cdot \theta_g(s)
\end{equation}

where $\theta_g(s) = \omega_g(s)/s$ is the integrated gyroscope signal.

\textbf{Intuition}:
\begin{itemize}
    \item At low frequencies ($\omega \ll 1/\tau$): $G \approx 1$, $H \approx 0$ $\Rightarrow$ use accelerometer
    \item At high frequencies ($\omega \gg 1/\tau$): $G \approx 0$, $H \approx 1$ $\Rightarrow$ use gyroscope
    \item At crossover ($\omega = 1/\tau$): Equal contribution from both
\end{itemize}

\subsection{Time Domain Implementation}

Converting to discrete time with sampling period $h$:

\begin{equation}
\boxed{\hat\theta_k = \alpha \cdot (\hat\theta_{k-1} + h \cdot \omega_{g,k}) + (1-\alpha) \cdot \theta_{a,k}}
\end{equation}

where:
\begin{equation}
\alpha = \frac{\tau}{\tau + h}
\end{equation}

%----------------------------------------------------------------------
% FIGURE: Discrete-Time Complementary Filter
%----------------------------------------------------------------------
\begin{figure}[htbp]
\centering
\begin{tikzpicture}[
    block/.style={draw, rectangle, minimum height=2em, minimum width=2.5em,
                  align=center, rounded corners=2pt},
    gain/.style={draw, circle, minimum size=1.8em, inner sep=0pt, font=\small},
    sum/.style={draw, circle, minimum size=1.5em, inner sep=0pt},
    >=Stealth
]
    % Gyroscope branch
    \node[block, fill=blue!15] (gyro) at (0,1.5) {$\omega_{g,k}$};
    \node[gain, fill=blue!25] (h) at (2,1.5) {$h$};
    \node[sum] (sum1) at (4,1.5) {$+$};
    \node[gain, fill=blue!35] (alpha) at (6,1.5) {$\alpha$};

    % Accelerometer branch
    \node[block, fill=red!15] (accel) at (0,-0.5) {$\theta_{a,k}$};
    \node[gain, fill=red!35] (oneminusalpha) at (6,-0.5) {$1{-}\alpha$};

    % Output summing and delay
    \node[sum] (sum2) at (8,0.5) {$+$};
    \node[block, fill=green!20] (output) at (10,0.5) {$\hat\theta_k$};
    \node[block, fill=gray!20] (delay) at (4,-1.8) {$z^{-1}$};

    % Connections - gyro branch
    \draw[->] (gyro) -- (h);
    \draw[->] (h) -- (sum1);
    \draw[->] (sum1) -- (alpha) node[midway, above, font=\scriptsize] {predict};
    \draw[->] (alpha) -- (sum2.north west);

    % Connections - accel branch
    \draw[->] (accel) -- (oneminusalpha);
    \draw[->] (oneminusalpha) -- (sum2.south west);

    % Output and feedback
    \draw[->] (sum2) -- (output);
    \draw[->] (output.east) -- ++(0.5,0) |- (delay.east);
    \draw[->] (delay.west) -| (sum1.south);

    % Labels
    \node[font=\scriptsize] at (4,-1.2) {$\hat\theta_{k-1}$};
    \node[blue!70!black, font=\footnotesize, align=center] at (3,2.8)
        {Predict from gyro\\(short-term trust)};
    \node[red!70!black, font=\footnotesize, align=center] at (3,-1.6)
        {Correct toward accel\\(long-term reference)};

    % Typical values box
    \node[draw=gray, fill=yellow!10, rounded corners, font=\scriptsize,
          align=left, text width=4.5cm] at (10.5,-1.5) {
        \textbf{Typical values:}\\
        At 100 Hz ($h{=}0.01$s), $\tau{=}1$s:\\
        $\alpha = 0.99$ (99\% gyro, 1\% accel)
    };
\end{tikzpicture}
\caption{Discrete-time complementary filter block diagram. Each iteration predicts from the gyroscope (multiply rate by $h$, add to previous estimate), then corrects toward the accelerometer using a weighted average. The parameter $\alpha$ controls the trust balance.}
\label{fig:discrete-complementary-filter}
\end{figure}

\textbf{Intuition}: Each step:
\begin{enumerate}
    \item Predict using gyroscope: $\hat\theta_{predict} = \hat\theta_{k-1} + h \cdot \omega_{g,k}$
    \item Correct toward accelerometer: blend prediction with accelerometer reading
\end{enumerate}

\subsection{Parameter Selection}

The only tuning parameter is $\tau$ (or equivalently $\alpha$):

\begin{center}
\begin{tabular}{lcc}
\toprule
$\tau$ & $\alpha$ & Behavior \\
\midrule
Large (e.g., 2 s) & $\approx 0.99$ & Trust gyro more, slow correction, less noise \\
Small (e.g., 0.2 s) & $\approx 0.95$ & Trust accel more, fast correction, more noise \\
\bottomrule
\end{tabular}
\end{center}

\begin{notebox}[title=Typical Values]
For quadrotors with MEMS IMUs:
\begin{itemize}
    \item $\tau = 0.5$ to $2$ seconds works well
    \item At 100 Hz ($h = 0.01$ s): $\alpha \approx 0.98$ to $0.995$
\end{itemize}
Start with $\alpha = 0.98$ and tune from there.
\end{notebox}

\subsection{Why Does This Work?}

Consider the error sources:
\begin{itemize}
    \item \textbf{Gyroscope bias}: Creates a slowly-growing error (low frequency). The low-pass filter on the accelerometer corrects this.
    \item \textbf{Accelerometer noise}: High frequency. The high-pass filter on the gyroscope ignores this.
    \item \textbf{Accelerations}: Corrupt accelerometer at high frequencies. The high-pass filter on gyroscope handles fast motion.
\end{itemize}

The complementary filter automatically separates these error sources by frequency!

\section{Extension to 3D: The Mahony Filter}
\index{Mahony filter}

The 1D filter extends to 3D using quaternions. The Mahony filter~\cite{mahony2008nonlinear}\index{Mahony filter|textbf} is a widely-used, computationally efficient approach that provides theoretical guarantees on stability.

\subsection{Key Idea}

Instead of filtering angles directly (which has gimbal lock issues), we:
\begin{enumerate}
    \item Represent orientation as a quaternion
    \item Compute an error vector from the accelerometer
    \item Use this error to correct the gyroscope
    \item Integrate the corrected gyroscope to update the quaternion
\end{enumerate}

\subsection{Error Computation}

\textbf{Step 1}: Normalize accelerometer reading (assuming it measures gravity):
\[
\hat{\mathbf{a}} = \frac{\mathbf{a}_{meas}}{\|\mathbf{a}_{meas}\|}
\]

\textbf{Step 2}: Predict gravity direction in body frame from current quaternion:
\[
\hat{\mathbf{g}}_B = \begin{bmatrix}
2(q_1 q_3 - q_0 q_2) \\
2(q_0 q_1 + q_2 q_3) \\
q_0^2 - q_1^2 - q_2^2 + q_3^2
\end{bmatrix}
\]

This is the third column of the rotation matrix $R^T$ (gravity direction in body frame).

\textbf{Step 3}: Compute error as cross product:
\[
\mathbf{e} = \hat{\mathbf{a}} \times \hat{\mathbf{g}}_B
\]

%----------------------------------------------------------------------
% FIGURE: Mahony Filter Error Computation (Cross Product)
%----------------------------------------------------------------------
\begin{figure}[htbp]
\centering
\begin{tikzpicture}[scale=1.2,
    x={(0.866cm,-0.35cm)}, y={(0.866cm,0.35cm)}, z={(0cm,0.9cm)},
    >=Stealth
]
    % Unit sphere (represented as ellipse)
    \draw[gray!40, thick] (0,0,0) circle (2cm);
    \draw[gray!30, dashed] (2,0,0) arc (0:180:2cm and 0.6cm);
    \draw[gray!40] (-2,0,0) arc (180:360:2cm and 0.6cm);

    % Coordinate axes (faint)
    \draw[gray!30, ->] (0,0,0) -- (2.5,0,0);
    \draw[gray!30, ->] (0,0,0) -- (0,2.5,0);
    \draw[gray!30, ->] (0,0,0) -- (0,0,2.5);

    % Measured gravity vector (accelerometer) - red
    \draw[very thick, red, ->] (0,0,0) -- (0.3,0.5,1.7)
        node[above right, font=\small] {$\hat{\mathbf{a}}$};

    % Predicted gravity vector (from quaternion) - blue
    \draw[very thick, blue, ->] (0,0,0) -- (-0.4,0.7,1.6)
        node[above left, font=\small] {$\hat{\mathbf{g}}_B$};

    % Error vector (cross product) - green, perpendicular to both
    \draw[very thick, green!60!black, ->] (0,0,0) -- (1.5,1.2,0.2)
        node[right, font=\small] {$\mathbf{e} = \hat{\mathbf{a}} \times \hat{\mathbf{g}}_B$};

    % Angle arc between vectors
    \draw[thick, orange] (0.12,0.2,0.68) arc (60:75:0.8);
    \node[orange, font=\footnotesize] at (0.5,0.4,1.1) {$\theta_{err}$};

    % Arc showing rotation path
    \draw[dashed, green!60!black, thick] (0.3,0.5,1.7) arc (35:55:2);

    % Key insight box
    \node[draw=green!50!black, fill=green!5, rounded corners,
          font=\footnotesize, align=left, text width=5cm] at (5.5,0,1.5) {
        \textbf{Cross product gives:}\\
        \textbf{Direction:} Axis to rotate around\\
        \textbf{Magnitude:} $\sin(\theta_{err}) \approx \theta_{err}$
    };

    % Labels
    \node[red!70!black, font=\footnotesize, align=center] at (1.5,-1,2) {Measured\\(from accel)};
    \node[blue!70!black, font=\footnotesize, align=center] at (-1.5,-0.5,2) {Predicted\\(from $q$)};

    % Perfect alignment note
    \node[gray, font=\scriptsize, align=center] at (0,-2.5,0)
        {When $\hat{\mathbf{a}} = \hat{\mathbf{g}}_B$: $\mathbf{e} = 0$ (no correction needed)};
\end{tikzpicture}
\caption{Mahony filter error computation using cross product. The measured gravity $\hat{\mathbf{a}}$ (red) and predicted gravity $\hat{\mathbf{g}}_B$ (blue) are both unit vectors. Their cross product $\mathbf{e}$ (green) gives the axis of rotation needed to align them, with magnitude proportional to the angular error.}
\label{fig:mahony-cross-product}
\end{figure}

\textbf{Intuition}: The cross product measures how much the measured and predicted gravity directions differ. If they align, error is zero. If they differ, the error vector points along the axis of rotation needed to align them.

\subsection{PI Correction}

Apply proportional-integral control to correct the gyroscope:
\begin{align}
\mathbf{e}_{int} &\leftarrow \mathbf{e}_{int} + \mathbf{e} \cdot h \\
\boldsymbol{\omega}_{corrected} &= \boldsymbol{\omega}_{gyro} + K_p \mathbf{e} + K_i \mathbf{e}_{int}
\end{align}

\begin{itemize}
    \item $K_p$: Proportional gain. Higher = faster correction but more noise.
    \item $K_i$: Integral gain. Slowly corrects gyroscope bias.
\end{itemize}

\subsection{Quaternion Integration}

Update quaternion using the corrected angular velocity:
\begin{equation}
q_{k+1} = q_k + \frac{h}{2} q_k \otimes \begin{bmatrix} 0 \\ \boldsymbol{\omega}_{corrected} \end{bmatrix}
\end{equation}

Renormalize to maintain unit quaternion:
\begin{equation}
q_{k+1} \leftarrow \frac{q_{k+1}}{\|q_{k+1}\|}
\end{equation}

%----------------------------------------------------------------------
% FIGURE: Mahony Filter Block Diagram
%----------------------------------------------------------------------
\begin{figure}[htbp]
\centering
\begin{tikzpicture}[
    block/.style={draw, rectangle, minimum height=2em, minimum width=2.8em,
                  align=center, rounded corners=2pt, font=\small},
    smallblock/.style={draw, rectangle, minimum height=1.8em, minimum width=2.2em,
                  align=center, rounded corners=2pt, font=\scriptsize},
    gain/.style={draw, circle, minimum size=1.5em, inner sep=0pt, font=\scriptsize},
    sum/.style={draw, circle, minimum size=1.3em, inner sep=0pt, font=\scriptsize},
    >=Stealth,
    scale=0.95, transform shape
]
    % === INPUTS (left side) ===
    \node[block, fill=blue!15] (gyro) at (0,0) {$\boldsymbol{\omega}_{gyro}$};
    \node[block, fill=red!15] (accel) at (0,3) {$\mathbf{a}_{meas}$};

    % === ACCELEROMETER BRANCH (top) ===
    \node[smallblock, fill=red!25] (norm) at (2.5,3) {Normalize};
    \draw[->, red!70!black] (accel) -- (norm);

    % === ERROR COMPUTATION ===
    \node[smallblock, fill=green!25] (cross) at (5,2) {$\times$\\(cross)};
    \node[smallblock, fill=gray!20] (grav) at (2.5,1) {$\hat{\mathbf{g}}_B$\\from $q$};

    \draw[->, red!70!black] (norm) -| (cross.north) node[pos=0.7, above, font=\scriptsize] {$\hat{\mathbf{a}}$};
    \draw[->, gray] (grav) -| (cross.south);

    % === PI CONTROLLER ===
    \node[gain, fill=green!30] (kp) at (7,2.5) {$K_p$};
    \node[smallblock, fill=green!20] (int) at (7,1) {$\int$};
    \node[gain, fill=green!30] (ki) at (8.5,1) {$K_i$};
    \node[sum] (sumpi) at (9.5,2) {$+$};

    \draw[->, green!60!black] (cross) -- ++(1,0) |- (kp);
    \draw[->, green!60!black] (cross) -- ++(1,0) |- (int);
    \draw[->, green!60!black] (int) -- (ki);
    \draw[->, green!60!black] (kp) -| (sumpi);
    \draw[->, green!60!black] (ki) -| (sumpi);

    % === GYRO CORRECTION ===
    \node[sum] (sumgyro) at (11,0) {$+$};
    \draw[->, blue!70!black] (gyro) -- (sumgyro) node[midway, below, font=\scriptsize] {};
    \draw[->, green!60!black] (sumpi) |- (sumgyro);

    % === QUATERNION INTEGRATION ===
    \node[smallblock, fill=purple!20] (qdot) at (13,0) {$\dot{q} = \frac{1}{2}q \otimes$\\$[0;\boldsymbol{\omega}]$};
    \node[smallblock, fill=purple!25] (qint) at (15.5,0) {$\int$};
    \node[smallblock, fill=purple!30] (qnorm) at (17.5,0) {Norm};

    \draw[->] (sumgyro) -- (qdot) node[midway, above, font=\scriptsize] {$\boldsymbol{\omega}_{corr}$};
    \draw[->] (qdot) -- (qint);
    \draw[->] (qint) -- (qnorm);

    % === OUTPUT ===
    \node[block, fill=purple!15] (output) at (19.5,0) {$q$};
    \draw[->] (qnorm) -- (output);

    % === FEEDBACK LOOP ===
    \draw[->] (output.south) -- ++(0,-1) -| (grav.south);
    \draw[->] (output.south) -- ++(0,-1) -| (13,-1) -- (13,-0.5);

    % === LABELS ===
    \node[blue!70!black, font=\scriptsize] at (0,-0.8) {Gyroscope};
    \node[red!70!black, font=\scriptsize] at (0,3.7) {Accelerometer};
    \node[purple!70!black, font=\scriptsize] at (19.5,0.8) {Output};

    % Color legend
    \node[font=\tiny, align=left] at (5,-1.8) {
        \textcolor{red!70!black}{Red: Accel path} \quad
        \textcolor{blue!70!black}{Blue: Gyro path} \quad
        \textcolor{green!60!black}{Green: Correction} \quad
        \textcolor{purple!70!black}{Purple: Quaternion}
    };

    % Error label
    \node[green!60!black, font=\scriptsize] at (6,2) {$\mathbf{e}$};
\end{tikzpicture}
\caption{Complete Mahony filter block diagram. The accelerometer provides a gravity reference (red path), which is compared to the predicted gravity from the quaternion estimate. The cross product error drives a PI controller (green path) that corrects the gyroscope. The corrected angular velocity integrates the quaternion (purple path), which feeds back for the next iteration.}
\label{fig:mahony-block-diagram}
\end{figure}

\subsection{Complete Algorithm}

The complete Mahony filter implementation is provided in Listing~\ref{lst:mahony-filter} (Appendix~\ref{app:code-listings}). The algorithm performs the following steps each iteration:
\begin{enumerate}
    \item Normalize accelerometer reading to unit vector
    \item Compute predicted gravity direction from current quaternion
    \item Calculate cross-product error between measured and predicted gravity
    \item Apply PI correction to gyroscope readings
    \item Integrate quaternion using corrected angular velocity
    \item Renormalize quaternion to maintain unit magnitude
\end{enumerate}

\subsection{Tuning}

\begin{center}
\begin{tabular}{lll}
\toprule
\textbf{Parameter} & \textbf{Effect} & \textbf{Typical Range} \\
\midrule
$K_p$ & Speed of convergence, noise sensitivity & 0.5 -- 2.0 \\
$K_i$ & Gyro bias correction rate & 0.001 -- 0.01 \\
\bottomrule
\end{tabular}
\end{center}

\begin{notebox}[title=Assumptions for Mahony Filter]
The filter assumes:
\begin{enumerate}
    \item Accelerometer measures primarily gravity (body not accelerating violently)
    \item Gyroscope is the primary source of orientation rate information
    \item Sensor biases are slowly varying
\end{enumerate}
During aggressive maneuvers, consider temporarily increasing $K_p$ or reducing accelerometer trust.
\end{notebox}

\section{Comparison with Kalman Filter}

\begin{center}
\begin{tabular}{lcc}
\toprule
& \textbf{Complementary/Mahony} & \textbf{Extended Kalman Filter} \\
\midrule
Computational cost & Low ($O(1)$ per update) & High (matrix operations) \\
Memory & Minimal & Stores covariance matrix \\
Tuning parameters & 2 ($K_p$, $K_i$) & Many (process/measurement noise) \\
Theoretical basis & Heuristic (frequency separation) & Optimal (for linear Gaussian) \\
Performance & Very good & Excellent \\
\bottomrule
\end{tabular}
\end{center}

\begin{keyidea}[title=When to Use What]
\begin{itemize}
    \item \textbf{Complementary/Mahony filter}: Excellent choice for embedded systems with limited computation. Used in many commercial drones.
    \item \textbf{Extended Kalman Filter}: Preferred when fusing additional sensors (GPS, barometer, vision) or when optimal estimation is required.
\end{itemize}
For this course, the Mahony filter is sufficient and provides excellent intuition about sensor fusion. For readers who want a deeper understanding of optimal estimation, the next chapter develops the Kalman filter from first principles.
\end{keyidea}

%======================================================================
% CHAPTER: Optimal State Estimation
%======================================================================
\chapter{Optimal State Estimation}
\label{ch:kalman-filter}
\index{Kalman filter}

The Mahony filter provides excellent attitude estimates with minimal computation. But what makes it work? Is there a principled way to derive such filters? And how do we extend sensor fusion when we add GPS, barometers, or other sensors?

This chapter develops the \textbf{Kalman filter}~\cite{kalman1960new}---the optimal state estimator for linear systems with Gaussian noise. We then extend it to nonlinear systems via the \textbf{Extended Kalman Filter (EKF)}, which is widely used in aerospace, robotics, and autonomous vehicles. Along the way, we'll discover that the Mahony filter is closely related to the steady-state Kalman filter, providing theoretical justification for its excellent performance. For a comprehensive treatment of Kalman filtering, see Brown and Hwang~\cite{brown2012introduction}.

\section{Why Go Beyond Mahony?}

The Mahony filter works well for attitude estimation, but it has limitations:

\begin{itemize}
    \item \textbf{Fixed sensor fusion structure}: The filter assumes a specific combination of gyroscope and accelerometer. Adding new sensors (GPS, magnetometer, barometer) requires redesigning the filter.

    \item \textbf{No uncertainty quantification}: The Mahony filter provides a point estimate $\hat{q}$ but no measure of confidence. We don't know if our estimate is accurate to 1° or 10°.

    \item \textbf{Heuristic tuning}: The gains $K_p$ and $K_i$ are tuned empirically. There's no systematic way to choose them based on sensor specifications.

    \item \textbf{Suboptimal for varying conditions}: The fixed gains cannot adapt when sensor noise characteristics change (e.g., accelerometer becomes unreliable during aggressive maneuvers).
\end{itemize}

The Kalman filter addresses all these limitations by providing:
\begin{itemize}
    \item A \textbf{modular framework} where adding sensors only requires specifying their measurement equations
    \item \textbf{Uncertainty tracking} via the covariance matrix $P$
    \item \textbf{Optimal gains} computed automatically from sensor noise specifications
    \item \textbf{Adaptive weighting} that automatically trusts sensors less when they become unreliable
\end{itemize}

\section{Probability Foundations}
\label{sec:probability-foundations}

The Kalman filter is fundamentally a probabilistic algorithm. Before deriving it, we need to establish the necessary probability concepts.

\subsection{Random Variables and Expectation}

A \textbf{random variable} $X$ is a variable whose value is determined by a random process. We characterize it by its probability distribution.

\begin{definition}[Expected Value]
The \textbf{expected value} (or mean) of a random variable $X$ is:
\[
\mathbb{E}[X] = \mu_X = \int_{-\infty}^{\infty} x \, p(x) \, dx
\]
where $p(x)$ is the probability density function (PDF).
\end{definition}

\textbf{Intuition}: The expected value is the ``center of mass'' of the probability distribution---the average value we'd observe over many samples.

\begin{definition}[Variance]
The \textbf{variance} measures the spread of a random variable around its mean:
\[
\text{Var}(X) = \sigma_X^2 = \mathbb{E}[(X - \mu_X)^2] = \mathbb{E}[X^2] - \mu_X^2
\]
The \textbf{standard deviation} $\sigma_X = \sqrt{\text{Var}(X)}$ has the same units as $X$.
\end{definition}

\textbf{Intuition}: Variance quantifies uncertainty. A gyroscope with $\sigma = 0.01$ rad/s is more precise than one with $\sigma = 0.1$ rad/s.

\subsection{Gaussian Distribution}

The \textbf{Gaussian} (or normal) distribution is central to Kalman filtering because:
\begin{enumerate}
    \item Many physical noise sources are approximately Gaussian (Central Limit Theorem)
    \item Gaussian distributions remain Gaussian under linear transformations
    \item The math works out elegantly (closed-form solutions)
\end{enumerate}

\begin{definition}[Gaussian Distribution]
A random variable $X$ is \textbf{Gaussian} with mean $\mu$ and variance $\sigma^2$, written $X \sim \mathcal{N}(\mu, \sigma^2)$, if its PDF is:
\[
p(x) = \frac{1}{\sqrt{2\pi\sigma^2}} \exp\left(-\frac{(x-\mu)^2}{2\sigma^2}\right)
\]
\end{definition}

\begin{center}
\begin{tikzpicture}
    \begin{axis}[
        width=10cm, height=6cm,
        xlabel={$x$},
        ylabel={$p(x)$},
        domain=-4:4,
        samples=100,
        ymin=0, ymax=0.45,
        legend pos=north east,
        grid=major,
    ]
    \addplot[blue, thick] {exp(-x^2/2)/sqrt(2*pi)};
    \addlegendentry{$\mathcal{N}(0, 1)$}
    \addplot[red, thick, dashed] {exp(-(x-1)^2/0.5)/sqrt(pi)};
    \addlegendentry{$\mathcal{N}(1, 0.5)$}
    \addplot[green!60!black, thick, dotted] {exp(-x^2/8)/sqrt(4*pi)};
    \addlegendentry{$\mathcal{N}(0, 4)$}
    \end{axis}
\end{tikzpicture}
\end{center}

\subsection{Random Vectors and Covariance}

For state estimation, we work with \textbf{random vectors}---vectors whose elements are random variables.

\begin{definition}[Random Vector]
A random vector $\mathbf{x} \in \mathbb{R}^n$ has:
\begin{itemize}
    \item \textbf{Mean vector}: $\boldsymbol{\mu} = \mathbb{E}[\mathbf{x}] = [\mathbb{E}[x_1], \ldots, \mathbb{E}[x_n]]^T$
    \item \textbf{Covariance matrix}: $\mathbf{P} = \mathbb{E}[(\mathbf{x} - \boldsymbol{\mu})(\mathbf{x} - \boldsymbol{\mu})^T]$
\end{itemize}
\end{definition}

The covariance matrix $\mathbf{P}$ is an $n \times n$ symmetric positive semi-definite matrix:
\[
\mathbf{P} = \begin{bmatrix}
\text{Var}(x_1) & \text{Cov}(x_1, x_2) & \cdots & \text{Cov}(x_1, x_n) \\
\text{Cov}(x_2, x_1) & \text{Var}(x_2) & \cdots & \text{Cov}(x_2, x_n) \\
\vdots & \vdots & \ddots & \vdots \\
\text{Cov}(x_n, x_1) & \text{Cov}(x_n, x_2) & \cdots & \text{Var}(x_n)
\end{bmatrix}
\]

where $\text{Cov}(x_i, x_j) = \mathbb{E}[(x_i - \mu_i)(x_j - \mu_j)]$.

\textbf{Interpretation}:
\begin{itemize}
    \item \textbf{Diagonal elements} $P_{ii} = \text{Var}(x_i)$: uncertainty in each state
    \item \textbf{Off-diagonal elements} $P_{ij} = \text{Cov}(x_i, x_j)$: correlation between states
\end{itemize}

\begin{example}[Attitude Estimation Covariance]
For attitude represented as Euler angles $\mathbf{x} = [\phi, \theta, \psi]^T$, a covariance matrix might be:
\[
\mathbf{P} = \begin{bmatrix}
0.01 & 0.002 & 0 \\
0.002 & 0.01 & 0 \\
0 & 0 & 0.04
\end{bmatrix} \text{ rad}^2
\]
This indicates:
\begin{itemize}
    \item Roll and pitch uncertainty: $\sigma_\phi = \sigma_\theta = 0.1$ rad $\approx 5.7°$
    \item Yaw uncertainty: $\sigma_\psi = 0.2$ rad $\approx 11.5°$ (larger because magnetometer is noisy)
    \item Roll-pitch correlation: slightly positive (tilting forward often couples with roll)
    \item No yaw-roll/pitch correlation: yaw errors are independent
\end{itemize}
\end{example}

\subsection{Linear Transformations of Gaussians}

A key property that makes Kalman filtering tractable:

\begin{theorem}[Linear Transformation of Gaussians]
If $\mathbf{x} \sim \mathcal{N}(\boldsymbol{\mu}, \mathbf{P})$ and $\mathbf{y} = \mathbf{A}\mathbf{x} + \mathbf{b}$, then:
\[
\mathbf{y} \sim \mathcal{N}(\mathbf{A}\boldsymbol{\mu} + \mathbf{b}, \mathbf{A}\mathbf{P}\mathbf{A}^T)
\]
\end{theorem}

\textbf{Proof sketch}: The mean transforms linearly: $\mathbb{E}[\mathbf{y}] = \mathbf{A}\mathbb{E}[\mathbf{x}] + \mathbf{b}$. For covariance:
\begin{align*}
\text{Cov}(\mathbf{y}) &= \mathbb{E}[(\mathbf{y} - \mathbb{E}[\mathbf{y}])(\mathbf{y} - \mathbb{E}[\mathbf{y}])^T] \\
&= \mathbb{E}[\mathbf{A}(\mathbf{x} - \boldsymbol{\mu})(\mathbf{x} - \boldsymbol{\mu})^T\mathbf{A}^T] \\
&= \mathbf{A}\mathbb{E}[(\mathbf{x} - \boldsymbol{\mu})(\mathbf{x} - \boldsymbol{\mu})^T]\mathbf{A}^T = \mathbf{A}\mathbf{P}\mathbf{A}^T
\end{align*}

\begin{keyidea}[title=Why This Matters]
This theorem is the foundation of Kalman filtering:
\begin{itemize}
    \item \textbf{Prediction step}: State evolves via $\mathbf{x}_{k+1} = \mathbf{A}\mathbf{x}_k + \ldots$, so covariance transforms as $\mathbf{P}_{k+1} = \mathbf{A}\mathbf{P}_k\mathbf{A}^T + \ldots$
    \item \textbf{Measurement}: Observation is $\mathbf{y} = \mathbf{C}\mathbf{x} + \ldots$, so predicted measurement covariance is $\mathbf{C}\mathbf{P}\mathbf{C}^T + \ldots$
\end{itemize}
Gaussian distributions remain Gaussian through all linear operations, enabling closed-form optimal estimation.
\end{keyidea}

\subsection{Conditional Gaussians and Bayes' Theorem}

When we receive a measurement, we want to update our belief about the state. This is formalized by Bayes' theorem.

\begin{theorem}[Bayes' Theorem]
\[
p(\mathbf{x} | \mathbf{y}) = \frac{p(\mathbf{y} | \mathbf{x}) \, p(\mathbf{x})}{p(\mathbf{y})}
\]
where:
\begin{itemize}
    \item $p(\mathbf{x})$: \textbf{prior}---our belief about $\mathbf{x}$ before seeing $\mathbf{y}$
    \item $p(\mathbf{y} | \mathbf{x})$: \textbf{likelihood}---probability of observing $\mathbf{y}$ given $\mathbf{x}$
    \item $p(\mathbf{x} | \mathbf{y})$: \textbf{posterior}---our updated belief after seeing $\mathbf{y}$
\end{itemize}
\end{theorem}

For jointly Gaussian variables, the posterior has a beautiful closed form:

\begin{theorem}[Gaussian Conditioning]
\label{thm:gaussian-conditioning}
If $\mathbf{x}$ and $\mathbf{y}$ are jointly Gaussian:
\[
\begin{bmatrix} \mathbf{x} \\ \mathbf{y} \end{bmatrix} \sim \mathcal{N}\left(
\begin{bmatrix} \boldsymbol{\mu}_x \\ \boldsymbol{\mu}_y \end{bmatrix},
\begin{bmatrix} \mathbf{P}_{xx} & \mathbf{P}_{xy} \\ \mathbf{P}_{yx} & \mathbf{P}_{yy} \end{bmatrix}
\right)
\]
then the conditional distribution $\mathbf{x} | \mathbf{y}$ is also Gaussian:
\[
\mathbf{x} | \mathbf{y} \sim \mathcal{N}\left(\boldsymbol{\mu}_x + \mathbf{P}_{xy}\mathbf{P}_{yy}^{-1}(\mathbf{y} - \boldsymbol{\mu}_y), \; \mathbf{P}_{xx} - \mathbf{P}_{xy}\mathbf{P}_{yy}^{-1}\mathbf{P}_{yx}\right)
\]
\end{theorem}

\textbf{Interpretation}:
\begin{itemize}
    \item The posterior mean is the prior mean plus a correction term
    \item The correction is proportional to the ``surprise'' $(\mathbf{y} - \boldsymbol{\mu}_y)$---how much the measurement differs from expectation
    \item The posterior covariance is always smaller than the prior (measurements reduce uncertainty)
    \item The matrix $\mathbf{K} = \mathbf{P}_{xy}\mathbf{P}_{yy}^{-1}$ is the \textbf{Kalman gain}
\end{itemize}

\section{The Linear Kalman Filter}
\label{sec:linear-kalman}

We now derive the Kalman filter for linear systems. This provides the foundation for understanding the Extended Kalman Filter used in attitude estimation.

\subsection{Problem Setup}

Consider a discrete-time linear system:

\begin{align}
\mathbf{x}_{k+1} &= \mathbf{A}\mathbf{x}_k + \mathbf{B}\mathbf{u}_k + \mathbf{w}_k & \text{(process model)} \label{eq:kf-process} \\
\mathbf{y}_k &= \mathbf{C}\mathbf{x}_k + \mathbf{v}_k & \text{(measurement model)} \label{eq:kf-measurement}
\end{align}

where:
\begin{itemize}
    \item $\mathbf{x}_k \in \mathbb{R}^n$: state vector (unknown)
    \item $\mathbf{u}_k \in \mathbb{R}^m$: control input (known)
    \item $\mathbf{y}_k \in \mathbb{R}^p$: measurement (observed)
    \item $\mathbf{w}_k \sim \mathcal{N}(\mathbf{0}, \mathbf{Q})$: process noise
    \item $\mathbf{v}_k \sim \mathcal{N}(\mathbf{0}, \mathbf{R})$: measurement noise
\end{itemize}

The matrices $\mathbf{Q}$ and $\mathbf{R}$ are the \textbf{process noise covariance} and \textbf{measurement noise covariance}, respectively.

\begin{example}[1D Attitude Estimation]
\label{ex:1d-kalman}
Consider estimating pitch angle $\theta$ from a gyroscope (measures $\dot{\theta}$) and accelerometer (measures $\theta$ directly, with noise):

\textbf{State}: $x = \theta$ (pitch angle)

\textbf{Process model}: $\theta_{k+1} = \theta_k + h \cdot \omega_k + w_k$

where $\omega_k$ is gyroscope measurement (treated as input $u_k$) and $w_k \sim \mathcal{N}(0, Q)$ captures gyroscope noise/bias.

\textbf{Measurement model}: $y_k = \theta_k + v_k$

where $y_k = \arctan(a_x / a_z)$ from accelerometer and $v_k \sim \mathcal{N}(0, R)$ is accelerometer noise.

In matrix form: $A = 1$, $B = h$, $C = 1$.
\end{example}

\subsection{The Kalman Filter Algorithm}

The Kalman filter alternates between two steps:

\begin{enumerate}
    \item \textbf{Predict}: Use the process model to predict the state at the next time step
    \item \textbf{Update}: Incorporate the measurement to correct the prediction
\end{enumerate}

\begin{tcolorbox}[colback=blue!5!white, colframe=blue!75!black, title=Kalman Filter Algorithm]
\textbf{Initialize}: $\hat{\mathbf{x}}_{0|0} = \mathbb{E}[\mathbf{x}_0]$, $\mathbf{P}_{0|0} = \text{Cov}(\mathbf{x}_0)$

\textbf{For each time step $k$}:

\textit{Predict (time update)}:
\begin{align}
\hat{\mathbf{x}}_{k+1|k} &= \mathbf{A}\hat{\mathbf{x}}_{k|k} + \mathbf{B}\mathbf{u}_k \label{eq:kf-predict-state} \\
\mathbf{P}_{k+1|k} &= \mathbf{A}\mathbf{P}_{k|k}\mathbf{A}^T + \mathbf{Q} \label{eq:kf-predict-cov}
\end{align}

\textit{Update (measurement update)}:
\begin{align}
\mathbf{K}_{k+1} &= \mathbf{P}_{k+1|k}\mathbf{C}^T(\mathbf{C}\mathbf{P}_{k+1|k}\mathbf{C}^T + \mathbf{R})^{-1} \label{eq:kf-gain} \\
\hat{\mathbf{x}}_{k+1|k+1} &= \hat{\mathbf{x}}_{k+1|k} + \mathbf{K}_{k+1}(\mathbf{y}_{k+1} - \mathbf{C}\hat{\mathbf{x}}_{k+1|k}) \label{eq:kf-update-state} \\
\mathbf{P}_{k+1|k+1} &= (\mathbf{I} - \mathbf{K}_{k+1}\mathbf{C})\mathbf{P}_{k+1|k} \label{eq:kf-update-cov}
\end{align}
\end{tcolorbox}

\textbf{Notation}: $\hat{\mathbf{x}}_{k|j}$ denotes the estimate of $\mathbf{x}_k$ given measurements up to time $j$:
\begin{itemize}
    \item $\hat{\mathbf{x}}_{k|k-1}$: predicted estimate (before measurement at time $k$)
    \item $\hat{\mathbf{x}}_{k|k}$: filtered estimate (after measurement at time $k$)
\end{itemize}

\subsection{Understanding the Kalman Gain}

The Kalman gain $\mathbf{K}$ determines how much to trust the measurement versus the prediction:

\[
\mathbf{K} = \mathbf{P}_{k|k-1}\mathbf{C}^T(\mathbf{C}\mathbf{P}_{k|k-1}\mathbf{C}^T + \mathbf{R})^{-1}
\]

\textbf{Extreme cases}:

\begin{itemize}
    \item \textbf{Perfect measurement} ($\mathbf{R} \to \mathbf{0}$): $\mathbf{K} \to \mathbf{C}^{-1}$ (if $\mathbf{C}$ is invertible)

    The estimate jumps directly to the measurement: $\hat{\mathbf{x}}_{k|k} \to \mathbf{C}^{-1}\mathbf{y}_k$

    \item \textbf{Useless measurement} ($\mathbf{R} \to \infty$): $\mathbf{K} \to \mathbf{0}$

    The measurement is ignored: $\hat{\mathbf{x}}_{k|k} \to \hat{\mathbf{x}}_{k|k-1}$

    \item \textbf{Very uncertain prediction} ($\mathbf{P}_{k|k-1} \to \infty$): $\mathbf{K} \to \mathbf{C}^{-1}$

    Trust the measurement completely

    \item \textbf{Very certain prediction} ($\mathbf{P}_{k|k-1} \to \mathbf{0}$): $\mathbf{K} \to \mathbf{0}$

    Ignore the measurement, trust the prediction
\end{itemize}

\begin{keyidea}[title=Kalman Gain Intuition]
The Kalman gain automatically balances trust between prediction and measurement:
\[
\text{Kalman gain} \propto \frac{\text{prediction uncertainty}}{\text{prediction uncertainty} + \text{measurement uncertainty}}
\]
When we're uncertain about our prediction, we trust the measurement more. When the measurement is noisy, we trust our prediction more.
\end{keyidea}

\subsection{Scalar Example: 1D Attitude Filter}

Continuing Example~\ref{ex:1d-kalman}, let's trace through the Kalman filter for pitch estimation.

\textbf{Parameters}:
\begin{itemize}
    \item Sample time: $h = 0.01$ s (100 Hz)
    \item Process noise: $Q = (0.01)^2$ rad$^2$ (gyro drift/noise)
    \item Measurement noise: $R = (0.1)^2$ rad$^2$ (accelerometer noise)
    \item Initial estimate: $\hat{\theta}_0 = 0$, $P_0 = (0.5)^2$ rad$^2$
\end{itemize}

\textbf{Scalar Kalman filter equations}:
\begin{align*}
\text{Predict}: \quad \hat{\theta}_{k|k-1} &= \hat{\theta}_{k-1|k-1} + h \cdot \omega_k \\
P_{k|k-1} &= P_{k-1|k-1} + Q \\[0.5em]
\text{Update}: \quad K_k &= \frac{P_{k|k-1}}{P_{k|k-1} + R} \\
\hat{\theta}_{k|k} &= \hat{\theta}_{k|k-1} + K_k(\theta_{accel,k} - \hat{\theta}_{k|k-1}) \\
P_{k|k} &= (1 - K_k) P_{k|k-1}
\end{align*}

\textbf{Steady-state analysis}: After many iterations, $P$ converges to a constant value. Setting $P_{k|k} = P_{k-1|k-1} = P_{ss}$:
\begin{align*}
P_{ss} + Q &= P_{predict} \\
P_{ss} &= (1 - K_{ss}) P_{predict} = (1 - K_{ss})(P_{ss} + Q)
\end{align*}

Solving: $K_{ss} = \frac{P_{ss} + Q}{P_{ss} + Q + R}$, which gives:
\[
K_{ss} = \frac{-R + \sqrt{R^2 + 4QR + 4Q^2}}{2(Q + R)} \approx \frac{\sqrt{Q}}{\sqrt{Q} + \sqrt{R}} \text{ for small } Q
\]

With our parameters: $K_{ss} \approx 0.09$.

\begin{keyidea}[title=Connection to Complementary Filter]
The steady-state Kalman filter for 1D attitude is equivalent to the complementary filter!

Kalman update: $\hat{\theta}_k = \hat{\theta}_{k|k-1} + K_{ss}(\theta_{accel} - \hat{\theta}_{k|k-1})$

Rewriting: $\hat{\theta}_k = (1 - K_{ss})\hat{\theta}_{k|k-1} + K_{ss} \cdot \theta_{accel}$

This matches the complementary filter $\hat{\theta}_k = \alpha \cdot \hat{\theta}_{gyro} + (1-\alpha) \cdot \theta_{accel}$ with $\alpha = 1 - K_{ss}$.

The complementary filter parameter $\alpha$ corresponds to the steady-state Kalman gain, which the Kalman filter derives automatically from the noise covariances $Q$ and $R$.
\end{keyidea}

\subsection{Optimality of the Kalman Filter}

\begin{theorem}[Kalman Filter Optimality]
For linear systems with Gaussian noise, the Kalman filter provides the \textbf{minimum variance unbiased estimate}:
\[
\hat{\mathbf{x}}_{k|k} = \arg\min_{\hat{\mathbf{x}}} \mathbb{E}\left[\|\mathbf{x}_k - \hat{\mathbf{x}}\|^2 \,|\, \mathbf{y}_{1:k}\right]
\]
where $\mathbf{y}_{1:k} = \{\mathbf{y}_1, \ldots, \mathbf{y}_k\}$ is all measurements up to time $k$.
\end{theorem}

\textbf{Proof sketch}:
\begin{enumerate}
    \item By Theorem~\ref{thm:gaussian-conditioning}, the posterior $p(\mathbf{x}_k | \mathbf{y}_{1:k})$ is Gaussian
    \item For a Gaussian, the mean equals the mode equals the minimum-variance estimate
    \item The Kalman filter recursively computes this posterior mean
\end{enumerate}

\begin{notebox}[title=When is the Kalman Filter Optimal?]
The Kalman filter is optimal \textbf{only} when:
\begin{itemize}
    \item The system is \textbf{linear} (equations \eqref{eq:kf-process}--\eqref{eq:kf-measurement})
    \item Noise is \textbf{Gaussian} with known covariances $\mathbf{Q}$, $\mathbf{R}$
    \item Initial state is \textbf{Gaussian} with known mean and covariance
\end{itemize}
For nonlinear systems (like attitude estimation with quaternions), the Extended Kalman Filter provides an approximation that is no longer guaranteed optimal.
\end{notebox}

\section{The Extended Kalman Filter}
\label{sec:ekf}

Attitude estimation involves nonlinear equations:
\begin{itemize}
    \item Quaternion kinematics: $\dot{\mathbf{q}} = \frac{1}{2}\mathbf{q} \otimes \boldsymbol{\omega}$
    \item Accelerometer measurement: $\mathbf{a} = \mathbf{R}(\mathbf{q})^T \mathbf{g}$
\end{itemize}

The \textbf{Extended Kalman Filter (EKF)} handles nonlinear systems by linearizing around the current estimate at each time step.

\subsection{Nonlinear System Model}

Consider the general nonlinear system:
\begin{align}
\mathbf{x}_{k+1} &= f(\mathbf{x}_k, \mathbf{u}_k) + \mathbf{w}_k \label{eq:ekf-process} \\
\mathbf{y}_k &= h(\mathbf{x}_k) + \mathbf{v}_k \label{eq:ekf-measurement}
\end{align}

where $f(\cdot)$ is the nonlinear process model and $h(\cdot)$ is the nonlinear measurement function.

\subsection{Linearization via Jacobians}

The EKF linearizes these functions around the current estimate using first-order Taylor expansion:
\begin{align*}
f(\mathbf{x}_k, \mathbf{u}_k) &\approx f(\hat{\mathbf{x}}_{k|k}, \mathbf{u}_k) + \mathbf{F}_k(\mathbf{x}_k - \hat{\mathbf{x}}_{k|k}) \\
h(\mathbf{x}_k) &\approx h(\hat{\mathbf{x}}_{k|k-1}) + \mathbf{H}_k(\mathbf{x}_k - \hat{\mathbf{x}}_{k|k-1})
\end{align*}

where the \textbf{Jacobian matrices} are:
\[
\mathbf{F}_k = \frac{\partial f}{\partial \mathbf{x}}\bigg|_{\hat{\mathbf{x}}_{k|k}}, \qquad
\mathbf{H}_k = \frac{\partial h}{\partial \mathbf{x}}\bigg|_{\hat{\mathbf{x}}_{k|k-1}}
\]

\subsection{EKF Algorithm}

\begin{tcolorbox}[colback=blue!5!white, colframe=blue!75!black, title=Extended Kalman Filter Algorithm]
\textbf{Initialize}: $\hat{\mathbf{x}}_{0|0}$, $\mathbf{P}_{0|0}$

\textbf{For each time step $k$}:

\textit{Predict}:
\begin{align}
\hat{\mathbf{x}}_{k+1|k} &= f(\hat{\mathbf{x}}_{k|k}, \mathbf{u}_k) \label{eq:ekf-predict-state} \\
\mathbf{F}_k &= \frac{\partial f}{\partial \mathbf{x}}\bigg|_{\hat{\mathbf{x}}_{k|k}} \\
\mathbf{P}_{k+1|k} &= \mathbf{F}_k\mathbf{P}_{k|k}\mathbf{F}_k^T + \mathbf{Q} \label{eq:ekf-predict-cov}
\end{align}

\textit{Update}:
\begin{align}
\mathbf{H}_{k+1} &= \frac{\partial h}{\partial \mathbf{x}}\bigg|_{\hat{\mathbf{x}}_{k+1|k}} \\
\mathbf{K}_{k+1} &= \mathbf{P}_{k+1|k}\mathbf{H}_{k+1}^T(\mathbf{H}_{k+1}\mathbf{P}_{k+1|k}\mathbf{H}_{k+1}^T + \mathbf{R})^{-1} \label{eq:ekf-gain} \\
\hat{\mathbf{x}}_{k+1|k+1} &= \hat{\mathbf{x}}_{k+1|k} + \mathbf{K}_{k+1}(\mathbf{y}_{k+1} - h(\hat{\mathbf{x}}_{k+1|k})) \label{eq:ekf-update-state} \\
\mathbf{P}_{k+1|k+1} &= (\mathbf{I} - \mathbf{K}_{k+1}\mathbf{H}_{k+1})\mathbf{P}_{k+1|k} \label{eq:ekf-update-cov}
\end{align}
\end{tcolorbox}

The key differences from the linear Kalman filter:
\begin{enumerate}
    \item State prediction uses the \textbf{nonlinear} function $f(\cdot)$, not $\mathbf{A}\hat{\mathbf{x}}$
    \item Measurement prediction uses $h(\hat{\mathbf{x}})$, not $\mathbf{C}\hat{\mathbf{x}}$
    \item Jacobians $\mathbf{F}_k$ and $\mathbf{H}_k$ must be recomputed at each time step
    \item Covariance propagation uses the Jacobians (linearized model)
\end{enumerate}

\subsection{EKF Limitations}

\begin{warningbox}[title=EKF is Not Optimal]
Unlike the linear Kalman filter, the EKF is \textbf{not} optimal:
\begin{itemize}
    \item Linearization introduces errors, especially for highly nonlinear systems
    \item The posterior is approximated as Gaussian, which may not be accurate
    \item The computed covariance $\mathbf{P}$ may not reflect true uncertainty
\end{itemize}

The EKF can \textbf{diverge} (estimates become increasingly wrong) if:
\begin{itemize}
    \item Initial estimate is far from truth (linearization invalid)
    \item System is highly nonlinear
    \item Noise covariances $\mathbf{Q}$, $\mathbf{R}$ are poorly specified
\end{itemize}
\end{warningbox}

\section{Attitude EKF Implementation}
\label{sec:attitude-ekf}

We now develop a complete EKF for attitude estimation using quaternions, suitable for implementation on the Crazyflie.

\subsection{State Representation}

We use the quaternion $\mathbf{q} = [q_0, q_1, q_2, q_3]^T$ as the state, where $q_0$ is the scalar part. The quaternion must satisfy the unit norm constraint $\|\mathbf{q}\| = 1$.

\textbf{State vector}: $\mathbf{x} = \mathbf{q} \in \mathbb{R}^4$

\textbf{Note}: Some implementations use a 3-element ``error state'' representation to avoid the constraint. We use the full quaternion for clarity, with explicit renormalization.

\subsection{Process Model}

The quaternion evolves according to the kinematic equation:
\[
\mathbf{q}_{k+1} = \mathbf{q}_k + \frac{h}{2}\mathbf{q}_k \otimes \begin{bmatrix} 0 \\ \boldsymbol{\omega}_k \end{bmatrix}
\]

where $\boldsymbol{\omega}_k = [p_k, q_k, r_k]^T$ is the angular velocity from the gyroscope.

This can be written as a matrix operation:
\[
\mathbf{q}_{k+1} = \left(\mathbf{I}_4 + \frac{h}{2}\boldsymbol{\Omega}(\boldsymbol{\omega}_k)\right)\mathbf{q}_k
\]

where $\boldsymbol{\Omega}(\boldsymbol{\omega})$ is the $4 \times 4$ matrix:
\[
\boldsymbol{\Omega}(\boldsymbol{\omega}) = \begin{bmatrix}
0 & -p & -q & -r \\
p & 0 & r & -q \\
q & -r & 0 & p \\
r & q & -p & 0
\end{bmatrix}
\]

\textbf{Process model}: $f(\mathbf{q}, \boldsymbol{\omega}) = \left(\mathbf{I}_4 + \frac{h}{2}\boldsymbol{\Omega}(\boldsymbol{\omega})\right)\mathbf{q}$

\textbf{Jacobian}: Since the model is linear in $\mathbf{q}$:
\[
\mathbf{F} = \frac{\partial f}{\partial \mathbf{q}} = \mathbf{I}_4 + \frac{h}{2}\boldsymbol{\Omega}(\boldsymbol{\omega})
\]

\subsection{Measurement Model}

The accelerometer measures the gravity vector in the body frame:
\[
\mathbf{a}_{meas} = \mathbf{R}(\mathbf{q})^T \mathbf{g} + \mathbf{v}
\]

where $\mathbf{g} = [0, 0, g]^T$ (in NED frame) and $\mathbf{R}(\mathbf{q})$ is the rotation matrix corresponding to quaternion $\mathbf{q}$.

The predicted gravity direction in body frame is:
\[
h(\mathbf{q}) = \mathbf{R}(\mathbf{q})^T \begin{bmatrix} 0 \\ 0 \\ g \end{bmatrix} = g \begin{bmatrix}
2(q_1 q_3 - q_0 q_2) \\
2(q_0 q_1 + q_2 q_3) \\
q_0^2 - q_1^2 - q_2^2 + q_3^2
\end{bmatrix}
\]

\textbf{Measurement Jacobian}:
\[
\mathbf{H} = \frac{\partial h}{\partial \mathbf{q}} = 2g \begin{bmatrix}
-q_2 & q_3 & -q_0 & q_1 \\
q_1 & q_0 & q_3 & q_2 \\
q_0 & -q_1 & -q_2 & q_3
\end{bmatrix}
\]

\subsection{Noise Covariance Matrices}

\textbf{Process noise} $\mathbf{Q}$: Represents gyroscope noise and bias drift.
\[
\mathbf{Q} = \sigma_\omega^2 h^2 \mathbf{I}_4
\]
where $\sigma_\omega$ is the gyroscope noise density (rad/s/$\sqrt{\text{Hz}}$).

Typical value for MEMS gyroscope: $\sigma_\omega \approx 0.01$ rad/s, giving $Q_{ii} \approx 10^{-8}$ for $h = 0.01$ s.

\textbf{Measurement noise} $\mathbf{R}$: Represents accelerometer noise.
\[
\mathbf{R} = \sigma_a^2 \mathbf{I}_3
\]
where $\sigma_a$ is the accelerometer noise in m/s² (or equivalently, the angular uncertainty in rad when measuring gravity direction).

Typical value: $\sigma_a \approx 0.5$ m/s², giving $R_{ii} \approx 0.25$ (m/s²)².

\subsection{Complete Algorithm}

The complete attitude EKF implementation is provided in Listing~\ref{lst:attitude-ekf} (Appendix~\ref{app:code-listings}). The code includes:
\begin{itemize}
    \item Data structure with quaternion state and covariance matrices
    \item Initialization with identity quaternion and diagonal covariances
    \item Prediction step using the $\boldsymbol{\Omega}$ matrix from gyroscope readings
    \item Update step using accelerometer measurement of gravity direction
    \item Proper normalization after both prediction and update
\end{itemize}

\subsection{Tuning Guidelines}

\begin{center}
\begin{tabular}{lll}
\toprule
\textbf{Parameter} & \textbf{Effect of Increasing} & \textbf{Typical Range} \\
\midrule
$\mathbf{Q}$ (process noise) & Trust measurements more & $10^{-8}$ to $10^{-4}$ \\
$\mathbf{R}$ (measurement noise) & Trust prediction more & $10^{-2}$ to $10^{1}$ \\
\bottomrule
\end{tabular}
\end{center}

\textbf{Tuning procedure}:
\begin{enumerate}
    \item Start with $\mathbf{Q}$ and $\mathbf{R}$ based on sensor datasheets
    \item If estimates are noisy, increase $\mathbf{R}$ (trust accelerometer less)
    \item If estimates drift, decrease $\mathbf{R}$ or increase $\mathbf{Q}$
    \item If estimates are slow to respond, decrease $\mathbf{R}$
\end{enumerate}

\section{Comparison: Mahony vs. EKF}

\begin{center}
\begin{tabular}{lll}
\toprule
\textbf{Aspect} & \textbf{Mahony Filter} & \textbf{Attitude EKF} \\
\midrule
\textbf{Derivation} & Heuristic (frequency separation) & Principled (Bayesian estimation) \\
\textbf{Tuning} & 2 parameters ($K_p$, $K_i$) & 2 matrices ($\mathbf{Q}$, $\mathbf{R}$) \\
\textbf{Computation} & $\sim$50 operations/update & $\sim$500 operations/update \\
\textbf{Memory} & $\sim$20 floats & $\sim$50 floats \\
\textbf{Uncertainty} & Not available & Covariance matrix $\mathbf{P}$ \\
\textbf{Sensor fusion} & Fixed (gyro + accel) & Easily extensible \\
\textbf{Bias estimation} & Integral term & Can add bias states \\
\textbf{Performance} & Excellent & Excellent \\
\bottomrule
\end{tabular}
\end{center}

\begin{keyidea}[title=When to Choose Each Filter]
\textbf{Use Mahony filter when}:
\begin{itemize}
    \item Computational resources are limited
    \item Only gyroscope and accelerometer are available
    \item Simple tuning is preferred
    \item Uncertainty quantification is not needed
\end{itemize}

\textbf{Use EKF when}:
\begin{itemize}
    \item Fusing additional sensors (GPS, magnetometer, barometer, vision)
    \item Uncertainty bounds are required (for planning, fault detection)
    \item Sensor noise characteristics are well-known
    \item Computational resources are adequate
\end{itemize}

For attitude-only estimation with IMU on a small quadrotor like Crazyflie, the Mahony filter is often the better choice due to its simplicity and efficiency.
\end{keyidea}

\section{Extensions and Advanced Topics}

\subsection{Multiplicative EKF (MEKF)}

The standard EKF with quaternion states has a subtle issue: the four quaternion elements are not independent (they must satisfy $\|\mathbf{q}\| = 1$). This leads to a rank-deficient covariance matrix~\cite{markley2003attitude}.

The \textbf{Multiplicative EKF} addresses this by:
\begin{enumerate}
    \item Maintaining a ``reference'' quaternion $\bar{\mathbf{q}}$ (the current estimate)
    \item Estimating a 3-element ``error rotation'' $\boldsymbol{\delta\theta}$
    \item True quaternion: $\mathbf{q} = \delta\mathbf{q}(\boldsymbol{\delta\theta}) \otimes \bar{\mathbf{q}}$
\end{enumerate}

The error state is always small, making linearization more accurate. After each update, the error is ``folded'' into the reference quaternion, and the error state is reset to zero.

\subsection{Unscented Kalman Filter (UKF)}

Instead of linearizing (which can be inaccurate for highly nonlinear systems), the \textbf{UKF}~\cite{julier1997new} propagates carefully chosen ``sigma points'' through the nonlinear functions and reconstructs the Gaussian approximation from these transformed points.

Advantages over EKF:
\begin{itemize}
    \item No Jacobian computation required
    \item More accurate for highly nonlinear systems
    \item Same computational complexity as EKF
\end{itemize}

\subsection{Multi-Sensor Fusion}

The EKF framework easily accommodates additional sensors by adding measurement update steps:

See Listing~\ref{lst:multisensor-ekf} (Appendix~\ref{app:code-listings}) for a multi-sensor EKF structure that demonstrates sequential sensor updates: prediction using gyroscope, then conditional updates from accelerometer (roll/pitch), magnetometer (yaw), GPS (position/velocity), and barometer (altitude).

Each sensor contributes information according to its noise characteristics ($\mathbf{R}$ matrix), and the Kalman gain automatically weights them appropriately.

\chapter{Quadrotor Dynamics}
\index{quadrotor!dynamics}

%----------------------------------------------------------------------
% FIGURE: Chapter Overview - Quadrotor Force/Torque Diagram
%----------------------------------------------------------------------
\begin{figure}[htbp]
\centering
\begin{tikzpicture}[scale=0.9,
    x={(0.75cm,-0.35cm)}, y={(0.75cm,0.35cm)}, z={(0cm,0.85cm)},
    >=Stealth
]
    % Quadrotor body (tilted slightly)
    \coordinate (center) at (0,0,2);

    % Arms
    \draw[thick, gray!70] ($(center)+(-2.5,0,0)$) -- ($(center)+(2.5,0,0)$);
    \draw[thick, gray!70] ($(center)+(0,-2.5,0)$) -- ($(center)+(0,2.5,0)$);

    % Central body
    \fill[gray!30] ($(center)+(-0.5,-0.5,0)$) -- ($(center)+(0.5,-0.5,0)$)
        -- ($(center)+(0.5,0.5,0)$) -- ($(center)+(-0.5,0.5,0)$) -- cycle;

    % Motors with rotation arrows
    % M1 (front) - CW
    \node[motor] (m1) at ($(center)+(2.5,0,0)$) {};
    \draw[->, thick, blue!60] ($(center)+(2.5,0,0.4)$) arc (90:330:0.3);
    \node[font=\scriptsize] at ($(center)+(2.5,0,0.8)$) {M1};

    % M2 (right) - CCW
    \node[motor] (m2) at ($(center)+(0,2.5,0)$) {};
    \draw[->, thick, red!60] ($(center)+(0,2.5,0.4)$) arc (90:-150:0.3);
    \node[font=\scriptsize] at ($(center)+(0,2.5,0.8)$) {M2};

    % M3 (back) - CW
    \node[motor] (m3) at ($(center)+(-2.5,0,0)$) {};
    \draw[->, thick, blue!60] ($(center)+(-2.5,0,0.4)$) arc (90:330:0.3);
    \node[font=\scriptsize] at ($(center)+(-2.5,0,0.8)$) {M3};

    % M4 (left) - CCW
    \node[motor] (m4) at ($(center)+(0,-2.5,0)$) {};
    \draw[->, thick, red!60] ($(center)+(0,-2.5,0.4)$) arc (90:-150:0.3);
    \node[font=\scriptsize] at ($(center)+(0,-2.5,0.8)$) {M4};

    % Thrust vectors (different heights to show unequal thrust)
    \draw[->, very thick, green!60!black] (m1.center) -- ++(0,0,1.8) node[right, font=\scriptsize] {$T_1$};
    \draw[->, very thick, green!60!black] (m2.center) -- ++(0,0,1.3) node[right, font=\scriptsize] {$T_2$};
    \draw[->, very thick, green!60!black] (m3.center) -- ++(0,0,1.2) node[left, font=\scriptsize] {$T_3$};
    \draw[->, very thick, green!60!black] (m4.center) -- ++(0,0,1.5) node[left, font=\scriptsize] {$T_4$};

    % Weight vector
    \draw[->, very thick, purple] (center) -- ++(0,0,-2.5) node[left, font=\scriptsize] {$mg$};

    % Body frame axes
    \draw[xaxisstyle] (center) -- ++(1.5,0,0) node[below, font=\scriptsize] {$x_B$};
    \draw[yaxisstyle] (center) -- ++(0,1.5,0) node[below, font=\scriptsize] {$y_B$};
    \draw[zaxisstyle] (center) -- ++(0,0,-1.2) node[left, font=\scriptsize] {$z_B$};

    % Torque indicators
    % Roll (around x)
    \draw[->, thick, orange, dashed] ($(center)+(1,0.8,0.3)$) arc (45:135:0.6);
    \node[orange, font=\scriptsize] at ($(center)+(1.5,0,0.8)$) {$\tau_\phi$};

    % Pitch (around y)
    \draw[->, thick, cyan, dashed] ($(center)+(0.8,1,0.3)$) arc (45:135:0.6);
    \node[cyan, font=\scriptsize] at ($(center)+(0,1.5,0.8)$) {$\tau_\theta$};

    % Legend
    \node[font=\scriptsize, align=left] at (5,0,0) {
        \textcolor{blue!60}{CW}: M1, M3\\
        \textcolor{red!60}{CCW}: M2, M4
    };
\end{tikzpicture}
\caption{Forces and torques acting on a quadrotor. Each motor produces thrust $T_i$ (green arrows). Differential thrusts create roll ($\tau_\phi$) and pitch ($\tau_\theta$) torques. Motor spin directions (CW/CCW) enable yaw control through reaction torques. Gravity ($mg$) acts downward.}
\label{fig:quadrotor-force-torque}
\end{figure}

\section{Introduction: Why Study Dynamics?}

We've covered:
\begin{itemize}
    \item How to represent orientation (frames, rotations, quaternions)
    \item How to measure/estimate orientation (IMU, complementary filter)
\end{itemize}

Now we ask: \textbf{how does the quadrotor move?} Given motor commands, what happens to position and orientation?

\textbf{Why this matters}:
\begin{itemize}
    \item \textbf{Simulation}: Before flying, test controllers in simulation
    \item \textbf{Control design}: Controllers are designed based on the dynamics model
    \item \textbf{State estimation}: Some estimation algorithms predict future states using dynamics
\end{itemize}

\section{Physical Description}

\subsection{Configurations}

%----------------------------------------------------------------------
% FIGURE: Quadrotor + vs X Configuration
%----------------------------------------------------------------------
\begin{figure}[htbp]
\centering
\begin{tikzpicture}[>=Stealth, scale=0.85]
    % === Plus Configuration (left) ===
    \begin{scope}[shift={(-5,0)}]
        % Arms
        \draw[thick, gray!70] (0,-2) -- (0,2);
        \draw[thick, gray!70] (-2,0) -- (2,0);

        % Central body
        \fill[gray!30] (-0.3,-0.3) rectangle (0.3,0.3);

        % Motors
        \node[motor] (m1p) at (0,2) {};
        \node[motor] (m2p) at (2,0) {};
        \node[motor] (m3p) at (0,-2) {};
        \node[motor] (m4p) at (-2,0) {};

        % Motor labels and rotation
        \node[font=\scriptsize] at (0,2.5) {M1};
        \node[font=\scriptsize] at (2.5,0) {M2};
        \node[font=\scriptsize] at (0,-2.5) {M3};
        \node[font=\scriptsize] at (-2.5,0) {M4};

        % Rotation arrows
        \draw[->, blue!60, thick] (0.3,2) arc (0:270:0.3);
        \draw[->, red!60, thick] (2,0.3) arc (90:-180:0.3);
        \draw[->, blue!60, thick] (0.3,-2) arc (0:270:0.3);
        \draw[->, red!60, thick] (-2,0.3) arc (90:-180:0.3);

        % Axes
        \draw[->, xaxis, thick] (0,0) -- (0,1.3) node[right, font=\scriptsize] {$x_B$};
        \draw[->, yaxis, thick] (0,0) -- (1.3,0) node[above, font=\scriptsize] {$y_B$};

        % Nose marker
        \fill[black] (0,1.7) -- (-0.15,1.4) -- (0.15,1.4) -- cycle;

        % Arm length
        \draw[<->, gray] (0.2,0) -- (0.2,2) node[midway, right, font=\tiny] {$d$};

        % Label
        \node[font=\small\bfseries] at (0,-3.3) {+ Configuration};
        \node[font=\tiny, gray] at (0,-3.8) {Forward along arm};
    \end{scope}

    % === X Configuration (right) ===
    \begin{scope}[shift={(5,0)}]
        % Arms (rotated 45°)
        \draw[thick, gray!70] (-1.4,-1.4) -- (1.4,1.4);
        \draw[thick, gray!70] (-1.4,1.4) -- (1.4,-1.4);

        % Central body
        \fill[gray!30] (-0.3,-0.3) rectangle (0.3,0.3);

        % Motors
        \node[motor] (m1x) at (1.4,1.4) {};
        \node[motor] (m2x) at (1.4,-1.4) {};
        \node[motor] (m3x) at (-1.4,-1.4) {};
        \node[motor] (m4x) at (-1.4,1.4) {};

        % Motor labels
        \node[font=\scriptsize] at (1.9,1.6) {M1};
        \node[font=\scriptsize] at (1.9,-1.6) {M2};
        \node[font=\scriptsize] at (-1.9,-1.6) {M3};
        \node[font=\scriptsize] at (-1.9,1.6) {M4};

        % Rotation arrows
        \draw[->, blue!60, thick] (1.7,1.4) arc (0:270:0.3);
        \draw[->, red!60, thick] (1.4,-1.1) arc (90:-180:0.3);
        \draw[->, blue!60, thick] (-1.1,-1.4) arc (0:270:0.3);
        \draw[->, red!60, thick] (-1.4,1.7) arc (90:-180:0.3);

        % Axes (between arms)
        \draw[->, xaxis, thick] (0,0) -- (0,1.3) node[right, font=\scriptsize] {$x_B$};
        \draw[->, yaxis, thick] (0,0) -- (1.3,0) node[above, font=\scriptsize] {$y_B$};

        % Nose marker (between M1 and M4)
        \fill[black] (0,1.7) -- (-0.15,1.4) -- (0.15,1.4) -- cycle;

        % Label
        \node[font=\small\bfseries] at (0,-3.3) {X Configuration};
        \node[font=\tiny, gray] at (0,-3.8) {Forward between arms};
    \end{scope}

    % Legend
    \node[font=\tiny, align=center] at (0,-3) {
        \textcolor{blue!60}{CW}: M1, M3 \quad
        \textcolor{red!60}{CCW}: M2, M4
    };
\end{tikzpicture}
\caption{Quadrotor configurations viewed from above. Left: Plus (+) configuration with forward direction along one arm. Right: X configuration (45° rotated) with forward direction between arms. Both use alternating CW/CCW motor directions.}
\label{fig:plus-vs-x-config}
\end{figure}

\begin{center}
\begin{tabular}{lcc}
\toprule
& \textbf{+ Configuration} & \textbf{X Configuration} \\
\midrule
Forward direction & Along one arm & Between two arms \\
Mixing equations & Simpler & Rotated 45° \\
Common use & Educational & Commercial \\
\bottomrule
\end{tabular}
\end{center}

We derive dynamics for the + configuration. The X configuration is identical after a 45° rotation of the mixing matrix.

\subsection{Motor Arrangement}

For + configuration with motors $M_1$ (front), $M_2$ (right), $M_3$ (back), $M_4$ (left):
\begin{itemize}
    \item $M_1$ and $M_3$ spin clockwise (CW)
    \item $M_2$ and $M_4$ spin counter-clockwise (CCW)
\end{itemize}

\textbf{Why alternate directions?} Each spinning propeller creates a reaction torque on the body. By having half spin each way, these torques cancel during hover, allowing yaw control through differential speed.

\subsection{Underactuation}

\begin{definition}
A system is \textbf{underactuated} if it has fewer control inputs than degrees of freedom.
\end{definition}

%----------------------------------------------------------------------
% FIGURE: Underactuation - Why Quadrotors Must Tilt to Move
%----------------------------------------------------------------------
\begin{figure}[htbp]
\centering
\begin{tikzpicture}[>=Stealth, scale=0.9]
    % === Panel 1: Hover ===
    \begin{scope}[shift={(-5,0)}]
        % Quadrotor body (level)
        \draw[thick, gray!70] (-1.5,0) -- (1.5,0);
        \fill[gray!40] (-0.3,-0.15) rectangle (0.3,0.15);
        \node[motor, minimum size=5mm] at (-1.5,0) {};
        \node[motor, minimum size=5mm] at (1.5,0) {};

        % Thrust vector
        \draw[->, very thick, green!60!black] (0,0.2) -- (0,2)
            node[right, font=\small] {$T$};

        % Weight vector
        \draw[->, very thick, purple] (0,-0.2) -- (0,-2)
            node[right, font=\small] {$mg$};

        % Equal signs
        \node[font=\scriptsize] at (0.8,1) {$=$};
        \node[font=\scriptsize] at (0.8,-1) {};

        % Label
        \node[font=\small\bfseries] at (0,-3) {Hover};
        \node[font=\tiny, gray, align=center] at (0,-3.6) {$T = mg$\\No horizontal force};
    \end{scope}

    % === Panel 2: Tilted ===
    \begin{scope}[shift={(0,0)}]
        % Quadrotor body (tilted 20°)
        \draw[thick, gray!70, rotate=20] (-1.5,0) -- (1.5,0);
        \fill[gray!40, rotate=20] (-0.3,-0.15) rectangle (0.3,0.15);
        \node[motor, minimum size=5mm, rotate=20] at ({-1.5*cos(20)},{-1.5*sin(20)}) {};
        \node[motor, minimum size=5mm, rotate=20] at ({1.5*cos(20)},{1.5*sin(20)}) {};

        % Thrust vector (perpendicular to body)
        \draw[->, very thick, green!60!black] (0,0.2) -- ({-2*sin(20)},{2*cos(20)})
            node[above left, font=\small] {$T$};

        % Decompose thrust
        \draw[dashed, green!40!black] ({-2*sin(20)},{2*cos(20)}) -- ({-2*sin(20)},0);
        \draw[dashed, green!40!black] ({-2*sin(20)},{2*cos(20)}) -- (0,{2*cos(20)});

        % Components
        \draw[->, thick, blue] (0,0.2) -- (0,{2*cos(20)})
            node[right, font=\scriptsize, pos=0.7] {$T\cos\theta$};
        \draw[->, thick, red] (0,0.2) -- ({-2*sin(20)},0.2)
            node[below, font=\scriptsize] {$T\sin\theta$};

        % Weight
        \draw[->, very thick, purple] (0,-0.2) -- (0,-1.8)
            node[right, font=\small] {$mg$};

        % Angle arc
        \draw[thick] (0,1) arc (90:110:1) node[midway, above, font=\scriptsize] {$\theta$};

        % Label
        \node[font=\small\bfseries] at (0,-3) {Tilted};
        \node[font=\tiny, gray, align=center] at (0,-3.6) {$F_{horiz} = T\sin\theta$\\Accelerates left!};
    \end{scope}

    % === Panel 3: DOF diagram ===
    \begin{scope}[shift={(5.5,0)}]
        % DOF box
        \node[draw, rounded corners, fill=blue!10, minimum width=2.5cm,
              minimum height=1.5cm, align=center, font=\small] at (0,1.5)
            {6 DOF\\$x,y,z,\phi,\theta,\psi$};

        % Inputs box
        \node[draw, rounded corners, fill=orange!10, minimum width=2.5cm,
              minimum height=1.2cm, align=center, font=\small] at (0,-1)
            {4 Inputs\\$\omega_1,\omega_2,\omega_3,\omega_4$};

        % Arrow
        \draw[->, thick] (0,-0.3) -- (0,0.6);

        % Direct control
        \node[font=\tiny, align=left] at (2.5,1.5) {Direct:\\$z,\phi,\theta,\psi$};

        % Indirect control
        \node[font=\tiny, align=left, red!70!black] at (2.5,0.7) {Via tilt:\\$x, y$};

        % Label
        \node[font=\small\bfseries] at (0,-3) {Underactuated};
        \node[font=\tiny, gray, align=center] at (0,-3.6) {6 DOF, 4 inputs\\Must tilt to translate};
    \end{scope}
\end{tikzpicture}
\caption{Quadrotor underactuation explained. Left: In hover, thrust balances gravity with no horizontal force. Center: To accelerate horizontally, the quadrotor must tilt, creating a horizontal thrust component. Right: With 6 degrees of freedom but only 4 control inputs, horizontal position must be controlled indirectly through attitude.}
\label{fig:underactuation}
\end{figure}

\begin{keyidea}[title=Quadrotor Underactuation]
\begin{itemize}
    \item Degrees of freedom: 6 (position $x, y, z$ and orientation $\phi, \theta, \psi$)
    \item Control inputs: 4 (motor speeds $\omega_1, \omega_2, \omega_3, \omega_4$)
\end{itemize}
\textbf{Consequence}: Cannot independently control all 6 DOF. To move horizontally, the quadrotor must first tilt, directing some thrust sideways. This is why attitude control is fundamental to position control.
\end{keyidea}

\section{Forces and Torques}
\index{thrust}\index{torque}

\subsection{Motor Thrust}

\begin{notebox}[title=Assumption: Quasi-Steady Aerodynamics]
We assume propeller thrust and torque are proportional to the square of propeller speed:
\[
T_i = b \cdot \omega_i^2, \quad Q_i = k \cdot \omega_i^2
\]
This is valid when:
\begin{itemize}
    \item Propeller speed changes slowly compared to air response
    \item Free-stream velocity is small (hover or slow flight)
    \item Ground effect is negligible
\end{itemize}
For aggressive maneuvers, these assumptions may break down.
\end{notebox}

Each motor produces thrust proportional to speed squared:
\begin{equation}
T_i = b \cdot \omega_i^2
\end{equation}

where $b$ is the thrust coefficient (determined experimentally or from propeller theory).

Total thrust:
\begin{equation}
T = b(\omega_1^2 + \omega_2^2 + \omega_3^2 + \omega_4^2)
\end{equation}

\subsection{Motor Drag Torque}

Each spinning propeller creates a reaction torque:
\begin{equation}
Q_i = k \cdot \omega_i^2
\end{equation}

where $k$ is the drag coefficient.

\subsection{Body Torques}

\textbf{Roll torque} (about $x_B$): Differential thrust between left and right motors:
\begin{equation}
\tau_\phi = d(T_4 - T_2) = db(\omega_4^2 - \omega_2^2)
\end{equation}

\textbf{Pitch torque} (about $y_B$): Differential thrust between front and back:
\begin{equation}
\tau_\theta = d(T_1 - T_3) = db(\omega_1^2 - \omega_3^2)
\end{equation}

\textbf{Yaw torque} (about $z_B$): Net reaction torque from all motors:
\begin{equation}
\tau_\psi = k(\omega_1^2 + \omega_3^2 - \omega_2^2 - \omega_4^2)
\end{equation}

where $d$ is the arm length from center to motor.

%----------------------------------------------------------------------
% FIGURE: Yaw Torque from Motor Reaction
%----------------------------------------------------------------------
\begin{figure}[htbp]
\centering
\begin{tikzpicture}[>=Stealth, scale=0.9]
    % === Left panel: Single motor physics ===
    \begin{scope}[shift={(-5,0)}]
        % Motor body (side view)
        \fill[gray!40] (-0.3,-1) rectangle (0.3,0);
        \draw[thick] (-0.3,-1) -- (-0.3,0) -- (0.3,0) -- (0.3,-1);

        % Propeller (top view indication)
        \draw[thick, fill=gray!20] (-1.5,0) -- (-0.3,0.1) -- (-0.3,-0.1) -- cycle;
        \draw[thick, fill=gray!20] (1.5,0) -- (0.3,0.1) -- (0.3,-0.1) -- cycle;

        % Propeller rotation (CW when viewed from above)
        \draw[->, thick, blue!60] (0.8,0.5) arc (30:330:0.5)
            node[right, pos=0.5, font=\scriptsize] {spin};

        % Drag torque on propeller (opposes rotation)
        \draw[->, very thick, red] (-0.6,0.8) arc (150:30:0.6)
            node[above, font=\scriptsize, pos=0.5] {$Q_{drag}$};

        % Reaction torque on body
        \draw[->, very thick, green!60!black] (0,-1.5) arc (-90:90:0.4)
            node[right, font=\scriptsize] {$Q_{react}$};

        % Newton's third law annotation
        \node[font=\scriptsize, align=center] at (0,-2.5) {Reaction torque\\on airframe};

        % Label
        \node[font=\small\bfseries] at (0,2) {Single Motor};
    \end{scope}

    % === Right panel: Four motors (top view) ===
    \begin{scope}[shift={(3,0)}]
        % Arms
        \draw[thick, gray!70] (0,-2) -- (0,2);
        \draw[thick, gray!70] (-2,0) -- (2,0);

        % Central body
        \fill[gray!30] (-0.4,-0.4) rectangle (0.4,0.4);

        % Motors
        \node[motor, minimum size=6mm] (m1) at (0,2) {};
        \node[motor, minimum size=6mm] (m2) at (2,0) {};
        \node[motor, minimum size=6mm] (m3) at (0,-2) {};
        \node[motor, minimum size=6mm] (m4) at (-2,0) {};

        % Motor labels
        \node[font=\scriptsize] at (0,2.6) {M1 (CW)};
        \node[font=\scriptsize] at (2.8,0) {M2 (CCW)};
        \node[font=\scriptsize] at (0,-2.6) {M3 (CW)};
        \node[font=\scriptsize] at (-2.8,0) {M4 (CCW)};

        % Reaction torque arrows (opposite to motor spin)
        % CW motors -> CCW reaction on body
        \draw[->, thick, blue!60] (0.3,2) arc (0:-270:0.3);
        \draw[->, thick, blue!60] (0.3,-2) arc (0:-270:0.3);
        % CCW motors -> CW reaction on body
        \draw[->, thick, red!60] (2,0.3) arc (90:360:0.3);
        \draw[->, thick, red!60] (-2,0.3) arc (90:360:0.3);

        % Net yaw torque (example: yaw CW)
        \draw[->, very thick, orange] (0.6,0) arc (0:-300:0.6)
            node[below right, font=\scriptsize, pos=0.4] {$\tau_\psi$};

        % Legend
        \node[font=\tiny, align=left, blue!60] at (4,1.5) {CW spin\\$\rightarrow$ CCW reaction};
        \node[font=\tiny, align=left, red!60] at (4,0.5) {CCW spin\\$\rightarrow$ CW reaction};
    \end{scope}

    % Key insight box
    \node[draw=orange!70!black, fill=orange!5, rounded corners,
          font=\scriptsize, align=center, text width=5cm] at (3,-3.5) {
        \textbf{To yaw CW:} Speed up M2, M4 (CCW motors)\\
        More CW reaction torque $\rightarrow$ body yaws CW
    };
\end{tikzpicture}
\caption{Yaw control via motor reaction torques. Left: Each spinning propeller experiences aerodynamic drag; by Newton's third law, this creates a reaction torque on the airframe. Right: CW and CCW motors create opposite reaction torques. Yaw is controlled by varying the balance between CW and CCW motor speeds.}
\label{fig:yaw-torque}
\end{figure}

\textbf{Intuition}: Yaw is controlled by the \emph{difference in total torque} between CW and CCW motors, not by thrust differences.

\section{Control Allocation (Mixing)}
\index{mixing matrix}\index{control allocation|see{mixing matrix}}

\subsection{The Mixing Matrix}

We want to map desired total thrust $T$ and torques $(\tau_\phi, \tau_\theta, \tau_\psi)$ to motor speed squares:

\begin{equation}
\begin{bmatrix} T \\ \tau_\phi \\ \tau_\theta \\ \tau_\psi \end{bmatrix} =
\underbrace{\begin{bmatrix}
b & b & b & b \\
0 & -db & 0 & db \\
db & 0 & -db & 0 \\
k & -k & k & -k
\end{bmatrix}}_{M}
\begin{bmatrix} \omega_1^2 \\ \omega_2^2 \\ \omega_3^2 \\ \omega_4^2 \end{bmatrix}
\end{equation}

\subsection{Inverse Mixing}

The controller computes desired $(T, \tau_\phi, \tau_\theta, \tau_\psi)$. We need motor speeds:

\begin{equation}
\begin{bmatrix} \omega_1^2 \\ \omega_2^2 \\ \omega_3^2 \\ \omega_4^2 \end{bmatrix} = M^{-1}
\begin{bmatrix} T \\ \tau_\phi \\ \tau_\theta \\ \tau_\psi \end{bmatrix}
\end{equation}

For the + configuration:
\begin{equation}
M^{-1} = \begin{bmatrix}
\frac{1}{4b} & 0 & \frac{1}{2db} & \frac{1}{4k} \\
\frac{1}{4b} & -\frac{1}{2db} & 0 & -\frac{1}{4k} \\
\frac{1}{4b} & 0 & -\frac{1}{2db} & \frac{1}{4k} \\
\frac{1}{4b} & \frac{1}{2db} & 0 & -\frac{1}{4k}
\end{bmatrix}
\end{equation}

\section{Equations of Motion}

\subsection{Translational Dynamics}

Newton's second law in the world frame:
\begin{equation}
m \ddot{\mathbf{r}} = \mathbf{F}_{gravity} + \mathbf{F}_{thrust} + \mathbf{F}_{drag}
\end{equation}

\begin{notebox}[title=Assumptions for Translational Dynamics]
\begin{itemize}
    \item Rigid body (no structural deformation)
    \item Point mass (mass concentrated at COG)
    \item Aerodynamic drag is linear in velocity (valid for low speeds)
    \item No wind
\end{itemize}
\end{notebox}

Expanding:
\begin{equation}
m \begin{bmatrix} \ddot{x} \\ \ddot{y} \\ \ddot{z} \end{bmatrix} =
\begin{bmatrix} 0 \\ 0 \\ mg \end{bmatrix} +
R^W_B \begin{bmatrix} 0 \\ 0 \\ -T \end{bmatrix} -
\begin{bmatrix} A_x \dot{x} \\ A_y \dot{y} \\ A_z \dot{z} \end{bmatrix}
\end{equation}

\textbf{Interpretation}:
\begin{itemize}
    \item Gravity pulls down (positive $z$ in NED)
    \item Thrust is along body $z$-axis (up in body frame, rotated to world by $R^W_B$)
    \item Drag opposes velocity
\end{itemize}

\subsection{Rotational Dynamics}

Euler's equation for a rotating rigid body:
\begin{equation}
\mathbf{J} \dot{\boldsymbol{\omega}} = -\boldsymbol{\omega} \times (\mathbf{J} \boldsymbol{\omega}) + \boldsymbol{\tau}
\end{equation}

\begin{notebox}[title=Assumptions for Rotational Dynamics]
\begin{itemize}
    \item Rigid body with constant inertia matrix $\mathbf{J}$
    \item Principal axes aligned with body axes (diagonal $\mathbf{J}$)
    \item Negligible rotor gyroscopic effects (valid for small, fast rotors)
\end{itemize}
\end{notebox}

For a symmetric quadrotor with $\mathbf{J} = \text{diag}(J_x, J_y, J_z)$:
\begin{align}
J_x \dot{p} &= (J_y - J_z) qr + \tau_\phi \\
J_y \dot{q} &= (J_z - J_x) pr + \tau_\theta \\
J_z \dot{r} &= (J_x - J_y) pq + \tau_\psi
\end{align}

\textbf{Interpretation}: The $(J_y - J_z)qr$ terms are \emph{gyroscopic coupling}---rotation about one axis affects rotation about others. For symmetric quadrotors ($J_x \approx J_y$), this effect is small.

\section{Complete State-Space Model}
\label{sec:state-space-model}

This section presents the complete mathematical model of the quadrotor in state-space form. We carefully define the notation, present both the full nonlinear model and its linearization around hover, and explain the relationship between them.

\subsection{State-Space Notation}

A \textbf{state-space model} represents a dynamical system as a set of first-order differential equations. The general form is:
\begin{equation}
\dot{\mathbf{x}} = f(\mathbf{x}, \mathbf{u}), \quad \mathbf{y} = h(\mathbf{x}, \mathbf{u})
\label{eq:nonlinear-ss}
\end{equation}
where:
\begin{itemize}
    \item $\mathbf{x} \in \mathbb{R}^n$ is the \textbf{state vector}---the minimum set of variables needed to completely describe the system's condition
    \item $\mathbf{u} \in \mathbb{R}^m$ is the \textbf{input vector}---the control signals we can manipulate
    \item $\mathbf{y} \in \mathbb{R}^p$ is the \textbf{output vector}---the quantities we measure or care about
    \item $f: \mathbb{R}^n \times \mathbb{R}^m \to \mathbb{R}^n$ is the \textbf{state transition function}
    \item $h: \mathbb{R}^n \times \mathbb{R}^m \to \mathbb{R}^p$ is the \textbf{output function}
\end{itemize}

For \textbf{linear} systems, these functions are matrix multiplications:
\begin{equation}
\dot{\mathbf{x}} = A\mathbf{x} + B\mathbf{u}, \quad \mathbf{y} = C\mathbf{x} + D\mathbf{u}
\label{eq:linear-ss}
\end{equation}
where $A \in \mathbb{R}^{n \times n}$, $B \in \mathbb{R}^{n \times m}$, $C \in \mathbb{R}^{p \times n}$, $D \in \mathbb{R}^{p \times m}$.

\subsection{State Vector Definition}
\label{sec:state-vector}

The quadrotor's state must capture everything needed to predict its future motion. This requires 12 variables:

\begin{equation}
\mathbf{x} = \begin{bmatrix} \mathbf{p} \\ \boldsymbol{\Theta} \\ \mathbf{v} \\ \boldsymbol{\omega} \end{bmatrix}
= \begin{bmatrix} x \\ y \\ z \\ \phi \\ \theta \\ \psi \\ v_x \\ v_y \\ v_z \\ p \\ q \\ r \end{bmatrix} \in \mathbb{R}^{12}
\label{eq:state-vector}
\end{equation}

\textbf{Component explanation}:
\begin{center}
\begin{tabular}{cllll}
\toprule
\textbf{Index} & \textbf{Symbol} & \textbf{Name} & \textbf{Frame} & \textbf{Units} \\
\midrule
1--3 & $\mathbf{p} = [x, y, z]^T$ & Position & World (NED) & m \\
4--6 & $\boldsymbol{\Theta} = [\phi, \theta, \psi]^T$ & Euler angles (roll, pitch, yaw) & Body-to-World & rad \\
7--9 & $\mathbf{v} = [v_x, v_y, v_z]^T$ & Linear velocity & World (NED) & m/s \\
10--12 & $\boldsymbol{\omega} = [p, q, r]^T$ & Angular velocity & Body & rad/s \\
\bottomrule
\end{tabular}
\end{center}

\begin{notebox}[title=Frame Conventions]
\begin{itemize}
    \item Position $\mathbf{p}$ and velocity $\mathbf{v}$ are expressed in the \textbf{world frame} (NED): $x$ points North, $y$ points East, $z$ points Down
    \item Angular velocity $\boldsymbol{\omega}$ is expressed in the \textbf{body frame}: $p$ is roll rate (about $x_B$), $q$ is pitch rate (about $y_B$), $r$ is yaw rate (about $z_B$)
    \item Euler angles $\boldsymbol{\Theta}$ define the rotation from world to body frame using ZYX convention
\end{itemize}
\end{notebox}

\subsection{Input Vector Definition}

The control inputs are the total thrust and three body-axis torques:
\begin{equation}
\mathbf{u} = \begin{bmatrix} T \\ \tau_\phi \\ \tau_\theta \\ \tau_\psi \end{bmatrix} \in \mathbb{R}^{4}
\label{eq:input-vector}
\end{equation}

where:
\begin{itemize}
    \item $T$ [N]: Total thrust magnitude (sum of all motor thrusts, along body $-z_B$ axis)
    \item $\tau_\phi$ [N$\cdot$m]: Roll torque (about body $x_B$ axis)
    \item $\tau_\theta$ [N$\cdot$m]: Pitch torque (about body $y_B$ axis)
    \item $\tau_\psi$ [N$\cdot$m]: Yaw torque (about body $z_B$ axis)
\end{itemize}

These inputs are computed from motor speeds via the mixing matrix (Section~\ref{sec:control-allocation}).

\subsection{Full Nonlinear State-Space Model}
\label{sec:nonlinear-model}

The complete nonlinear dynamics can be written as $\dot{\mathbf{x}} = f(\mathbf{x}, \mathbf{u})$, which expands to 12 first-order differential equations:

\textbf{Position kinematics} (states 1--3):
\begin{equation}
\dot{\mathbf{p}} = \mathbf{v} \quad \Leftrightarrow \quad
\begin{cases}
\dot{x} = v_x \\
\dot{y} = v_y \\
\dot{z} = v_z
\end{cases}
\label{eq:position-kinematics}
\end{equation}

These are simply kinematic relations: velocity is the time derivative of position.

\textbf{Attitude kinematics} (states 4--6):
\begin{equation}
\dot{\boldsymbol{\Theta}} = W(\boldsymbol{\Theta}) \boldsymbol{\omega} \quad \Leftrightarrow \quad
\begin{cases}
\dot{\phi} = p + (q \sin\phi + r\cos\phi)\tan\theta \\
\dot{\theta} = q\cos\phi - r\sin\phi \\
\dot{\psi} = (q\sin\phi + r\cos\phi)\sec\theta
\end{cases}
\label{eq:attitude-kinematics}
\end{equation}

where $W(\boldsymbol{\Theta})$ is the transformation from body angular velocity to Euler angle rates:
\begin{equation}
W(\boldsymbol{\Theta}) = \begin{bmatrix}
1 & \sin\phi\tan\theta & \cos\phi\tan\theta \\
0 & \cos\phi & -\sin\phi \\
0 & \sin\phi\sec\theta & \cos\phi\sec\theta
\end{bmatrix}
\label{eq:euler-rate-matrix}
\end{equation}

\begin{warningbox}[title=Singularity]
The matrix $W(\boldsymbol{\Theta})$ becomes singular when $\theta = \pm 90°$ (gimbal lock). This is a limitation of the Euler angle representation, not the physical system. For simulation near vertical orientations, use quaternions instead.
\end{warningbox}

\textbf{Translational dynamics} (states 7--9):
\begin{equation}
m\dot{\mathbf{v}} = \mathbf{F}_g + \mathbf{F}_T + \mathbf{F}_D
\label{eq:translational-dynamics}
\end{equation}

Expanding in the world frame (NED):
\begin{equation}
\begin{cases}
\dot{v}_x = -\dfrac{T}{m}(\cos\psi\sin\theta\cos\phi + \sin\psi\sin\phi) - \dfrac{A_x}{m}v_x \\[10pt]
\dot{v}_y = -\dfrac{T}{m}(\sin\psi\sin\theta\cos\phi - \cos\psi\sin\phi) - \dfrac{A_y}{m}v_y \\[10pt]
\dot{v}_z = g - \dfrac{T}{m}\cos\theta\cos\phi - \dfrac{A_z}{m}v_z
\end{cases}
\label{eq:velocity-dynamics}
\end{equation}

where:
\begin{itemize}
    \item The thrust terms come from $R^W_B \cdot [0, 0, -T]^T$ (thrust along body $-z_B$, rotated to world frame)
    \item $A_x, A_y, A_z$ are aerodynamic drag coefficients (typically small for slow flight)
    \item $g = 9.81$ m/s² is gravitational acceleration (positive in NED, since $z$ points down)
\end{itemize}

\textbf{Rotational dynamics} (states 10--12):
\begin{equation}
J\dot{\boldsymbol{\omega}} = -\boldsymbol{\omega} \times (J\boldsymbol{\omega}) + \boldsymbol{\tau}
\label{eq:rotational-dynamics}
\end{equation}

For a symmetric quadrotor with diagonal inertia matrix $J = \text{diag}(J_x, J_y, J_z)$:
\begin{equation}
\begin{cases}
\dot{p} = \dfrac{J_y - J_z}{J_x}qr + \dfrac{\tau_\phi}{J_x} \\[10pt]
\dot{q} = \dfrac{J_z - J_x}{J_y}pr + \dfrac{\tau_\theta}{J_y} \\[10pt]
\dot{r} = \dfrac{J_x - J_y}{J_z}pq + \dfrac{\tau_\psi}{J_z}
\end{cases}
\label{eq:angular-dynamics}
\end{equation}

The terms $(J_y - J_z)qr/J_x$, etc., are \textbf{gyroscopic coupling} terms arising from the cross product $\boldsymbol{\omega} \times (J\boldsymbol{\omega})$.

\subsection{Compact Matrix Form}

The full nonlinear model can be written compactly as:
\begin{equation}
\boxed{
\dot{\mathbf{x}} = f(\mathbf{x}, \mathbf{u}) = \begin{bmatrix}
\mathbf{v} \\
W(\boldsymbol{\Theta})\boldsymbol{\omega} \\
\frac{1}{m}\left( \mathbf{F}_g + R^W_B(\boldsymbol{\Theta})\mathbf{F}_T^B - D\mathbf{v} \right) \\
J^{-1}\left( -\boldsymbol{\omega} \times (J\boldsymbol{\omega}) + \boldsymbol{\tau} \right)
\end{bmatrix}
}
\label{eq:full-nonlinear}
\end{equation}

where:
\begin{itemize}
    \item $\mathbf{F}_g = [0, 0, mg]^T$ (gravity in NED world frame)
    \item $\mathbf{F}_T^B = [0, 0, -T]^T$ (thrust in body frame)
    \item $R^W_B(\boldsymbol{\Theta})$ is the rotation matrix from body to world (function of $\phi, \theta, \psi$)
    \item $D = \text{diag}(A_x, A_y, A_z)$ is the drag matrix
    \item $\boldsymbol{\tau} = [\tau_\phi, \tau_\theta, \tau_\psi]^T$
\end{itemize}

\begin{keyidea}[title=Structure of the Nonlinear Model]
The 12-state model has a clear structure:
\begin{enumerate}
    \item \textbf{Kinematics} (6 equations): Geometric relationships between positions/angles and velocities---no forces involved
    \item \textbf{Dynamics} (6 equations): Newton/Euler equations relating forces/torques to accelerations
\end{enumerate}
The nonlinearities come from:
\begin{itemize}
    \item Trigonometric functions in $W(\boldsymbol{\Theta})$ and $R^W_B(\boldsymbol{\Theta})$
    \item Gyroscopic coupling $\boldsymbol{\omega} \times (J\boldsymbol{\omega})$
    \item Product of thrust $T$ with orientation-dependent direction
\end{itemize}
\end{keyidea}

\section{Linearization Around Hover}
\index{linearization}\index{hover}

The nonlinear model~\eqref{eq:full-nonlinear} is accurate but difficult to analyze and design controllers for. \textbf{Linearization} approximates the nonlinear system by a linear one, valid near a chosen operating point. This enables powerful linear control design techniques.

\subsection{The Linearization Procedure}

Given a nonlinear system $\dot{\mathbf{x}} = f(\mathbf{x}, \mathbf{u})$, linearization proceeds as follows:

\textbf{Step 1: Choose an equilibrium point} $(\mathbf{x}_0, \mathbf{u}_0)$ satisfying $f(\mathbf{x}_0, \mathbf{u}_0) = \mathbf{0}$.

\textbf{Step 2: Define perturbation variables}:
\begin{equation}
\delta\mathbf{x} = \mathbf{x} - \mathbf{x}_0, \quad \delta\mathbf{u} = \mathbf{u} - \mathbf{u}_0
\end{equation}

\textbf{Step 3: Taylor expand} $f$ around $(\mathbf{x}_0, \mathbf{u}_0)$:
\begin{equation}
\dot{\mathbf{x}} = f(\mathbf{x}_0, \mathbf{u}_0) + \frac{\partial f}{\partial \mathbf{x}}\bigg|_0 \delta\mathbf{x} + \frac{\partial f}{\partial \mathbf{u}}\bigg|_0 \delta\mathbf{u} + \text{higher order terms}
\end{equation}

\textbf{Step 4: Neglect higher-order terms} to obtain the linear model:
\begin{equation}
\delta\dot{\mathbf{x}} = A\,\delta\mathbf{x} + B\,\delta\mathbf{u}
\label{eq:linearized-model}
\end{equation}

where the \textbf{Jacobian matrices} are:
\begin{equation}
A = \frac{\partial f}{\partial \mathbf{x}}\bigg|_{(\mathbf{x}_0, \mathbf{u}_0)} \in \mathbb{R}^{12 \times 12}, \quad
B = \frac{\partial f}{\partial \mathbf{u}}\bigg|_{(\mathbf{x}_0, \mathbf{u}_0)} \in \mathbb{R}^{12 \times 4}
\end{equation}

\subsection{Hover Equilibrium Point}

For a quadrotor, the natural equilibrium is \textbf{hover}---stationary flight with level attitude.

\begin{definition}[Hover Equilibrium]
The hover equilibrium $(\mathbf{x}_0, \mathbf{u}_0)$ is defined by:
\begin{align}
\mathbf{x}_0 &= [x_0, y_0, z_0, 0, 0, \psi_0, 0, 0, 0, 0, 0, 0]^T \label{eq:hover-state} \\
\mathbf{u}_0 &= [mg, 0, 0, 0]^T \label{eq:hover-input}
\end{align}
\end{definition}

\textbf{Interpretation}:
\begin{itemize}
    \item \textbf{Position} $(x_0, y_0, z_0)$: Any fixed point in space (the hover location)
    \item \textbf{Attitude} $(\phi_0, \theta_0, \psi_0) = (0, 0, \psi_0)$: Level orientation (roll and pitch zero), heading $\psi_0$ can be arbitrary
    \item \textbf{Velocities}: All zero (stationary)
    \item \textbf{Thrust} $T_0 = mg$: Exactly balances gravity
    \item \textbf{Torques} $\tau_{\phi,0} = \tau_{\theta,0} = \tau_{\psi,0} = 0$: No rotation commanded
\end{itemize}

\textbf{Verification}: Substituting into the nonlinear model~\eqref{eq:full-nonlinear}:
\begin{itemize}
    \item Position: $\dot{\mathbf{p}} = \mathbf{v}_0 = \mathbf{0}$ \checkmark
    \item Attitude: $\dot{\boldsymbol{\Theta}} = W(\boldsymbol{\Theta}_0)\boldsymbol{\omega}_0 = \mathbf{0}$ \checkmark
    \item Velocity: $\dot{\mathbf{v}} = [0, 0, g - T_0\cos(0)\cos(0)/m]^T = [0, 0, g - g]^T = \mathbf{0}$ \checkmark
    \item Angular velocity: $\dot{\boldsymbol{\omega}} = J^{-1}(\mathbf{0} + \mathbf{0}) = \mathbf{0}$ \checkmark
\end{itemize}

\subsection{Small-Perturbation Variables}

We define perturbations from hover:
\begin{equation}
\delta\mathbf{x} = \begin{bmatrix}
\delta x \\ \delta y \\ \delta z \\ \delta\phi \\ \delta\theta \\ \delta\psi \\
\delta v_x \\ \delta v_y \\ \delta v_z \\ \delta p \\ \delta q \\ \delta r
\end{bmatrix}, \quad
\delta\mathbf{u} = \begin{bmatrix}
\delta T \\ \delta\tau_\phi \\ \delta\tau_\theta \\ \delta\tau_\psi
\end{bmatrix}
\label{eq:perturbation-vars}
\end{equation}

where $T = mg + \delta T$ (thrust perturbation from hover thrust).

\textbf{Small-angle approximations} (valid for $|\phi|, |\theta| \ll 1$ radian, say $< 15°$):
\begin{equation}
\sin\alpha \approx \alpha, \quad \cos\alpha \approx 1, \quad \tan\alpha \approx \alpha
\label{eq:small-angle}
\end{equation}

\subsection{Computing the Jacobian Matrices}

Evaluating the partial derivatives at hover (with $\psi_0 = 0$ for simplicity):

\textbf{System matrix $A$} (12 × 12):

The structure of $A$ reflects the physical coupling between states:

\begin{equation}
A = \left[\begin{array}{c|c|c|c}
\mathbf{0}_{3\times3} & \mathbf{0}_{3\times3} & I_{3\times3} & \mathbf{0}_{3\times3} \\
\hline
\mathbf{0}_{3\times3} & \mathbf{0}_{3\times3} & \mathbf{0}_{3\times3} & I_{3\times3} \\
\hline
\mathbf{0}_{3\times3} & A_{v\Theta} & A_{vv} & \mathbf{0}_{3\times3} \\
\hline
\mathbf{0}_{3\times3} & \mathbf{0}_{3\times3} & \mathbf{0}_{3\times3} & \mathbf{0}_{3\times3}
\end{array}\right]
\label{eq:A-matrix-structure}
\end{equation}

where the non-zero blocks are:
\begin{equation}
A_{v\Theta} = \frac{\partial \dot{\mathbf{v}}}{\partial \boldsymbol{\Theta}}\bigg|_0 = \begin{bmatrix}
0 & -g & 0 \\
g & 0 & 0 \\
0 & 0 & 0
\end{bmatrix}, \quad
A_{vv} = -\frac{1}{m}\begin{bmatrix}
A_x & 0 & 0 \\
0 & A_y & 0 \\
0 & 0 & A_z
\end{bmatrix}
\label{eq:A-submatrices}
\end{equation}

\textbf{Physical interpretation of $A$}:
\begin{itemize}
    \item Row 1--3: Position changes with velocity ($\dot{\mathbf{p}} = \mathbf{v}$)
    \item Row 4--6: Euler angles change with angular velocity (at small angles: $\dot{\boldsymbol{\Theta}} \approx \boldsymbol{\omega}$)
    \item Row 7--9: Velocity changes with attitude (tilt causes horizontal acceleration via gravity) and with velocity itself (drag)
    \item Row 10--12: Angular velocity has no linear dependence on state at hover (gyroscopic terms vanish when $\boldsymbol{\omega}_0 = \mathbf{0}$)
\end{itemize}

\textbf{Input matrix $B$} (12 × 4):

\begin{equation}
B = \frac{\partial f}{\partial \mathbf{u}}\bigg|_0 = \begin{bmatrix}
\mathbf{0}_{3\times4} \\
\mathbf{0}_{3\times4} \\
B_v \\
B_\omega
\end{bmatrix}
\label{eq:B-matrix}
\end{equation}

where:
\begin{equation}
B_v = \begin{bmatrix}
0 & 0 & 0 & 0 \\
0 & 0 & 0 & 0 \\
-1/m & 0 & 0 & 0
\end{bmatrix}, \quad
B_\omega = \begin{bmatrix}
0 & 1/J_x & 0 & 0 \\
0 & 0 & 1/J_y & 0 \\
0 & 0 & 0 & 1/J_z
\end{bmatrix}
\label{eq:B-submatrices}
\end{equation}

\textbf{Physical interpretation of $B$}:
\begin{itemize}
    \item Thrust $\delta T$ directly affects vertical acceleration $\dot{v}_z$ (with gain $-1/m$; negative because $+z$ is down in NED)
    \item Torques $\tau_\phi, \tau_\theta, \tau_\psi$ directly affect angular accelerations $\dot{p}, \dot{q}, \dot{r}$
\end{itemize}

\subsection{Full Linearized State-Space Model}

Combining the above, the linearized model is:

\begin{equation}
\boxed{
\delta\dot{\mathbf{x}} = A\,\delta\mathbf{x} + B\,\delta\mathbf{u}
}
\label{eq:linearized-full}
\end{equation}

with explicit matrices (neglecting drag, i.e., $A_x = A_y = A_z = 0$):

\begin{equation}
A = \begin{bmatrix}
0 & 0 & 0 & 0 & 0 & 0 & 1 & 0 & 0 & 0 & 0 & 0 \\
0 & 0 & 0 & 0 & 0 & 0 & 0 & 1 & 0 & 0 & 0 & 0 \\
0 & 0 & 0 & 0 & 0 & 0 & 0 & 0 & 1 & 0 & 0 & 0 \\
0 & 0 & 0 & 0 & 0 & 0 & 0 & 0 & 0 & 1 & 0 & 0 \\
0 & 0 & 0 & 0 & 0 & 0 & 0 & 0 & 0 & 0 & 1 & 0 \\
0 & 0 & 0 & 0 & 0 & 0 & 0 & 0 & 0 & 0 & 0 & 1 \\
0 & 0 & 0 & 0 & -g & 0 & 0 & 0 & 0 & 0 & 0 & 0 \\
0 & 0 & 0 & g & 0 & 0 & 0 & 0 & 0 & 0 & 0 & 0 \\
0 & 0 & 0 & 0 & 0 & 0 & 0 & 0 & 0 & 0 & 0 & 0 \\
0 & 0 & 0 & 0 & 0 & 0 & 0 & 0 & 0 & 0 & 0 & 0 \\
0 & 0 & 0 & 0 & 0 & 0 & 0 & 0 & 0 & 0 & 0 & 0 \\
0 & 0 & 0 & 0 & 0 & 0 & 0 & 0 & 0 & 0 & 0 & 0 \\
\end{bmatrix}
\label{eq:A-explicit}
\end{equation}

\begin{equation}
B = \begin{bmatrix}
0 & 0 & 0 & 0 \\
0 & 0 & 0 & 0 \\
0 & 0 & 0 & 0 \\
0 & 0 & 0 & 0 \\
0 & 0 & 0 & 0 \\
0 & 0 & 0 & 0 \\
0 & 0 & 0 & 0 \\
0 & 0 & 0 & 0 \\
-1/m & 0 & 0 & 0 \\
0 & 1/J_x & 0 & 0 \\
0 & 0 & 1/J_y & 0 \\
0 & 0 & 0 & 1/J_z \\
\end{bmatrix}
\label{eq:B-explicit}
\end{equation}

\subsection{Decoupled Subsystem Interpretation}

A remarkable property of the linearized quadrotor model is that it \textbf{decouples} into independent subsystems:

\textbf{1. Altitude subsystem} (states: $z$, $v_z$; input: $\delta T$):
\begin{equation}
\begin{bmatrix} \delta\dot{z} \\ \delta\dot{v}_z \end{bmatrix} =
\begin{bmatrix} 0 & 1 \\ 0 & 0 \end{bmatrix}
\begin{bmatrix} \delta z \\ \delta v_z \end{bmatrix} +
\begin{bmatrix} 0 \\ -1/m \end{bmatrix} \delta T
\label{eq:altitude-subsystem}
\end{equation}

This is a \textbf{double integrator}: $m\ddot{z} = -\delta T$.

\textbf{2. Longitudinal subsystem} (states: $x$, $\theta$, $v_x$, $q$; input: $\tau_\theta$):
\begin{equation}
\begin{bmatrix} \delta\dot{x} \\ \delta\dot{\theta} \\ \delta\dot{v}_x \\ \delta\dot{q} \end{bmatrix} =
\begin{bmatrix} 0 & 0 & 1 & 0 \\ 0 & 0 & 0 & 1 \\ 0 & -g & 0 & 0 \\ 0 & 0 & 0 & 0 \end{bmatrix}
\begin{bmatrix} \delta x \\ \delta\theta \\ \delta v_x \\ \delta q \end{bmatrix} +
\begin{bmatrix} 0 \\ 0 \\ 0 \\ 1/J_y \end{bmatrix} \tau_\theta
\label{eq:longitudinal-subsystem}
\end{equation}

The key coupling: pitch angle $\theta$ causes horizontal acceleration via gravity ($\dot{v}_x = -g\theta$).

\textbf{3. Lateral subsystem} (states: $y$, $\phi$, $v_y$, $p$; input: $\tau_\phi$):
\begin{equation}
\begin{bmatrix} \delta\dot{y} \\ \delta\dot{\phi} \\ \delta\dot{v}_y \\ \delta\dot{p} \end{bmatrix} =
\begin{bmatrix} 0 & 0 & 1 & 0 \\ 0 & 0 & 0 & 1 \\ 0 & g & 0 & 0 \\ 0 & 0 & 0 & 0 \end{bmatrix}
\begin{bmatrix} \delta y \\ \delta\phi \\ \delta v_y \\ \delta p \end{bmatrix} +
\begin{bmatrix} 0 \\ 0 \\ 0 \\ 1/J_x \end{bmatrix} \tau_\phi
\label{eq:lateral-subsystem}
\end{equation}

Similar structure: roll angle $\phi$ causes lateral acceleration ($\dot{v}_y = g\phi$).

\textbf{4. Yaw subsystem} (states: $\psi$, $r$; input: $\tau_\psi$):
\begin{equation}
\begin{bmatrix} \delta\dot{\psi} \\ \delta\dot{r} \end{bmatrix} =
\begin{bmatrix} 0 & 1 \\ 0 & 0 \end{bmatrix}
\begin{bmatrix} \delta\psi \\ \delta r \end{bmatrix} +
\begin{bmatrix} 0 \\ 1/J_z \end{bmatrix} \tau_\psi
\label{eq:yaw-subsystem}
\end{equation}

Another \textbf{double integrator}: $J_z\ddot{\psi} = \tau_\psi$.

\begin{keyidea}[title=Why Decoupling Matters]
The 12-state system decomposes into four independent subsystems:
\begin{center}
\begin{tabular}{lccc}
\toprule
\textbf{Subsystem} & \textbf{States} & \textbf{Input} & \textbf{Structure} \\
\midrule
Altitude & 2 & $\delta T$ & Double integrator \\
Longitudinal ($x$-$\theta$) & 4 & $\tau_\theta$ & Cascaded double integrators \\
Lateral ($y$-$\phi$) & 4 & $\tau_\phi$ & Cascaded double integrators \\
Yaw & 2 & $\tau_\psi$ & Double integrator \\
\bottomrule
\end{tabular}
\end{center}
This decoupling means we can design four independent controllers, greatly simplifying the control problem.
\end{keyidea}

\subsection{Physical Insight: Tilt-to-Accelerate}

The linearized model reveals a fundamental principle:

\begin{equation}
\boxed{
\ddot{x} \approx -g\theta, \quad \ddot{y} \approx g\phi
}
\label{eq:tilt-accelerate}
\end{equation}

\textbf{To move North} (positive $x$): pitch nose-down ($\theta < 0$), creating a horizontal thrust component.

\textbf{To move East} (positive $y$): roll right ($\phi > 0$), tilting the thrust vector eastward.

This is why quadrotor position control uses a \textbf{cascaded architecture}:
\begin{enumerate}
    \item \textbf{Outer loop} (position controller): Computes desired attitude angles $(\phi_d, \theta_d)$ from position error
    \item \textbf{Inner loop} (attitude controller): Tracks the desired attitude by commanding torques
\end{enumerate}

\subsection{Validity of the Linearized Model}

The linearized model~\eqref{eq:linearized-full} is accurate when:
\begin{itemize}
    \item Roll and pitch angles satisfy $|\phi|, |\theta| < 15°$ (small-angle assumption)
    \item Angular velocities are small ($|p|, |q|, |r| < 1$ rad/s)
    \item Thrust is close to hover ($|T - mg| < 0.3mg$)
\end{itemize}

For aggressive maneuvers violating these assumptions, use the full nonlinear model~\eqref{eq:full-nonlinear} or gain-scheduled linearizations at multiple operating points.

%----------------------------------------------------------------------
% FIGURE: Cascaded Control Architecture
%----------------------------------------------------------------------
\begin{figure}[htbp]
\centering
\begin{tikzpicture}[
    block/.style={draw, rectangle, minimum height=2.2em, minimum width=2.8em,
                  align=center, rounded corners=2pt, font=\small},
    sum/.style={draw, circle, minimum size=1.3em, inner sep=0pt, font=\scriptsize},
    >=Stealth,
    scale=0.85, transform shape
]
    % === Position Control (outer loop - blue) ===
    \node[block, fill=blue!15] (rdes) at (0,0) {$\mathbf{r}_{des}$\\$(x,y,z)$};
    \node[sum] (sumpos) at (2,0) {$-$};
    \node[block, fill=blue!30, minimum width=3.5em] (posctrl) at (4.5,0)
        {Position\\Controller};

    % === Attitude Control (middle loop - green) ===
    \node[sum] (sumatt) at (7,0) {$-$};
    \node[block, fill=green!30, minimum width=3.5em] (attctrl) at (9.5,0)
        {Attitude\\Controller};

    % === Motor Mixing (orange) ===
    \node[block, fill=orange!30] (mix) at (12.5,0) {$M^{-1}$\\Mixing};

    % === Plant (gray) ===
    \node[block, fill=gray!30, minimum width=4em, minimum height=3em] (plant) at (15.5,0)
        {Quadrotor\\Dynamics};

    % Forward path connections
    \draw[->] (rdes) -- (sumpos);
    \draw[->, blue!70!black] (sumpos) -- (posctrl)
        node[midway, above, font=\tiny] {$e_{pos}$};
    \draw[->, blue!70!black] (posctrl) -- (sumatt)
        node[midway, above, font=\tiny] {$\phi_d,\theta_d,T$};
    \draw[->, green!70!black] (sumatt) -- (attctrl)
        node[midway, above, font=\tiny] {$e_{att}$};
    \draw[->, green!70!black] (attctrl) -- (mix)
        node[midway, above, font=\tiny] {$\boldsymbol{\tau}$};
    \draw[->, orange!70!black] (mix) -- (plant)
        node[midway, above, font=\tiny] {$\omega_i^2$};

    % Output
    \draw[->] (plant.east) -- ++(1,0) node[right, font=\small] {$\mathbf{r},\boldsymbol{\Theta}$};

    % Feedback paths
    % Attitude feedback (fast inner loop)
    \draw[->, green!60!black] (plant.south) -- ++(0,-1) -| (sumatt.south)
        node[pos=0.25, below, font=\tiny] {Attitude (IMU)};

    % Position feedback (slow outer loop)
    \draw[->, blue!60!black] (plant.south) -- ++(0,-1.8) -| (sumpos.south)
        node[pos=0.2, below, font=\tiny] {Position (GPS)};

    % Sample rate annotations
    \node[font=\tiny, blue!70!black] at (4.5,-1.8) {50--100 Hz};
    \node[font=\tiny, green!70!black] at (9.5,-1.8) {250--1000 Hz};

    % Loop labels
    \draw[blue!50, thick, dashed, rounded corners]
        (1.3,1.2) rectangle (5.8,-2.3);
    \node[blue!70!black, font=\tiny] at (3.5,1.5) {Outer Loop (Position)};

    \draw[green!50, thick, dashed, rounded corners]
        (6.3,0.9) rectangle (11,-1.5);
    \node[green!70!black, font=\tiny] at (8.7,1.2) {Inner Loop (Attitude)};

    % Key insight
    \node[draw=gray, fill=gray!5, rounded corners, font=\tiny,
          align=center, text width=4cm] at (15.5,-2.5) {
        Inner loop 5--10$\times$ faster\\
        Makes quadrotor ``easy to control''
    };
\end{tikzpicture}
\caption{Cascaded control architecture for quadrotors. The fast inner loop (green, 250--1000 Hz) controls attitude, making the quadrotor track desired roll/pitch angles. The slower outer loop (blue, 50--100 Hz) controls position by commanding attitude setpoints. This separation of timescales simplifies control design.}
\label{fig:cascaded-control}
\end{figure}

\begin{keyidea}[title=Cascaded Control Architecture]
The linearized model suggests:
\begin{enumerate}
    \item \textbf{Inner loop} (attitude control): Fast, high bandwidth (250--1000 Hz)
    \item \textbf{Outer loop} (position control): Slower, commands attitude (50--100 Hz)
\end{enumerate}
The inner loop makes the quadrotor ``easy to control''---it tracks desired attitudes. The outer loop decides what attitudes lead to the desired position.
\end{keyidea}

%======================================================================
% CHAPTER: Optimal Control
%======================================================================
\chapter{Optimal Control}
\label{ch:optimal-control}
\index{optimal control}
\index{LQR}

The cascaded PID architecture described in the previous chapter works well in practice, but requires tuning 12 or more gains empirically. How do we know if our gains are ``good''? Is there a systematic way to compute gains that achieve a desired trade-off between performance and control effort?

This chapter introduces \textbf{Linear Quadratic Regulator (LQR)} design~\cite{anderson2007optimal}---an optimal control method that computes state-feedback gains by minimizing a cost function. We show how LQR provides a principled approach to gain selection, and how it connects to PID tuning. Combined with the Kalman filter from Chapter~\ref{ch:kalman-filter}, this leads to the \textbf{Linear Quadratic Gaussian (LQG)} controller---the optimal controller for linear systems with Gaussian noise.

\section{The Optimal Control Problem}

\subsection{From Heuristic to Optimal}

Consider the attitude control problem. A PD controller for roll is:
\[
\tau_\phi = K_p(\phi_d - \phi) + K_d(p_d - p)
\]

\textbf{Questions that PID tuning doesn't answer}:
\begin{itemize}
    \item What values of $K_p$ and $K_d$ are ``best''?
    \item How do we trade off fast response (large gains) vs. control effort (small gains)?
    \item If we want to penalize roll error twice as much as roll rate error, what gains achieve this?
    \item Are there gains that guarantee stability and good robustness margins?
\end{itemize}

Optimal control answers these questions by:
\begin{enumerate}
    \item Defining a \textbf{cost function} that quantifies what ``good'' means
    \item Computing the \textbf{optimal gains} that minimize this cost
\end{enumerate}

\subsection{Cost Function Formulation}

For a linear system $\dot{\mathbf{x}} = \mathbf{A}\mathbf{x} + \mathbf{B}\mathbf{u}$, we define the \textbf{quadratic cost}:
\begin{equation}
J = \int_0^\infty \left(\mathbf{x}^T \mathbf{Q} \mathbf{x} + \mathbf{u}^T \mathbf{R} \mathbf{u}\right) dt
\label{eq:lqr-cost}
\end{equation}

where:
\begin{itemize}
    \item $\mathbf{Q} \succeq 0$: \textbf{State cost matrix} (symmetric positive semi-definite)
    \item $\mathbf{R} \succ 0$: \textbf{Control cost matrix} (symmetric positive definite)
\end{itemize}

\textbf{Interpretation}:
\begin{itemize}
    \item $\mathbf{x}^T \mathbf{Q} \mathbf{x}$: Penalty for being away from the origin (large states are bad)
    \item $\mathbf{u}^T \mathbf{R} \mathbf{u}$: Penalty for control effort (large inputs are expensive)
    \item The integral sums these penalties over all time
\end{itemize}

\begin{example}[Attitude Cost Function]
For the roll subsystem with state $\mathbf{x} = [\phi, p]^T$ and input $u = \tau_\phi$:
\[
J = \int_0^\infty \left(q_1 \phi^2 + q_2 p^2 + r \tau_\phi^2\right) dt
\]

In matrix form: $\mathbf{Q} = \begin{bmatrix} q_1 & 0 \\ 0 & q_2 \end{bmatrix}$, $\mathbf{R} = r$.

\textbf{Physical meaning}:
\begin{itemize}
    \item $q_1$ large: Penalize roll angle errors heavily (stay level!)
    \item $q_2$ large: Penalize angular velocity (smooth motion)
    \item $r$ large: Penalize torque (save motor effort, reduce vibration)
\end{itemize}
\end{example}

\subsection{The LQR Problem}

\begin{definition}[Linear Quadratic Regulator]
The \textbf{LQR problem} is to find the control law $\mathbf{u}(t)$ that minimizes the cost \eqref{eq:lqr-cost} subject to the dynamics $\dot{\mathbf{x}} = \mathbf{A}\mathbf{x} + \mathbf{B}\mathbf{u}$.
\end{definition}

The remarkable result is that the optimal control law is \textbf{linear state feedback}:
\begin{equation}
\mathbf{u}^* = -\mathbf{K}\mathbf{x}
\label{eq:lqr-law}
\end{equation}

where $\mathbf{K}$ is a constant gain matrix that depends only on $\mathbf{A}$, $\mathbf{B}$, $\mathbf{Q}$, and $\mathbf{R}$.

\section{Solution via the Riccati Equation}

\subsection{The Algebraic Riccati Equation}

The optimal gain matrix is:
\begin{equation}
\mathbf{K} = \mathbf{R}^{-1}\mathbf{B}^T\mathbf{P}
\label{eq:lqr-gain}
\end{equation}

where $\mathbf{P}$ is the unique positive definite solution to the \textbf{Algebraic Riccati Equation (ARE)}:
\begin{equation}
\mathbf{A}^T\mathbf{P} + \mathbf{P}\mathbf{A} - \mathbf{P}\mathbf{B}\mathbf{R}^{-1}\mathbf{B}^T\mathbf{P} + \mathbf{Q} = \mathbf{0}
\label{eq:are}
\end{equation}

\textbf{Note}: The ARE is a matrix equation where the unknown $\mathbf{P}$ appears both linearly and quadratically. Solving it analytically is possible only for small systems; numerical methods are used in practice.

\subsection{Proof Sketch}

The derivation uses calculus of variations or dynamic programming. Here we provide intuition:

\textbf{Key insight}: Define the \textbf{cost-to-go} from state $\mathbf{x}$:
\[
V(\mathbf{x}) = \min_{\mathbf{u}(\cdot)} \int_0^\infty \left(\mathbf{x}^T \mathbf{Q} \mathbf{x} + \mathbf{u}^T \mathbf{R} \mathbf{u}\right) dt
\]

For linear-quadratic problems, this has the form $V(\mathbf{x}) = \mathbf{x}^T \mathbf{P} \mathbf{x}$ for some matrix $\mathbf{P}$.

\textbf{Hamilton-Jacobi-Bellman equation}: The cost-to-go satisfies:
\[
0 = \min_{\mathbf{u}} \left[\mathbf{x}^T \mathbf{Q} \mathbf{x} + \mathbf{u}^T \mathbf{R} \mathbf{u} + \frac{\partial V}{\partial \mathbf{x}} (\mathbf{A}\mathbf{x} + \mathbf{B}\mathbf{u})\right]
\]

Substituting $V = \mathbf{x}^T \mathbf{P} \mathbf{x}$ and minimizing over $\mathbf{u}$:
\begin{enumerate}
    \item Take derivative w.r.t. $\mathbf{u}$: $2\mathbf{R}\mathbf{u} + \mathbf{B}^T(2\mathbf{P}\mathbf{x}) = 0$
    \item Solve: $\mathbf{u}^* = -\mathbf{R}^{-1}\mathbf{B}^T\mathbf{P}\mathbf{x}$
    \item Substitute back to get the ARE for $\mathbf{P}$
\end{enumerate}

\subsection{Closed-Form Solution for Second-Order Systems}

For a second-order system (like roll dynamics), the ARE can be solved analytically. Consider:
\[
\mathbf{A} = \begin{bmatrix} 0 & 1 \\ 0 & 0 \end{bmatrix}, \quad
\mathbf{B} = \begin{bmatrix} 0 \\ b \end{bmatrix}, \quad
\mathbf{Q} = \begin{bmatrix} q_1 & 0 \\ 0 & q_2 \end{bmatrix}, \quad
\mathbf{R} = r
\]

This is a double integrator with input gain $b$. The ARE solution gives:
\begin{align}
K_1 &= \sqrt{\frac{q_1}{r}} \label{eq:lqr-k1} \\
K_2 &= \sqrt{\frac{q_2 + 2\sqrt{q_1 r}}{r}} = \sqrt{\frac{q_2}{r} + 2K_1} \label{eq:lqr-k2}
\end{align}

\begin{keyidea}[title=LQR Design Rule for Double Integrators]
For a double-integrator system (position and velocity states), the LQR gains are:
\begin{itemize}
    \item \textbf{Position gain}: $K_1 = \sqrt{q_1/r}$ (depends only on position weight and control cost)
    \item \textbf{Velocity gain}: $K_2 = \sqrt{q_2/r + 2K_1}$ (depends on velocity weight and couples to position gain)
\end{itemize}
Increasing $q_1/r$ (penalize position more) increases both gains. Increasing $q_2/r$ (penalize velocity more) primarily increases the derivative gain.
\end{keyidea}

\section{LQR Design for Quadrotor Attitude}
\label{sec:lqr-attitude}

We now apply LQR to design attitude controllers for the quadrotor, using the decoupled subsystems from Section~\ref{sec:linearization}.

\subsection{Roll Subsystem Model}

From the linearized dynamics, the roll subsystem is:
\[
\begin{bmatrix} \dot{\phi} \\ \dot{p} \end{bmatrix} =
\begin{bmatrix} 0 & 1 \\ 0 & 0 \end{bmatrix}
\begin{bmatrix} \phi \\ p \end{bmatrix} +
\begin{bmatrix} 0 \\ 1/J_x \end{bmatrix} \tau_\phi
\]

This is a double integrator with $b = 1/J_x$.

\subsection{Design Procedure}

\textbf{Step 1: Choose cost weights.}

Start with physical reasoning:
\begin{itemize}
    \item Roll angle: Want $|\phi| < 15°$ during hover disturbances
    \item Angular rate: Want $|p| < 1$ rad/s for smooth flight
    \item Torque: Limited by motor saturation, say $|\tau_\phi| < 0.001$ Nm
\end{itemize}

Normalize by maximum acceptable values:
\[
q_1 = \frac{1}{(15° \cdot \pi/180)^2} \approx 15 \text{ rad}^{-2}, \quad
q_2 = \frac{1}{1^2} = 1 \text{ (rad/s)}^{-2}, \quad
r = \frac{1}{0.001^2} = 10^6 \text{ (Nm)}^{-2}
\]

\textbf{Step 2: Solve the ARE.}

Using MATLAB:
\begin{lstlisting}[language=Matlab]
Jx = 1.4e-5;  % Roll moment of inertia [kg*m^2]
A = [0 1; 0 0];
B = [0; 1/Jx];
Q = diag([15, 1]);
R = 1e6;

[K, P, e] = lqr(A, B, Q, R);
\end{lstlisting}

Result: $\mathbf{K} = [K_\phi, K_p] = [3.87 \times 10^{-3}, \; 5.57 \times 10^{-3}]$

Wait---these gains seem very small! This is because we're computing $\tau_\phi = -K_\phi \phi - K_p p$ directly, and the small moment of inertia $J_x$ means small torques produce large angular accelerations.

\textbf{Step 3: Verify closed-loop behavior.}

The closed-loop poles are:
\begin{lstlisting}[language=Matlab]
eig(A - B*K)
% ans = -138.5 + 138.5i
%       -138.5 - 138.5i
\end{lstlisting}

This gives a natural frequency $\omega_n = 196$ rad/s and damping ratio $\zeta = 0.707$---a fast, well-damped response typical for attitude control.

\subsection{Effect of Cost Weights}

\begin{center}
\begin{tikzpicture}
\begin{axis}[
    width=12cm, height=7cm,
    xlabel={Time (s)},
    ylabel={Roll angle $\phi$ (deg)},
    xmin=0, xmax=0.1,
    ymin=0, ymax=12,
    grid=major,
    legend pos=north east,
]
% High q1 (aggressive)
\addplot[blue, thick, domain=0:0.1, samples=50]
    {10*exp(-300*x)*cos(300*x*180/3.14159)};
\addlegendentry{High $q_1/r$: aggressive}

% Medium (balanced)
\addplot[red, thick, dashed, domain=0:0.1, samples=50]
    {10*exp(-150*x)*cos(150*x*180/3.14159)};
\addlegendentry{Medium: balanced}

% Low q1 (gentle)
\addplot[green!60!black, thick, dotted, domain=0:0.1, samples=50]
    {10*exp(-75*x)*cos(75*x*180/3.14159)};
\addlegendentry{Low $q_1/r$: gentle}

\end{axis}
\end{tikzpicture}
\end{center}

\begin{center}
\begin{tabular}{lccc}
\toprule
\textbf{Design} & $q_1/r$ & $\omega_n$ (rad/s) & Settling time (ms) \\
\midrule
Aggressive & $10^{-4}$ & 300 & 15 \\
Balanced & $10^{-5}$ & 150 & 30 \\
Gentle & $10^{-6}$ & 75 & 60 \\
\bottomrule
\end{tabular}
\end{center}

\textbf{Trade-off}: Higher $q_1/r$ gives faster response but requires more torque (larger gains), potentially causing motor saturation during aggressive maneuvers.

\section{LQR-Tuned PID: Bridging Theory and Practice}

While LQR provides optimal state feedback, practical implementations often use PID controllers for several reasons:
\begin{itemize}
    \item PID has integral action for steady-state error rejection
    \item PID operates on error, not absolute state
    \item PID is familiar to practitioners and easier to implement
\end{itemize}

The \textbf{LQR-tuned PID} approach combines the best of both worlds.

\subsection{From LQR Gains to PD Gains}

For the roll subsystem, LQR gives:
\[
\tau_\phi = -K_\phi \phi - K_p p
\]

A PD controller (without setpoint) is:
\[
\tau_\phi = K_P (\phi_d - \phi) + K_D (p_d - p)
\]

For regulation ($\phi_d = 0$, $p_d = 0$), these are identical with:
\begin{equation}
K_P = K_\phi, \quad K_D = K_p
\label{eq:lqr-to-pd}
\end{equation}

\subsection{Adding Integral Action}

LQR on the basic system has no integral action, leading to steady-state error under constant disturbances (e.g., center-of-mass offset, wind).

\textbf{Solution 1: Augment the state.}

Add an integrator state $\xi = \int \phi \, dt$ to get the augmented system:
\[
\begin{bmatrix} \dot{\phi} \\ \dot{p} \\ \dot{\xi} \end{bmatrix} =
\begin{bmatrix} 0 & 1 & 0 \\ 0 & 0 & 0 \\ 1 & 0 & 0 \end{bmatrix}
\begin{bmatrix} \phi \\ p \\ \xi \end{bmatrix} +
\begin{bmatrix} 0 \\ 1/J_x \\ 0 \end{bmatrix} \tau_\phi
\]

Design LQR with $\mathbf{Q} = \text{diag}(q_1, q_2, q_3)$ where $q_3$ weights the integrator state. This yields gains $[K_\phi, K_p, K_\xi]$ corresponding to a PID controller.

\textbf{Solution 2: Add integral term heuristically.}

Start with LQR-computed $K_P$ and $K_D$, then add $K_I$ empirically:
\begin{itemize}
    \item Start small: $K_I = 0.01 \cdot K_P$
    \item Increase until steady-state error is eliminated
    \item Ensure integral doesn't cause oscillation
\end{itemize}

\begin{lstlisting}[language=C, caption=LQR-tuned PID implementation]
// LQR-computed gains (from MATLAB)
#define KP_ROLL  3.87e-3f   // From LQR K[0]
#define KD_ROLL  5.57e-3f   // From LQR K[1]
#define KI_ROLL  0.05e-3f   // Added empirically

typedef struct {
    float integral;
    float prev_error;
} PIDState;

float pid_roll(PIDState* state, float phi_d, float phi, float p, float dt) {
    float error = phi_d - phi;

    // Anti-windup: limit integral
    state->integral += error * dt;
    state->integral = clamp(state->integral, -0.5f, 0.5f);

    // PID output
    float tau = KP_ROLL * error
              + KI_ROLL * state->integral
              + KD_ROLL * (0.0f - p);  // Assume p_d = 0

    return tau;
}
\end{lstlisting}

\subsection{Design Workflow}

\begin{enumerate}
    \item \textbf{Model}: Extract linearized subsystem (A, B matrices)
    \item \textbf{Specify}: Choose Q, R based on acceptable errors and actuator limits
    \item \textbf{Solve}: Compute LQR gains using MATLAB's \texttt{lqr()} or Python's \texttt{scipy.linalg.solve\_continuous\_are()}
    \item \textbf{Convert}: Map LQR gains to PD gains via \eqref{eq:lqr-to-pd}
    \item \textbf{Augment}: Add integral gain empirically or via augmented LQR
    \item \textbf{Implement}: Deploy as standard PID
    \item \textbf{Validate}: Test on hardware, fine-tune if needed
\end{enumerate}

\section{Full-State LQR vs. Cascaded PID}

For completeness, we compare full-state LQR (using all 12 states) with the cascaded PID architecture.

\subsection{Full-State LQR Design}

Using the full linearized model from Section~\ref{sec:linearization}:

\begin{lstlisting}[language=Matlab]
% Full 12-state model (simplified, without drag)
A = zeros(12,12);
A(1:3, 7:9) = eye(3);           % dp/dt = v
A(4:6, 10:12) = eye(3);         % dTheta/dt = omega (small angle)
A(7,5) = -g; A(8,4) = g;        % dv/dt depends on attitude

B = zeros(12,4);
B(9,1) = -1/m;                  % vz from thrust
B(10,2) = 1/Jx;                 % p from tau_phi
B(11,3) = 1/Jy;                 % q from tau_theta
B(12,4) = 1/Jz;                 % r from tau_psi

% LQR design
Q = diag([10 10 10 100 100 10 1 1 1 1 1 1]);  % Penalize attitude more
R = diag([1e6 1e6 1e6 1e6]);

K = lqr(A, B, Q, R);  % 4x12 gain matrix
\end{lstlisting}

The resulting 4×12 gain matrix $\mathbf{K}$ provides direct mapping from all states to all inputs---no cascade structure.

\subsection{Comparison}

\begin{center}
\begin{tabular}{lcc}
\toprule
\textbf{Aspect} & \textbf{Cascaded PID} & \textbf{Full-State LQR} \\
\midrule
\textbf{Gains to tune/specify} & 12--16 (4 PIDs) & 16 ($\mathbf{Q}$, $\mathbf{R}$ diagonals) \\
\textbf{Decoupling} & Manual (by design) & Automatic (MIMO) \\
\textbf{Physical intuition} & High (each loop clear) & Low (coupled gains) \\
\textbf{Implementation} & Simple (4 PIDs) & Complex (matrix multiply) \\
\textbf{States needed} & Errors only & Full state vector \\
\textbf{Integral action} & Built-in & Requires augmentation \\
\textbf{Robustness} & Adjustable per loop & Guaranteed margins \\
\textbf{Performance} & Very good & Optimal (by construction) \\
\textbf{Aggressive maneuvers} & May need gain scheduling & Handles coupling naturally \\
\bottomrule
\end{tabular}
\end{center}

\begin{keyidea}[title=When to Use Each Approach]
\textbf{Cascaded PID} is preferred when:
\begin{itemize}
    \item Implementation must be simple (limited MCU resources)
    \item Physical intuition and independent tuning are important
    \item Flight envelope is primarily hover and gentle maneuvers
    \item Integral action is essential (wind, weight variations)
\end{itemize}

\textbf{Full-State LQR} is preferred when:
\begin{itemize}
    \item Aggressive maneuvers with significant axis coupling
    \item Full state is available (e.g., from EKF)
    \item Systematic, reproducible tuning is needed
    \item Formal performance/robustness guarantees required
\end{itemize}

For most quadrotor applications, \textbf{LQR-tuned cascaded PID} provides an excellent compromise: principled gain selection with practical implementation.
\end{keyidea}

\section{LQG: Combining Kalman Filter and LQR}
\label{sec:lqg}

Chapter~\ref{ch:kalman-filter} developed the Kalman filter for state estimation; this chapter developed LQR for state feedback control. The \textbf{Linear Quadratic Gaussian (LQG)} controller combines them.

\subsection{The Separation Principle}

\begin{theorem}[Separation Principle]
For linear systems with Gaussian noise, the optimal controller consists of:
\begin{enumerate}
    \item A \textbf{Kalman filter} that computes $\hat{\mathbf{x}}$ from measurements
    \item An \textbf{LQR controller} that computes $\mathbf{u} = -\mathbf{K}\hat{\mathbf{x}}$
\end{enumerate}
These can be designed \textbf{independently}: the optimal estimator gain doesn't depend on the control cost, and the optimal control gain doesn't depend on the noise statistics.
\end{theorem}

\textbf{Intuition}: The Kalman filter provides the best estimate $\hat{\mathbf{x}}$ of the true state. Given this estimate, the best thing to do is apply the optimal control law as if $\hat{\mathbf{x}}$ were the true state.

\subsection{LQG Structure}

\begin{center}
\begin{tikzpicture}[
    block/.style={draw, rectangle, minimum height=2.5em, minimum width=4em, align=center},
    sum/.style={draw, circle, minimum size=1.5em, inner sep=0pt},
    >=Stealth
]
    % Plant
    \node[block] (plant) at (0,0) {Plant\\$\dot{\mathbf{x}} = \mathbf{A}\mathbf{x} + \mathbf{B}\mathbf{u}$};

    % Kalman filter
    \node[block, fill=blue!10] (kf) at (0,-2.5) {Kalman Filter\\$\hat{\mathbf{x}}$};

    % LQR
    \node[block, fill=green!10] (lqr) at (-4,-2.5) {LQR\\$\mathbf{u} = -\mathbf{K}\hat{\mathbf{x}}$};

    % Signals
    \draw[->] (plant.east) -- ++(1.5,0) node[right] {$\mathbf{y}$} |- (kf.east);
    \draw[->] (kf.west) -- (lqr.east) node[midway, above] {$\hat{\mathbf{x}}$};
    \draw[->] (lqr.west) -- ++(-1,0) |- (plant.west) node[near start, left] {$\mathbf{u}$};

    % Noise
    \draw[->] (1,1) node[above] {$\mathbf{w}$ (process)} -- (plant.north -| 0.5,0);
    \draw[->] (2,-1) node[right] {$\mathbf{v}$ (measurement)} -- (1.5,-1) -- (1.5,-0.5);

    % Annotation
    \node[draw, dashed, fit=(kf)(lqr), inner sep=10pt, label=below:LQG Controller] {};
\end{tikzpicture}
\end{center}

\subsection{Design Procedure}

\begin{enumerate}
    \item \textbf{Model}: Specify $\mathbf{A}$, $\mathbf{B}$, $\mathbf{C}$ matrices
    \item \textbf{Kalman filter}: Choose $\mathbf{Q}_{\text{KF}}$ (process noise), $\mathbf{R}_{\text{KF}}$ (measurement noise)
    \item \textbf{LQR}: Choose $\mathbf{Q}_{\text{LQR}}$ (state cost), $\mathbf{R}_{\text{LQR}}$ (control cost)
    \item \textbf{Compute}: Solve two Riccati equations (one for KF, one for LQR)
    \item \textbf{Implement}: Run Kalman filter, apply LQR control law
\end{enumerate}

\begin{lstlisting}[language=Matlab, caption=LQG design in MATLAB]
% System model
A = [0 1; 0 0];  B = [0; 1/Jx];  C = [1 0];

% Kalman filter design (estimate roll from noisy measurement)
Qkf = 1e-4;   % Process noise (gyro integration drift)
Rkf = 0.01;   % Measurement noise (accelerometer)
[Kf, ~, ~] = lqe(A, eye(2), C, Qkf*eye(2), Rkf);

% LQR design (control based on estimated state)
Qlqr = diag([10, 1]);  % State cost
Rlqr = 1e5;            % Control cost
[Kr, ~, ~] = lqr(A, B, Qlqr, Rlqr);

% Combined LQG: use Kr with estimated state from Kalman filter
% u = -Kr * x_hat
\end{lstlisting}

\subsection{LQG for Quadrotor Attitude}

Combining the attitude EKF from Chapter~\ref{ch:kalman-filter} with attitude LQR from Section~\ref{sec:lqr-attitude}:

See Listing~\ref{lst:lqg-controller} (Appendix~\ref{app:code-listings}) for the complete LQG attitude controller implementation that combines EKF state estimation with LQR feedback control.

\section{Practical Considerations}

\subsection{Gain Scheduling}

The linearized model is valid only near hover. For aggressive maneuvers:
\begin{itemize}
    \item Large angles invalidate small-angle approximations
    \item Thrust significantly different from $mg$
    \item Aerodynamic effects become nonlinear
\end{itemize}

\textbf{Solution}: \textbf{Gain scheduling}---use different LQR gains for different operating regions:

\begin{lstlisting}[language=C]
void select_gains(float phi, float theta, float thrust_ratio) {
    if (fabsf(phi) < 0.2f && fabsf(theta) < 0.2f &&
        thrust_ratio > 0.7f && thrust_ratio < 1.3f) {
        // Near hover: use nominal gains
        K = K_hover;
    } else {
        // Aggressive: use more conservative gains
        K = K_aggressive;
    }
}
\end{lstlisting}

\subsection{Actuator Saturation}

LQR assumes unconstrained control. When motors saturate:
\begin{itemize}
    \item Computed torques cannot be achieved
    \item Closed-loop behavior differs from design
    \item Integrators (if present) wind up
\end{itemize}

\textbf{Solutions}:
\begin{enumerate}
    \item \textbf{Conservative design}: Choose $\mathbf{R}$ to keep control within limits
    \item \textbf{Anti-windup}: Limit integrator states when saturated
    \item \textbf{Control allocation}: Prioritize attitude over position when saturated
\end{enumerate}

\subsection{Robustness Margins}

A key advantage of LQR is guaranteed stability margins:

\begin{theorem}[LQR Robustness]
For single-input LQR with $\mathbf{R} = r > 0$, the closed-loop system has:
\begin{itemize}
    \item \textbf{Gain margin}: $[\frac{1}{2}, \infty)$ (can double the gain without instability)
    \item \textbf{Phase margin}: $\geq 60°$
\end{itemize}
\end{theorem}

This robustness is ``free''---it comes from the optimization, not explicit design.

\begin{warningbox}[title=LQG Does Not Inherit LQR Robustness]
While LQR alone has excellent robustness margins, the combined LQG controller (LQR + Kalman filter) does \textbf{not} guarantee these margins. The Kalman filter introduces phase lag that can reduce stability margins.

\textbf{Mitigation}: Design the Kalman filter to be faster than the control bandwidth, or use loop transfer recovery (LTR) techniques.
\end{warningbox}

\section{Chapter Summary}

This chapter introduced optimal control methods for quadrotor attitude and position control:

\begin{itemize}
    \item \textbf{LQR} provides a systematic method for computing state-feedback gains by minimizing a quadratic cost function

    \item The \textbf{Algebraic Riccati Equation} gives optimal gains in terms of the cost weights $\mathbf{Q}$ (state penalty) and $\mathbf{R}$ (control penalty)

    \item \textbf{LQR-tuned PID} combines principled gain selection with practical integral action and familiar implementation

    \item \textbf{LQG} combines the Kalman filter (Chapter~\ref{ch:kalman-filter}) with LQR via the separation principle

    \item For most quadrotor applications, \textbf{cascaded PID with LQR-tuned gains} provides an excellent balance of performance, simplicity, and robustness
\end{itemize}

\begin{center}
\begin{tabular}{lll}
\toprule
\textbf{Method} & \textbf{Advantages} & \textbf{Best For} \\
\midrule
Hand-tuned PID & Simple, intuitive & Prototyping, one-off designs \\
LQR-tuned PID & Systematic, principled & Production systems \\
Full-state LQR & Optimal, handles coupling & Aggressive flight \\
LQG & Optimal estimation + control & Sensor fusion + control \\
\bottomrule
\end{tabular}
\end{center}

%======================================================================
\chapter{Parameter Identification}
%======================================================================

The mathematical models developed in this module contain physical parameters that must be determined for each specific quadrotor. This chapter presents systematic methods for identifying these parameters from measurements.

\section{Parameters to Identify}

The quadrotor dynamics model requires the following parameters:

\begin{center}
\begin{tabular}{llcc}
\toprule
\textbf{Parameter} & \textbf{Description} & \textbf{Units} & \textbf{Typical Value} \\
\midrule
$m$ & Total mass & kg & 0.027 (Crazyflie) \\
$I_{xx}$ & Roll moment of inertia & kg$\cdot$m$^2$ & $1.4 \times 10^{-5}$ \\
$I_{yy}$ & Pitch moment of inertia & kg$\cdot$m$^2$ & $1.4 \times 10^{-5}$ \\
$I_{zz}$ & Yaw moment of inertia & kg$\cdot$m$^2$ & $2.2 \times 10^{-5}$ \\
$L$ & Arm length (center to motor) & m & 0.046 \\
$C_T$ & Thrust coefficient & -- & 0.005 \\
$C_Q$ & Torque coefficient & -- & 0.0003 \\
$K_m$ & Motor constant & N/(rad/s)$^2$ & $3.5 \times 10^{-8}$ \\
$\tau_m$ & Motor time constant & s & 0.05 \\
\bottomrule
\end{tabular}
\end{center}

\begin{keyidea}[title=Parameter Identification Philosophy]
Some parameters can be measured directly (mass, arm length). Others require indirect methods based on dynamic response. The key is to design experiments that isolate individual parameters.
\end{keyidea}

\section{Mass and Geometry}

\subsection{Mass Measurement}

Total mass is measured directly using a precision scale:

\begin{lstlisting}[language=C, caption=Recording mass for flight log]
// Record mass as metadata for flight log
typedef struct {
    float total_mass_kg;     // Measured with scale
    float battery_mass_kg;   // Separate battery measurement
    float payload_mass_kg;   // Additional payload if any
} MassData;

// Total mass changes with battery charge level (minimal but measurable)
// For Crazyflie: ~27g without battery, ~34g with battery
\end{lstlisting}

\begin{notebox}[title=Mass Variation]
For accurate control, account for mass variation during flight:
\begin{itemize}
    \item Battery discharge: typically 1--2\% mass loss over flight
    \item Payload changes: if carrying variable loads
    \item Temperature effects: negligible for most applications
\end{itemize}
\end{notebox}

\subsection{Arm Length Measurement}

The arm length $L$ is the perpendicular distance from the center of mass to each rotor axis. For symmetric X-configuration quadrotors:

\[
L = \frac{d}{\sqrt{2}}
\]

where $d$ is the motor-to-motor diagonal distance.

\textbf{Measurement procedure}:
\begin{enumerate}
    \item Measure diagonal distance $d$ between opposite motor centers
    \item For X-configuration: $L = d / \sqrt{2}$
    \item For +-configuration: $L = d / 2$
    \item Verify center of mass is at geometric center (hang test)
\end{enumerate}

\section{Moment of Inertia Identification}
\index{moment of inertia}

The moments of inertia\index{moment of inertia|textbf} ($I_{xx}$, $I_{yy}$, $I_{zz}$) determine how quickly the quadrotor can rotate. Several methods exist for measurement.

\subsection{Bifilar Pendulum Method}

The bifilar pendulum is a classic technique for measuring moments of inertia:

%----------------------------------------------------------------------
% FIGURE: Bifilar Pendulum Setup
%----------------------------------------------------------------------
% Description: Quadrotor suspended by two parallel strings.
%
% Show:
%   - Quadrotor hanging level from two vertical strings
%   - String separation distance 'b'
%   - String length 'l'
%   - Angular displacement theta when twisted
%
% Dimensions: Half column width, ~5cm height.
%----------------------------------------------------------------------

The quadrotor is suspended by two parallel strings of length $l$, separated by distance $b$. When twisted and released, it oscillates with period $T$:

\begin{equation}
I = \frac{m g b^2 T^2}{16 \pi^2 l}
\label{eq:bifilar}
\end{equation}

See Listing~\ref{lst:bifilar} (Appendix~\ref{app:code-listings}) for the MATLAB implementation of Equation~\eqref{eq:bifilar}, including example measurements for a Crazyflie quadrotor.

\subsection{CAD-Based Estimation}

For initial estimates, CAD software can calculate moments of inertia if a detailed model exists with accurate material densities.

\subsection{Dynamic Identification from Flight Data}

Moments of inertia can also be identified from flight data by analyzing angular acceleration response to known torques:

\begin{equation}
I_{xx} \dot{\omega}_x = \tau_x \quad \Rightarrow \quad I_{xx} = \frac{\tau_x}{\dot{\omega}_x}
\end{equation}

See Listing~\ref{lst:dynamic-inertia} (Appendix~\ref{app:code-listings}) for a MATLAB script that identifies moment of inertia from step response flight data by analyzing angular acceleration.

\section{Thrust and Torque Coefficients}

The propulsion parameters $C_T$ and $C_Q$ (or equivalently $k_f$ and $k_\tau$) relate motor speed to thrust and torque:

\begin{align}
T &= C_T \rho D^4 \omega^2 = k_f \omega^2 \\
Q &= C_Q \rho D^5 \omega^2 = k_\tau \omega^2
\end{align}

\subsection{Thrust Stand Measurement}

The most reliable method uses a thrust test stand:

See Listing~\ref{lst:thrust-coeff} (Appendix~\ref{app:code-listings}) for a MATLAB script that fits $T = k_f \omega^2$ to thrust stand measurements and plots the fit quality.

\subsection{Torque Coefficient from Motor Constant}

The torque coefficient relates to thrust coefficient through the propeller geometry. A common approximation:

\[
\frac{k_\tau}{k_f} = \frac{C_Q}{C_T} \cdot D \approx c \cdot D
\]

where $c \approx 0.01$ for typical quadrotor propellers.

Alternatively, measure reaction torque directly:

See Listing~\ref{lst:torque-coeff} (Appendix~\ref{app:code-listings}) for a MATLAB script that identifies the torque coefficient from reaction torque measurements.

\section{Motor Dynamics Identification}

Motors have first-order dynamics that limit how fast thrust can change:

\[
\tau_m \dot{\omega}_m + \omega_m = \omega_m^{\text{cmd}}
\]

\subsection{Step Response Method}

Apply a step command and measure the rise time:

See Listing~\ref{lst:motor-timeconstant} (Appendix~\ref{app:code-listings}) for a MATLAB script that identifies the motor time constant from step response data using both the 63.2\% rise time method and nonlinear curve fitting.

\section{System Identification from Flight Data}

For complete system identification, frequency-domain methods provide robust results.

\subsection{Frequency Sweep Testing}

Inject sinusoidal commands of varying frequency and measure response:

See Listing~\ref{lst:freq-sweep} (Appendix~\ref{app:code-listings}) for a MATLAB script that processes frequency sweep flight test data to produce an experimental Bode plot using cross-correlation to extract gain and phase.

\subsection{Grey-Box System Identification}

Use MATLAB's System Identification Toolbox to fit a model structure with physical parameters:

See Listing~\ref{lst:greybox} (Appendix~\ref{app:code-listings}) for a complete grey-box system identification example using MATLAB's System Identification Toolbox. The code defines a parameterized state-space model for roll dynamics and estimates physical parameters ($I_{xx}$, $k_f$, $L$, $\tau_m$) from flight data.

\section{Validation}

Parameter identification is only useful if the results are validated.

\subsection{Cross-Validation}

Test the identified model on data not used for identification:

See Listing~\ref{lst:model-validate} (Appendix~\ref{app:code-listings}) for a MATLAB script that performs cross-validation by splitting data into training and validation sets.

\subsection{Physical Reasonableness}

Check that identified parameters are physically reasonable:

See Listing~\ref{lst:param-check} (Appendix~\ref{app:code-listings}) for a MATLAB function that validates identified parameters against physical constraints (reasonable inertia scaling, hover RPM range, motor time constant bounds).

\begin{keyidea}[title=Parameter Identification Best Practices]
\begin{enumerate}
    \item \textbf{Start simple}: Measure mass and geometry directly before trying dynamic methods
    \item \textbf{Design informative experiments}: Step responses and frequency sweeps excite dynamics
    \item \textbf{Use multiple methods}: Cross-check bifilar pendulum with dynamic identification
    \item \textbf{Validate thoroughly}: Test on held-out data and check physical reasonableness
    \item \textbf{Document everything}: Record test conditions, battery level, temperature
    \item \textbf{Iterate}: Initial flight tests may require updated parameters
\end{enumerate}
\end{keyidea}

\begin{warningbox}[title=Common Identification Pitfalls]
\begin{itemize}
    \item \textbf{Noisy data}: Filter appropriately before differentiation
    \item \textbf{Saturation}: Motor commands near limits give nonlinear response
    \item \textbf{Coupling}: Pure axis tests difficult; cross-coupling affects results
    \item \textbf{Air effects}: Ground effect, propeller wash affect hover tests
    \item \textbf{Battery variation}: Thrust changes with battery voltage
\end{itemize}
\end{warningbox}

\section{Module Summary}

This module has developed the complete framework for quadrotor orientation estimation:

\begin{enumerate}
    \item \textbf{Coordinate Frames and Transformations}
    \begin{itemize}
        \item World and body frames; NED convention
        \item Rotation matrices and their properties
        \item Euler angles: intuitive but singular (gimbal lock)
        \item Quaternions: robust and efficient
    \end{itemize}

    \item \textbf{Inertial Sensors}
    \begin{itemize}
        \item Gyroscope: accurate rates, but drifts
        \item Accelerometer: absolute roll/pitch reference, noisy during motion
        \item Magnetometer: yaw reference, susceptible to disturbances
    \end{itemize}

    \item \textbf{Orientation Estimation}
    \begin{itemize}
        \item Complementary filter: fuses sensors by frequency
        \item Mahony filter: practical 3D implementation
        \item Key insight: trust gyro for fast changes, accelerometer for long-term
    \end{itemize}

    \item \textbf{Quadrotor Dynamics}
    \begin{itemize}
        \item Forces and torques from motors
        \item Control allocation (mixing matrix)
        \item 12-state nonlinear model
        \item Linearization reveals cascaded control structure
    \end{itemize}
\end{enumerate}

In subsequent modules, you will implement these concepts on real hardware and design controllers for attitude and position regulation.


% ===== MODULE 2 =====
\part{Module 2: Simulation and Hybrid Systems}
%   - Equation-Based Modelling
%   - Hybrid Systems
%   - Numerical Simulation Methods
%   - Simulation Fidelity and Model Validity
%   - Multi-Rate Control Systems
%======================================================================
% MODULE 2: Equation-Based Modelling and Hybrid Systems
%
% Learning Objectives:
% After completing this module, students will be able to:
% - Explain the advantages of acausal equation-based modelling
% - Build multi-domain models using Modelica or Simscape
% - Define and analyze hybrid systems with discrete modes and continuous dynamics
% - Identify and prevent Zeno behavior in hybrid models
% - Select appropriate numerical solvers for quadrotor simulation
% - Implement flight mode state machines in Stateflow
%======================================================================

\chapter{Equation-Based Modelling}
\index{equation-based modelling}

\section{Introduction: Why Equation-Based Modelling?}

In Module 1, we derived the equations of motion for a quadrotor: Newton's laws for translation, Euler's equations for rotation, and motor thrust models. These equations describe the physics. But how do we turn them into a working simulation?

This question is more subtle than it might appear. The equations we write on paper are \textbf{declarative}---they state relationships between quantities. But a computer simulation is \textbf{procedural}---it needs a sequence of instructions that compute outputs from inputs. Bridging this gap is the central challenge of simulation.

\subsection{The Traditional Approach: Block Diagrams}

The traditional approach is to manually rearrange equations into \emph{explicit} form:
\[
\dot{x} = f(x, u)
\]
and then implement this as a block diagram in Simulink, where each block computes one part of $f$.

\textbf{Why this is the traditional approach}: Block diagrams mirror the way we think about causality---inputs flow through blocks to produce outputs. This worked well for simple control systems where the signal flow is clear (e.g., reference $\to$ controller $\to$ plant $\to$ output).

\textbf{Problems with this approach}:
\begin{itemize}
    \item \textbf{Manual algebra}: You must solve for derivatives by hand, which is error-prone for complex systems. For a quadrotor with 12 states and 4 inputs, this involves substantial manipulation.
    \item \textbf{Fixed causality}: Once you decide that voltage is input and current is output, you cannot easily reverse this. But physics doesn't care about causality---Ohm's law $v = Ri$ relates $v$ and $i$ without specifying which is ``input.''
    \item \textbf{Poor reusability}: A motor model designed for one context may not work in another without re-derivation. If your motor model outputs torque from voltage, you cannot directly use it in a context where you need voltage from torque.
    \item \textbf{Multi-domain difficulty}: Connecting electrical, mechanical, and thermal subsystems requires careful manual bookkeeping of energy flow and conservation laws.
\end{itemize}

\subsection{The Equation-Based Approach}

\begin{keyidea}[title=The Core Idea]
Instead of telling the computer \emph{how} to compute (assignments), we tell it \emph{what} relationships hold (equations). The computer automatically determines how to solve them.

\textbf{Traditional (causal)}:
\begin{lstlisting}[language=C]
i = v / R;  // Current is computed FROM voltage
\end{lstlisting}

\textbf{Equation-based (acausal)}:
\begin{lstlisting}[language=Pascal]
R * i = v;  // Relationship holds; causality determined later
\end{lstlisting}
\end{keyidea}

\textbf{Why this matters for quadrotors}:
\begin{itemize}
    \item A motor model can be reused whether you're computing torque from voltage or back-EMF from speed.
    \item Connecting a battery, ESC, motor, and propeller ``just works''---the tool figures out the math.
    \item Physical constraints (conservation laws) are automatically enforced.
\end{itemize}

\section{Acausal Modelling}

\begin{definition}[Acausal Model]
\index{acausal model}
An \textbf{acausal model}\index{acausal model|textbf} describes a system using equations without specifying which variables are inputs and which are outputs. The causality is determined automatically when the model is compiled, based on how components are connected.
\end{definition}

\textbf{Why ``acausal''?} The term means ``without cause''---not in the philosophical sense, but in the computational sense. We don't specify what causes what; we only specify what relationships must hold. The compiler figures out the causal (computational) structure from the model topology.

\textbf{Intuition}: Think of physical components. A resistor doesn't ``know'' whether it's being driven by a voltage source or a current source. It simply obeys Ohm's law. Acausal modelling captures this physical reality directly.

\subsection{A Simple Example: The Resistor}

Consider Ohm's law for a resistor:
\[
v = R \cdot i
\]

In a \textbf{causal} (traditional) implementation, you must choose:
\begin{itemize}
    \item $i = v/R$ (voltage in, current out), OR
    \item $v = R \cdot i$ (current in, voltage out)
\end{itemize}

In an \textbf{acausal} implementation, you simply write the equation:
\begin{lstlisting}[language=Pascal]
R * i = v;
\end{lstlisting}

The tool determines causality based on what's connected. If this resistor is connected to a voltage source, current becomes the output. If connected to a current source, voltage becomes the output.

\begin{keyidea}[title=Reusability Through Acausality]
The same resistor model works in \emph{any} circuit, regardless of what it's connected to. This dramatically improves model reusability and reduces errors from manual re-derivation.
\end{keyidea}

\subsection{Intuition: Physical vs. Computational Thinking}

\textbf{Physical thinking}: ``A resistor obeys $v = Ri$. Period. It doesn't care which variable is 'input' or 'output'.''

\textbf{Computational thinking}: ``I need to compute something. What's given? What do I solve for?''

Acausal modelling lets you think physically while the computer handles the computational details.

\section{Languages and Tools}

Two major equation-based modelling ecosystems exist:

\subsection{Modelica}
\index{Modelica}

Modelica~\cite{fritzson2014principles} is an open, object-oriented modeling language for multi-domain physical systems.

\begin{itemize}
    \item \textbf{Open standard}: The language specification is freely available.
    \item \textbf{Multiple implementations}:
    \begin{itemize}
        \item Commercial: Dymola (Dassault), Wolfram SystemModeler
        \item Open-source: OpenModelica, JModelica
    \end{itemize}
    \item \textbf{Extensive libraries}: Modelica Standard Library includes electrical, mechanical, thermal, fluid domains.
    \item \textbf{Industry adoption}: Automotive (Ford, Toyota), aerospace (Airbus), energy systems.
\end{itemize}

\textbf{History}: Initiated in 1996 by Hilding Elmqvist (Lund, Sweden), building on ideas from his PhD thesis on structured modelling.

\subsection{Simscape (MATLAB/Simulink)}
\index{Simscape}

\begin{itemize}
    \item \textbf{Tight integration}: Works seamlessly with Simulink, Stateflow, and MATLAB.
    \item \textbf{Same concepts}: Acausal modelling, physical connections, automatic equation sorting.
    \item \textbf{Domain libraries}: Simscape Electrical, Simscape Multibody, Simscape Fluids.
    \item \textbf{Vendor lock-in}: Proprietary to MathWorks.
\end{itemize}

\begin{notebox}[title=Choosing Between Modelica and Simscape]
\textbf{Use Modelica} if:
\begin{itemize}
    \item You need an open, portable model format
    \item You're collaborating with partners using different tools
    \item You want access to community-developed libraries
\end{itemize}

\textbf{Use Simscape} if:
\begin{itemize}
    \item Your control design is in Simulink/MATLAB
    \item You need tight integration with code generation
    \item Your organization already uses the MathWorks toolchain
\end{itemize}

For this course, we use Simscape due to its integration with Simulink for control design.
\end{notebox}

\section{Modelica Language Fundamentals}

Even if using Simscape, understanding Modelica syntax helps read documentation and examples.

\subsection{Basic Structure}

\begin{lstlisting}[language=Pascal, caption=Basic Modelica class structure]
class ClassName "Description string"
  // Declarations
  Real x(start = 0) "State variable";
  parameter Real k = 1.0 "Constant parameter";

equation
  // Equations (not assignments!)
  der(x) = -k * x;  // der() is the derivative

end ClassName;
\end{lstlisting}

Key elements:
\begin{itemize}
    \item \texttt{Real}: Continuous variable type
    \item \texttt{parameter}: Constant during simulation (can be changed between runs)
    \item \texttt{start}: Initial condition
    \item \texttt{der(x)}: Time derivative $\dot{x}$
    \item \texttt{equation}: Section containing equations (not assignments)
\end{itemize}

\subsection{Example: Van der Pol Oscillator}

\begin{lstlisting}[language=Pascal, caption=Van der Pol oscillator---a nonlinear system]
class VanDerPol "Van der Pol oscillator model"
  Real x(start = 1) "First state";
  Real y(start = 1) "Second state";
  parameter Real lambda = 0.3 "Nonlinearity parameter";
equation
  der(x) = y;
  der(y) = -x + lambda*(1 - x*x)*y;
end VanDerPol;
\end{lstlisting}

\textbf{Interpretation}: This is a second-order nonlinear ODE. The \texttt{equation} section directly mirrors the mathematical formulation---no manual rearrangement needed.

\section{Connectors and Physical Connections}

The real power of acausal modelling comes from \emph{connectors}---interfaces that enforce physical laws automatically.

\textbf{Motivation}: When you connect two electrical components, Kirchhoff's laws must hold at the connection point. When you connect two mechanical elements, forces must balance. These are not arbitrary rules---they express conservation of energy and charge. A good modelling framework should enforce these automatically, not require the modeller to add them manually (which is error-prone).

\subsection{Connector Variables: Potential and Flow}

Physical systems exchange energy through pairs of conjugate variables. This classification comes from bond graph theory and energy-based modelling.

\begin{notebox}[title=Assumption: Lumped Parameter Systems]
The potential-flow classification assumes \textbf{lumped parameter} systems where energy storage and dissipation are localized. This is valid when the physical dimensions are small compared to the wavelength of signals. For a quadrotor motor running at 20,000 RPM with electrical dynamics under 10 kHz, this assumption holds well. It would not hold for radio-frequency circuits or long transmission lines.
\end{notebox}

\begin{center}
\begin{tabular}{llll}
\toprule
\textbf{Domain} & \textbf{Potential (Across)} & \textbf{Flow (Through)} & \textbf{Power} \\
\midrule
Electrical & Voltage $v$ & Current $i$ & $P = v \cdot i$ \\
Translational & Velocity $v$ & Force $F$ & $P = v \cdot F$ \\
Rotational & Angular velocity $\omega$ & Torque $\tau$ & $P = \omega \cdot \tau$ \\
Hydraulic & Pressure $p$ & Volume flow $\dot{V}$ & $P = p \cdot \dot{V}$ \\
Thermal & Temperature $T$ & Heat flow $\dot{Q}$ & $P = T \cdot \dot{Q}$ \\
\bottomrule
\end{tabular}
\end{center}

\begin{keyidea}[title=Connection Rules]
When connectors are joined:
\begin{itemize}
    \item \textbf{Potential variables} are set \textbf{equal} (same voltage, same velocity)
    \item \textbf{Flow variables} sum to \textbf{zero} (Kirchhoff's current law, force balance)
\end{itemize}
These rules automatically enforce conservation of energy.
\end{keyidea}

\subsection{Connector Definitions in Modelica}

\begin{lstlisting}[language=Pascal, caption=Electrical and mechanical connectors]
connector ElectricalPin
  Voltage v;        // Potential variable
  flow Current i;   // Flow variable (note 'flow' keyword)
end ElectricalPin;

connector MechanicalFlange
  Angle phi;        // Potential: angular position
  flow Torque tau;  // Flow: torque
end MechanicalFlange;
\end{lstlisting}

The \texttt{flow} keyword tells the compiler to use sum-to-zero equations for that variable.

\subsection{The Connect Equation}

\begin{lstlisting}[language=Pascal]
connect(component1.port, component2.port);
\end{lstlisting}

This generates:
\begin{itemize}
    \item Equality equations for potential variables: $v_1 = v_2$
    \item Sum-to-zero for flow variables: $i_1 + i_2 = 0$
\end{itemize}

\section{Multi-Domain Modelling: DC Motor Example}

A DC motor connects electrical and mechanical domains---current produces torque, and rotation produces back-EMF.

\begin{lstlisting}[language=Pascal, caption=DC motor model in Modelica]
model DCMotor
  // Components
  Resistor R(R = 0.5);           // Armature resistance
  Inductor L(L = 0.01);          // Armature inductance
  EMF emf(k = 0.1);              // Electromechanical coupling
  Inertia J(J = 0.001);          // Rotor inertia
  Damper b(d = 0.0001);          // Viscous friction
  Ground elecGnd;                // Electrical ground
  Fixed mechGnd;                 // Mechanical ground (fixed frame)

  // Connectors for external connection
  ElectricalPin p, n;            // Positive and negative terminals
  MechanicalFlange shaft;        // Output shaft

equation
  // Electrical connections
  connect(p, R.p);
  connect(R.n, L.p);
  connect(L.n, emf.p);
  connect(emf.n, n);
  connect(n, elecGnd.p);

  // Mechanical connections
  connect(emf.flange, J.flange_a);
  connect(J.flange_b, shaft);
  connect(J.flange_a, b.flange_a);
  connect(b.flange_b, mechGnd.flange);

end DCMotor;
\end{lstlisting}

\textbf{What the tool does automatically}:
\begin{enumerate}
    \item Collects all equations from components (Ohm's law, Faraday's law, Newton's second law)
    \item Generates connection equations (voltage equalities, current sums)
    \item Sorts equations into computable order
    \item Solves algebraic loops if needed
    \item Generates efficient simulation code
\end{enumerate}

\section{Application: Quadrotor Motor-Propeller System}

Let us apply equation-based modelling to the motor-propeller subsystem that converts PWM commands to thrust. This is a multi-domain system: electrical (motor windings), mechanical (rotor dynamics), and aerodynamic (propeller thrust and drag).

\textbf{Why model this in detail?} For basic simulation, a simple $T = b\omega^2$ model suffices. But understanding the full motor-propeller dynamics is essential for:
\begin{itemize}
    \item Predicting transient response (how fast can thrust change?)
    \item Designing motor controllers that improve response
    \item Understanding failure modes (motor saturation, thermal limits)
    \item Optimizing propeller-motor matching for efficiency
\end{itemize}

\subsection{System Overview}

\begin{center}
\begin{tabular}{ccccc}
PWM & $\rightarrow$ & ESC & $\rightarrow$ & Motor \\
(0-100\%) & & (voltage modulation) & & (electrical $\rightarrow$ mechanical) \\
\\
& & & & $\downarrow$ \\
\\
Thrust & $\leftarrow$ & Propeller & $\leftarrow$ & Shaft \\
(force + torque) & & (mechanical $\rightarrow$ aerodynamic) & & (torque, speed)
\end{tabular}
\end{center}

\subsection{Component Models}

\textbf{ESC (Electronic Speed Controller)}:
\begin{notebox}[title=Assumption: Ideal ESC]
We model the ESC as an ideal voltage source proportional to the PWM duty cycle:
\[
V_{motor} = \text{PWM} \cdot V_{battery}
\]
This ignores switching dynamics and ESC current limits. For most flight conditions, this is adequate.
\end{notebox}

\textbf{Brushless DC Motor}:
\begin{align}
L \frac{di}{dt} &= V - Ri - k_e \omega \quad \text{(electrical dynamics)} \\
J \frac{d\omega}{dt} &= k_t i - \tau_{load} - b\omega \quad \text{(mechanical dynamics)}
\end{align}

where:
\begin{itemize}
    \item $k_e$: Back-EMF constant [V/(rad/s)]
    \item $k_t$: Torque constant [Nm/A]
    \item For an ideal motor: $k_e = k_t$ (from energy conservation)
\end{itemize}

\begin{notebox}[title=Assumptions in the Motor Model]
\begin{enumerate}
    \item \textbf{Linear magnetics}: The motor operates in the linear region of its B-H curve (no saturation). This is valid for normal operating currents.
    \item \textbf{No cogging torque}: We ignore the position-dependent torque ripple from motor pole interactions. This is a good approximation for time-averaged behavior.
    \item \textbf{Averaged three-phase}: We model the three-phase motor as an equivalent single-phase DC motor, valid when the ESC commutation is much faster than mechanical dynamics.
    \item \textbf{Constant parameters}: $R$, $L$, $k_e$, $k_t$ are constant. In reality, $R$ increases with temperature (by $\sim$0.4\%/°C for copper).
\end{enumerate}
\end{notebox}

\textbf{Intuition}: The electrical equation is Kirchhoff's voltage law around the motor circuit: applied voltage equals resistive drop plus inductive voltage plus back-EMF (a voltage source proportional to speed). The mechanical equation is Newton's second law for rotation: inertia times angular acceleration equals motor torque minus load torque minus friction.

\begin{notebox}[title=Assumption: First-Order Motor Dynamics]
For small motors with low inductance, the electrical time constant $\tau_e = L/R$ is very small (sub-millisecond). We can often simplify to first-order dynamics:
\[
\tau_m \dot{\omega} + \omega = K \cdot \text{PWM}
\]
where $\tau_m$ is the mechanical time constant and $K$ is the steady-state gain.
\end{notebox}

\textbf{Propeller}:
\begin{align}
T &= C_T \rho D^4 \omega^2 \quad \text{(thrust)} \\
Q &= C_Q \rho D^5 \omega^2 \quad \text{(drag torque)}
\end{align}

This matches Module 1's $T = b\omega^2$ and $Q = k\omega^2$ with:
\[
b = C_T \rho D^4, \quad k = C_Q \rho D^5
\]

\begin{notebox}[title=Assumptions in the Propeller Model]
\begin{enumerate}
    \item \textbf{Hover or low-speed flight}: These equations assume the propeller operates in nearly static air. In forward flight, the advance ratio changes, and $C_T$, $C_Q$ become functions of the inflow.
    \item \textbf{No ground effect}: Near the ground (within about one rotor diameter), thrust increases significantly due to air being trapped between the propeller and ground.
    \item \textbf{Quasi-steady aerodynamics}: The propeller instantly produces thrust proportional to $\omega^2$. This is valid when aerodynamic time scales are much faster than mechanical time scales.
    \item \textbf{Constant air density}: We assume $\rho$ is constant at sea-level standard conditions. At altitude, lower $\rho$ reduces thrust for the same RPM.
\end{enumerate}
\end{notebox}

\textbf{Intuition}: Why $\omega^2$? Thrust is proportional to the momentum imparted to air. The momentum flow rate is (mass flow rate) $\times$ (exit velocity). Both increase linearly with $\omega$, giving $\omega^2$ dependence. The diameter appears as $D^4$ because thrust scales with disk area ($D^2$) and with the square of tip speed ($D^2$).

\subsection{Simscape Implementation}

This section provides a complete, working Simscape implementation of the motor-propeller system.

\subsubsection{Custom Propeller Component}

First, we create a custom Simscape component for the propeller aerodynamics. Save this as \texttt{Propeller.ssc}:

\begin{lstlisting}[language=Matlab, caption=Propeller.ssc - Custom Simscape propeller component]
component Propeller
% Propeller
% Converts rotational motion to thrust and drag torque
% using aerodynamic coefficients.
%
% Thrust: T = CT * rho * D^4 * omega^2
% Torque: Q = CQ * rho * D^5 * omega^2

nodes
    R = foundation.mechanical.rotational.rotational;  % Shaft:right
end

outputs
    T = {0, 'N'};         % Thrust:right
end

parameters
    CT = {0.1, '1'};           % Thrust coefficient
    CQ = {0.01, '1'};          % Torque coefficient
    D = {0.1, 'm'};            % Propeller diameter
    rho = {1.225, 'kg/m^3'};   % Air density
    direction = {1, '1'};      % Rotation direction (+1 or -1)
end

variables
    w = {0, 'rad/s'};          % Angular velocity
    tau = {0, 'N*m'};          % Reaction torque
end

branches
    tau : R.t -> *;            % Torque flows into the shaft
end

equations
    w == R.w;                                          % Get angular velocity
    T == direction * CT * rho * D^4 * w * abs(w);     % Thrust (signed)
    tau == CQ * rho * D^5 * w * abs(w);               % Drag torque
end
end
\end{lstlisting}

\subsubsection{ESC Component}

The Electronic Speed Controller converts PWM commands to motor voltage:

\begin{lstlisting}[language=Matlab, caption=ESC.ssc - Electronic Speed Controller]
component ESC
% Electronic Speed Controller
% Converts PWM command (0-1) to voltage output

nodes
    p = foundation.electrical.electrical;    % +:right
    n = foundation.electrical.electrical;    % -:right
end

inputs
    pwm = {0, '1'};                          % PWM command (0-1)
end

parameters
    V_batt = {3.7, 'V'};                     % Battery voltage
end

variables
    v = {0, 'V'};                            % Output voltage
    i = {0, 'A'};                            % Current
end

branches
    i : p.i -> n.i;
end

equations
    v == p.v - n.v;
    v == pwm * V_batt;
end
end
\end{lstlisting}

\subsubsection{Complete Motor-Propeller Subsystem}

Now we assemble the complete system in a Simscape model. The following MATLAB script creates the model programmatically:

See Listing~\ref{lst:simscape-model} (Appendix~\ref{app:code-listings}) for a complete MATLAB script that programmatically creates the motor-propeller Simscape model, including DC motor, propeller, sensors, and all connections.

\subsubsection{Testing the Motor-Propeller Model}

See Listing~\ref{lst:simscape-test} (Appendix~\ref{app:code-listings}) for a MATLAB test script that validates the motor-propeller model by applying a step input and analyzing the step response.

\subsubsection{Parameter Tuning to Match Hardware}

To match the Simscape model to actual Crazyflie motors:

See Listing~\ref{lst:simscape-param-id} (Appendix~\ref{app:code-listings}) for a MATLAB script that back-calculates motor parameters ($K_e$, $J$, $C_T$) from measured step response data.

\begin{keyidea}[title=Simscape Model Benefits]
The Simscape motor-propeller model automatically handles:
\begin{itemize}
    \item \textbf{Bidirectional coupling}: Propeller load affects motor speed
    \item \textbf{Transient dynamics}: Electrical and mechanical time constants
    \item \textbf{Energy conservation}: Power flow is consistent across domains
    \item \textbf{Parameterization}: Easy to adjust for different motors/propellers
\end{itemize}

This model can be connected to the quadrotor rigid-body dynamics for complete system simulation.
\end{keyidea}

\subsection{Ground Effect Modeling}
\index{ground effect}

When a quadrotor hovers close to the ground, the propeller thrust increases beyond what the standard thrust equation predicts. This phenomenon, called \textbf{ground effect}\index{ground effect|textbf}, significantly impacts low-altitude flight and landing.

\subsubsection{Physical Explanation}

Ground effect occurs because the ground surface restricts the downward flow of air from the propellers:

\begin{itemize}
    \item Air cannot accelerate downward indefinitely---the ground redirects it
    \item Higher pressure builds up beneath the propeller
    \item The propeller effectively operates at a higher induced velocity
    \item For the same power input, more thrust is generated
\end{itemize}

%----------------------------------------------------------------------
% FIGURE: Ground Effect Flow Visualization
%----------------------------------------------------------------------
% Description: Side-by-side comparison of propeller flow patterns.
%
% Left panel: "Free air" - Propeller with streamlines showing
%   downward flow accelerating away, vortex ring at tip.
%
% Right panel: "Ground effect" - Same propeller close to ground,
%   streamlines curving outward along ground surface, higher
%   pressure region indicated beneath propeller.
%
% Show height 'z' and propeller diameter 'D' for scale.
% Dimensions: Full width, ~5cm height.
%----------------------------------------------------------------------

\subsubsection{Ground Effect Model}

The standard thrust model gives:
\[
T_\infty = C_T \rho D^4 \omega^2
\]
where $T_\infty$ is the thrust in free air (far from the ground).

Near the ground, thrust is amplified by a factor that depends on the height-to-diameter ratio:
\[
T_{\text{ground}} = T_\infty \cdot \sigma(z/D)
\]

Several empirical models exist for $\sigma(z/D)$. A commonly used model based on blade element momentum theory is:

\begin{equation}
\sigma(z/D) = \frac{1}{1 - \left(\frac{R}{4z}\right)^2} \approx \frac{1}{1 - \left(\frac{D}{8z}\right)^2}
\label{eq:ground_effect}
\end{equation}

where $R = D/2$ is the propeller radius and $z$ is the height above ground.

\begin{example}[Ground Effect Magnitude]
For a quadrotor with 46 mm propellers ($D = 0.046$ m) at various heights:

\begin{center}
\begin{tabular}{ccc}
\toprule
Height $z$ (cm) & $z/D$ & Thrust Multiplier $\sigma$ \\
\midrule
2.3 (0.5D) & 0.5 & 1.066 \\
4.6 (1D) & 1.0 & 1.016 \\
9.2 (2D) & 2.0 & 1.004 \\
23 (5D) & 5.0 & 1.0006 \\
\bottomrule
\end{tabular}
\end{center}

At one diameter above ground, thrust is about 1.6\% higher than in free air. Ground effect becomes negligible above $z > 3D$.
\end{example}

\subsubsection{Alternative Ground Effect Models}

Several alternative formulations are used in the literature:

\textbf{Cheeseman-Bennett Model} (classic helicopter theory):
\[
\sigma(z/R) = \frac{1}{1 - (R/4z)^2}
\]

\textbf{Hayden Model} (better for small $z$):
\[
\sigma(z/R) = 1 + \frac{R^2}{8z^2} - \frac{R^4}{128z^4}
\]

\textbf{Sanchez-Cuevas et al.} (fitted for quadrotors):
\[
\sigma(z/R) = 1 + A \exp(-B \cdot z/R)
\]
with typical values $A \approx 0.4$ and $B \approx 3$.

\begin{notebox}[title=Model Selection]
The exponential model is often preferred for quadrotors because:
\begin{itemize}
    \item It smoothly approaches 1 as $z \to \infty$
    \item Parameters can be fitted to experimental data
    \item No singularity at $z = D/8$ (unlike the classical model)
\end{itemize}
\end{notebox}

\subsubsection{Implementation in Simulation}

Adding ground effect to our Simscape propeller model:

See Listing~\ref{lst:propeller-ge} (Appendix~\ref{app:code-listings}) for the complete Simscape component that implements the exponential ground effect model.

\subsubsection{Impact on Control}

Ground effect has several implications for flight control:

\begin{itemize}
    \item \textbf{Landing}: The quadrotor requires less throttle near the ground. If the controller doesn't account for this, it may descend faster than expected and bounce.

    \item \textbf{Takeoff}: Thrust appears to decrease as the quadrotor climbs out of ground effect, which can feel like the motors are losing power.

    \item \textbf{Low-altitude hover}: Without compensation, position control will have a steady-state error dependent on height.
\end{itemize}

\textbf{Feedforward Compensation}:

See Listing~\ref{lst:ground-effect-ctrl} (Appendix~\ref{app:code-listings}) for a C implementation of ground effect feedforward compensation in the altitude control loop.

\begin{warningbox}[title=Height Measurement Requirements]
Ground effect compensation requires accurate height-above-ground measurement. GPS altitude is not suitable because:
\begin{itemize}
    \item GPS accuracy is typically 2--5 m (ground effect occurs within 10 cm)
    \item GPS measures altitude above sea level, not above ground
\end{itemize}
Use a downward-facing range sensor (ultrasonic, infrared, or LiDAR) for ground effect compensation.
\end{warningbox}

\subsubsection{Experimental Identification}

To characterize ground effect for a specific quadrotor:

See Listing~\ref{lst:ground-effect-id} (Appendix~\ref{app:code-listings}) for a MATLAB script that identifies ground effect parameters from experimental thrust measurements at various heights.

\begin{keyidea}[title=When Ground Effect Matters]
Include ground effect in your simulation if:
\begin{itemize}
    \item Testing landing/takeoff control algorithms
    \item Operating primarily at low altitudes (indoor, warehouse)
    \item Requiring high position accuracy near the ground
    \item Validating simulation against real flight data
\end{itemize}

For general flight testing above 1 meter, ground effect can usually be neglected.
\end{keyidea}

\section{Model Translation Process}

Understanding how equation-based models become simulations:

\begin{enumerate}
    \item \textbf{Flattening}: Expand all component hierarchies into one large set of equations
    \item \textbf{Equation analysis}: Identify variables, equations, and their dependencies
    \item \textbf{Index reduction}: Handle higher-index DAEs (constraints that require differentiation)
    \item \textbf{Causality assignment}: Determine which variable is solved from which equation
    \item \textbf{Sorting}: Order equations so each uses only previously computed values
    \item \textbf{Algebraic loop handling}: Solve simultaneous equations where needed
    \item \textbf{Code generation}: Produce efficient C code for simulation
\end{enumerate}

\begin{notebox}[title=What Can Go Wrong]
\begin{itemize}
    \item \textbf{Structurally singular}: More unknowns than equations (under-constrained)
    \item \textbf{Over-constrained}: More equations than unknowns (contradictory constraints)
    \item \textbf{High-index DAE}: Constraints require repeated differentiation (numerical difficulties)
\end{itemize}
The tool will report these errors during compilation.
\end{notebox}

\section{Chapter Summary}

Equation-based modelling with Modelica or Simscape offers significant advantages for quadrotor development:

\begin{itemize}
    \item \textbf{Physical fidelity}: Write equations as they appear in textbooks
    \item \textbf{Reusability}: Same motor model works in any context
    \item \textbf{Multi-domain}: Naturally connects electrical, mechanical, and aerodynamic subsystems
    \item \textbf{Automatic solving}: No manual causality derivation
    \item \textbf{Error checking}: Tool catches missing equations or constraints
\end{itemize}

The tradeoff is less direct control over the solution process and a steeper learning curve for the tools.


\chapter{Hybrid Systems}
\index{hybrid system}

\section{Introduction: Why Hybrid Systems?}

So far, we have modelled systems with continuous dynamics: differential equations that evolve smoothly in time. But real quadrotor systems\index{hybrid system|textbf} have \emph{discrete} aspects too:

\begin{itemize}
    \item \textbf{Flight modes}: The controller switches between Idle, Armed, Hover, Position Control, Land
    \item \textbf{Sensor modes}: GPS available vs. GPS denied triggers different estimation algorithms
    \item \textbf{Failsafes}: Low battery triggers Return-to-Home; signal loss triggers Land
    \item \textbf{Physical events}: Landing gear contact, propeller stall, motor saturation
\end{itemize}

\textbf{Why study hybrid systems formally?} One might think: ``I'll just add some if-statements to my simulation.'' This works for simple cases but leads to problems:
\begin{itemize}
    \item \textbf{Numerical issues}: Standard ODE solvers assume smooth dynamics. Sudden changes can cause solver failures or inaccurate results.
    \item \textbf{Analysis difficulty}: How do you prove stability when the controller switches between different modes? Standard Lyapunov theory doesn't directly apply.
    \item \textbf{Subtle bugs}: What happens if two mode transitions are triggered simultaneously? The informal approach doesn't answer this.
    \item \textbf{Zeno behavior}: Under certain conditions, infinitely many mode switches occur in finite time---a pathological situation that stops simulation.
\end{itemize}

A formal hybrid systems framework addresses these issues systematically. The theory of hybrid automata was developed by Henzinger~\cite{henzinger1996theory} and provides a rigorous foundation for modeling and analyzing such systems. For a comprehensive treatment, see Goebel et al.~\cite{goebel2012hybrid}.

\begin{keyidea}[title=The Hybrid Nature of Quadrotors]
A quadrotor is inherently a \textbf{hybrid system}:
\begin{itemize}
    \item \textbf{Continuous dynamics}: Position, velocity, orientation evolve according to ODEs
    \item \textbf{Discrete modes}: Flight controller state machine switches between behaviors
    \item \textbf{Mode-dependent dynamics}: Different control laws apply in different modes
    \item \textbf{Discrete events}: Takeoff, landing, failsafe activation
\end{itemize}
Understanding hybrid systems is essential for designing robust flight software.
\end{keyidea}

\section{Formal Definition of Hybrid Systems}
\label{sec:hybrid-automaton-def}

\begin{definition}[Hybrid Automaton]
\index{hybrid automaton}
A \textbf{hybrid automaton}\index{hybrid automaton|textbf} $H$ is defined by the tuple:
\[
H = (Q, X, f, \text{Init}, D, E, G, R)
\]
where:
\begin{itemize}
    \item $Q$: Finite set of \textbf{discrete modes}
    \item $X \subseteq \mathbb{R}^n$: \textbf{Continuous state space}
    \item $f: Q \times X \rightarrow \mathbb{R}^n$: \textbf{Vector field} (dynamics in each mode)
    \item $\text{Init} \subseteq Q \times X$: \textbf{Initial states}
    \item $D: Q \rightarrow 2^X$: \textbf{Domain} (invariant) for each mode
    \item $E \subseteq Q \times Q$: \textbf{Edges} (allowed mode transitions)
    \item $G: E \rightarrow 2^X$: \textbf{Guard conditions} (when transitions may occur)
    \item $R: E \times X \rightarrow 2^X$: \textbf{Reset map} (how continuous state changes at transitions)
\end{itemize}
\end{definition}

\textbf{Semantics}: The system evolves continuously in mode $q$ according to $\dot{x} = f(q, x)$ while $x \in D(q)$. When $x \in G(e)$ for some edge $e = (q, q')$, a discrete transition \emph{may} occur, with the new state satisfying $x^+ \in R(e, x^-)$. If $x$ reaches the boundary of $D(q)$ and no guard is enabled, the system blocks (this indicates a modeling error).

\textbf{Understanding each component}:
\begin{itemize}
    \item \textbf{Modes $Q$}: Operating regimes with distinct dynamics. A quadrotor has modes like Hover, Land, Emergency.

    \item \textbf{Continuous state $X$}: Physical variables evolving according to ODEs---position, velocity, orientation.

    \item \textbf{Vector field $f$}: Mode-dependent dynamics. In Hover mode, thrust compensates gravity; in Emergency mode, thrust is reduced.

    \item \textbf{Domain $D$}: The region where continuous evolution is valid in each mode. The system \emph{must} leave mode $q$ before $x$ exits $D(q)$.

    \item \textbf{Guards $G$}: Conditions that \emph{enable} a transition. A transition may occur when $x \in G(e)$, but is not forced until $x$ exits the domain.

    \item \textbf{Reset map $R$}: How continuous state changes at a transition. Often $R(e, x) = \{x\}$ (no reset), but integrator states may be zeroed.
\end{itemize}

\begin{keyidea}[title=Domain vs. Guard]
The distinction between domain and guard is crucial:
\begin{itemize}
    \item \textbf{Domain $D(q)$}: Where the system \emph{can} stay in mode $q$. Leaving the domain \emph{forces} a transition.
    \item \textbf{Guard $G(e)$}: Where a transition \emph{may} occur. The transition is optional while inside the domain.
\end{itemize}
Example: In Landing mode, the domain might be $D(\text{Landing}) = \{x : z \geq -0.5\}$ (above 0.5m). The guard to Settled might be $G(\text{Landing} \to \text{Settled}) = \{x : z > -0.05 \land |v_z| < 0.1\}$. The quadrotor can transition to Settled when near the ground and slow, but \emph{must} transition (or to another mode) before exceeding the domain boundary.
\end{keyidea}

\section{Example: Quadrotor Landing Sequence}
\label{sec:landing-hybrid}

Rather than the classical bouncing ball example, we develop a hybrid automaton for the \textbf{quadrotor landing sequence}---a practically relevant scenario that connects to the dynamics and control material throughout this book.

\subsection{Physical Description}

Landing a quadrotor involves several distinct phases with different dynamics:
\begin{enumerate}
    \item \textbf{Descent}: The quadrotor descends at a controlled rate toward the landing zone
    \item \textbf{Ground effect}: As altitude decreases below approximately one rotor diameter, aerodynamic ground effect increases lift
    \item \textbf{Touchdown}: The landing gear contacts the ground, introducing contact forces
    \item \textbf{Settled}: Motors spin down and the quadrotor rests on the ground
\end{enumerate}

Each phase has fundamentally different dynamics, making this a natural hybrid system.

\subsection{Hybrid Automaton Formulation}

We define the landing hybrid automaton with continuous state $x = [z, \dot{z}, I_z]^T$ where $z$ is altitude (NED, so $z < 0$ means above ground), $\dot{z}$ is vertical velocity, and $I_z$ is the altitude controller integrator state.

Following the formal definition from Section~\ref{sec:hybrid-automaton-def}, we specify:

\begin{itemize}
    \item $Q = \{\text{Descent}, \text{GroundEffect}, \text{Contact}, \text{Settled}\}$
    \item $X = \mathbb{R}^3$ (altitude, velocity, integrator state)
    \item $\text{Init} = \{(\text{Descent}, x) : z < -h_{ge}, \dot{z} = 0\}$ (start above ground effect region)
\end{itemize}

\textbf{Vector fields} (simplified to vertical dynamics):

\textbf{Descent mode}:
\[
f_{\text{Descent}}(x) = \begin{bmatrix} \dot{z} \\ g - \frac{T}{m} \\ z_{target} - z \end{bmatrix}
\quad \text{where} \quad T = mg + K_p(z_{target} - z) + K_d(\dot{z}_{target} - \dot{z}) + K_i I_z
\]

\textbf{GroundEffect mode} (thrust augmented by ground effect factor $\sigma(z)$):
\[
f_{\text{GE}}(x) = \begin{bmatrix} \dot{z} \\ g - \frac{T \cdot \sigma(z)}{m} \\ z_{target} - z \end{bmatrix}
\quad \text{where} \quad \sigma(z) = \frac{1}{1 - (r/4z)^2}
\]
Here $r$ is the rotor radius and $z$ is height above ground (recall from Chapter~\ref{ch:quadrotor-dynamics} that $\sigma > 1$ near the ground).

\textbf{Contact mode} (spring-damper ground interaction):
\[
f_{\text{Contact}}(x) = \begin{bmatrix} \dot{z} \\ g - \frac{T}{m} + \frac{1}{m}(k_g z - c_g \dot{z}) \\ 0 \end{bmatrix}
\]
where $k_g$ and $c_g$ are ground stiffness and damping. Note the integrator is frozen.

\textbf{Settled mode}:
\[
f_{\text{Settled}}(x) = \begin{bmatrix} 0 \\ 0 \\ 0 \end{bmatrix}
\]

\textbf{Domains} (invariants for each mode):
\begin{align*}
D(\text{Descent}) &= \{x \in X : z \leq -h_{ge}\} \\
D(\text{GroundEffect}) &= \{x \in X : -h_{ge} < z < \delta\} \\
D(\text{Contact}) &= \{x \in X : z \geq -\delta\} \\
D(\text{Settled}) &= X \quad \text{(terminal mode)}
\end{align*}
The domain specifies where the system \emph{must} remain while in each mode. When the state reaches the boundary of the domain, a transition \emph{must} occur (if a guard is enabled) or the model is ill-posed.

\textbf{Edge set}:
\[
E = \{(\text{Descent}, \text{GroundEffect}), (\text{GroundEffect}, \text{Contact}), (\text{Contact}, \text{Settled}), (\text{Contact}, \text{GroundEffect})\}
\]

\textbf{Guard conditions}:
\begin{center}
\begin{tabular}{lll}
\toprule
\textbf{Transition} & \textbf{Guard $G$} & \textbf{Physical meaning} \\
\midrule
Descent $\to$ GroundEffect & $z > -h_{ge}$ & Altitude below ground effect threshold \\
GroundEffect $\to$ Contact & $z \geq 0$ & Landing gear touches ground \\
Contact $\to$ Settled & $|z| < \epsilon \land |\dot{z}| < v_{thresh}$ & Motion stopped \\
Contact $\to$ GroundEffect & $z < -\delta$ & Bounced back up \\
\bottomrule
\end{tabular}
\end{center}
where $h_{ge} \approx 2r$ is the ground effect onset altitude (typically 0.1--0.3 m for small quadrotors).

\textbf{Reset maps}:
\begin{itemize}
    \item Descent $\to$ GroundEffect: $R(x) = x$ (no reset, continuous transition)
    \item GroundEffect $\to$ Contact: $R(x) = [0, \dot{z}, 0]^T$ (clamp altitude, reset integrator)
    \item Contact $\to$ Settled: $R(x) = [0, 0, 0]^T$ (zero all states)
\end{itemize}

The integrator reset at ground contact prevents wind-up from accumulated error during descent.

\subsection{Implementation in Modelica}

See Listing~\ref{lst:modelica-landing} (Appendix~\ref{app:code-listings}) for the complete Modelica implementation of the landing hybrid automaton with ground effect, contact dynamics, and integrator reset.

Key hybrid constructs:
\begin{itemize}
    \item \texttt{when}: Triggers on discrete events (guard condition becomes true)
    \item \texttt{reinit}: Resets continuous variable to new value (reset map)
    \item Mode-dependent \texttt{if}: Different dynamics in each mode
\end{itemize}

\begin{notebox}[title=Connection to Earlier Material]
The ground effect factor $\sigma(z)$ comes from the aerodynamic model in Chapter~\ref{ch:quadrotor-dynamics}. The PID controller structure matches the altitude controller discussed in the control design sections. This example shows how these components combine in a realistic hybrid system.
\end{notebox}

\section{Example: Flight Mode State Machine}

Let's model a simplified quadrotor flight mode system as a hybrid automaton.

\subsection{Hybrid Automaton Specification}

Following the formal definition:

\begin{itemize}
    \item $Q = \{\text{Disarmed}, \text{Armed}, \text{TakingOff}, \text{Hovering}, \text{Landing}, \text{Emergency}\}$
    \item $X = \mathbb{R}^{12}$ (full quadrotor state: position, velocity, orientation, angular velocity)
    \item $\text{Init} = \{(\text{Disarmed}, x) : z = 0, v = 0\}$ (start on ground, stationary)
\end{itemize}

\begin{center}
\begin{tabular}{ll}
\toprule
\textbf{Mode} & \textbf{Description} \\
\midrule
Disarmed & Motors off, waiting for arm command \\
Armed & Motors idle, ready for takeoff \\
TakingOff & Ascending to target altitude \\
Hovering & Maintaining position \\
Landing & Descending to ground \\
Emergency & Failsafe engaged, controlled descent \\
\bottomrule
\end{tabular}
\end{center}

\subsection{Domains}

\begin{align*}
D(\text{Disarmed}) &= \{x : z \geq z_{ground} - \delta\} \\
D(\text{Armed}) &= \{x : z \geq z_{ground} - \delta\} \\
D(\text{TakingOff}) &= \{x : z \geq z_{ground}\} \\
D(\text{Hovering}) &= \{x : z \geq z_{min}\} \quad \text{(minimum safe altitude)} \\
D(\text{Landing}) &= \{x : z \geq z_{ground} - \delta\} \\
D(\text{Emergency}) &= X \quad \text{(no restriction---emergency handles all states)}
\end{align*}

\subsection{Edge Set and Guards}

\textbf{Edge set}:
\begin{align*}
E = \{&(\text{Disarmed}, \text{Armed}), (\text{Armed}, \text{TakingOff}), (\text{TakingOff}, \text{Hovering}), \\
&(\text{Hovering}, \text{Landing}), (\text{Landing}, \text{Disarmed})\} \cup \{(q, \text{Emergency}) : q \in Q\}
\end{align*}

\textbf{Guard conditions}:
\begin{center}
\begin{tabular}{lll}
\toprule
\textbf{Transition} & \textbf{Guard $G$} & \textbf{Meaning} \\
\midrule
Disarmed $\to$ Armed & arm\_cmd $\land$ battery\_ok & Pilot arms, battery sufficient \\
Armed $\to$ TakingOff & throttle $>$ threshold & Pilot commands takeoff \\
TakingOff $\to$ Hovering & $|z - z_{target}| < \epsilon$ & Reached target altitude \\
Hovering $\to$ Landing & land\_cmd & Pilot commands landing \\
Landing $\to$ Disarmed & $z < z_{ground} + \delta$ & Touched down \\
$* \to$ Emergency & $\neg$signal\_ok $\lor$ battery\_critical & Failsafe triggered \\
\bottomrule
\end{tabular}
\end{center}

\subsection{Vector Fields}

Each mode has different dynamics $f(q, x)$. For brevity, we show only the thrust component:

\textbf{Disarmed/Armed}:
\[
T = 0, \quad \tau = 0
\]

\textbf{TakingOff}:
\[
T = mg + K_p(z_{target} - z) + K_d(0 - \dot{z})
\]

\textbf{Hovering}:
\[
T = mg + K_p(z_{cmd} - z) + K_d(\dot{z}_{cmd} - \dot{z}) + K_i \int (z_{cmd} - z) dt
\]

\textbf{Landing}:
\[
T = mg + K_p(z_{ground} - z) + K_d(v_{descent} - \dot{z})
\]

\textbf{Emergency}:
\[
T = T_{emergency} \quad \text{(fixed low thrust for controlled descent)}
\]

\subsection{Reset Maps}

Most transitions have identity reset $R(e, x) = x$ (position, velocity remain continuous). Exceptions:

\begin{align*}
R((\text{Armed}, \text{TakingOff}), x) &= x \text{ with } I := 0 \quad \text{(reset integrators)} \\
R((q, \text{Emergency}), x) &= x \text{ with } I := 0 \quad \text{(reset to failsafe state)}
\end{align*}

\section{Zeno Behavior}
\index{Zeno behavior}

\begin{warningbox}[title=The Zeno Problem]
A \textbf{Zeno execution}\index{Zeno behavior|textbf} is an infinite sequence of discrete transitions occurring in finite time.

\textbf{Quadrotor example}: Consider a landing controller that switches between GroundEffect and Contact modes based on altitude:
\begin{itemize}
    \item GroundEffect $\to$ Contact when $z \geq 0$
    \item Contact $\to$ GroundEffect when $z < 0$
\end{itemize}

If the quadrotor oscillates rapidly around $z = 0$ (e.g., due to aggressive control gains or insufficient damping), the mode switches become increasingly frequent. With proportional control and ground compliance, oscillation frequency can increase as amplitude decreases, potentially causing infinitely many switches in finite time.
\end{warningbox}

The name ``Zeno'' comes from the ancient Greek philosopher Zeno of Elea, who posed paradoxes about infinite divisibility. The mathematical resolution is that infinite sums can converge to finite values. But for simulation, this creates a real problem: the solver takes infinitely many steps without time advancing.

\textbf{Intuition}: Imagine a poorly-tuned altitude controller causing the quadrotor to ``chatter'' at the ground boundary---rapidly switching between Contact and GroundEffect modes. Each oscillation has smaller amplitude but higher frequency. If the controller gain is too high relative to the damping, the total time for infinitely many oscillations can be finite, causing the simulation to hang.

\subsection{Classic Example: Bouncing Ball}

The canonical Zeno example is a bouncing ball with coefficient of restitution $\alpha < 1$. Each bounce loses energy, so bounces become more frequent:
\[
t_i = \frac{2\alpha^i v_0}{g}, \quad T_\infty = \sum_{i=0}^{\infty} t_i = \frac{2v_0}{g} \cdot \frac{1}{1-\alpha} < \infty
\]

The ball bounces infinitely often in finite time---an absurd situation revealing a modeling flaw. The quadrotor landing scenario has the same mathematical structure when oscillating at the ground boundary.

\subsection{Why Zeno is a Problem}

\begin{itemize}
    \item \textbf{Mathematical}: The solution is undefined beyond the Zeno time
    \item \textbf{Numerical}: Simulation takes infinitely small timesteps, never progressing
    \item \textbf{Physical}: Real systems don't exhibit true Zeno behavior (there's always some regularizing effect)
\end{itemize}

\subsection{Detecting Zeno Behavior}

Warning signs in your model:
\begin{itemize}
    \item Simulation slows dramatically near certain states
    \item Event count increases without bound
    \item Time stops advancing despite events occurring
\end{itemize}

\subsection{Preventing Zeno Behavior}

\textbf{Strategy 1: Minimum dwell time}

Require a minimum time between mode switches:
\begin{lstlisting}[language=Pascal]
when z >= 0 and mode == 2 and time > lastSwitch + minDwell then
  mode := 3;  // Contact
  lastSwitch := time;
end when;
\end{lstlisting}

For quadrotor landing, a dwell time of 50--100 ms prevents chattering while still allowing responsive transitions.

\textbf{Strategy 2: Hysteresis in guard conditions}

Use different thresholds for entering and leaving a mode:
\begin{lstlisting}[language=Pascal]
// Enter Contact when clearly on ground
when z >= 0.005 and mode == 2 then mode := 3; end when;
// Return to GroundEffect only if clearly airborne
when z < -0.01 and mode == 3 then mode := 2; end when;
\end{lstlisting}

The 5 mm / 10 mm hysteresis band prevents rapid switching due to sensor noise or small oscillations.

\textbf{Strategy 3: Absorbing states}

Define conditions under which the system settles into a final mode:
\begin{lstlisting}[language=Pascal]
when abs(z) < 0.001 and abs(vz) < 0.01 and mode == 3 then
  mode := 4;  // Settled (absorbing state)
  reinit(vz, 0);
end when;
\end{lstlisting}

Once in the Settled state, no further transitions occur.

\textbf{Strategy 4: Compliant contact model (regularization)}

The landing hybrid automaton in Section~\ref{sec:landing-hybrid} already uses this approach: the Contact mode includes spring-damper dynamics ($k_g$, $c_g$) that physically damp oscillations, preventing Zeno behavior through the physics rather than artificial guards.

\begin{keyidea}[title=Best Practice for Quadrotor Landing]
Combine multiple strategies:
\begin{enumerate}
    \item Use compliant ground contact model (regularization)
    \item Add hysteresis to altitude-based guards (±5--10 mm)
    \item Include absorbing Settled state with velocity threshold
    \item Log mode transitions to detect chattering during testing
\end{enumerate}
\end{keyidea}

\section{Example: GPS-Denied Mode Switching}
\label{sec:gps-denied-hybrid}

A critical hybrid behavior in outdoor quadrotors is the transition between GPS-based position control and GPS-denied attitude-only control. This occurs when GPS signal quality degrades (urban canyons, interference, indoor flight).

\subsection{Hybrid Automaton Formulation}

Following the formal definition, we specify:

\begin{itemize}
    \item $Q = \{\text{PositionControl}, \text{AttitudeHold}\}$
    \item $X$: Full quadrotor state space $x = [p, v, q, \omega]^T$
    \item $\text{Init} = \{(\text{PositionControl}, x) : \gamma \geq \gamma_{low}\}$ (start with valid GPS)
    \item External signal: GPS quality metric $\gamma \in [0, 1]$
\end{itemize}

\textbf{Mode dynamics} (vector fields $f$):

\textbf{PositionControl}: Full 3D position tracking with GPS feedback
\[
\begin{aligned}
\phi_{cmd} &= K_p^{pos}(y_{target} - y) + K_d^{pos}(\dot{y}_{target} - \dot{y}) \\
\theta_{cmd} &= -K_p^{pos}(x_{target} - x) - K_d^{pos}(\dot{x}_{target} - \dot{x}) \\
T &= mg + K_p^z(z_{target} - z) + K_d^z(\dot{z}_{target} - \dot{z}) + K_i^z I_z
\end{aligned}
\]

\textbf{AttitudeHold}: Maintains level attitude, no position feedback
\[
\begin{aligned}
\phi_{cmd} &= 0, \quad \theta_{cmd} = 0 \quad \text{(level flight)} \\
T &= T_{manual} \quad \text{(pilot throttle input)}
\end{aligned}
\]

\textbf{Domains} (invariants):
\begin{align*}
D(\text{PositionControl}) &= \{x : \gamma \geq \gamma_{low}\} \\
D(\text{AttitudeHold}) &= \{x : \gamma \leq \gamma_{high}\}
\end{align*}
The overlapping domains (when $\gamma_{low} \leq \gamma \leq \gamma_{high}$) allow hysteresis: the system remains in its current mode until the guard for transition is triggered.

\textbf{Edge set}:
\[
E = \{(\text{PositionControl}, \text{AttitudeHold}), (\text{AttitudeHold}, \text{PositionControl})\}
\]

\textbf{Guard conditions} (with hysteresis):
\begin{center}
\begin{tabular}{ll}
\toprule
\textbf{Transition} & \textbf{Guard $G$} \\
\midrule
PositionControl $\to$ AttitudeHold & $\gamma < \gamma_{low} = 0.3$ \\
AttitudeHold $\to$ PositionControl & $\gamma > \gamma_{high} = 0.5$ \textbf{and} $|\dot{p}| < v_{max}$ \\
\bottomrule
\end{tabular}
\end{center}

The hysteresis ($\gamma_{low} \neq \gamma_{high}$) prevents rapid switching when GPS quality fluctuates near the threshold. The velocity check before re-entering PositionControl ensures the position controller doesn't receive a large initial error.

\textbf{Reset maps}:
\begin{align*}
R((\text{PositionControl}, \text{AttitudeHold}), x) &= x \quad \text{(identity)} \\
R((\text{AttitudeHold}, \text{PositionControl}), x) &= x \text{ with } I_z := 0, p_{target} := p_{current}
\end{align*}

The integrator reset and setpoint update prevent jump discontinuities in the control signal.

\begin{warningbox}[title=Safety Consideration]
GPS-denied transitions must be tested thoroughly:
\begin{itemize}
    \item What happens if GPS is lost during aggressive maneuvering?
    \item Can the pilot maintain control in AttitudeHold mode?
    \item Is there sufficient battery for return-to-home if GPS recovers?
\end{itemize}
Real flight controllers often trigger automatic landing or return-to-home when GPS is lost.
\end{warningbox}

\section{Example: Motor Saturation Hybrid Mode}
\label{sec:motor-saturation-hybrid}

Motor saturation occurs when the commanded thrust or torque exceeds what the motors can physically produce. This is particularly important during aggressive maneuvers or when carrying payloads near the thrust limit.

\subsection{Problem Description}

Each motor has physical limits:
\[
0 \leq \omega_i \leq \omega_{max} \quad \Rightarrow \quad 0 \leq T_i \leq T_{max} = b \cdot \omega_{max}^2
\]

When the controller commands a thrust combination that violates these limits, the actual thrust differs from the commanded thrust, potentially causing:
\begin{itemize}
    \item Altitude loss (if total thrust is limited)
    \item Attitude errors (if differential thrust is limited)
    \item Integrator windup (controller integrates an error it cannot correct)
\end{itemize}

\subsection{Hybrid Automaton Formulation}

Following the formal definition:

\begin{itemize}
    \item $Q = \{\text{Normal}, \text{Saturated}\}$
    \item $X$: Quadrotor state including integrator states $I = [I_\phi, I_\theta, I_\psi, I_z]^T$
    \item $\text{Init} = \{(\text{Normal}, x) : \forall i, \omega_{min} < \omega_i^{cmd} < \omega_{max}\}$
\end{itemize}

\textbf{Domains}:
\begin{align*}
D(\text{Normal}) &= \{x : \forall i, \omega_i^{cmd} \leq \omega_{max} \land \omega_i^{cmd} \geq \omega_{min}\} \\
D(\text{Saturated}) &= X \quad \text{(no restriction---saturated mode handles all states)}
\end{align*}

\textbf{Edge set}:
\[
E = \{(\text{Normal}, \text{Saturated}), (\text{Saturated}, \text{Normal})\}
\]

\textbf{Guard conditions}:
\begin{center}
\begin{tabular}{ll}
\toprule
\textbf{Transition} & \textbf{Guard} \\
\midrule
Normal $\to$ Saturated & $\exists i: \omega_i^{cmd} > \omega_{max}$ or $\omega_i^{cmd} < \omega_{min}$ \\
Saturated $\to$ Normal & $\forall i: \omega_{min} < \omega_i^{cmd} < \omega_{max}$ for duration $t_{dwell}$ \\
\bottomrule
\end{tabular}
\end{center}

\textbf{Mode-dependent control}:

\textbf{Normal mode}: Standard PID with integrator active
\[
\dot{I} = e \quad \text{(integrator accumulates error)}
\]

\textbf{Saturated mode}: Anti-windup active
\[
\dot{I} = 0 \quad \text{(integrator frozen)}
\]

Additionally, in Saturated mode, the controller may:
\begin{itemize}
    \item Prioritize attitude over altitude (sacrifice height to maintain level)
    \item Reduce position controller aggressiveness
    \item Alert the pilot or trigger return-to-home
\end{itemize}

\textbf{Reset maps}:
\begin{align*}
R((\text{Normal}, \text{Saturated}), x) &= x \quad \text{(identity---no reset)} \\
R((\text{Saturated}, \text{Normal}), x) &= x \quad \text{(identity---integrators retain value)}
\end{align*}
Note: Some implementations reset integrators when exiting saturation to prevent overshoot.

\subsection{Implementation Consideration}

Rather than discrete mode switching, many controllers implement \textbf{continuous anti-windup}:
\begin{lstlisting}[language=C, caption=Back-calculation anti-windup]
// Compute unconstrained motor commands
omega_cmd = inverse_mixing(T_cmd, tau_cmd);

// Apply saturation
omega_actual = clamp(omega_cmd, OMEGA_MIN, OMEGA_MAX);

// Back-calculate actual thrust/torque
[T_actual, tau_actual] = mixing(omega_actual);

// Anti-windup: reduce integrator based on saturation
for (int i = 0; i < 4; i++) {
    I[i] -= K_aw * (T_cmd - T_actual);  // Back-calculation
}
\end{lstlisting}

This continuous approach avoids discrete mode switching but achieves the same goal: preventing integrator windup when actuators saturate.

\begin{keyidea}[title=Hybrid vs. Continuous Anti-Windup]
The choice between discrete hybrid modes and continuous anti-windup depends on:
\begin{itemize}
    \item \textbf{Discrete}: Cleaner analysis, explicit mode-dependent behavior, easier to verify
    \item \textbf{Continuous}: Smoother transitions, no chattering risk, simpler implementation
\end{itemize}
Many production flight controllers use continuous anti-windup but model it as a hybrid system for analysis and verification.
\end{keyidea}

\section{Stability of Hybrid Systems}

\begin{warningbox}[title=Switching Can Destabilize Stable Systems]
Even if every individual mode is stable, switching between them can make the overall system unstable! Conversely, switching between unstable modes can sometimes produce a stable system. This section develops the mathematical framework for analyzing hybrid system stability.
\end{warningbox}

\subsection{Formal Definitions}

We consider switched systems of the form $\dot{x} = f_q(x)$ where $q \in Q$ is the active mode, with equilibrium at $x^* = 0$.

\begin{definition}[Stability of Hybrid Systems]
\index{stability!hybrid systems}
The equilibrium $x^* = 0$ of a hybrid system is:
\begin{itemize}
    \item \textbf{Stable}: For all $\epsilon > 0$, there exists $\delta > 0$ such that $\|x(0)\| < \delta$ implies $\|x(t)\| < \epsilon$ for all $t \geq 0$ and all switching signals
    \item \textbf{Asymptotically stable}: Stable and $\lim_{t \to \infty} x(t) = 0$
    \item \textbf{Globally asymptotically stable (GAS)}: Asymptotically stable for all initial conditions $x(0) \in \mathbb{R}^n$
\end{itemize}
\end{definition}

\textbf{Key distinction}: For hybrid systems, we must consider stability under \emph{all admissible switching signals}, not just a single trajectory. A system may be stable for some switching patterns but unstable for others.

\begin{definition}[Switching Signal]
A \textbf{switching signal} $\sigma: [0, \infty) \to Q$ specifies the active mode at each time. The switching times are $t_0 = 0 < t_1 < t_2 < \cdots$ where $\sigma$ changes value.
\end{definition}

\subsection{Common Lyapunov Function}

The strongest stability guarantee comes from finding a single Lyapunov function that works for all modes. For background on Lyapunov stability theory, see Khalil~\cite{khalil2002nonlinear}. The extension to switched systems is treated by Branicky~\cite{branicky1998multiple}.

\begin{theorem}[Common Lyapunov Function]
\index{Lyapunov function!common}
\label{thm:common-lyapunov}
Consider a switched system $\dot{x} = f_q(x)$ with modes $q \in Q$ and equilibrium $x^* = 0$. If there exists a continuously differentiable function $V: \mathbb{R}^n \to \mathbb{R}$ such that:
\begin{enumerate}
    \item $V(0) = 0$ and $V(x) > 0$ for all $x \neq 0$ \hfill (positive definite)
    \item $V(x) \to \infty$ as $\|x\| \to \infty$ \hfill (radially unbounded)
    \item $\dot{V}_q(x) := \nabla V(x)^T f_q(x) < 0$ for all $q \in Q$ and $x \neq 0$ \hfill (decreasing in all modes)
\end{enumerate}
Then the equilibrium $x = 0$ is \textbf{globally asymptotically stable under arbitrary switching}.
\end{theorem}

\begin{proof}[Proof sketch]
Since $V$ decreases along trajectories in every mode, $V(x(t))$ is monotonically decreasing regardless of when or how often switching occurs. By radial unboundedness and positive definiteness, $V(x(t)) \to 0$ implies $x(t) \to 0$.
\end{proof}

\begin{example}[Common Quadratic Lyapunov Function]
For linear modes $\dot{x} = A_q x$, we seek $V(x) = x^T P x$ with $P \succ 0$ such that:
\[
\dot{V}_q = x^T (A_q^T P + P A_q) x < 0 \quad \text{for all } q \in Q
\]
This requires $A_q^T P + P A_q \prec 0$ for all $q$---a set of Linear Matrix Inequalities (LMIs) that can be solved numerically.
\end{example}

\textbf{Limitation}: A common Lyapunov function often does not exist, even when the hybrid system is stable. We need more general tools.

\subsection{Multiple Lyapunov Functions}

When no common Lyapunov function exists, we can use mode-dependent Lyapunov functions with constraints on how they relate at switches.

\begin{definition}[Class-$\mathcal{K}$ Function]
A function $\alpha: [0, \infty) \to [0, \infty)$ is \textbf{class-$\mathcal{K}$} if it is continuous, strictly increasing, and $\alpha(0) = 0$.
\end{definition}

\begin{theorem}[Multiple Lyapunov Functions with Dwell Time]
\index{dwell time}
\index{Lyapunov function!multiple}
\label{thm:dwell-time}
Consider a switched system with modes $q \in Q$. Suppose each mode has a Lyapunov function $V_q: \mathbb{R}^n \to \mathbb{R}$ satisfying:
\begin{enumerate}
    \item \textbf{Bounds}: There exist class-$\mathcal{K}$ functions $\alpha_1, \alpha_2$ such that
    \[
    \alpha_1(\|x\|) \leq V_q(x) \leq \alpha_2(\|x\|) \quad \text{for all } q \in Q
    \]

    \item \textbf{Exponential decay}: There exists $\lambda > 0$ such that
    \[
    \dot{V}_q(x) \leq -\lambda V_q(x) \quad \text{for all } q \in Q, \; x \neq 0
    \]

    \item \textbf{Bounded jump at switches}: There exists $\mu \geq 1$ such that for any switch $q \to q'$:
    \[
    V_{q'}(x) \leq \mu \, V_q(x)
    \]
\end{enumerate}
Then the system is \textbf{globally asymptotically stable} if the \textbf{dwell time} $\tau_D$ between consecutive switches satisfies:
\begin{equation}
\boxed{\tau_D > \frac{\ln \mu}{\lambda}}
\label{eq:dwell-time}
\end{equation}
\end{theorem}

\begin{proof}[Proof sketch]
Between switches, $V_q$ decreases exponentially: $V_q(t) \leq V_q(t_k) e^{-\lambda(t-t_k)}$ where $t_k$ is the last switch time.

At a switch, $V$ may jump by factor $\mu$: $V_{q'}(t_k^+) \leq \mu V_q(t_k^-)$.

After dwell time $\tau_D$: $V_{q'}(t_k + \tau_D) \leq \mu V_q(t_k^-) e^{-\lambda \tau_D}$.

For overall decrease, we need $\mu e^{-\lambda \tau_D} < 1$, which gives $\tau_D > \frac{\ln \mu}{\lambda}$.
\end{proof}

\textbf{Physical interpretation}:
\begin{itemize}
    \item $\lambda$ is the \textbf{decay rate}---how fast energy dissipates within each mode
    \item $\mu$ is the \textbf{energy jump ratio}---worst-case energy increase at a switch
    \item $\tau_D$ is the \textbf{minimum dwell time}---how long the system must stay in each mode
\end{itemize}

The dwell time condition says: \emph{stay in each mode long enough to dissipate the energy that was (potentially) added at the last switch}.

\subsection{Example: Two Stable Modes, Unstable Hybrid System}
\label{sec:unstable-example}

We now analyze why stable modes can produce an unstable hybrid system.

Consider two linear systems:
\begin{align}
\text{Mode 1}: \quad \dot{x} &= A_1 x = \begin{bmatrix} -1 & -100 \\ 10 & -1 \end{bmatrix} x \\
\text{Mode 2}: \quad \dot{x} &= A_2 x = \begin{bmatrix} -1 & 10 \\ -100 & -1 \end{bmatrix} x
\end{align}

\textbf{Step 1: Verify each mode is stable.}

Both matrices have characteristic polynomial $\det(\lambda I - A_i) = (\lambda + 1)^2 + 1000$, giving eigenvalues $\lambda = -1 \pm j\sqrt{1000} \approx -1 \pm 31.6j$.

Since $\text{Re}(\lambda) = -1 < 0$, both modes are asymptotically stable with decay rate approximately $\lambda = 1$.

\textbf{Step 2: Find mode-specific Lyapunov functions.}

For each mode, we solve $A_i^T P_i + P_i A_i = -I$ to get:
\[
P_1 = \frac{1}{2} \begin{bmatrix} 1 & 0 \\ 0 & 1 \end{bmatrix} = \frac{1}{2}I, \quad
P_2 = \frac{1}{2} \begin{bmatrix} 1 & 0 \\ 0 & 1 \end{bmatrix} = \frac{1}{2}I
\]

So $V_1(x) = V_2(x) = \frac{1}{2}\|x\|^2$---the same Lyapunov function works for both modes!

\textbf{Step 3: Why is there no common Lyapunov function for stability under arbitrary switching?}

The issue is subtle. While $V = \frac{1}{2}\|x\|^2$ decreases in both modes, the \emph{rate} of decrease varies with direction. Along certain directions, the trajectory can ``pick up'' energy when switching.

Consider the state-dependent switching:
\begin{itemize}
    \item Mode 1 $\to$ Mode 2 when $x_2 = -0.2 x_1$ (switching surface $S_1$)
    \item Mode 2 $\to$ Mode 1 when $x_2 = 5 x_1$ (switching surface $S_2$)
\end{itemize}

\textbf{Step 4: Compute energy change along a switching cycle.}

Starting at $x_0 = [1, -0.2]^T$ on $S_1$:
\begin{enumerate}
    \item In Mode 2, the trajectory spirals and reaches $S_2$ at some $x_1 = [a, 5a]^T$
    \item Switch to Mode 1, trajectory spirals and returns to $S_1$ at $x_2 = [b, -0.2b]^T$
\end{enumerate}

Detailed calculation (omitted) shows $\|x_2\| > \|x_0\|$---energy has increased over one cycle.

\textbf{Geometric interpretation}: In Mode 1, trajectories spiral clockwise. In Mode 2, trajectories spiral counter-clockwise. The switching surfaces are positioned such that each switch ``resets'' the trajectory to a higher-energy orbit in the new mode.

\begin{figure}[htbp]
\centering
\begin{tikzpicture}[scale=1.5]
    % Axes
    \draw[->] (-2.2,0) -- (2.2,0) node[right] {$x_1$};
    \draw[->] (0,-2.2) -- (0,2.2) node[above] {$x_2$};

    % Switching surfaces
    \draw[dashed, blue, thick] (-2,-0.4) -- (2,0.4) node[right] {$S_1: x_2 = -0.2x_1$};
    \draw[dashed, red, thick] (-0.4,-2) -- (0.4,2) node[above right] {$S_2: x_2 = 5x_1$};

    % Spiral in Mode 1 (clockwise, from S1 toward S2)
    \draw[thick, blue, ->] (1.5, -0.3) arc[start angle=-11, end angle=70, radius=1.5];
    \node[blue] at (1.8, 0.8) {Mode 1};

    % Spiral in Mode 2 (counter-clockwise, from S2 toward S1)
    \draw[thick, red, ->] (0.25, 1.25) arc[start angle=78, end angle=160, radius=1.3];
    \node[red] at (-1.0, 1.3) {Mode 2};

    % Energy increase indication
    \draw[<->, thick, green!50!black] (1.5, -0.5) -- (1.8, -0.6);
    \node[green!50!black, below] at (1.65, -0.7) {$\|x\|$ grows};
\end{tikzpicture}
\caption{Phase portrait showing how switching between two stable modes can cause instability. Each mode is stable (spiraling inward), but the switching surfaces are positioned to increase energy at each transition.}
\label{fig:unstable-switching}
\end{figure}

\subsection{The Converse: Stabilization by Switching}

\begin{keyidea}[title=Stabilization by Switching]
Switching between \textbf{unstable} modes can produce a \textbf{stable} hybrid system, if the switches are timed correctly.

This is the basis of \textbf{switched-mode control}---a powerful technique used in power electronics, where transistors switch between configurations to achieve stable average behavior.
\end{keyidea}

\subsection{Quadrotor Application: Hovering to Landing Stability}
\label{sec:hover-land-stability}

We apply the dwell time theorem to analyze the Hovering $\to$ Landing mode transition.

\textbf{Altitude dynamics in each mode} (simplified to 1D):

\textbf{Hovering mode}: PID control to maintain $z = z_{ref}$
\[
m\ddot{z} = -K_p^H(z - z_{ref}) - K_d^H \dot{z} - K_i^H \int_0^t (z - z_{ref}) d\tau
\]

Ignoring the integral term for this analysis, with state $\xi = [z - z_{ref}, \dot{z}]^T$:
\[
\dot{\xi} = A_H \xi = \begin{bmatrix} 0 & 1 \\ -K_p^H/m & -K_d^H/m \end{bmatrix} \xi
\]

\textbf{Landing mode}: PD control to descend to $z = 0$
\[
m\ddot{z} = -K_p^L z - K_d^L \dot{z}
\]

With state $\eta = [z, \dot{z}]^T$:
\[
\dot{\eta} = A_L \eta = \begin{bmatrix} 0 & 1 \\ -K_p^L/m & -K_d^L/m \end{bmatrix} \eta
\]

\textbf{Step 1: Compute decay rates.}

For a second-order system with natural frequency $\omega_n$ and damping ratio $\zeta$:
\[
\omega_n^H = \sqrt{K_p^H/m}, \quad \zeta^H = \frac{K_d^H}{2\sqrt{K_p^H m}}
\]

The eigenvalues are $\lambda = -\zeta\omega_n \pm j\omega_n\sqrt{1-\zeta^2}$, so the decay rate is:
\[
\lambda_H = \zeta^H \omega_n^H = \frac{K_d^H}{2m}
\]

Similarly, $\lambda_L = K_d^L / (2m)$.

\textbf{Step 2: Compute energy jump at mode switch.}

Define quadratic Lyapunov functions:
\[
V_H(\xi) = \frac{1}{2}m\dot{z}^2 + \frac{1}{2}K_p^H(z - z_{ref})^2, \quad
V_L(\eta) = \frac{1}{2}m\dot{z}^2 + \frac{1}{2}K_p^L z^2
\]

At the switch (when landing command is issued at altitude $z_s$ with velocity $\dot{z}_s$):
\[
\mu = \frac{V_L(\eta^+)}{V_H(\xi^-)} = \frac{\frac{1}{2}m\dot{z}_s^2 + \frac{1}{2}K_p^L z_s^2}{\frac{1}{2}m\dot{z}_s^2 + \frac{1}{2}K_p^H(z_s - z_{ref})^2}
\]

If we switch when the quadrotor is at the hover setpoint ($z_s = z_{ref}$, $\dot{z}_s \approx 0$), then the energy ``before'' is nearly zero, and we must be careful. Instead, consider the setpoint change as a disturbance.

\textbf{Step 3: Practical design.}

For the Crazyflie with typical values:
\begin{center}
\begin{tabular}{lcc}
\toprule
\textbf{Parameter} & \textbf{Hovering} & \textbf{Landing} \\
\midrule
$K_p$ & 25000 (stiff) & 10000 (soft) \\
$K_d$ & 500 & 300 \\
$m$ & 0.03 kg & 0.03 kg \\
\midrule
$\omega_n$ & 28.9 rad/s & 18.3 rad/s \\
$\zeta$ & 0.29 & 0.27 \\
$\lambda$ & 8.3 s$^{-1}$ & 5.0 s$^{-1}$ \\
\bottomrule
\end{tabular}
\end{center}

If the worst-case energy jump ratio is $\mu = 1.5$ (from setpoint change and gain mismatch):
\[
\tau_D > \frac{\ln(1.5)}{5.0} = \frac{0.405}{5.0} = 0.081 \text{ s} = 81 \text{ ms}
\]

\begin{keyidea}[title=Design Rule for Mode Transitions]
For the Hovering $\to$ Landing transition:
\begin{enumerate}
    \item Ensure damping ratio $\zeta > 0.5$ in both modes for fast decay
    \item Use similar gains to minimize energy jump ratio $\mu$
    \item Ramp setpoints over 100--200 ms rather than stepping them
    \item Enforce minimum dwell time $\tau_D > 100$ ms between mode changes
\end{enumerate}
\end{keyidea}

\subsection{Stability Analysis Methods: Summary}

\begin{center}
\begin{tabular}{lp{5cm}p{5cm}}
\toprule
\textbf{Method} & \textbf{Condition} & \textbf{Guarantees} \\
\midrule
Common Lyapunov & Single $V(x)$ decreasing in all modes & GAS under arbitrary switching \\
\addlinespace
Multiple Lyapunov + Dwell Time & $V_q(x)$ per mode, $\tau_D > \frac{\ln\mu}{\lambda}$ & GAS if dwell time respected \\
\addlinespace
Average Dwell Time & Allows occasional fast switches if average is slow enough & GAS with some flexibility \\
\addlinespace
State-dependent & Analyze specific switching surfaces & Stability for given switching law \\
\bottomrule
\end{tabular}
\end{center}

\begin{notebox}[title=Practical Implication for Quadrotors]
When designing flight mode transitions:
\begin{itemize}
    \item Compute decay rates $\lambda$ from closed-loop eigenvalues
    \item Estimate energy jump ratio $\mu$ from gain changes and setpoint discontinuities
    \item Enforce $\tau_D > \frac{\ln\mu}{\lambda}$ in the state machine (typically 50--200 ms)
    \item Ramp setpoints to reduce $\mu$ toward 1
    \item Test edge cases: What happens with rapid mode cycling?
\end{itemize}
\end{notebox}

\subsection{Real-Time Considerations for Hybrid Stability}

In embedded flight controllers, hybrid system behavior interacts with real-time scheduling constraints. This connection between hybrid systems theory and real-time systems (covered in Module 3) is crucial for practical stability.

\textbf{Mode switching and task timing}:

Consider the Crazyflie's multi-rate control architecture:
\begin{itemize}
    \item Attitude control: 500 Hz (2 ms period)
    \item Position control: 100 Hz (10 ms period)
    \item State estimation: 1000 Hz (1 ms period)
\end{itemize}

A mode switch (e.g., Hovering $\to$ Landing) may occur:
\begin{enumerate}
    \item Between attitude control executions
    \item During a position control computation
    \item While sensor data is being processed
\end{enumerate}

\textbf{Timing-induced instability}: If mode switching is not synchronized with control task execution, transient inconsistencies can occur:
\begin{itemize}
    \item Position controller computes setpoint for old mode
    \item Mode switch occurs
    \item Attitude controller uses new mode gains with old setpoint
\end{itemize}

\textbf{Dwell time and control period}: The minimum dwell time $\tau_{dwell}$ for stability must satisfy:
\[
\tau_{dwell} \geq \max(T_{position}, T_{estimation}) + \epsilon
\]
where $T_{position}$ is the position control period and $\epsilon$ accounts for worst-case execution time variability.

For the Crazyflie: $\tau_{dwell} \geq 10\text{ ms} + 2\text{ ms} = 12\text{ ms}$ minimum.

\textbf{Jitter effects on guard conditions}: Consider a guard condition $z < z_{threshold}$. With sensor jitter $\sigma_z$:
\begin{itemize}
    \item Apparent threshold crossings may occur due to noise
    \item Hysteresis band should exceed $3\sigma_z$ to prevent false triggers
    \item Mode switch latency adds to effective guard uncertainty
\end{itemize}

\begin{keyidea}[title=Integrating Hybrid and Real-Time Analysis]
A complete stability argument for an embedded flight controller must consider:
\begin{enumerate}
    \item \textbf{Hybrid stability}: Lyapunov analysis for mode-dependent dynamics
    \item \textbf{Dwell time}: Minimum time between switches for energy dissipation
    \item \textbf{Task scheduling}: Mode switches synchronized with control periods
    \item \textbf{Timing jitter}: Guard condition hysteresis exceeds timing uncertainty
\end{enumerate}
Module 3 provides the real-time analysis tools; this chapter provides the hybrid systems framework. Together they enable rigorous analysis of flight controller stability.
\end{keyidea}

\section{Verification with Signal Temporal Logic}
\label{sec:stl-hybrid}

Signal Temporal Logic (STL) provides a formal language for specifying and verifying hybrid system requirements. This connects to the testing and verification methods in Module 4.

\subsection{STL Specifications for Flight Modes}

\textbf{Safety property} (always maintain safe attitude):
\[
\Box \left( |\phi| < 45° \land |\theta| < 45° \right)
\]
``At all times, roll and pitch remain within 45 degrees.''

\textbf{Liveness property} (eventually land after command):
\[
\Box \left( \text{land\_cmd} \rightarrow \Diamond_{[0, 30s]} (\text{mode} = \text{Disarmed} \land z > -0.1) \right)
\]
``Whenever land command is issued, within 30 seconds the quadrotor is disarmed and on the ground.''

\textbf{Mode sequencing} (correct landing sequence):
\[
\Box \left( \text{mode} = \text{Landing} \rightarrow \neg(\text{mode} = \text{TakingOff}) \; \mathcal{U} \; (\text{mode} = \text{Disarmed}) \right)
\]
``Once in Landing mode, don't return to TakingOff before reaching Disarmed.''

\textbf{Dwell time constraint}:
\[
\Box \left( \text{mode\_change} \rightarrow \Box_{[0, 50ms]} \neg\text{mode\_change} \right)
\]
``After any mode change, no other mode change occurs for at least 50 ms.''

\subsection{Robustness Metrics}

STL specifications have quantitative \textbf{robustness} values indicating how strongly a property is satisfied:
\begin{itemize}
    \item $\rho > 0$: Property satisfied, with margin $\rho$
    \item $\rho < 0$: Property violated, by margin $|\rho|$
    \item $\rho = 0$: Boundary case
\end{itemize}

For the attitude safety property:
\[
\rho = \min_t \left( \min(45° - |\phi(t)|, 45° - |\theta(t)|) \right)
\]

A robustness of $\rho = 10°$ means the quadrotor always stayed at least 10° away from the attitude limits---a comfortable margin. A robustness of $\rho = 2°$ suggests the system is close to violating safety.

\begin{notebox}[title=Connection to Module 4]
Module 4 covers falsification-based testing using STL specifications. The hybrid system model from this chapter, combined with STL requirements, enables systematic testing:
\begin{enumerate}
    \item Define STL specifications for all flight modes and transitions
    \item Use falsification tools (e.g., Breach, S-TaLiRo) to search for violations
    \item Analyze robustness to identify weak points in the design
\end{enumerate}
\end{notebox}

\section{Implementation in Stateflow}

Stateflow is MATLAB's tool for designing state machines, with native support for hybrid systems. It integrates with Simulink for continuous dynamics and supports code generation for embedded targets.

\subsection{Key Concepts}

\begin{itemize}
    \item \textbf{States}: Modes with entry/during/exit actions
    \item \textbf{Transitions}: Arrows with guard conditions and actions
    \item \textbf{Continuous dynamics}: \texttt{du:} syntax for ODEs within states
    \item \textbf{Events}: Trigger transitions based on signals crossing thresholds
    \item \textbf{Hierarchy}: States can contain substates for complex logic
\end{itemize}

\subsection{Crazyflie Flight Controller State Machine}

We implement a complete flight mode state machine for the Crazyflie quadrotor, matching the modes used in real firmware:

See Listing~\ref{lst:stateflow-flight} (Appendix~\ref{app:code-listings}) for the complete Stateflow implementation of the Crazyflie flight mode state machine, including Disarmed, Armed, TakingOff, Hovering, Landing, and Emergency states with all transitions and guards.

\textbf{Key design decisions}:
\begin{itemize}
    \item \textbf{Arm safety}: Requires low throttle and minimum battery voltage
    \item \textbf{Takeoff delay}: 1 second after arming prevents accidental takeoff
    \item \textbf{Low battery triggers landing}: Automatic at 3.3V, emergency at 2.9V
    \item \textbf{Attitude limits}: Emergency if roll/pitch exceeds 60° (loss of control)
    \item \textbf{Emergency timeout}: Auto-disarm after 10 seconds in emergency
\end{itemize}

\subsection{Best Practices for Flight Controllers}

\begin{enumerate}
    \item \textbf{Explicit state enumeration}: Define all modes clearly
    \item \textbf{Default transitions}: Every state should have an outgoing path
    \item \textbf{Guard completeness}: Ensure guards cover all cases (use \texttt{else})
    \item \textbf{Failsafe priority}: Emergency transitions should have highest priority
    \item \textbf{Logging}: Record mode transitions for debugging
\end{enumerate}

\section{Chapter Summary}

Hybrid systems combine continuous dynamics with discrete mode switching. This chapter developed the theory using quadrotor-specific examples throughout:

\textbf{Core concepts}:
\begin{itemize}
    \item \textbf{Hybrid automata}: Formal model $(Q, X, f, \text{Init}, D, E, G, R)$ with modes $Q$, continuous state $X$, vector fields $f$, initial states Init, domains $D$, edges $E$, guards $G$, and reset maps $R$
    \item \textbf{Zeno behavior}: Infinite mode switches in finite time---prevented through hysteresis, dwell times, and regularization
    \item \textbf{Stability}: Requires analysis beyond individual mode stability; dwell time conditions ensure convergence
\end{itemize}

\textbf{Quadrotor applications developed in this chapter}:
\begin{itemize}
    \item \textbf{Landing sequence} (Section~\ref{sec:landing-hybrid}): Four-mode hybrid automaton with ground effect, contact dynamics, and integrator resets
    \item \textbf{GPS-denied switching} (Section~\ref{sec:gps-denied-hybrid}): Hysteresis-based mode transitions for position control loss
    \item \textbf{Motor saturation} (Section~\ref{sec:motor-saturation-hybrid}): Anti-windup as hybrid mode or continuous compensation
    \item \textbf{Flight controller state machine}: Complete Crazyflie implementation with safety guards
\end{itemize}

\textbf{Connections to other modules}:
\begin{itemize}
    \item \textbf{Module 1}: Ground effect model, motor dynamics, and sensor fusion inform mode-dependent dynamics
    \item \textbf{Module 3}: Real-time scheduling constraints affect dwell times and guard condition timing
    \item \textbf{Module 4}: STL specifications enable formal verification of hybrid properties
\end{itemize}

The hybrid systems framework provides the foundation for designing robust flight software that handles the inevitable discrete events in real-world operation: mode changes, sensor failures, actuator saturation, and ground contact.


\chapter{Numerical Simulation}
\index{numerical simulation}

\section{Introduction: Why Simulation Matters}

Before flying a real quadrotor with new control software, you simulate it. This is not optional---it is how engineers develop complex systems safely and efficiently.

\textbf{The fundamental problem}: We have differential equations describing the quadrotor dynamics, but most differential equations cannot be solved analytically. Even when analytical solutions exist (like for simple linear systems), they are often too complex to be useful. Numerical simulation computes approximate solutions by stepping forward in time.

\textbf{Why simulation matters}:
\begin{itemize}
    \item \textbf{Find bugs safely}: A crash in simulation costs nothing; a crash in reality can destroy expensive hardware or injure people
    \item \textbf{Test edge cases}: Low battery, sensor failure, extreme maneuvers---scenarios that are dangerous or impractical to test on real hardware
    \item \textbf{Iterate quickly}: Simulation runs faster than real-time; you can test hundreds of scenarios overnight
    \item \textbf{Validate models}: Compare simulation predictions to flight data to verify that your models capture the essential physics
    \item \textbf{Regulatory compliance}: Safety-critical systems require evidence that software behaves correctly across operating conditions
\end{itemize}

\begin{keyidea}[title=The Simulation Workflow]
\begin{enumerate}
    \item \textbf{Model-in-the-Loop (MIL)}: Test control algorithms against plant models in continuous time
    \item \textbf{Software-in-the-Loop (SIL)}: Test actual C code against plant models
    \item \textbf{Hardware-in-the-Loop (HIL)}: Test embedded hardware against plant models (real-time)
    \item \textbf{Flight testing}: Test everything against reality
\end{enumerate}
Each stage catches different types of errors. Most debugging happens in MIL/SIL.
\end{keyidea}

\section{Numerical Integration of ODEs}

\subsection{The Problem}

Given an initial value problem (IVP):
\[
\dot{x} = f(x, t), \quad x(0) = x_0
\]

Find $x(t)$ for $t \in [0, T]$.

\begin{notebox}[title=Assumption: Existence and Uniqueness]
We assume the ODE has a unique solution. This requires $f$ to be Lipschitz continuous (roughly: bounded derivative). Physical systems generally satisfy this, but numerical models can violate it (e.g., division by zero).
\end{notebox}

\subsection{Exact Solution (Usually Impossible)}

The exact solution satisfies:
\[
x(t) = x_0 + \int_0^t f(x(\tau), \tau) \, d\tau
\]

For most nonlinear systems (including quadrotors), this integral cannot be computed analytically.

\subsection{Numerical Approximation}

Instead of the exact solution, we compute approximate values at discrete times:
\[
x_0, x_1, x_2, \ldots, x_N \quad \text{where} \quad x_k \approx x(t_k)
\]

The art is making this approximation accurate and efficient.

\section{Euler's Method}
\index{Euler method}

The simplest numerical integration approach: assume the derivative is constant over each timestep.

\textbf{Historical note}: Leonhard Euler introduced this method in the 18th century. Despite its simplicity (and limitations), understanding Euler's method is essential because: (1) more sophisticated methods build on the same idea, and (2) it illustrates the fundamental tradeoffs in numerical integration.

\subsection{Derivation}

The idea is to approximate the continuous solution by discrete steps. Partition $[0, T]$ into $N$ intervals of length $h = T/N$:
\[
t_k = kh, \quad k = 0, 1, \ldots, N
\]

From the fundamental theorem of calculus:
\[
x(t_{k+1}) = x(t_k) + \int_{t_k}^{t_{k+1}} f(x(\tau), \tau) \, d\tau
\]

The integral is unknown (that is what we are trying to compute!), so we approximate it. The simplest approximation uses a rectangle with height $f(x(t_k), t_k)$:
\[
\int_{t_k}^{t_{k+1}} f(x(\tau), \tau) \, d\tau \approx h \cdot f(x(t_k), t_k)
\]

This gives Euler's method:
\[
x_{k+1} = x_k + h \cdot f(x_k, t_k)
\]

\subsection{Algorithm}

\begin{lstlisting}[language=C, caption=Euler's method]
x = x0;
t = 0;
while (t < T) {
    x = x + h * f(x, t);  // Euler step
    t = t + h;
}
\end{lstlisting}

\subsection{Geometric Interpretation}

At each step, we follow the tangent line for time $h$. The solution is approximated by a piecewise linear function.

\subsection{Error Analysis}

\textbf{Local truncation error} (error in one step):
\[
\epsilon_{local} = O(h^2)
\]

\textbf{Global error} (accumulated over $N = T/h$ steps):
\[
\epsilon_{global} = O(h)
\]

Euler's method is \textbf{first-order accurate}: halving $h$ halves the error (but doubles computation time).

\begin{warningbox}[title=Euler's Method is Rarely Sufficient]
For quadrotor simulation:
\begin{itemize}
    \item Attitude dynamics are fast (eigenvalues with large imaginary parts)
    \item Position dynamics are slower
    \item The disparity requires either very small $h$ (slow) or a better method
\end{itemize}
Use Euler only for quick prototyping, never for final validation.
\end{warningbox}

\section{Runge-Kutta Methods}
\index{Runge-Kutta method}

Runge-Kutta methods~\cite{butcher2016numerical} improve accuracy by evaluating $f$ at multiple points within each timestep.

\subsection{The Idea}

Instead of:
\[
x_{k+1} = x_k + h \cdot f(x_k, t_k)
\]

Use:
\[
x_{k+1} = x_k + h \cdot \sum_{i=1}^{s} b_i k_i
\]

where $k_i$ are evaluations of $f$ at carefully chosen points.

\subsection{Fourth-Order Runge-Kutta (RK4)}

The most famous method:

\begin{align}
k_1 &= f(x_k, t_k) \\
k_2 &= f(x_k + \frac{h}{2}k_1, t_k + \frac{h}{2}) \\
k_3 &= f(x_k + \frac{h}{2}k_2, t_k + \frac{h}{2}) \\
k_4 &= f(x_k + h \cdot k_3, t_k + h) \\
x_{k+1} &= x_k + \frac{h}{6}(k_1 + 2k_2 + 2k_3 + k_4)
\end{align}

\textbf{Interpretation}:
\begin{itemize}
    \item $k_1$: Slope at the beginning
    \item $k_2$: Slope at midpoint, using $k_1$ to get there
    \item $k_3$: Slope at midpoint, using $k_2$ to get there
    \item $k_4$: Slope at the end, using $k_3$ to get there
    \item Final step: Weighted average (Simpson's rule)
\end{itemize}

\subsection{Error and Efficiency}

\textbf{Local error}: $O(h^5)$

\textbf{Global error}: $O(h^4)$

RK4 is \textbf{fourth-order accurate}: halving $h$ reduces error by factor of 16.

\textbf{Cost}: 4 function evaluations per step (vs. 1 for Euler).

\begin{keyidea}[title=RK4 Trade-off]
RK4 with step size $h$ has similar accuracy to Euler with step size $h/16$, but RK4 is much faster because:
\[
\frac{\text{Euler work}}{\text{RK4 work}} = \frac{16 \text{ steps} \times 1 \text{ eval}}{1 \text{ step} \times 4 \text{ evals}} = 4
\]
For smooth problems, higher-order methods win decisively.
\end{keyidea}

\section{Variable-Step Methods}

\subsection{Motivation}

Fixed-step methods waste effort:
\begin{itemize}
    \item When dynamics are slow, large steps suffice
    \item When dynamics are fast, small steps are needed
    \item Optimal step size varies over the simulation
\end{itemize}

\subsection{Adaptive Step Size Control}

\textbf{Basic idea}: Estimate the error at each step and adjust $h$ to maintain a target accuracy.

\textbf{Error estimation}: Compute the step with two methods of different orders. The difference estimates the error.

\textbf{Step adjustment}:
\begin{itemize}
    \item If error $<$ tolerance: Accept step, try larger $h$ next time
    \item If error $>$ tolerance: Reject step, retry with smaller $h$
\end{itemize}

\subsection{Common Variable-Step Methods}

\begin{center}
\begin{tabular}{lll}
\toprule
\textbf{MATLAB Name} & \textbf{Method} & \textbf{Best For} \\
\midrule
\texttt{ode45} & Dormand-Prince (RK 4/5) & Most problems (default) \\
\texttt{ode23} & Bogacki-Shampine (RK 2/3) & Rough solutions, events \\
\texttt{ode113} & Adams-Bashforth-Moulton & Smooth, expensive $f$ \\
\texttt{ode15s} & BDF (implicit) & Stiff problems \\
\texttt{ode23t} & Trapezoidal (implicit) & Moderately stiff, DAEs \\
\bottomrule
\end{tabular}
\end{center}

\subsection{What is ``Stiff''?}
\index{stiff ODE}

For a comprehensive treatment of stiff ODEs and implicit methods, see Hairer and Wanner~\cite{hairer1996solving}.

\begin{definition}[Stiff ODE]
A system is \textbf{stiff}\index{stiff ODE|textbf} if it has dynamics on vastly different timescales---some very fast, some very slow.
\end{definition}

\textbf{Intuition}: Imagine a pot of water on a stove. Initially, the water temperature changes rapidly as heat transfers from the burner. But eventually, the water reaches boiling point and the temperature becomes nearly constant. A stiff system has both ``transient'' (fast) and ``steady-state'' (slow) behaviors coexisting.

\textbf{The problem with explicit methods}: Stability requires the step size $h$ to satisfy $|1 + h\lambda| < 1$ for all eigenvalues $\lambda$ of the linearized system. For stiff systems, $\lambda_{fast}$ has a large magnitude, forcing $h$ to be small. But once the fast dynamics have settled, we are taking tiny steps just to remain stable, even though nothing interesting is happening at that timescale.

\textbf{Example}: Quadrotor motor electrical dynamics ($\tau_e = L/R \sim 0.1$ ms) coupled with position dynamics ($\tau_p \sim 1$ s). The ratio is $10^4$---an explicit solver would need $10^4$ steps per slow time constant just for stability.

\textbf{Solution}: Implicit methods (ode15s, ode23t) are \textbf{unconditionally stable}---they remain stable for any step size $h$. This comes at the cost of solving a nonlinear equation at each step, but for stiff systems, this cost is far less than taking $10^4$ explicit steps.

\begin{notebox}[title=Practical Guidance for Quadrotors]
\begin{itemize}
    \item Start with \texttt{ode45} (Simulink: \texttt{ode45 (Dormand-Prince)})
    \item If simulation is very slow, try \texttt{ode15s} or \texttt{ode23t}
    \item If you need events (zero-crossings), \texttt{ode23} or \texttt{ode45} work well
    \item For code generation, you'll need fixed-step (see below)
\end{itemize}
\end{notebox}

\section{Fixed-Step Methods for Real-Time}

\subsection{Why Fixed-Step?}

Variable-step methods are optimal for offline simulation but problematic for:
\begin{itemize}
    \item \textbf{Real-time simulation}: Must complete each step in bounded time
    \item \textbf{Code generation}: Embedded targets need predictable execution
    \item \textbf{HIL testing}: Synchronization with hardware requires fixed timing
\end{itemize}

\subsection{Choosing Fixed-Step Size}

The step size must be small enough for stability and accuracy:

\begin{itemize}
    \item \textbf{Stability}: $h < h_{max}$ where $h_{max}$ depends on the fastest eigenvalue
    \item \textbf{Accuracy}: $h$ small enough that errors are acceptable
\end{itemize}

\textbf{Rule of thumb}: For a system with fastest time constant $\tau_{min}$:
\[
h \leq \frac{\tau_{min}}{5} \quad \text{(for RK4)}
\]

\subsection{Fixed-Step Methods in Simulink}

\begin{center}
\begin{tabular}{ll}
\toprule
\textbf{Solver} & \textbf{Description} \\
\midrule
\texttt{ode1} & Euler (1st order) \\
\texttt{ode2} & Heun (2nd order) \\
\texttt{ode3} & Bogacki-Shampine (3rd order) \\
\texttt{ode4} & RK4 (4th order) \\
\texttt{ode5} & Dormand-Prince (5th order) \\
\texttt{ode8} & 8th order \\
\bottomrule
\end{tabular}
\end{center}

\begin{keyidea}[title=Recommended Practice]
\begin{enumerate}
    \item Develop and debug with variable-step (\texttt{ode45})
    \item Validate with fixed-step (\texttt{ode4}) at target sample rate
    \item Generate code with fixed-step solver
\end{enumerate}
If fixed-step results differ significantly from variable-step, reduce $h$.
\end{keyidea}

\section{Zero-Crossing Detection}
\label{sec:zero-crossing}

Simulating hybrid systems requires detecting when continuous states cross thresholds that trigger discrete transitions. This \emph{event detection} problem connects the continuous ODE solver to the discrete mode-switching logic of the hybrid automaton.

\subsection{Mathematical Formulation}

\begin{definition}[Zero-Crossing Function]
A \textbf{zero-crossing function} (or event function) is a smooth function $g: X \to \mathbb{R}$ that defines an event surface. An \textbf{event} occurs at time $t^*$ when:
\[
g(x(t^*)) = 0
\]
The event is a \textbf{crossing} if $g$ changes sign, i.e., $g(x(t^* - \epsilon)) \cdot g(x(t^* + \epsilon)) < 0$ for small $\epsilon > 0$.
\end{definition}

\textbf{Connection to hybrid automata}: Zero-crossing functions encode guards and domain boundaries from Section~\ref{sec:hybrid-automaton-def}:
\begin{itemize}
    \item \textbf{Guard} $G(e) = \{x : g_e(x) \leq 0\}$: Transition $e$ enabled when $g_e(x) \leq 0$
    \item \textbf{Domain boundary} $\partial D(q) = \{x : d_q(x) = 0\}$: Must exit mode $q$ when $d_q(x) = 0$
\end{itemize}

For the landing sequence example, the ground contact event has:
\[
g_{\text{contact}}(x) = -z \quad \text{(event when } z = 0 \text{, i.e., altitude reaches ground)}
\]

\subsection{The Event Detection Problem}

Given an ODE $\dot{x} = f(x)$ and zero-crossing function $g$, the solver computes $x(t_k)$ at discrete times. Between steps, the state evolves continuously but is not directly observed.

\textbf{Problem}: The solver computes $x(t_k)$ with $g(x(t_k)) > 0$ and $x(t_{k+1})$ with $g(x(t_{k+1})) < 0$. Find $t^* \in (t_k, t_{k+1})$ such that $g(x(t^*)) = 0$.

\begin{example}[Landing Detection]
During quadrotor landing with $\dot{z} = v_z$ and $\dot{v}_z = g - T/m$:
\begin{itemize}
    \item At $t_k = 1.000$ s: $z(t_k) = -0.02$ m (2 cm above ground), so $g(x(t_k)) = 0.02 > 0$
    \item At $t_{k+1} = 1.010$ s: $z(t_{k+1}) = +0.01$ m (1 cm ``into'' ground), so $g(x(t_{k+1})) = -0.01 < 0$
\end{itemize}
The contact event occurred at some $t^* \in (1.000, 1.010)$ s. We must find $t^*$ precisely to apply the correct reset and switch to Contact mode.
\end{example}

\textbf{Why precision matters}: If we simply detect the event at $t_{k+1}$ and apply the reset there:
\begin{itemize}
    \item The state $z = +0.01$ m is physically impossible (quadrotor inside ground)
    \item Energy is not conserved correctly
    \item Subsequent dynamics start from an incorrect state
    \item With stiff ground models, this can cause numerical instability
\end{itemize}

\subsection{Root-Finding Algorithms}

Event detection reduces to finding a root of $\phi(t) = g(x(t))$ on the interval $[t_k, t_{k+1}]$. Since evaluating $\phi(t)$ requires integrating the ODE from $t_k$ to $t$, efficiency is important.

\subsubsection{Bisection Method}

The simplest robust method is bisection, which repeatedly halves the interval containing the root.

\begin{lstlisting}[language=Python, caption=Bisection for event localization]
def locate_event(t_low, t_high, x_low, g, f, tol):
    """Find t* where g(x(t*)) = 0 using bisection.

    Args:
        t_low, t_high: Bracket with g(x(t_low)) > 0, g(x(t_high)) < 0
        x_low: State at t_low
        g: Zero-crossing function
        f: ODE right-hand side
        tol: Time tolerance for event localization
    """
    g_low = g(x_low)  # Known to be positive

    while (t_high - t_low) > tol:
        t_mid = (t_low + t_high) / 2
        x_mid = integrate_ode(f, x_low, t_low, t_mid)
        g_mid = g(x_mid)

        if g_mid * g_low > 0:  # Same sign: root in upper half
            t_low, x_low, g_low = t_mid, x_mid, g_mid
        else:  # Different sign: root in lower half
            t_high = t_mid

    return (t_low + t_high) / 2, x_mid
\end{lstlisting}

\textbf{Convergence}: After $n$ iterations, the interval width is $(t_{k+1} - t_k)/2^n$. To achieve tolerance $\epsilon_t$:
\[
n = \left\lceil \log_2 \frac{t_{k+1} - t_k}{\epsilon_t} \right\rceil
\]

For a typical step size of 10 ms and tolerance of 1 $\mu$s, this requires $\lceil \log_2(10000) \rceil = 14$ iterations.

\subsubsection{Regula Falsi (False Position)}

Regula falsi uses linear interpolation instead of midpoint selection:
\[
t_{new} = t_{low} - g_{low} \cdot \frac{t_{high} - t_{low}}{g_{high} - g_{low}}
\]

This converges faster than bisection when $g$ is approximately linear, but can stagnate if $g$ is highly curved. The \textbf{Illinois algorithm} modifies regula falsi to guarantee superlinear convergence by halving function values when one endpoint persists.

\subsubsection{Brent's Method}

Production-quality solvers typically use \textbf{Brent's method}, which combines:
\begin{itemize}
    \item Bisection (guaranteed convergence)
    \item Secant method (fast convergence when applicable)
    \item Inverse quadratic interpolation (even faster for smooth functions)
\end{itemize}

Brent's method automatically selects the best approach at each iteration, achieving superlinear convergence while maintaining the robustness of bisection.

\begin{center}
\begin{tabular}{llll}
\toprule
\textbf{Method} & \textbf{Convergence Rate} & \textbf{Robustness} & \textbf{Iterations (typical)} \\
\midrule
Bisection & Linear, $O(\log(1/\epsilon))$ & Very high & 10--20 \\
Regula Falsi & Superlinear & Medium & 5--15 \\
Illinois & Superlinear & High & 4--10 \\
Brent & Superlinear & Very high & 4--8 \\
\bottomrule
\end{tabular}
\end{center}

\subsection{Edge Cases and Difficulties}

Several situations complicate event detection beyond the simple sign-change case.

\subsubsection{Grazing Contact}

A \textbf{grazing} (or tangent) event occurs when the trajectory touches the event surface without crossing:
\[
g(x(t^*)) = 0 \quad \text{and} \quad \dot{g}(x(t^*)) = \frac{d}{dt}g(x(t)) \big|_{t=t^*} = 0
\]

\begin{example}[Quadrotor Near-Miss]
A quadrotor descending with $v_z > 0$ (downward in NED) might have its descent arrested by increased thrust just as $z$ approaches zero. If the minimum altitude is exactly $z = 0$, this is a grazing contact---the quadrotor touches down instantaneously with zero vertical velocity.
\end{example}

\textbf{Numerical difficulty}: Since $g$ doesn't change sign, standard detection fails. Even if detected, the post-event behavior is ambiguous (does the trajectory continue in the same mode or switch?).

\textbf{Mitigation}:
\begin{itemize}
    \item Check for $|g| < \epsilon$ even without sign change
    \item Use domain constraints: if $D(q) = \{x : g(x) \geq 0\}$, then $g < 0$ forces a transition regardless of sign change
    \item Physical regularization: add slight damping or compliance to prevent exact tangency
\end{itemize}

\subsubsection{Simultaneous Events}

When multiple zero-crossing functions reach zero at the same instant (or within numerical tolerance), the order of event handling affects the outcome.

\begin{example}[Landing with Multiple Guards]
During landing, these conditions might become true simultaneously:
\begin{enumerate}
    \item Ground contact: $g_1 = -z = 0$
    \item Velocity threshold: $g_2 = |v_z| - v_{thresh} = 0$
    \item Battery critical: $g_3 = V_{batt} - V_{crit} = 0$
\end{enumerate}
\end{example}

\textbf{Resolution strategies}:
\begin{itemize}
    \item \textbf{Priority ordering}: Assign priorities to events; handle highest priority first (Stateflow approach)
    \item \textbf{Deterministic tie-breaking}: Use consistent ordering (e.g., by guard index)
    \item \textbf{Simultaneous reset}: Apply all applicable resets in defined order
\end{itemize}

The hybrid automaton semantics should specify event priorities to ensure deterministic simulation.

\subsubsection{Chattering (Rapid Successive Events)}

\textbf{Chattering} occurs when the system rapidly switches back and forth across an event surface, potentially approaching Zeno behavior.

\begin{example}[Altitude Controller Chatter]
A poorly-tuned altitude controller near the ground might cause:
\begin{enumerate}
    \item $z$ crosses 0 from above $\to$ switch to Contact mode
    \item Ground reaction force pushes quadrotor up
    \item $z$ crosses 0 from below $\to$ switch back to GroundEffect mode
    \item Cycle repeats with decreasing amplitude but increasing frequency
\end{enumerate}
\end{example}

\textbf{Detection}: Monitor the number of events per unit time. Simulink limits consecutive zero-crossings (default: 1000) and issues a warning.

\textbf{Mitigation}:
\begin{itemize}
    \item Add hysteresis to guards (different thresholds for each direction)
    \item Use dwell-time constraints (minimum time in each mode)
    \item Regularize the physical model (compliant contact instead of rigid switching)
\end{itemize}

\subsubsection{Sliding Modes}

A \textbf{sliding mode} occurs when the vector fields on both sides of a switching surface point toward the surface, trapping the trajectory on the boundary.

Formally, at the surface $\{x : g(x) = 0\}$ with modes $q_1$ (where $g > 0$) and $q_2$ (where $g < 0$):
\[
\nabla g \cdot f_{q_1}(x) < 0 \quad \text{and} \quad \nabla g \cdot f_{q_2}(x) > 0
\]

The trajectory cannot leave the surface in either direction.

\textbf{Resolution}: Filippov's solution defines the sliding dynamics as a convex combination:
\[
\dot{x} = \alpha f_{q_1}(x) + (1-\alpha) f_{q_2}(x)
\]
where $\alpha \in [0,1]$ is chosen so that $\dot{g} = 0$ (trajectory stays on surface).

\begin{notebox}[title=Sliding Modes in Quadrotors]
Sliding modes are uncommon in well-designed quadrotor controllers but can occur with:
\begin{itemize}
    \item Saturation limits where both increasing and decreasing commands saturate
    \item Mode-switching altitude controllers with discontinuous thrust
    \item Hybrid position/velocity control with conflicting objectives
\end{itemize}
If sliding is expected, use specialized hybrid solvers that detect and handle sliding modes explicitly.
\end{notebox}

\subsection{Error Analysis}

Event detection introduces errors beyond the ODE solver's intrinsic error.

\subsubsection{Event Localization Error}

Let $\epsilon_t$ be the tolerance for locating $t^*$. The state error at the event time is:
\[
\|x(t^*) - x(\hat{t}^*)\| \leq \epsilon_t \cdot \max_{t \in [t^* - \epsilon_t, t^* + \epsilon_t]} \|f(x(t))\|
\]

For quadrotor landing with descent velocity $v_z \approx 0.3$ m/s and $\epsilon_t = 1$ $\mu$s:
\[
\|z(t^*) - z(\hat{t}^*)\| \leq 10^{-6} \times 0.3 = 0.3 \text{ } \mu\text{m}
\]

This is negligible compared to physical tolerances.

\subsubsection{Accumulated Error from Multiple Events}

With $N$ events, each introducing error $\epsilon$, the total accumulated error can grow as $O(N \cdot \epsilon)$ in the worst case, or $O(\sqrt{N} \cdot \epsilon)$ if errors are random.

\textbf{Quadrotor implication}: A 5-minute flight with mode switches every 100 ms has $N \approx 3000$ events. With $\epsilon_t = 1$ $\mu$s, accumulated timing error is at most 3 ms---still negligible.

\subsubsection{Tolerance Selection}

\begin{center}
\begin{tabular}{lll}
\toprule
\textbf{Parameter} & \textbf{Typical Value} & \textbf{Guidance} \\
\midrule
Event time tolerance $\epsilon_t$ & $10^{-6}$ to $10^{-10}$ s & 1 $\mu$s sufficient for quadrotors \\
State tolerance at event & RelTol $\times |g|$ & Match ODE solver tolerance \\
Maximum events per step & 100--1000 & Detect Zeno/chattering \\
\bottomrule
\end{tabular}
\end{center}

\begin{keyidea}[title=Tolerance Hierarchy]
For consistent simulation, maintain:
\[
\epsilon_t \ll h_{min} \ll \tau_{physical}
\]
where $h_{min}$ is the minimum solver step size and $\tau_{physical}$ is the fastest physical time constant. For quadrotors: $\epsilon_t \approx 1$ $\mu$s, $h_{min} \approx 10$ $\mu$s, $\tau_{physical} \approx 10$ ms (motor dynamics).
\end{keyidea}

\subsection{Example: Landing Event Detection}

We trace through complete event detection for ground contact during landing.

\textbf{Setup}:
\begin{itemize}
    \item State: $x = [z, v_z]^T$ (altitude and vertical velocity, NED convention)
    \item Dynamics: $\dot{z} = v_z$, $\dot{v}_z = g - T/m$ with $T = 0.28$ N, $m = 0.03$ kg
    \item Zero-crossing function: $g(x) = -z$ (event when $z = 0$)
    \item Event tolerance: $\epsilon_t = 10^{-6}$ s
\end{itemize}

\textbf{Solver state at detection}:
\begin{align*}
t_k &= 1.0000 \text{ s}: & z &= -0.0200 \text{ m}, & v_z &= 0.250 \text{ m/s}, & g &= 0.0200 \\
t_{k+1} &= 1.0100 \text{ s}: & z &= +0.0023 \text{ m}, & v_z &= 0.247 \text{ m/s}, & g &= -0.0023
\end{align*}

Sign change detected: $g(x(t_k)) > 0$ and $g(x(t_{k+1})) < 0$.

\textbf{Bisection iterations}:
\begin{center}
\begin{tabular}{ccccc}
\toprule
\textbf{Iter} & $t_{low}$ (s) & $t_{high}$ (s) & $t_{mid}$ (s) & $g(x(t_{mid}))$ \\
\midrule
1 & 1.000000 & 1.010000 & 1.005000 & $-0.00112$ \\
2 & 1.000000 & 1.005000 & 1.002500 & $+0.00569$ \\
3 & 1.002500 & 1.005000 & 1.003750 & $+0.00228$ \\
4 & 1.003750 & 1.005000 & 1.004375 & $+0.00058$ \\
$\vdots$ & $\vdots$ & $\vdots$ & $\vdots$ & $\vdots$ \\
14 & 1.004519 & 1.004520 & 1.004520 & $< 10^{-6}$ \\
\bottomrule
\end{tabular}
\end{center}

\textbf{Result}: Event localized at $t^* = 1.004520$ s with $z(t^*) \approx 0$ to within numerical precision.

\textbf{Post-event action}:
\begin{enumerate}
    \item Record event time: $t_{event} = 1.004520$ s
    \item Apply reset: $R((\text{GroundEffect}, \text{Contact}), x) = [0, v_z, 0]^T$
    \item Switch mode: $q \leftarrow \text{Contact}$
    \item Restart integration from $(t^*, x^+)$ with Contact dynamics
\end{enumerate}

\subsection{Implementation in Simulink}

Simulink provides automatic zero-crossing detection for common blocks.

\textbf{Blocks with built-in detection}:
\begin{itemize}
    \item Saturation, Dead Zone (output limiting)
    \item Switch, Multiport Switch (signal routing)
    \item Relational Operator, Compare To Zero (logical conditions)
    \item Stateflow charts (guard conditions on transitions)
    \item Hit Crossing block (explicit event detection)
\end{itemize}

\textbf{Configuration}: Simulation $\rightarrow$ Model Configuration Parameters $\rightarrow$ Solver $\rightarrow$ Zero-crossing options

\begin{center}
\begin{tabular}{ll}
\toprule
\textbf{Option} & \textbf{Description} \\
\midrule
\texttt{Use local settings} & Each block controls its own detection \\
\texttt{Enable all} & Detect all possible crossings (most accurate) \\
\texttt{Disable all} & Ignore crossings (faster but may miss events) \\
\texttt{Adaptive} & Automatically adjust based on model behavior \\
\bottomrule
\end{tabular}
\end{center}

\textbf{Algorithm selection}: Simulink uses a variant of Brent's method for event localization, with additional logic for:
\begin{itemize}
    \item Consecutive zero-crossing limits (default: 1000)
    \item Adaptive signal threshold (noise rejection)
    \item Algebraic loop handling at events
\end{itemize}

\begin{warningbox}[title=Zero-Crossing and Fixed-Step Solvers]
Fixed-step solvers cannot precisely locate zero-crossings---they only detect sign changes at fixed sample times. Consequences:
\begin{itemize}
    \item \textbf{Delayed detection}: Up to one sample period ($h$)
    \item \textbf{State error}: Proportional to $h \cdot \|\dot{x}\|$ at the event
    \item \textbf{Missed events}: If crossing occurs entirely between samples
    \item \textbf{Chattering}: Rapid switching when state oscillates near threshold
\end{itemize}

\textbf{Mitigation for fixed-step}:
\begin{enumerate}
    \item Add hysteresis to guard conditions: different thresholds for up/down crossings
    \item Use dwell-time constraints: minimum time before allowing reverse transition
    \item Reduce step size near expected events
    \item Accept bounded timing error if physical tolerances allow
\end{enumerate}

For the quadrotor landing example with $h = 1$ ms and $v_z = 0.3$ m/s, the maximum position error at ground contact is 0.3 mm---acceptable for most applications.
\end{warningbox}

\section{Practical Simulation Workflow}

\subsection{Step 1: Model Development (MIL)}

\begin{itemize}
    \item Use Simulink with variable-step solver (\texttt{ode45})
    \item Enable all diagnostics and zero-crossing detection
    \item Focus on correct behavior, not performance
\end{itemize}

\subsection{Step 2: Controller Discretization}

\begin{itemize}
    \item Convert continuous controllers to discrete (e.g., Tustin, ZOH)
    \item Choose sample rates: attitude (500 Hz), position (50 Hz)
    \item Verify discrete controller matches continuous behavior
\end{itemize}

\subsection{Step 3: Fixed-Step Validation}

\begin{itemize}
    \item Switch to fixed-step solver (\texttt{ode4})
    \item Set step size to match target hardware (e.g., 1 ms)
    \item Compare results to variable-step simulation
    \item Add hysteresis to mode switches if needed
\end{itemize}

\subsection{Step 4: Code Generation (SIL)}

\begin{itemize}
    \item Generate C code from Simulink model
    \item Run generated code against plant model (still in Simulink)
    \item Verify numerical equivalence with step 3
\end{itemize}

\subsection{Step 5: Hardware Testing (HIL)}

\begin{itemize}
    \item Run generated code on target hardware
    \item Plant model runs on real-time simulator
    \item Test timing, communication, sensor interfaces
\end{itemize}

\section{Common Simulation Pitfalls}

\begin{warningbox}[title=Things That Go Wrong]
\begin{enumerate}
    \item \textbf{Algebraic loops}: Circular dependencies that can't be solved
    \begin{itemize}
        \item Symptom: ``Algebraic loop'' error or very slow simulation
        \item Fix: Add a unit delay or restructure the model
    \end{itemize}

    \item \textbf{Stiff systems with explicit solver}: Motor dynamics too fast
    \begin{itemize}
        \item Symptom: Simulation runs but gives wrong results or is very slow
        \item Fix: Use implicit solver or simplify motor model
    \end{itemize}

    \item \textbf{Chattering}: Rapid mode switching near guard boundaries
    \begin{itemize}
        \item Symptom: Thousands of mode switches per second
        \item Fix: Add hysteresis to guards
    \end{itemize}

    \item \textbf{Division by zero}: Undefined operations at certain states
    \begin{itemize}
        \item Symptom: NaN or Inf in outputs
        \item Fix: Add saturation or conditional logic
    \end{itemize}
\end{enumerate}
\end{warningbox}

\section{Chapter Summary}

Numerical simulation is essential for quadrotor development:

\begin{itemize}
    \item \textbf{ODE solvers}: Euler (simple, inaccurate), RK4 (good general choice), variable-step (efficient, accurate)
    \item \textbf{Stiff systems}: Use implicit solvers (ode15s, ode23t) when timescales differ greatly
    \item \textbf{Fixed-step}: Required for real-time and code generation; choose $h$ based on fastest dynamics
    \item \textbf{Zero-crossing}: Critical for hybrid systems; use variable-step when possible
    \item \textbf{Workflow}: MIL $\rightarrow$ SIL $\rightarrow$ HIL catches errors at increasing fidelity
\end{itemize}

\begin{keyidea}[title=The Bottom Line]
Simulation is not just ``running the model.'' It requires:
\begin{itemize}
    \item Understanding your system's timescales
    \item Choosing appropriate solvers
    \item Validating results at each stage
    \item Bridging the gap from continuous math to discrete code
\end{itemize}
Time invested in simulation setup pays off in safer, faster development.
\end{keyidea}

%======================================================================
\chapter{Simulation Fidelity and Model Validity}
\index{model fidelity}
%======================================================================

\section{What Makes a Good Test Model}

A test passes or fails based on the \textbf{model's} behavior, not the real system's. The value of testing depends entirely on how well the model represents reality.

\begin{definition}[Model Fidelity]
\textbf{Fidelity}\index{model fidelity|textbf} measures how accurately a model reproduces the behavior of the real system. Higher fidelity means smaller differences between simulated and actual responses.
\end{definition}

\subsection{Sources of Model Error}

\begin{center}
\begin{tabular}{p{3.5cm}p{8cm}}
\toprule
\textbf{Error Source} & \textbf{Example for Quadrotor} \\
\midrule
Parameter uncertainty & Actual mass differs from nominal by 5\% \\
Unmodeled dynamics & Blade flapping, motor dynamics, flexibility \\
Simplifying assumptions & Rigid body assumption, no ground effect \\
Environmental factors & Air density variation, temperature effects \\
Sensor models & Idealized sensors vs. real noise characteristics \\
Actuator models & Idealized motors vs. nonlinear response \\
\bottomrule
\end{tabular}
\end{center}

\subsection{Fidelity vs. Computational Cost}

\begin{center}
\begin{tikzpicture}
    \begin{axis}[
        width=10cm, height=6cm,
        xlabel={Model Complexity}, ylabel={},
        xmin=0, xmax=10, ymin=0, ymax=10,
        xtick=\empty, ytick=\empty,
        legend pos=north west
    ]
    \addplot[blue, thick, domain=0:10] {8*(1-exp(-0.5*x))};
    \addlegendentry{Fidelity}

    \addplot[red, thick, domain=0:10] {0.5*x^1.5};
    \addlegendentry{Computation time}

    \draw[dashed, gray] (axis cs:4,0) -- (axis cs:4,10);
    \node at (axis cs:4,9) {\small Sweet spot};
    \end{axis}
\end{tikzpicture}
\end{center}

There's a trade-off:
\begin{itemize}
    \item Simple models run fast but may miss important behaviors
    \item Complex models are more accurate but slow (limiting test coverage)
    \item The ``sweet spot'' depends on what behaviors you're testing
\end{itemize}

\section{Validating Simulation Against Hardware}

Before trusting simulation results, validate the model against real hardware:

\begin{enumerate}
    \item \textbf{Static validation}: Compare parameters (mass, inertia, motor constants)
    \item \textbf{Open-loop validation}: Apply same inputs to simulation and hardware, compare outputs
    \item \textbf{Closed-loop validation}: Run same controller on both, compare trajectories
\end{enumerate}

\begin{example}[Validation Procedure]
\begin{enumerate}
    \item Apply a step input to the real quadrotor's attitude controller
    \item Record the attitude response (from motion capture or onboard sensors)
    \item Apply the same step input to the simulation
    \item Compare the responses:
    \begin{itemize}
        \item Rise time should match within 10\%
        \item Overshoot should match within 20\%
        \item Steady-state error should match within 5\%
    \end{itemize}
\end{enumerate}

If discrepancies exist, tune model parameters or add missing dynamics until responses match acceptably.
\end{example}

\section{When to Trust Simulation Results}

\begin{keyidea}[title=Simulation Trust Guidelines]
\begin{itemize}
    \item \textbf{High confidence}: Behaviors that depend on well-modeled dynamics (e.g., controller stability for known parameters)
    \item \textbf{Medium confidence}: Behaviors near operating limits where model accuracy decreases
    \item \textbf{Low confidence}: Behaviors involving unmodeled effects (ground effect, wind turbulence, sensor interference)
\end{itemize}

A test that passes in simulation is \textbf{necessary but not sufficient} for real-world success.
\end{keyidea}

\section{Safety Margins for Uncertainty}

Given model uncertainty, design with margins:

\begin{itemize}
    \item If the spec requires altitude $< 10$ m, design for altitude $< 8$ m in simulation
    \item If the spec allows 30° roll, ensure simulation stays below 25°
    \item General rule: Simulation margin should exceed expected model error
\end{itemize}

\[
\text{Simulation limit} = \text{Spec limit} - \text{Safety margin}
\]
\[
\text{Safety margin} \geq \text{Expected model error}
\]

%======================================================================
\chapter{Multi-Rate Control Systems}
\index{multi-rate control}
%======================================================================

\section{Introduction: The Multi-Rate Problem}

Quadrotor flight controllers are inherently \textbf{multi-rate systems}\index{multi-rate control|textbf}: different control loops run at different frequencies.

\begin{center}
\begin{tabular}{lcc}
\toprule
\textbf{Control Loop} & \textbf{Typical Rate} & \textbf{Rationale} \\
\midrule
Attitude control (inner) & 250--500 Hz & Fast dynamics, stability \\
Rate gyro filtering & 500--2000 Hz & Noise rejection \\
Position control (outer) & 50--100 Hz & Slower dynamics \\
Altitude hold & 50--100 Hz & Barometer/sonar rate \\
Mission planning & 1--10 Hz & High-level decisions \\
\bottomrule
\end{tabular}
\end{center}

\textbf{Why different rates?}

\begin{itemize}
    \item \textbf{Dynamics timescales}: Attitude dynamics are fast (time constant ~50 ms); position dynamics are slow (time constant ~1 s). Control theory dictates that sampling rate should be 10--20$\times$ faster than the closed-loop bandwidth.

    \item \textbf{Sensor rates}: IMU can output at 1 kHz+; GPS outputs at 5--10 Hz. Control loops must match sensor availability.

    \item \textbf{Computational cost}: Higher rates require more CPU. Running all loops at the fastest rate wastes resources.
\end{itemize}

\section{Cascaded Control Architecture}

The standard quadrotor control architecture is \textbf{cascaded}: outer loops generate setpoints for inner loops.

\begin{center}
\begin{tikzpicture}[node distance=1.5cm, >=Stealth]
    \node[draw, rectangle, minimum width=2.5cm, minimum height=1cm] (pos) {Position Controller (50 Hz)};
    \node[draw, rectangle, minimum width=2.5cm, minimum height=1cm, right=of pos] (att) {Attitude Controller (500 Hz)};
    \node[right=of att] (motor) {Motors};
    \node[left=of pos] (cmd) {Position Cmd};

    \draw[->] (cmd) -- (pos);
    \draw[->] (pos) -- node[above] {Attitude Setpoint} (att);
    \draw[->] (att) -- node[above] {Motor Cmd} (motor);
\end{tikzpicture}
\end{center}

\textbf{Why cascade?}
\begin{enumerate}
    \item \textbf{Timescale separation}: Inner loops are much faster than outer loops, allowing them to be designed independently.
    \item \textbf{Physical structure}: Position changes through attitude (you tilt to move), so attitude control is a natural inner loop.
    \item \textbf{Tuning simplicity}: Tune inner loop first (assuming outer loop is constant), then tune outer loop (assuming inner loop is ideal).
\end{enumerate}

\begin{keyidea}[title=Timescale Separation Principle]
For cascaded control to work well, the inner loop must be significantly faster than the outer loop (typically 5--10$\times$). This allows the outer loop to ``see'' the inner loop as nearly instantaneous.

For quadrotors:
\begin{itemize}
    \item Attitude bandwidth: 10--20 rad/s
    \item Position bandwidth: 1--2 rad/s
    \item Ratio: ~10$\times$ \checkmark
\end{itemize}
\end{keyidea}

\section{Multi-Rate Task Design}

\subsection{Rate Relationships}

The cleanest multi-rate designs use \textbf{harmonic rates}: each slow rate is an integer divisor of faster rates.

\begin{example}[Harmonic Rate Design]
Base rate: 1 ms (1000 Hz)
\begin{itemize}
    \item IMU sampling: every 1 ms (1000 Hz)
    \item Attitude control: every 2 ms (500 Hz)
    \item Position control: every 20 ms (50 Hz)
    \item Telemetry: every 100 ms (10 Hz)
\end{itemize}

All rates are integer multiples of the base rate, ensuring consistent timing relationships.
\end{example}

\subsection{Rate Transitions}

When data flows between tasks running at different rates, we need \textbf{rate transition} handling:

\textbf{Fast-to-Slow (Downsampling)}:
The slow task reads data produced by the fast task. Options:
\begin{itemize}
    \item \textbf{Latest value}: Use the most recent value (simple, but aliases fast changes)
    \item \textbf{Averaging}: Average values over the slow period (filters noise)
\end{itemize}

\textbf{Slow-to-Fast (Upsampling)}:
The fast task uses data produced by the slow task. Options:
\begin{itemize}
    \item \textbf{Hold}: Use the same value until the slow task updates (creates steps)
    \item \textbf{Interpolation}: Linearly interpolate between slow updates (smoother but adds delay)
\end{itemize}

\subsection{Data Consistency}

\begin{warningbox}[title=Multi-Rate Data Consistency]
When slow and fast tasks share data, ensure consistency:
\begin{enumerate}
    \item The slow task should update all related variables atomically
    \item The fast task should copy all related variables atomically
    \item Never read some variables, release lock, then read more related variables
\end{enumerate}
\end{warningbox}

\section{Timing Analysis for Multi-Rate Systems}

Consider a system with three rates:

\begin{itemize}
    \item Task A: Period 2 ms, WCET 0.5 ms (attitude control)
    \item Task B: Period 10 ms, WCET 2 ms (position control)
    \item Task C: Period 20 ms, WCET 3 ms (state estimation)
\end{itemize}

At $t = 0$, all tasks are released simultaneously (the \textbf{critical instant}). With RMS priorities:

\begin{enumerate}
    \item Task A (highest priority) runs first: completes at 0.5 ms
    \item Task A releases again at 2 ms, preempts whatever is running
    \item Task B runs when A is not ready
    \item Task C (lowest priority) runs in remaining gaps
\end{enumerate}

The critical instant produces the worst-case response times used in schedulability analysis.

\section{Design Guidelines}

\begin{keyidea}[title=Multi-Rate Design Guidelines]
\begin{enumerate}
    \item Use \textbf{harmonic rates} (integer multiples) for predictable timing
    \item Ensure \textbf{timescale separation} (5--10$\times$) between cascaded loops
    \item Handle \textbf{rate transitions} explicitly (hold, interpolate, or average)
    \item Verify \textbf{schedulability} considering all rates
    \item Test at the \textbf{critical instant} when all tasks release simultaneously
\end{enumerate}
\end{keyidea}


% ===== MODULE 3 =====
\part{Module 3: Real-Time Embedded Systems}
%   - Why Real-Time Matters
%   - RTOS Fundamentals and FreeRTOS
%   - Concurrency and Synchronization
%   - Deadlocks and Priority Inversion
%   - Advanced Scheduling Theory
%   - Latency and Jitter in Control Systems
%   - Worst-Case Execution Time Estimation
%   - Fault Detection and Recovery
%======================================================================
% MODULE 3: Real-Time Embedded Systems
%
% Learning Objectives:
% After completing this module, students will be able to:
% - Explain why real-time guarantees are essential for flight control
% - Describe the role of an RTOS in managing concurrent tasks
% - Configure FreeRTOS tasks with appropriate priorities and stack sizes
% - Use task notifications, queues, and timers for inter-task communication
% - Identify and prevent race conditions in concurrent control software
% - Apply synchronization primitives correctly for flight controller tasks
% - Analyze schedulability using RMS and response-time analysis
% - Estimate worst-case execution times for embedded control code
% - Implement fault detection and recovery mechanisms
%======================================================================

\chapter{Real-Time Operating Systems}
\index{RTOS (Real-Time Operating System)}

%----------------------------------------------------------------------
% FIGURE: Module Overview - Quadrotor Software Architecture
%----------------------------------------------------------------------
% Description: High-level view of the software layers in a quadrotor.
%
% Layered diagram (bottom to top):
%   - Hardware layer: MCU, IMU, Motors, Radio
%   - RTOS layer: Scheduler, Memory, Interrupts
%   - Middleware: Sensor drivers, PWM drivers, Communication stack
%   - Application: Control loops, State estimation, Mission logic
%
% Show timing requirements at each level.
% Dimensions: Column width, ~6cm height.
%----------------------------------------------------------------------

\section{Why Real-Time Matters for Quadrotor Control}

In previous modules, we developed the mathematical models for quadrotor dynamics and orientation estimation. Now we face a critical question: \textbf{how do we implement these algorithms on an embedded processor so that they execute reliably and on time?}

This question matters because control theory assumes perfect timing. When we design a controller using discrete-time methods, we assume the controller executes exactly once per sample period $T_s$, computes instantaneously, and outputs immediately. In reality:
\begin{itemize}
    \item Computation takes time---the control law does not execute ``instantaneously''
    \item Multiple tasks compete for the same processor
    \item External events (sensor data, communication) arrive asynchronously
    \item Memory and peripheral access can cause unpredictable delays
\end{itemize}

A Real-Time Operating System (RTOS) provides the infrastructure to manage these complexities systematically.

\subsection{The Multi-Rate Control Problem}

A quadrotor flight controller must run multiple control loops simultaneously, each with different timing requirements:

\begin{center}
\begin{tabular}{lccc}
\toprule
\textbf{Task} & \textbf{Rate} & \textbf{Period} & \textbf{Deadline} \\
\midrule
Attitude control (inner loop) & 250--1000 Hz & 1--4 ms & Hard \\
Sensor fusion (Mahony filter) & 250--500 Hz & 2--4 ms & Hard \\
Position control (outer loop) & 50--100 Hz & 10--20 ms & Firm \\
State estimation (Kalman filter) & 50--200 Hz & 5--20 ms & Firm \\
Communication (telemetry) & 10--50 Hz & 20--100 ms & Soft \\
Logging (SD card) & 1--10 Hz & 100--1000 ms & Soft \\
\bottomrule
\end{tabular}
\end{center}

%----------------------------------------------------------------------
% FIGURE: Multi-Rate Control Timing
%----------------------------------------------------------------------
% Description: Timeline showing multiple periodic tasks at different rates.
%
% Time axis (horizontal), multiple task rows:
%   - Row 1: Attitude control - short boxes every 2ms
%   - Row 2: Sensor fusion - boxes every 2ms (synchronized with attitude)
%   - Row 3: Position control - longer boxes every 20ms
%   - Row 4: Communication - boxes every 50ms
%
% Show how tasks must interleave without missing deadlines.
% Highlight a moment where multiple tasks want to run simultaneously.
% Dimensions: Full width, ~5cm height.
%----------------------------------------------------------------------

\begin{keyidea}[title=The Fundamental Challenge]
Multiple tasks with different periods must share a single processor. If the attitude control loop misses its 2 ms deadline, the quadrotor may become unstable. How do we guarantee that critical tasks always meet their deadlines?
\end{keyidea}

\subsection{Why Timing Precision Affects Stability}

Control systems are designed assuming a specific sampling rate $f_s = 1/h$. When the actual execution deviates from this assumption, control performance degrades:

\textbf{Jitter} is the variation in the time between consecutive executions of a periodic task:
\[
\text{Jitter} = \max_k |t_k - t_{k-1} - h|
\]

where $t_k$ is the actual execution time of iteration $k$ and $h$ is the intended period.

\begin{warningbox}[title=Effect of Jitter on Control]
For a control system with crossover frequency $\omega_c$, timing jitter $\Delta t$ introduces additional phase lag:
\[
\Delta \phi \approx \omega_c \cdot \Delta t \quad \text{(radians)}
\]

For a quadrotor with attitude bandwidth $\omega_c = 20$ rad/s and jitter $\Delta t = 500$ $\mu$s:
\[
\Delta \phi = 20 \times 0.0005 = 0.01 \text{ rad} \approx 0.6°
\]

This may seem small, but cumulative effects and worst-case scenarios can significantly erode stability margins. Professional flight controllers aim for jitter below 50 $\mu$s.
\end{warningbox}

\subsection{From Super-Loop to RTOS}

The simplest embedded architecture is the \textbf{super-loop} (also called bare-metal or polling):

\begin{lstlisting}[language=C, caption=Super-loop architecture (problematic)]
int main(void) {
    init_hardware();
    while (1) {
        read_sensors();        // Variable time!
        run_attitude_control();
        run_position_control();
        send_telemetry();      // Blocks waiting for UART!
        write_to_sd_card();    // Very slow!
    }
}
\end{lstlisting}

\textbf{Problems with super-loop}:
\begin{itemize}
    \item Loop time varies depending on which branches execute
    \item Slow operations (SD card, communication) block fast operations (control)
    \item No way to prioritize critical tasks
    \item Adding new functionality changes timing of everything else
\end{itemize}

\textbf{Solution}: A Real-Time Operating System (RTOS) that:
\begin{itemize}
    \item Runs each task independently with its own timing
    \item Preempts low-priority tasks when high-priority tasks need to run
    \item Provides predictable, analyzable timing behavior
\end{itemize}

\section{Real-Time System Fundamentals}

\subsection{Response Time}

\begin{definition}[Response Time]
The \textbf{response time} of a system is the time between the presentation of a set of inputs and the realization of the required behavior, including the availability of all associated outputs.
\end{definition}

For a quadrotor, the response time of the attitude controller includes:
\begin{enumerate}
    \item Time for IMU interrupt to trigger
    \item Time to read sensor data (SPI/I2C transfer)
    \item Time to run sensor fusion algorithm
    \item Time to compute control law
    \item Time to update PWM outputs to motors
\end{enumerate}

\subsection{Hard, Firm, and Soft Real-Time}

\begin{definition}[Hard Real-Time]
\index{hard real-time}
A \textbf{hard real-time}\index{hard real-time|textbf} system is one in which failure to meet even a single deadline may lead to complete or catastrophic system failure.
\end{definition}

\textit{Example}: Anti-lock braking system (ABS). Missing a deadline could cause wheel lockup during braking.

\begin{definition}[Firm Real-Time]
A \textbf{firm real-time} system is one in which a few missed deadlines will not lead to total failure, but missing more than a few may lead to complete system failure.
\end{definition}

\textit{Example}: Quadrotor attitude control. Missing one deadline causes a small disturbance; missing several consecutive deadlines leads to loss of control.

\begin{definition}[Soft Real-Time]
\index{soft real-time}
A \textbf{soft real-time}\index{soft real-time|textbf} system is one in which performance is degraded but not destroyed by failure to meet response-time constraints.
\end{definition}

\textit{Example}: Video streaming. Dropped frames reduce quality but don't cause failure.

\begin{notebox}[title=Quadrotor Real-Time Classification]
\begin{itemize}
    \item \textbf{Attitude control}: Firm real-time (a few misses tolerable, many cause crash)
    \item \textbf{Position control}: Firm real-time (more tolerant than attitude)
    \item \textbf{Telemetry}: Soft real-time (degraded but functional)
    \item \textbf{Logging}: Soft real-time (data loss acceptable)
\end{itemize}
\end{notebox}

\subsection{Events and Timing}

\begin{definition}[Event]
Any occurrence that causes the program counter to change non-sequentially is an \textbf{event}. Events can be:
\begin{itemize}
    \item \textbf{Synchronous}: Occur at predictable times (e.g., system call, exception)
    \item \textbf{Asynchronous}: Occur at unpredictable times (e.g., hardware interrupt)
\end{itemize}
\end{definition}

Events can also be classified by their temporal pattern:

\begin{definition}[Periodic, Aperiodic, Sporadic Events]
\begin{itemize}
    \item \textbf{Periodic}: Occur at regular intervals (e.g., timer tick every 1 ms)
    \item \textbf{Aperiodic}: Occur irregularly with no minimum inter-arrival time
    \item \textbf{Sporadic}: Occur irregularly but with a minimum inter-arrival time
\end{itemize}
\end{definition}

\textbf{Quadrotor examples}:
\begin{itemize}
    \item \textbf{Periodic}: Control loop timer, sensor sampling
    \item \textbf{Sporadic}: Pilot command from radio (human reaction time sets minimum interval)
    \item \textbf{Aperiodic}: GPS position fix (varies with satellite visibility)
\end{itemize}

\subsection{Release Time and Deadline}

\begin{definition}[Release Time]
The \textbf{release time} $r_i$ is the time at which a task instance becomes ready to execute.
\end{definition}

\begin{definition}[Deadline]
\index{deadline}
The \textbf{deadline}\index{deadline|textbf} $d_i$ is the time by which a task instance must complete execution.
\end{definition}

\begin{definition}[Worst-Case Execution Time (WCET)]
\index{WCET (Worst-Case Execution Time)}
The \textbf{WCET}\index{WCET (Worst-Case Execution Time)|textbf} $C_i$ is the maximum time a task can take to execute, considering all possible execution paths and inputs.
\end{definition}

For a task to be schedulable, we need:
\[
C_i \leq d_i - r_i
\]

%----------------------------------------------------------------------
% FIGURE: Task Timing Terminology
%----------------------------------------------------------------------
% Description: Timeline showing release time, execution, deadline.
%
% Horizontal time axis with:
%   - Release time r_i marked
%   - Execution block (width = C_i)
%   - Deadline d_i marked
%   - Response time shown as r_i to completion
%   - Slack time shown (if any)
%
% Show both "meeting deadline" and "missing deadline" scenarios.
% Dimensions: Column width, ~4cm height.
%----------------------------------------------------------------------

\section{RTOS Architecture}

For a comprehensive treatment of real-time scheduling theory, see Buttazzo~\cite{buttazzo2011hard}.

\subsection{What is an Operating System?}

An operating system provides:
\begin{itemize}
    \item \textbf{Abstraction}: Hide hardware details from application code
    \item \textbf{Resource management}: CPU time, memory, peripherals
    \item \textbf{Multitasking}: Run multiple tasks ``simultaneously''
    \item \textbf{Services}: File systems, networking, device drivers
\end{itemize}

\subsection{Real-Time Operating System (RTOS)}

An RTOS is designed specifically for systems with timing constraints:

\begin{itemize}
    \item \textbf{Deterministic}: Bounded, predictable response times
    \item \textbf{Preemptive}: High-priority tasks can interrupt low-priority tasks
    \item \textbf{Priority-based}: Tasks assigned priorities; scheduler always runs highest-priority ready task
    \item \textbf{Minimal overhead}: Small memory footprint, fast context switch
\end{itemize}

\textbf{Common RTOS examples}:
\begin{center}
\begin{tabular}{lll}
\toprule
\textbf{RTOS} & \textbf{License} & \textbf{Common Use} \\
\midrule
FreeRTOS & MIT (open source) & Hobbyist, commercial embedded \\
Zephyr & Apache 2.0 & IoT, wearables \\
ChibiOS & GPL/commercial & Robotics, drones \\
NuttX & Apache 2.0 & PX4 flight controller \\
VxWorks & Commercial & Aerospace, defense \\
QNX & Commercial & Automotive, medical \\
\bottomrule
\end{tabular}
\end{center}

\subsection{Tasks (Threads)}
\index{task}

\begin{definition}[Task]
A \textbf{task}\index{task|textbf} (or thread) is an independent sequence of instructions that can be scheduled by the RTOS. Each task has its own:
\begin{itemize}
    \item \textbf{Stack}: Local variables, function call history
    \item \textbf{Registers}: Saved/restored during context switch (including PC, SP)
    \item \textbf{Priority}: Determines scheduling order
    \item \textbf{State}: Running, Ready, Blocked, or Suspended
\end{itemize}
\end{definition}

Tasks within the same process share:
\begin{itemize}
    \item Global variables and static data
    \item Heap memory
    \item Peripheral access
\end{itemize}

\begin{warningbox}[title=Shared Data Hazard]
Because tasks share memory, concurrent access to shared variables can cause \textbf{race conditions}. We will address this with synchronization primitives later in this module.
\end{warningbox}

\subsection{Task States}

%----------------------------------------------------------------------
% FIGURE: Task State Diagram
%----------------------------------------------------------------------
% Description: State machine showing task states and transitions.
%
% States (boxes):
%   - Running (only one task at a time)
%   - Ready (waiting for CPU)
%   - Blocked (waiting for event)
%   - Suspended (explicitly paused)
%
% Transitions (arrows with labels):
%   - Ready -> Running: "Scheduler dispatch"
%   - Running -> Ready: "Preemption or yield"
%   - Running -> Blocked: "Wait for event"
%   - Blocked -> Ready: "Event occurs"
%   - Any -> Suspended: "vTaskSuspend()"
%   - Suspended -> Ready: "vTaskResume()"
%
% Highlight that only ONE task can be Running at a time.
% Dimensions: Column width, ~6cm height.
%----------------------------------------------------------------------

A task is always in one of four states:

\begin{enumerate}
    \item \textbf{Running}: Currently executing on the CPU. Only one task can be Running at any time (on a single-core processor).

    \item \textbf{Ready}: Able to run but waiting for the CPU (a higher-priority task is Running).

    \item \textbf{Blocked}: Waiting for an event (timer expiry, semaphore, queue data). Cannot run until the event occurs.

    \item \textbf{Suspended}: Explicitly paused by API call. Will not run until explicitly resumed.
\end{enumerate}

\textbf{State transitions}:
\begin{itemize}
    \item \textbf{Ready $\to$ Running}: Scheduler selects this task (highest priority ready task)
    \item \textbf{Running $\to$ Ready}: Higher-priority task becomes ready (preemption) or task yields
    \item \textbf{Running $\to$ Blocked}: Task calls blocking API (e.g., wait for semaphore)
    \item \textbf{Blocked $\to$ Ready}: Event occurs that the task was waiting for
\end{itemize}

\subsection{Context Switching}
\index{context switch}

\begin{definition}[Context Switch]
A \textbf{context switch}\index{context switch|textbf} is the process of saving the state of the currently running task and restoring the state of another task so it can run.
\end{definition}

During a context switch, the RTOS must:
\begin{enumerate}
    \item Save all CPU registers of the current task to its stack
    \item Save the stack pointer to the task control block (TCB)
    \item Select the next task to run (scheduler decision)
    \item Load the stack pointer from the new task's TCB
    \item Restore all CPU registers from the new task's stack
    \item Resume execution at the new task's saved program counter
\end{enumerate}

\begin{notebox}[title=Context Switch Overhead]
Context switching takes time during which no useful work is done. Typical context switch times:
\begin{itemize}
    \item ARM Cortex-M4 @ 168 MHz: 2--5 $\mu$s
    \item ARM Cortex-M7 @ 400 MHz: 1--3 $\mu$s
\end{itemize}
This overhead must be accounted for in timing analysis. If tasks switch too frequently, significant CPU time is lost to overhead.
\end{notebox}

\subsection{The Scheduler}
\index{scheduler}

\begin{definition}[Scheduler]
The \textbf{scheduler}\index{scheduler|textbf} is the kernel component that decides which task should run at any given time.
\end{definition}

\begin{definition}[Scheduling Policy]
The \textbf{scheduling policy} is the algorithm the scheduler uses to make this decision.
\end{definition}

Most RTOS use \textbf{fixed-priority preemptive scheduling}:
\begin{itemize}
    \item Each task has a fixed priority (assigned at creation)
    \item The scheduler always runs the highest-priority ready task
    \item If a higher-priority task becomes ready, it immediately preempts the running task
    \item Tasks of equal priority may use round-robin (time slicing)
\end{itemize}

\textbf{Scheduling decisions occur when}:
\begin{enumerate}
    \item A task blocks (waits for event)
    \item A task is created or resumed
    \item An event occurs that unblocks a task
    \item A periodic timer tick occurs (for time slicing)
    \item A task explicitly yields
\end{enumerate}

\section{Scheduling Theory}

Understanding scheduling theory helps us answer: \textit{Given a set of tasks with known execution times and periods, can we guarantee all deadlines will be met?}

\subsection{Task Model}

To analyze schedulability mathematically, we need a formal model of what tasks look like. We consider \textbf{periodic tasks} with the following parameters:
\begin{itemize}
    \item $T_i$: Period (time between consecutive releases)
    \item $C_i$: Worst-case execution time (WCET)
    \item $D_i$: Relative deadline (usually $D_i = T_i$)
    \item $\pi_i$: Priority
\end{itemize}

\begin{notebox}[title=Assumptions in the Standard Task Model]
The classical scheduling theory makes several simplifying assumptions:
\begin{enumerate}
    \item \textbf{Known, constant WCET}: We assume $C_i$ is known and does not vary between instances. In practice, execution time varies depending on input data and cache state.
    \item \textbf{Independent tasks}: Tasks do not share resources or communicate. We relax this assumption when discussing synchronization.
    \item \textbf{Negligible context switch time}: Switching between tasks takes zero time. In practice, context switches take 2--10 $\mu$s on typical microcontrollers.
    \item \textbf{Deadline equals period}: $D_i = T_i$, meaning each task instance must complete before the next instance is released.
    \item \textbf{No self-suspension}: Tasks do not voluntarily block during execution (other than at the end of each period).
    \item \textbf{Single processor}: We analyze a single CPU. Multi-core scheduling is significantly more complex.
\end{enumerate}
These assumptions let us derive clean theoretical results. Real systems violate some of these assumptions, requiring engineering margin in the timing budget.
\end{notebox}

\begin{example}[Quadrotor Task Set]
\begin{center}
\begin{tabular}{lcccc}
\toprule
\textbf{Task} & $T_i$ (ms) & $C_i$ (ms) & $D_i$ (ms) & Utilization \\
\midrule
Attitude control & 2 & 0.3 & 2 & 15\% \\
Sensor fusion & 2 & 0.2 & 2 & 10\% \\
Position control & 20 & 1.0 & 20 & 5\% \\
Communication & 50 & 2.0 & 50 & 4\% \\
Logging & 100 & 5.0 & 100 & 5\% \\
\midrule
\textbf{Total} & & & & \textbf{39\%} \\
\bottomrule
\end{tabular}
\end{center}
\end{example}

\subsection{CPU Utilization}

\begin{definition}[CPU Utilization]
The \textbf{CPU utilization} of a task set is the fraction of CPU time used by all tasks:
\[
U = \sum_{i=1}^{n} \frac{C_i}{T_i}
\]
\end{definition}

If $U > 1$, the task set is definitely \textbf{not schedulable}---there isn't enough CPU time.

If $U \leq 1$, the task set \textbf{might} be schedulable, depending on the scheduling algorithm.

\subsection{Rate-Monotonic Scheduling (RMS)}
\index{Rate-Monotonic Scheduling (RMS)}

\begin{definition}[Rate-Monotonic Scheduling]
\textbf{Rate-Monotonic Scheduling (RMS)}\index{Rate-Monotonic Scheduling (RMS)|textbf} is a fixed-priority scheduling algorithm where priorities are assigned based on period:
\[
\text{Shorter period} \Rightarrow \text{Higher priority}
\]
\end{definition}

\textbf{Intuition}: Why should shorter-period tasks have higher priority? Consider what happens if we do the opposite. Suppose a slow task (period 100 ms) has higher priority than a fast task (period 2 ms). Every time the slow task runs, it blocks the fast task for potentially its entire execution time. But the fast task has a deadline every 2 ms! If the slow task takes more than 2 ms to execute, the fast task misses its deadline.

By giving the fast task higher priority, we ensure it can always preempt the slow task and meet its tight deadline. The slow task may be delayed, but it has a long time until its deadline---50 times longer in this example.

\begin{theorem}[RMS Optimality]
For periodic tasks with deadlines equal to periods, RMS is \textbf{optimal} among fixed-priority algorithms. If any fixed-priority assignment can schedule the task set, RMS can too.
\end{theorem}

This theorem, proved by Liu and Layland~\cite{liu1973scheduling} in their seminal 1973 paper, means you never need to search for a ``better'' priority assignment---just assign by rate, and you have done the best possible.

\textbf{RMS Utilization Bound}:

\begin{theorem}[Liu \& Layland Bound]
A set of $n$ periodic tasks with deadlines equal to periods is guaranteed schedulable under RMS if:
\[
U = \sum_{i=1}^{n} \frac{C_i}{T_i} \leq n(2^{1/n} - 1)
\]
\end{theorem}

\begin{center}
\begin{tabular}{cc}
\toprule
$n$ (tasks) & Utilization bound \\
\midrule
1 & 100\% \\
2 & 82.8\% \\
3 & 78.0\% \\
4 & 75.7\% \\
5 & 74.3\% \\
$\infty$ & 69.3\% \\
\bottomrule
\end{tabular}
\end{center}

\begin{example}[Schedulability Check]
For our quadrotor task set with $U = 39\%$ and $n = 5$ tasks:
\[
U = 0.39 < 0.743 = 5(2^{1/5} - 1)
\]
The task set is \textbf{guaranteed schedulable} under RMS.
\end{example}

\begin{notebox}[title=The Bound is Sufficient but Not Necessary]
The Liu \& Layland bound is a \textit{sufficient} condition. Task sets with higher utilization may still be schedulable---the bound is pessimistic because it assumes worst-case (non-harmonic) period relationships. More precise analysis (response time analysis) can verify schedulability up to 100\% utilization in some cases. See Section~\ref{sec:liu-layland} for a proof sketch explaining why the bound takes this particular form.
\end{notebox}

\subsection{Priority Assignment for Quadrotors}

Applying RMS to our quadrotor:

\begin{center}
\begin{tabular}{lccc}
\toprule
\textbf{Task} & \textbf{Period} & \textbf{RMS Priority} & \textbf{FreeRTOS Priority} \\
\midrule
Attitude control & 2 ms & Highest & 5 \\
Sensor fusion & 2 ms & Highest & 5 \\
Position control & 20 ms & Medium-High & 4 \\
Communication & 50 ms & Medium & 3 \\
Logging & 100 ms & Low & 2 \\
Idle & --- & Lowest & 0 \\
\bottomrule
\end{tabular}
\end{center}

\begin{keyidea}[title=Why Attitude Control Has Highest Priority]
The attitude control loop has the shortest period (2 ms) and the most critical deadline. If it misses deadlines, the quadrotor becomes unstable. RMS correctly assigns it the highest priority, ensuring it can always preempt slower tasks.
\end{keyidea}

\subsection{Priority Inversion}
\index{priority inversion}

\begin{definition}[Priority Inversion]
\textbf{Priority inversion}\index{priority inversion|textbf} occurs when a high-priority task is blocked waiting for a resource held by a low-priority task, while a medium-priority task runs instead.
\end{definition}

%----------------------------------------------------------------------
% FIGURE: Priority Inversion Scenario
%----------------------------------------------------------------------
% Description: Timeline showing priority inversion problem.
%
% Three tasks: High (H), Medium (M), Low (L)
% Timeline showing:
%   1. L starts, acquires mutex
%   2. H arrives, needs mutex, blocks
%   3. M arrives (doesn't need mutex), preempts L
%   4. M runs for a long time
%   5. H is blocked even though it has higher priority than M!
%
% Show the "inversion" period where H waits for M.
% Dimensions: Full width, ~5cm height.
%----------------------------------------------------------------------

\begin{example}[Priority Inversion Scenario]
Consider three tasks: High (H), Medium (M), Low (L), and a shared resource protected by a mutex.

\begin{enumerate}
    \item Task L (low priority) acquires the mutex
    \item Task H (high priority) arrives and tries to acquire the mutex---blocks
    \item Task M (medium priority) arrives and preempts L (M doesn't need the mutex)
    \item Task M runs to completion
    \item Task L finally resumes, releases mutex
    \item Task H finally runs
\end{enumerate}

\textbf{Problem}: H is blocked not just by L (which holds the resource) but also by M (which doesn't even need the resource). This is \textbf{unbounded priority inversion}---any number of medium-priority tasks could delay H.
\end{example}

\begin{warningbox}[title=Mars Pathfinder Incident (1997)]
The Mars Pathfinder spacecraft experienced system resets due to priority inversion~\cite{jones1997mars}. A low-priority meteorological task held a mutex needed by a high-priority bus management task. Medium-priority communication tasks kept preempting the low-priority task, causing deadline misses and watchdog resets. The fix: enable priority inheritance (uploaded from Earth!).
\end{warningbox}

\subsection{Priority Inheritance Protocol}

The Priority Inheritance Protocol was developed by Sha, Rajkumar, and Lehoczky~\cite{sha1990priority} to address the priority inversion problem.

\begin{definition}[Priority Inheritance]
Under \textbf{priority inheritance}, when a high-priority task blocks on a mutex held by a low-priority task, the low-priority task temporarily inherits the high priority until it releases the mutex.
\end{definition}

This prevents medium-priority tasks from preempting the low-priority task while it holds a resource needed by a high-priority task.

\begin{keyidea}[title=Using Priority Inheritance]
In FreeRTOS, mutexes automatically implement priority inheritance. Always use \texttt{xSemaphoreCreateMutex()} instead of \texttt{xSemaphoreCreateBinary()} when protecting shared resources that might cause priority inversion.
\end{keyidea}

\section{FreeRTOS Implementation}
\index{FreeRTOS}

FreeRTOS~\cite{barry2016mastering}\index{FreeRTOS|textbf} is a popular open-source RTOS used in many embedded systems, including hobbyist and commercial flight controllers. This section covers practical implementation details.

\subsection{Task Creation}

\begin{lstlisting}[language=C, caption=Creating a FreeRTOS task]
// Task function prototype
void vAttitudeControlTask(void *pvParameters);

// Create the task
xTaskCreate(
    vAttitudeControlTask,    // Task function
    "AttCtrl",               // Name (for debugging)
    256,                     // Stack size (words)
    NULL,                    // Parameters
    5,                       // Priority (5 = high)
    &xAttitudeTaskHandle     // Task handle (optional)
);
\end{lstlisting}

\textbf{Parameters explained}:
\begin{itemize}
    \item \textbf{Stack size}: Each task needs its own stack. Size depends on local variables and call depth. Start with 256 words (1 KB on 32-bit) and increase if you get stack overflow.
    \item \textbf{Priority}: Higher number = higher priority (in FreeRTOS). Priority 0 is reserved for the idle task.
    \item \textbf{Task handle}: Optional reference to the task for later use (e.g., to delete or suspend it).
\end{itemize}

\subsection{Task Function Structure}

\begin{lstlisting}[language=C, caption=Typical control task structure]
void vAttitudeControlTask(void *pvParameters)
{
    // Local variables (each task instance gets its own copy)
    TickType_t xLastWakeTime;
    ImuData_t imuData;
    ControlOutput_t output;

    // Initialize timing
    xLastWakeTime = xTaskGetTickCount();

    // Task loop (runs forever)
    for (;;)
    {
        // 1. Read sensor data (from queue, not directly)
        xQueueReceive(xImuQueue, &imuData, portMAX_DELAY);

        // 2. Run sensor fusion
        MahonyUpdate(&imuData, &orientation);

        // 3. Compute control law
        ComputeAttitudeControl(&orientation, &setpoint, &output);

        // 4. Send motor commands
        SetMotorPWM(&output);

        // 5. Wait for next period
        vTaskDelayUntil(&xLastWakeTime, pdMS_TO_TICKS(2));
    }

    // Should never reach here
    vTaskDelete(NULL);
}
\end{lstlisting}

\subsection{Timing: vTaskDelay vs vTaskDelayUntil}

\textbf{vTaskDelay}: Delays for a \textit{relative} time from when the function is called.

\begin{lstlisting}[language=C]
vTaskDelay(pdMS_TO_TICKS(10));  // Delay 10 ms from NOW
\end{lstlisting}

\textbf{Problem}: The actual period becomes execution time + delay time, causing drift.

%----------------------------------------------------------------------
% FIGURE: vTaskDelay vs vTaskDelayUntil Timing
%----------------------------------------------------------------------
% Description: Two timelines comparing the two delay functions.
%
% Top timeline (vTaskDelay):
%   - Shows execution blocks of varying length
%   - Delay starts AFTER execution completes
%   - Period = execution_time + delay_time (variable!)
%   - Mark "drift" accumulating over time
%
% Bottom timeline (vTaskDelayUntil):
%   - Shows execution blocks of varying length
%   - Delay calculated from last wake time
%   - Period is constant regardless of execution time
%   - Mark consistent spacing
%
% Dimensions: Full width, ~5cm height.
%----------------------------------------------------------------------

\textbf{vTaskDelayUntil}: Delays until an \textit{absolute} time, maintaining consistent period.

\begin{lstlisting}[language=C]
TickType_t xLastWakeTime = xTaskGetTickCount();

for (;;) {
    // Do work...

    // Delay until 10 ms after last wake, not 10 ms from now
    vTaskDelayUntil(&xLastWakeTime, pdMS_TO_TICKS(10));
}
\end{lstlisting}

\begin{keyidea}[title=Always Use vTaskDelayUntil for Periodic Tasks]
For control loops that must run at a precise rate, always use \texttt{vTaskDelayUntil()}. The function automatically accounts for varying execution times, maintaining a constant period.
\end{keyidea}

\subsection{The Tick Interrupt}

FreeRTOS uses a periodic timer interrupt called the \textbf{tick} to:
\begin{itemize}
    \item Track elapsed time
    \item Wake tasks whose delay has expired
    \item Implement time slicing for equal-priority tasks
\end{itemize}

\begin{lstlisting}[language=C, caption=Tick configuration in FreeRTOSConfig.h]
#define configTICK_RATE_HZ    1000  // 1 kHz tick = 1 ms resolution
\end{lstlisting}

\textbf{Trade-offs}:
\begin{itemize}
    \item Higher tick rate $\Rightarrow$ finer timing resolution, more interrupt overhead
    \item Lower tick rate $\Rightarrow$ coarser resolution, less overhead
\end{itemize}

For quadrotor control at 500 Hz (2 ms period), a tick rate of 1000 Hz (1 ms) provides adequate resolution.

\subsection{The Idle Task}

FreeRTOS automatically creates an \textbf{idle task} at priority 0. This task:
\begin{itemize}
    \item Runs when no other task is ready
    \item Can be used for power management (sleep modes)
    \item Cleans up memory from deleted tasks
\end{itemize}

The idle task ensures the scheduler always has something to run.

\section{Synchronization and Communication}

Tasks often need to share data or coordinate their execution. Without proper synchronization, concurrent access to shared data causes \textbf{race conditions}---bugs that are difficult to reproduce and debug.

\textbf{Why synchronization is necessary}: Consider what ``concurrent access'' means on a single-core processor. Tasks do not actually run simultaneously---the scheduler interleaves them. The problem is that this interleaving can happen at any point, including in the middle of a multi-step operation. If one task is updating a quaternion (four values that must be consistent) and another task preempts it mid-update, the second task sees a partially updated, inconsistent quaternion.

\textbf{Why these bugs are hard to find}: Race conditions often depend on exact timing. Your code might work correctly 99.9\% of the time and fail only when tasks preempt at precisely the wrong moment. These bugs are notoriously difficult to reproduce in testing but can cause crashes in the field.

\subsection{The Shared Data Problem}

\begin{example}[Race Condition]
Two tasks access a shared orientation estimate:

\begin{lstlisting}[language=C]
// Shared global variable
Quaternion_t orientation;

// Task 1: Sensor fusion (writes orientation)
void vSensorFusionTask(void *pvParameters) {
    for (;;) {
        // Update all four quaternion components
        orientation.w = new_w;  // <-- What if preempted here?
        orientation.x = new_x;
        orientation.y = new_y;
        orientation.z = new_z;
    }
}

// Task 2: Control (reads orientation)
void vControlTask(void *pvParameters) {
    for (;;) {
        Quaternion_t q = orientation;  // May read partial update!
        // Use q for control...
    }
}
\end{lstlisting}

If the control task preempts the sensor fusion task between writes, it reads an \textbf{inconsistent} quaternion (some components old, some new). This can cause erratic control behavior.
\end{example}

\subsection{Critical Sections}

The simplest protection is to disable interrupts during shared data access:

\begin{lstlisting}[language=C]
taskENTER_CRITICAL();
// Access shared data (cannot be preempted)
orientation.w = new_w;
orientation.x = new_x;
orientation.y = new_y;
orientation.z = new_z;
taskEXIT_CRITICAL();
\end{lstlisting}

\begin{warningbox}[title=Critical Section Limitations]
Disabling interrupts blocks \textbf{all} tasks and interrupts. Keep critical sections as short as possible (microseconds). Long critical sections cause missed interrupts and timing jitter.
\end{warningbox}

\subsection{Mutexes}
\index{mutex}

\begin{definition}[Mutex]
A \textbf{mutex}\index{mutex|textbf} (mutual exclusion) is a synchronization primitive that ensures only one task can access a shared resource at a time.
\end{definition}

\begin{lstlisting}[language=C, caption=Using a mutex]
// Create mutex (once, at initialization)
SemaphoreHandle_t xOrientationMutex;
xOrientationMutex = xSemaphoreCreateMutex();

// Task 1: Writer
void vSensorFusionTask(void *pvParameters) {
    for (;;) {
        // Acquire mutex (blocks if another task holds it)
        xSemaphoreTake(xOrientationMutex, portMAX_DELAY);

        // Safe to write
        orientation.w = new_w;
        orientation.x = new_x;
        orientation.y = new_y;
        orientation.z = new_z;

        // Release mutex
        xSemaphoreGive(xOrientationMutex);
    }
}

// Task 2: Reader
void vControlTask(void *pvParameters) {
    Quaternion_t localQ;
    for (;;) {
        xSemaphoreTake(xOrientationMutex, portMAX_DELAY);
        localQ = orientation;  // Copy while holding mutex
        xSemaphoreGive(xOrientationMutex);

        // Use localQ (safe, it's a local copy)
        ComputeControl(&localQ);
    }
}
\end{lstlisting}

\begin{keyidea}[title=Mutex Best Practices]
\begin{itemize}
    \item Hold mutexes for the minimum time necessary
    \item Never call blocking functions while holding a mutex
    \item Use \texttt{xSemaphoreCreateMutex()} (includes priority inheritance)
    \item Copy shared data to local variables, then release the mutex
\end{itemize}
\end{keyidea}

\subsection{Semaphores}
\index{semaphore}

\begin{definition}[Semaphore]
A \textbf{semaphore}\index{semaphore|textbf} is a signaling mechanism. A \textbf{binary semaphore} can be ``given'' (signaled) by one task and ``taken'' (waited on) by another.
\end{definition}

Common use: Signaling that an event has occurred.

\begin{lstlisting}[language=C, caption=Binary semaphore for ISR-to-task signaling]
SemaphoreHandle_t xImuDataReady;

// Create semaphore
xImuDataReady = xSemaphoreCreateBinary();

// ISR: Signal that IMU data is ready
void IMU_DataReady_ISR(void) {
    BaseType_t xHigherPriorityTaskWoken = pdFALSE;

    xSemaphoreGiveFromISR(xImuDataReady, &xHigherPriorityTaskWoken);

    // Request context switch if a higher-priority task was woken
    portYIELD_FROM_ISR(xHigherPriorityTaskWoken);
}

// Task: Wait for IMU data
void vSensorTask(void *pvParameters) {
    for (;;) {
        // Block until ISR signals data ready
        xSemaphoreTake(xImuDataReady, portMAX_DELAY);

        // Read IMU data (data is definitely ready now)
        ReadImuRegisters(&imuData);
    }
}
\end{lstlisting}

\subsection{Queues}
\index{queue}

\begin{definition}[Queue]
A \textbf{queue}\index{queue|textbf} is a FIFO buffer that allows tasks to send and receive data safely. Queues handle all synchronization internally.
\end{definition}

Queues are ideal for passing data between tasks:

\begin{lstlisting}[language=C, caption=Queue for sensor data]
// Create queue that holds 5 ImuData_t items
QueueHandle_t xImuQueue;
xImuQueue = xQueueCreate(5, sizeof(ImuData_t));

// Producer task (sensor reading)
void vSensorTask(void *pvParameters) {
    ImuData_t data;
    for (;;) {
        ReadImuData(&data);

        // Send to queue (blocks if queue full)
        xQueueSend(xImuQueue, &data, portMAX_DELAY);
    }
}

// Consumer task (control)
void vControlTask(void *pvParameters) {
    ImuData_t data;
    for (;;) {
        // Receive from queue (blocks if queue empty)
        xQueueReceive(xImuQueue, &data, portMAX_DELAY);

        // Process data
        ProcessImuData(&data);
    }
}
\end{lstlisting}

\begin{keyidea}[title=When to Use Each Primitive]
\begin{itemize}
    \item \textbf{Mutex}: Protect shared data from concurrent access
    \item \textbf{Binary semaphore}: Signal events between tasks or from ISR to task
    \item \textbf{Queue}: Pass data between tasks (producer-consumer pattern)
    \item \textbf{Critical section}: Very short, time-critical protection
\end{itemize}
\end{keyidea}

\section{Interrupt Handling}
\index{interrupt}

Interrupts are essential for responsive embedded systems. They allow the processor to react immediately to hardware events without polling.

\textbf{Motivation}: Without interrupts, the processor would need to continuously check (``poll'') whether events have occurred. For a quadrotor with sensors producing data at 1 kHz, polling would waste significant CPU time just checking. Worse, if the processor is executing a long computation when data arrives, there might be a significant delay before detecting the event.

Interrupts solve this by allowing hardware to signal the processor immediately when an event occurs. The processor stops whatever it is doing, handles the event, and resumes its previous work.

\subsection{Hardware Interrupts}

\begin{definition}[Interrupt]
An \textbf{interrupt} is a hardware signal that causes the processor to suspend current execution and run a special function called an \textbf{Interrupt Service Routine (ISR)}.
\end{definition}

\textbf{What happens during an interrupt}:
\begin{enumerate}
    \item Hardware asserts an interrupt signal (e.g., IMU data ready pin goes high)
    \item At the next instruction boundary, the processor stops the current task
    \item Processor saves enough state (PC, flags) to resume later
    \item Processor jumps to the ISR address (from an interrupt vector table)
    \item ISR executes and returns
    \item Processor restores state and resumes interrupted code
\end{enumerate}

On ARM Cortex-M processors, this entire process takes 12--24 clock cycles, allowing response within microseconds.

\textbf{Common interrupt sources in quadrotor systems}:
\begin{itemize}
    \item Timer interrupt (control loop timing)
    \item IMU data-ready interrupt (new sensor data available)
    \item SPI/I2C transfer complete
    \item UART receive (radio commands)
    \item DMA complete (bulk data transfer done)
\end{itemize}

\subsection{Interrupt Service Routines}

\begin{lstlisting}[language=C, caption=Typical ISR structure]
void TIM2_IRQHandler(void)  // Timer 2 interrupt handler
{
    // 1. Clear interrupt flag (required!)
    TIM2->SR &= ~TIM_SR_UIF;

    // 2. Do minimal work
    // ...

    // 3. Signal task if needed (deferred processing)
    BaseType_t xHigherPriorityTaskWoken = pdFALSE;
    xSemaphoreGiveFromISR(xTimerSemaphore, &xHigherPriorityTaskWoken);
    portYIELD_FROM_ISR(xHigherPriorityTaskWoken);
}
\end{lstlisting}

\begin{warningbox}[title=ISR Rules]
\begin{enumerate}
    \item \textbf{Keep ISRs short}: Do minimal work; defer processing to tasks
    \item \textbf{No blocking}: Never call blocking functions (e.g., \texttt{vTaskDelay})
    \item \textbf{Use ISR-safe APIs}: Use \texttt{...FromISR()} versions of FreeRTOS functions
    \item \textbf{Clear interrupt flag}: Forgetting this causes infinite interrupt loop
    \item \textbf{Request context switch}: Call \texttt{portYIELD\_FROM\_ISR()} if you woke a task
\end{enumerate}
\end{warningbox}

\subsection{Deferred Interrupt Processing}

The recommended pattern is to keep ISRs minimal and do most work in tasks:

%----------------------------------------------------------------------
% FIGURE: Deferred Interrupt Processing
%----------------------------------------------------------------------
% Description: Sequence diagram showing ISR to task handoff.
%
% Timeline with two lanes: ISR, Task
%   1. Hardware event triggers ISR
%   2. ISR: clear flag, give semaphore, return (very fast)
%   3. Scheduler runs (if higher-priority task woken)
%   4. Task: take semaphore, do actual work
%
% Show timing: ISR is microseconds, task can take longer.
% Dimensions: Column width, ~5cm height.
%----------------------------------------------------------------------

\begin{lstlisting}[language=C, caption=Deferred interrupt processing pattern]
// ISR: Very short, just signals
void IMU_DataReady_ISR(void) {
    BaseType_t woken = pdFALSE;
    xSemaphoreGiveFromISR(xImuSemaphore, &woken);
    portYIELD_FROM_ISR(woken);
}

// Task: Does the actual work
void vImuProcessingTask(void *pvParameters) {
    for (;;) {
        // Wait for ISR to signal
        xSemaphoreTake(xImuSemaphore, portMAX_DELAY);

        // Do the work (can take longer, can use blocking calls)
        ReadImuOverSPI(&rawData);
        CalibrateImuData(&rawData, &calibratedData);
        xQueueSend(xImuQueue, &calibratedData, 0);
    }
}
\end{lstlisting}

\subsection{Interrupt Priority vs Task Priority}

These are \textbf{different} priority systems:

\begin{itemize}
    \item \textbf{Interrupt priority}: Hardware-managed, determines which ISR runs if multiple interrupts occur
    \item \textbf{Task priority}: RTOS-managed, determines which task runs when no ISR is active
\end{itemize}

\textbf{Key rule}: All interrupts (even low-priority ones) preempt all tasks (even high-priority ones).

In ARM Cortex-M processors, FreeRTOS uses two special priority levels:
\begin{itemize}
    \item \texttt{configLIBRARY\_MAX\_SYSCALL\_INTERRUPT\_PRIORITY}: Highest priority that can use FreeRTOS API
    \item Interrupts above this priority cannot call FreeRTOS functions but have lowest latency
\end{itemize}

\section{Timer and DMA Peripherals}

Modern microcontrollers include sophisticated timer and DMA peripherals that are essential for efficient flight controller implementation. Understanding these peripherals allows you to offload timing-critical operations from software, achieving both better performance and more predictable timing.

\subsection{Hardware Timers for Precise Timing}

Hardware timers count clock cycles independently of the CPU, providing precise timing without software overhead. For quadrotors, timers serve several critical functions:

\begin{itemize}
    \item \textbf{Control loop timing}: Triggering periodic control updates at exact intervals
    \item \textbf{PWM generation}: Producing motor drive signals with microsecond precision
    \item \textbf{Input capture}: Measuring RC receiver pulse widths
    \item \textbf{Timing measurements}: Profiling code execution time
\end{itemize}

\subsubsection{Timer Architecture}

A typical ARM Cortex-M timer (e.g., STM32 TIM2) contains:

\begin{itemize}
    \item \textbf{Counter register (CNT)}: 16 or 32-bit counter that increments each clock cycle
    \item \textbf{Prescaler (PSC)}: Divides the input clock to reduce counter frequency
    \item \textbf{Auto-reload register (ARR)}: Maximum count value; counter resets when reached
    \item \textbf{Capture/Compare registers (CCRx)}: For PWM generation or input capture
\end{itemize}

The timer frequency after the prescaler is:
\[
f_{\text{timer}} = \frac{f_{\text{clock}}}{(\text{PSC} + 1)}
\]

And the timer period (time between overflows) is:
\[
T_{\text{period}} = \frac{(\text{ARR} + 1) \times (\text{PSC} + 1)}{f_{\text{clock}}}
\]

\begin{example}[Timer Configuration for 1 kHz Control Loop]
For an STM32 running at 168 MHz, to generate a 1 kHz (1 ms period) interrupt:
\begin{align*}
f_{\text{clock}} &= 168\,\text{MHz} \\
T_{\text{desired}} &= 1\,\text{ms}
\end{align*}

Choose PSC = 167 (divides by 168):
\[
f_{\text{timer}} = \frac{168\,\text{MHz}}{168} = 1\,\text{MHz}
\]

Then ARR = 999 gives period:
\[
T = \frac{1000}{1\,\text{MHz}} = 1\,\text{ms} \quad \checkmark
\]
\end{example}

\begin{lstlisting}[language=C, caption=Timer initialization for 1 kHz interrupt]
void Timer_Init_1kHz(void)
{
    // Enable clock to TIM2
    RCC->APB1ENR |= RCC_APB1ENR_TIM2EN;

    // Configure timer: 168 MHz / 168 = 1 MHz tick
    TIM2->PSC = 167;     // Prescaler (divides by 168)
    TIM2->ARR = 999;     // Auto-reload (period of 1000 ticks = 1 ms)

    // Enable update interrupt
    TIM2->DIER |= TIM_DIER_UIE;

    // Enable interrupt in NVIC
    NVIC_SetPriority(TIM2_IRQn, 5);
    NVIC_EnableIRQ(TIM2_IRQn);

    // Start timer
    TIM2->CR1 |= TIM_CR1_CEN;
}

// Timer interrupt handler
void TIM2_IRQHandler(void)
{
    if (TIM2->SR & TIM_SR_UIF) {
        TIM2->SR &= ~TIM_SR_UIF;  // Clear interrupt flag

        // Signal control task
        BaseType_t woken = pdFALSE;
        xSemaphoreGiveFromISR(xControlSemaphore, &woken);
        portYIELD_FROM_ISR(woken);
    }
}
\end{lstlisting}

\subsubsection{PWM Generation for Motor Control}

PWM output uses the timer's capture/compare functionality. When the counter value matches CCRx, the output toggles:

See Listing~\ref{lst:pwm-motors} (Appendix~\ref{app:code-listings}) for complete PWM initialization and motor control functions using STM32 TIM1 for 400 Hz PWM generation.

\begin{notebox}[title=PWM Resolution and Frequency Trade-off]
PWM resolution depends on the timer frequency and PWM period:
\[
\text{Resolution} = \frac{1}{\text{ARR} + 1} = \frac{f_{\text{PWM}}}{f_{\text{timer}}}
\]

For motor control, you need enough resolution to smoothly vary thrust. With 1 MHz timer and 400 Hz PWM:
\[
\text{Resolution} = \frac{400}{1\,000\,000} = 0.04\%
\]

This provides 1000 discrete steps between minimum and maximum throttle (1000--2000 $\mu$s), which is more than adequate for motor control.
\end{notebox}

\subsubsection{Input Capture for RC Receivers}

Input capture measures the time between edges of an input signal. This is used to decode PWM signals from RC receivers:

\begin{lstlisting}[language=C, caption=Input capture for RC receiver]
volatile uint16_t rcPulseWidth[8];
volatile uint32_t risingEdge[8];

void InputCapture_Init(void)
{
    // Configure TIM3 for input capture
    RCC->APB1ENR |= RCC_APB1ENR_TIM3EN;

    // 1 MHz timer (1 us resolution)
    TIM3->PSC = 167;
    TIM3->ARR = 0xFFFF;

    // Channel 1: Input capture on rising and falling edges
    TIM3->CCMR1 |= TIM_CCMR1_CC1S_0;  // IC1 mapped to TI1
    TIM3->CCER |= TIM_CCER_CC1E | TIM_CCER_CC1P | TIM_CCER_CC1NP;

    // Enable capture interrupt
    TIM3->DIER |= TIM_DIER_CC1IE;
    NVIC_EnableIRQ(TIM3_IRQn);

    TIM3->CR1 |= TIM_CR1_CEN;
}

void TIM3_IRQHandler(void)
{
    if (TIM3->SR & TIM_SR_CC1IF) {
        TIM3->SR &= ~TIM_SR_CC1IF;

        uint16_t capture = TIM3->CCR1;

        if (GPIOA->IDR & GPIO_IDR_6) {  // Rising edge
            risingEdge[0] = capture;
        } else {  // Falling edge
            rcPulseWidth[0] = capture - risingEdge[0];
        }
    }
}
\end{lstlisting}

\subsection{Direct Memory Access (DMA)}
\index{DMA (Direct Memory Access)}

DMA\index{DMA (Direct Memory Access)|textbf} allows data transfer between peripherals and memory without CPU involvement. This is critical for efficient sensor data acquisition and motor control in flight controllers.

\subsubsection{Why DMA Matters for Quadrotors}

Consider reading the IMU over SPI at 1 kHz. Without DMA:
\begin{itemize}
    \item CPU must wait for each byte transfer (SPI at 10 MHz = 0.8 $\mu$s per byte)
    \item Reading 14 bytes (3-axis accel + 3-axis gyro + temp) takes $\approx$11 $\mu$s of CPU time
    \item At 1 kHz, this is 1.1\% CPU utilization just for data transfer
    \item CPU cannot do anything else during the transfer
\end{itemize}

With DMA:
\begin{itemize}
    \item CPU initiates transfer and continues other work
    \item DMA controller handles byte-by-byte transfer
    \item CPU interrupted only when entire transfer completes
    \item CPU utilization for transfer approaches zero
\end{itemize}

\begin{keyidea}[title=DMA Philosophy]
DMA moves data between memory and peripherals \textbf{in the background}, freeing the CPU for computation. This is especially important for real-time systems where predictable timing is essential.
\end{keyidea}

\subsubsection{DMA Architecture}

A DMA controller contains multiple \textbf{channels} (or \textbf{streams} on STM32F4). Each channel can be configured for:

\begin{itemize}
    \item \textbf{Source address}: Where to read data from (peripheral or memory)
    \item \textbf{Destination address}: Where to write data to
    \item \textbf{Transfer size}: Number of bytes/halfwords/words to transfer
    \item \textbf{Direction}: Peripheral-to-memory, memory-to-peripheral, or memory-to-memory
    \item \textbf{Mode}: Single transfer or circular (auto-restart)
\end{itemize}

See Listing~\ref{lst:dma-spi} (Appendix~\ref{app:code-listings}) for complete DMA configuration for SPI IMU reading, including TX/RX stream setup and interrupt handling with FreeRTOS signaling.

\subsubsection{Circular DMA for Continuous Acquisition}

For continuous sensor data acquisition, circular DMA automatically restarts when the transfer completes:

\begin{lstlisting}[language=C, caption=Circular DMA for ADC battery monitoring]
#define ADC_BUFFER_SIZE 16
uint16_t adcBuffer[ADC_BUFFER_SIZE];

void ADC_DMA_CircularInit(void)
{
    // Enable clocks
    RCC->AHB1ENR |= RCC_AHB1ENR_DMA2EN;
    RCC->APB2ENR |= RCC_APB2ENR_ADC1EN;

    // Configure DMA for circular mode
    DMA2_Stream0->CR = 0;
    while (DMA2_Stream0->CR & DMA_SxCR_EN);

    DMA2_Stream0->PAR = (uint32_t)&ADC1->DR;
    DMA2_Stream0->M0AR = (uint32_t)adcBuffer;
    DMA2_Stream0->NDTR = ADC_BUFFER_SIZE;

    DMA2_Stream0->CR = (0 << 25) |        // Channel 0
                       DMA_SxCR_MSIZE_0 |  // 16-bit memory
                       DMA_SxCR_PSIZE_0 |  // 16-bit peripheral
                       DMA_SxCR_MINC |     // Memory increment
                       DMA_SxCR_CIRC |     // Circular mode
                       DMA_SxCR_HTIE |     // Half-transfer interrupt
                       DMA_SxCR_TCIE;      // Full-transfer interrupt

    // Enable DMA stream
    DMA2_Stream0->CR |= DMA_SxCR_EN;

    // Configure ADC for continuous conversion with DMA
    ADC1->CR2 = ADC_CR2_DMA | ADC_CR2_DDS | ADC_CR2_CONT;
    ADC1->CR2 |= ADC_CR2_ADON;
    ADC1->CR2 |= ADC_CR2_SWSTART;
}
\end{lstlisting}

\subsubsection{Double Buffering}

For high-rate sensor fusion, double buffering prevents data corruption:

\begin{lstlisting}[language=C, caption=Double-buffered DMA for continuous IMU acquisition]
uint8_t imuBuffer0[14];
uint8_t imuBuffer1[14];
volatile uint8_t activeBuffer = 0;

void DMA_DoubleBuffer_Init(void)
{
    DMA2_Stream0->CR |= DMA_SxCR_DBM;  // Enable double buffer mode
    DMA2_Stream0->M0AR = (uint32_t)imuBuffer0;
    DMA2_Stream0->M1AR = (uint32_t)imuBuffer1;
}

void DMA2_Stream0_IRQHandler(void)
{
    if (DMA2->LISR & DMA_LISR_TCIF0) {
        DMA2->LIFCR = DMA_LIFCR_CTCIF0;

        // Determine which buffer just completed
        if (DMA2_Stream0->CR & DMA_SxCR_CT) {
            // Buffer 1 active, Buffer 0 complete - process buffer 0
            activeBuffer = 0;
        } else {
            // Buffer 0 active, Buffer 1 complete - process buffer 1
            activeBuffer = 1;
        }

        // Signal task with completed buffer
        BaseType_t woken = pdFALSE;
        xSemaphoreGiveFromISR(xImuDataReady, &woken);
        portYIELD_FROM_ISR(woken);
    }
}

// In processing task
void vImuTask(void *pvParameters)
{
    for (;;) {
        xSemaphoreTake(xImuDataReady, portMAX_DELAY);

        // Process the completed buffer (not the one DMA is filling)
        uint8_t *buffer = (activeBuffer == 0) ? imuBuffer0 : imuBuffer1;
        ProcessImuData(buffer);
    }
}
\end{lstlisting}

\subsection{Timer-Triggered DMA}

Combining timers with DMA enables precise periodic data transfer without CPU intervention:

\begin{lstlisting}[language=C, caption=Timer-triggered DMA for synchronized motor updates]
// Motor commands updated atomically via DMA
uint16_t motorCommands[4];

void TimerDMA_Init(void)
{
    // Configure TIM2 to trigger DMA at control rate (1 kHz)
    TIM2->PSC = 167;
    TIM2->ARR = 999;

    // Enable DMA request on update
    TIM2->DIER |= TIM_DIER_UDE;

    // Configure DMA to transfer motor commands to PWM registers
    // This ensures all 4 motors update simultaneously
    DMA1_Stream6->PAR = (uint32_t)&TIM1->CCR1;
    DMA1_Stream6->M0AR = (uint32_t)motorCommands;
    DMA1_Stream6->NDTR = 4;

    DMA1_Stream6->CR = DMA_SxCR_MSIZE_0 |  // 16-bit
                       DMA_SxCR_PSIZE_0 |
                       DMA_SxCR_MINC |
                       DMA_SxCR_PINC |      // Increment through CCR1-4
                       (1 << 6) |           // Memory to peripheral
                       DMA_SxCR_CIRC;       // Circular

    DMA1_Stream6->CR |= DMA_SxCR_EN;
    TIM2->CR1 |= TIM_CR1_CEN;
}

// Control task just updates the buffer
void vControlTask(void *pvParameters)
{
    for (;;) {
        xSemaphoreTake(xControlSemaphore, portMAX_DELAY);

        // Compute motor commands
        float thrust[4];
        ComputeMotorThrust(thrust);

        // Convert to PWM values (DMA transfers on next timer update)
        for (int i = 0; i < 4; i++) {
            motorCommands[i] = 1000 + (uint16_t)(thrust[i] * 1000.0f);
        }
    }
}
\end{lstlisting}

\begin{keyidea}[title=Complete Timing Chain]
For deterministic quadrotor control:
\begin{enumerate}
    \item Timer interrupt triggers at exact control rate (e.g., 1 kHz)
    \item DMA automatically reads sensor data without CPU
    \item Transfer-complete interrupt signals control task
    \item Control task computes new motor commands
    \item Timer-triggered DMA updates all motor PWM outputs simultaneously
\end{enumerate}

This approach minimizes jitter and ensures consistent timing regardless of other system activity.
\end{keyidea}

\subsection{Practical Considerations}

\subsubsection{DMA and Cache Coherency}

On processors with data cache (e.g., Cortex-M7), DMA can cause cache coherency issues:

\begin{lstlisting}[language=C, caption=Cache management for DMA buffers]
// Place DMA buffers in non-cacheable memory
__attribute__((section(".dma_buffer")))
uint8_t imuDmaBuffer[14];

// Or manually manage cache
void ProcessDmaBuffer(void)
{
    // Invalidate cache before reading DMA-filled buffer
    SCB_InvalidateDCache_by_Addr((uint32_t*)imuDmaBuffer, 14);

    // Now CPU reads correct data
    ProcessImuData(imuDmaBuffer);
}

void PrepareTxBuffer(void)
{
    // Fill buffer
    PrepareTelemetryData(txBuffer);

    // Clean cache to ensure DMA reads correct data
    SCB_CleanDCache_by_Addr((uint32_t*)txBuffer, TX_BUFFER_SIZE);

    // Start DMA
    StartTelemetryDma();
}
\end{lstlisting}

\subsubsection{DMA Transfer Errors}

DMA can encounter errors (bus error, FIFO overrun). Always check for errors:

\begin{lstlisting}[language=C, caption=DMA error handling]
void DMA2_Stream0_IRQHandler(void)
{
    // Check for errors first
    if (DMA2->LISR & (DMA_LISR_TEIF0 | DMA_LISR_DMEIF0 | DMA_LISR_FEIF0)) {
        // Clear error flags
        DMA2->LIFCR = DMA_LIFCR_CTEIF0 | DMA_LIFCR_CDMEIF0 | DMA_LIFCR_CFEIF0;

        // Handle error (log, attempt recovery, or trigger failsafe)
        HandleDmaError();
        return;
    }

    // Normal completion
    if (DMA2->LISR & DMA_LISR_TCIF0) {
        DMA2->LIFCR = DMA_LIFCR_CTCIF0;
        // Process data...
    }
}
\end{lstlisting}

\subsubsection{Timing Measurement with Timers}

Timers are invaluable for measuring code execution time:

\begin{lstlisting}[language=C, caption=Using timers for WCET measurement]
// Use TIM5 as a free-running microsecond counter
void Timing_Init(void)
{
    RCC->APB1ENR |= RCC_APB1ENR_TIM5EN;
    TIM5->PSC = 167;     // 1 MHz (1 us resolution)
    TIM5->ARR = 0xFFFFFFFF;  // 32-bit timer, wraps every ~71 minutes
    TIM5->CR1 |= TIM_CR1_CEN;
}

static inline uint32_t GetMicros(void)
{
    return TIM5->CNT;
}

// In control task
void vControlTask(void *pvParameters)
{
    uint32_t worstCase = 0;

    for (;;) {
        xSemaphoreTake(xControlSemaphore, portMAX_DELAY);

        uint32_t start = GetMicros();

        // Run control algorithm
        RunAttitudeController();

        uint32_t elapsed = GetMicros() - start;
        if (elapsed > worstCase) {
            worstCase = elapsed;
            LogWcet(worstCase);
        }
    }
}
\end{lstlisting}

\section{Quadrotor Flight Controller Architecture}

This section brings together all concepts into a complete flight controller design.

\subsection{Task Structure}

%----------------------------------------------------------------------
% FIGURE: Complete Flight Controller Task Diagram
%----------------------------------------------------------------------
% Description: Block diagram showing all tasks and data flow.
%
% Tasks as boxes with priorities:
%   - IMU ISR -> Sensor Task (via semaphore)
%   - Sensor Task -> Fusion Task (via queue)
%   - Fusion Task -> Attitude Control Task (via shared state + mutex)
%   - Position sensor -> Position Control Task
%   - Position Control -> Attitude Control (setpoint)
%   - Attitude Control -> Motor Mixing -> PWM output
%   - Radio ISR -> Command Task -> Position/Attitude setpoints
%   - Logging Task reads from circular buffer
%
% Show priorities and approximate rates.
% Dimensions: Full width, ~8cm height.
%----------------------------------------------------------------------

\begin{lstlisting}[language=C, caption=Flight controller task creation]
void CreateFlightControllerTasks(void)
{
    // Create synchronization primitives
    xImuSemaphore = xSemaphoreCreateBinary();
    xImuQueue = xQueueCreate(5, sizeof(ImuData_t));
    xStateMutex = xSemaphoreCreateMutex();

    // Create tasks (highest priority first)
    xTaskCreate(vAttitudeControlTask, "AttCtrl", 512, NULL, 5, NULL);
    xTaskCreate(vSensorFusionTask,    "SensFus", 512, NULL, 5, NULL);
    xTaskCreate(vPositionControlTask, "PosCtrl", 512, NULL, 4, NULL);
    xTaskCreate(vCommunicationTask,   "Comm",    256, NULL, 3, NULL);
    xTaskCreate(vLoggingTask,         "Log",     512, NULL, 2, NULL);

    // Start scheduler
    vTaskStartScheduler();
}
\end{lstlisting}

\subsection{Sensor Acquisition Task}

\begin{lstlisting}[language=C, caption=Sensor acquisition task]
void vSensorAcquisitionTask(void *pvParameters)
{
    ImuRawData_t rawData;
    ImuCalibratedData_t calData;

    for (;;) {
        // Wait for IMU data-ready interrupt
        xSemaphoreTake(xImuSemaphore, portMAX_DELAY);

        // Read raw data from IMU (SPI transfer)
        IMU_ReadRegisters(&rawData);

        // Apply calibration
        ApplyImuCalibration(&rawData, &calData);

        // Send to sensor fusion task
        xQueueSend(xImuQueue, &calData, 0);
    }
}
\end{lstlisting}

\subsection{Sensor Fusion Task}

\begin{lstlisting}[language=C, caption=Sensor fusion task (Mahony filter)]
void vSensorFusionTask(void *pvParameters)
{
    ImuCalibratedData_t imuData;
    Quaternion_t q = {1, 0, 0, 0};  // Initial orientation

    for (;;) {
        // Get calibrated IMU data
        xQueueReceive(xImuQueue, &imuData, portMAX_DELAY);

        // Run Mahony filter
        MahonyAHRSupdate(
            imuData.gx, imuData.gy, imuData.gz,
            imuData.ax, imuData.ay, imuData.az,
            &q
        );

        // Update shared state (protected by mutex)
        xSemaphoreTake(xStateMutex, portMAX_DELAY);
        g_orientation = q;
        xSemaphoreGive(xStateMutex);
    }
}
\end{lstlisting}

\subsection{Attitude Control Task}

\begin{lstlisting}[language=C, caption=Attitude control task]
void vAttitudeControlTask(void *pvParameters)
{
    TickType_t xLastWakeTime = xTaskGetTickCount();
    Quaternion_t q;
    AttitudeSetpoint_t setpoint;
    TorqueCommand_t torque;
    MotorCommand_t motors;

    for (;;) {
        // Get current orientation (protected read)
        xSemaphoreTake(xStateMutex, portMAX_DELAY);
        q = g_orientation;
        setpoint = g_attitudeSetpoint;
        xSemaphoreGive(xStateMutex);

        // Compute attitude error
        Quaternion_t qError = QuaternionError(&q, &setpoint.q);

        // PID control for each axis
        torque.roll  = PID_Update(&pidRoll,  qError.x);
        torque.pitch = PID_Update(&pidPitch, qError.y);
        torque.yaw   = PID_Update(&pidYaw,   qError.z);
        torque.thrust = setpoint.thrust;

        // Motor mixing
        MotorMix(&torque, &motors);

        // Output to motors
        SetMotorPWM(&motors);

        // Maintain 500 Hz rate
        vTaskDelayUntil(&xLastWakeTime, pdMS_TO_TICKS(2));
    }
}
\end{lstlisting}

\subsection{Position Control Task}

\begin{lstlisting}[language=C, caption=Position control task (outer loop)]
void vPositionControlTask(void *pvParameters)
{
    TickType_t xLastWakeTime = xTaskGetTickCount();
    Position_t currentPos, desiredPos;
    Velocity_t currentVel;
    AttitudeSetpoint_t attSetpoint;

    for (;;) {
        // Get current state
        GetPositionEstimate(&currentPos, &currentVel);
        GetDesiredPosition(&desiredPos);

        // Position control -> desired acceleration
        Vector3_t accelDesired;
        accelDesired.x = PID_Update(&pidX, desiredPos.x - currentPos.x);
        accelDesired.y = PID_Update(&pidY, desiredPos.y - currentPos.y);
        accelDesired.z = PID_Update(&pidZ, desiredPos.z - currentPos.z);

        // Convert acceleration to attitude setpoint
        // (This is the "control allocation" for position->attitude)
        attSetpoint.roll  = accelDesired.y / GRAVITY;
        attSetpoint.pitch = -accelDesired.x / GRAVITY;
        attSetpoint.thrust = (accelDesired.z + GRAVITY) * MASS;

        // Update attitude setpoint for inner loop
        xSemaphoreTake(xStateMutex, portMAX_DELAY);
        g_attitudeSetpoint = attSetpoint;
        xSemaphoreGive(xStateMutex);

        // Maintain 50 Hz rate (outer loop is slower)
        vTaskDelayUntil(&xLastWakeTime, pdMS_TO_TICKS(20));
    }
}
\end{lstlisting}

\subsection{Timing Budget}

\begin{center}
\begin{tabular}{lcccc}
\toprule
\textbf{Task} & \textbf{Period} & \textbf{WCET} & \textbf{Util.} & \textbf{Priority} \\
\midrule
IMU ISR & Event & 5 $\mu$s & <1\% & (HW) \\
Sensor Acquisition & 2 ms & 100 $\mu$s & 5\% & 5 \\
Sensor Fusion & 2 ms & 150 $\mu$s & 7.5\% & 5 \\
Attitude Control & 2 ms & 200 $\mu$s & 10\% & 5 \\
Position Control & 20 ms & 500 $\mu$s & 2.5\% & 4 \\
Communication & 50 ms & 2 ms & 4\% & 3 \\
Logging & 100 ms & 10 ms & 10\% & 2 \\
\midrule
\textbf{Total} & & & \textbf{39\%} & \\
\bottomrule
\end{tabular}
\end{center}

With 39\% CPU utilization, we have significant margin for:
\begin{itemize}
    \item WCET variation
    \item Additional features
    \item Debugging/tracing
\end{itemize}

\section{Debugging and Validation}

Real-time systems require special debugging techniques because timing bugs may not appear in normal debugging.

\subsection{Stack Usage Analysis}

Each task needs sufficient stack space. Too little causes stack overflow (often silent corruption); too much wastes memory.

\begin{lstlisting}[language=C, caption=Checking stack high water mark]
// Get minimum free stack (in words) since task started
UBaseType_t stackRemaining = uxTaskGetStackHighWaterMark(NULL);

// If this is small (< 50 words), increase stack size!
printf("Stack remaining: %lu words\n", stackRemaining);
\end{lstlisting}

Enable stack overflow checking in \texttt{FreeRTOSConfig.h}:
\begin{lstlisting}[language=C]
#define configCHECK_FOR_STACK_OVERFLOW  2
\end{lstlisting}

\subsection{CPU Utilization Monitoring}

\begin{lstlisting}[language=C, caption=Runtime statistics (requires configuration)]
#define configGENERATE_RUN_TIME_STATS   1
#define configUSE_STATS_FORMATTING_FUNCTIONS  1

// In your code:
char statsBuffer[512];
vTaskGetRunTimeStats(statsBuffer);
printf("%s", statsBuffer);
\end{lstlisting}

\subsection{Detecting Deadline Misses}

\begin{lstlisting}[language=C, caption=Simple deadline miss detection]
void vControlTask(void *pvParameters)
{
    TickType_t xLastWakeTime = xTaskGetTickCount();
    const TickType_t xPeriod = pdMS_TO_TICKS(2);

    for (;;) {
        // Do control work...

        // Check if we're late
        TickType_t now = xTaskGetTickCount();
        TickType_t expected = xLastWakeTime + xPeriod;

        if (now > expected) {
            // Deadline miss! Log it
            g_deadlineMissCount++;
        }

        vTaskDelayUntil(&xLastWakeTime, xPeriod);
    }
}
\end{lstlisting}

\subsection{Trace Tools}

For detailed timing analysis, use trace tools:

\begin{itemize}
    \item \textbf{Segger SystemView}: Free, works with FreeRTOS, shows task execution timeline
    \item \textbf{Tracealyzer}: Commercial, powerful visualization and analysis
    \item \textbf{Logic analyzer}: Hardware-based, toggle GPIO pins at task entry/exit
\end{itemize}

\begin{lstlisting}[language=C, caption=GPIO-based timing measurement]
#define TIMING_PIN_ATTITUDE  GPIO_PIN_0
#define TIMING_PIN_POSITION  GPIO_PIN_1

void vAttitudeControlTask(void *pvParameters)
{
    for (;;) {
        GPIO_SetPin(TIMING_PIN_ATTITUDE);   // Rising edge: start

        // Do control work...

        GPIO_ClearPin(TIMING_PIN_ATTITUDE); // Falling edge: end

        vTaskDelayUntil(&xLastWakeTime, pdMS_TO_TICKS(2));
    }
}
\end{lstlisting}

Use an oscilloscope or logic analyzer to measure the pulse width (execution time) and period.

%======================================================================
% PART II: Advanced Concurrency and Scheduling
%======================================================================

\chapter{Concurrency and Synchronization}

\section{Introduction: Why Concurrency Matters for Quadrotors}

In the previous chapter, we introduced the Real-Time Operating System (RTOS) that manages multiple tasks on a single processor. Now we examine the challenges that arise when these tasks need to share data and coordinate their execution.

\textbf{The fundamental problem}: A quadrotor flight controller has multiple tasks that must share information:
\begin{itemize}
    \item The sensor fusion task computes orientation estimates
    \item The attitude control task reads these estimates to compute motor commands
    \item The position control task reads position estimates and writes attitude setpoints
    \item The telemetry task reads all state data for transmission
\end{itemize}

These tasks run concurrently (interleaved by the scheduler) and access shared data. Without careful design, this concurrent access causes subtle bugs that can crash the quadrotor.

\begin{keyidea}[title=Concurrency vs. Parallelism]
On a single-core processor (like the Crazyflie's STM32), tasks do not run \textbf{in parallel}---only one task executes at any instant. However, tasks are \textbf{concurrent}: from each task's perspective, other tasks can make progress at unpredictable times (whenever the scheduler preempts).

This distinction matters because concurrency bugs can occur even without true parallelism. The scheduler can preempt a task at any machine instruction, creating interleaving patterns that cause incorrect behavior.
\end{keyidea}

\section{Race Conditions and Data Races}
\index{race condition}

\subsection{The Shared Data Problem}

Consider two tasks that both access a shared variable:

\begin{example}[Race Condition]
Suppose we have a shared counter \texttt{x = 0} and two tasks that each increment it:

\begin{center}
\begin{tabular}{cc}
\textbf{Task T1} & \textbf{Task T2} \\
\texttt{temp1 = x;} & \texttt{temp2 = x;} \\
\texttt{temp1 = temp1 + 1;} & \texttt{temp2 = temp2 + 1;} \\
\texttt{x = temp1;} & \texttt{x = temp2;} \\
\end{tabular}
\end{center}

After both tasks complete, what is the value of \texttt{x}?

\textbf{Expected}: 2 (each task increments once)

\textbf{Possible outcomes}: 1 or 2, depending on interleaving:
\begin{itemize}
    \item T1 runs completely, then T2: $x = 2$ \checkmark
    \item T1 reads (temp1=0), T2 runs completely (x=1), T1 writes (x=1): $x = 1$ \texttimes
\end{itemize}

The second outcome is a \textbf{race condition}---the result depends on the order of execution.
\end{example}

\textbf{Why this matters for quadrotors}: The increment example seems artificial, but the same pattern occurs in real flight controllers:

\begin{lstlisting}[language=C, caption=Race condition in flight controller]
// Shared state
typedef struct {
    float w, x, y, z;
} Quaternion_t;
Quaternion_t orientation;  // Written by sensor fusion, read by control

// Sensor fusion task (writer)
void SensorFusionTask(void) {
    Quaternion_t newQ = ComputeOrientation();
    // What if preempted here?
    orientation.w = newQ.w;
    orientation.x = newQ.x;  // <-- Preemption point
    orientation.y = newQ.y;
    orientation.z = newQ.z;
}

// Control task (reader)
void ControlTask(void) {
    Quaternion_t q = orientation;  // May read partial update!
    // q might have new w,x but old y,z -> invalid quaternion
    ComputeControl(&q);
}
\end{lstlisting}

If the control task preempts the sensor fusion task between writes, it reads an \textbf{inconsistent} quaternion---some components from the old estimate, some from the new. This inconsistent state can cause erratic control behavior or even crashes.

\begin{definition}[Data Race]
A \textbf{data race} occurs when:
\begin{enumerate}
    \item Two or more tasks access the same memory location
    \item At least one access is a write
    \item The accesses are not synchronized
\end{enumerate}
Data races cause \textbf{undefined behavior}---the program may work correctly most of the time but fail unpredictably.
\end{definition}

\subsection{Why Race Conditions Are Dangerous}

Race conditions are among the most insidious bugs in software:

\begin{itemize}
    \item \textbf{Non-deterministic}: The bug only manifests with specific timing, which may occur rarely
    \item \textbf{Hard to reproduce}: Running the same test twice may give different results
    \item \textbf{Timing-dependent}: Adding debug output (printf) can change timing enough to hide the bug
    \item \textbf{Platform-dependent}: May work on one processor speed but fail on another
    \item \textbf{Catastrophic}: In a flight controller, a race condition can cause a crash
\end{itemize}

\begin{warningbox}[title=The Heisenbug Phenomenon]
Race conditions often exhibit ``Heisenbug'' behavior: they disappear when you try to observe them. Adding logging, running in a debugger, or changing compiler optimization can all affect timing enough to mask the bug. This makes race conditions extremely difficult to diagnose.

The only reliable solution is to prevent race conditions through proper synchronization, not to try to find and fix them after the fact.
\end{warningbox}

\section{Critical Sections and Mutual Exclusion}

\subsection{The Critical Section Problem}

\begin{definition}[Critical Section]
A \textbf{critical section} is a code region that accesses shared resources and must not be executed by more than one task simultaneously.
\end{definition}

\begin{definition}[Mutual Exclusion]
\textbf{Mutual exclusion} ensures that when one task is executing in a critical section, no other task can enter its critical section for the same shared resource.
\end{definition}

\textbf{Requirements for a correct mutual exclusion solution}:
\begin{enumerate}
    \item \textbf{Safety (Mutual Exclusion)}: At most one task is in the critical section at any time
    \item \textbf{Liveness (Progress)}: If no task is in the critical section and some task wants to enter, it will eventually enter
    \item \textbf{Bounded Waiting}: A task waiting to enter the critical section will not wait forever
\end{enumerate}

\subsection{Hardware Support: Atomic Operations}

Modern processors provide hardware support for synchronization through \textbf{atomic operations}---operations that complete without interruption.

\begin{definition}[Test-and-Set]
The \textbf{Test-and-Set} instruction atomically:
\begin{enumerate}
    \item Reads the current value of a memory location
    \item Writes 1 to that location
    \item Returns the old value
\end{enumerate}
This happens as a single, uninterruptible operation.
\end{definition}

\begin{lstlisting}[language=C, caption=Test-and-Set conceptual implementation]
// This executes atomically (cannot be interrupted)
int TestAndSet(int *lock) {
    int old = *lock;
    *lock = 1;
    return old;
}

// Usage for mutual exclusion
void EnterCriticalSection(int *lock) {
    while (TestAndSet(lock) == 1) {
        // Spin until lock is free
    }
}

void ExitCriticalSection(int *lock) {
    *lock = 0;
}
\end{lstlisting}

\textbf{Why hardware support is necessary}: Without atomic operations, any software-only solution requires multiple instructions that can be interrupted. The Test-and-Set instruction is specifically designed to be uninterruptible, providing a foundation for building higher-level synchronization primitives.

ARM Cortex-M processors provide similar functionality through the \texttt{LDREX}/\texttt{STREX} (Load-Exclusive/Store-Exclusive) instructions.

\section{Advanced Synchronization Primitives}

\subsection{Semaphores}

\begin{definition}[Semaphore]
A \textbf{semaphore} is a synchronization primitive consisting of:
\begin{itemize}
    \item An integer counter (not directly accessible)
    \item Two atomic operations: \textbf{wait} (P/take) and \textbf{signal} (V/give)
\end{itemize}

\textbf{Wait (Take)}: If counter $> 0$, decrement and continue. If counter $\leq 0$, block until another task signals.

\textbf{Signal (Give)}: Increment counter. If tasks are waiting, wake one of them.
\end{definition}

\textbf{Intuition}: Think of a semaphore as a bucket of tokens. ``Take'' removes a token (blocking if none available). ``Give'' adds a token (potentially waking a waiting task).

The semaphore was invented by Edsger Dijkstra in 1965. The names P and V come from Dutch: ``P'' for \emph{proberen} (try) and ``V'' for \emph{verhogen} (increment).

\begin{lstlisting}[language=C, caption=FreeRTOS semaphore operations]
// Create a binary semaphore (initial value 0)
SemaphoreHandle_t sem = xSemaphoreCreateBinary();

// Create a counting semaphore (initial value n, max value m)
SemaphoreHandle_t sem = xSemaphoreCreateCounting(m, n);

// Wait (take) - blocks if semaphore is 0
xSemaphoreTake(sem, portMAX_DELAY);  // Wait forever
xSemaphoreTake(sem, pdMS_TO_TICKS(100));  // Wait up to 100ms

// Signal (give)
xSemaphoreGive(sem);

// From ISR (special versions that don't block)
xSemaphoreGiveFromISR(sem, &xHigherPriorityTaskWoken);
\end{lstlisting}

\subsection{Types of Semaphores}

\textbf{Binary Semaphore}: Counter is 0 or 1. Used for signaling and simple mutual exclusion.

\textbf{Counting Semaphore}: Counter can be any non-negative integer. Used for:
\begin{itemize}
    \item Counting events (e.g., number of items in a buffer)
    \item Managing multiple identical resources (e.g., pool of buffers)
\end{itemize}

\subsection{Mutexes}

\begin{definition}[Mutex]
A \textbf{mutex} (mutual exclusion) is a binary semaphore with additional semantics:
\begin{itemize}
    \item \textbf{Ownership}: Only the task that acquired the mutex can release it
    \item \textbf{Priority inheritance}: Prevents priority inversion (explained later)
    \item \textbf{Recursive locking}: Some implementations allow the owner to acquire multiple times
\end{itemize}
\end{definition}

\begin{lstlisting}[language=C, caption=FreeRTOS mutex usage]
// Create mutex (initial state: available)
SemaphoreHandle_t mutex = xSemaphoreCreateMutex();

// Acquire mutex
xSemaphoreTake(mutex, portMAX_DELAY);

// Critical section - access shared data
orientation.w = newQ.w;
orientation.x = newQ.x;
orientation.y = newQ.y;
orientation.z = newQ.z;

// Release mutex
xSemaphoreGive(mutex);
\end{lstlisting}

\begin{keyidea}[title=When to Use Mutex vs. Binary Semaphore]
\begin{itemize}
    \item \textbf{Mutex}: Protecting shared data (one task acquires, same task releases)
    \item \textbf{Binary Semaphore}: Signaling between tasks or from ISR to task (one task/ISR gives, different task takes)
\end{itemize}

In FreeRTOS, mutexes include priority inheritance but binary semaphores do not. Always use a mutex for protecting shared data to avoid priority inversion.
\end{keyidea}

\subsection{Queues}

\begin{definition}[Queue]
A \textbf{queue} (message queue) is a FIFO buffer that allows tasks to send and receive data items. Queues handle synchronization internally:
\begin{itemize}
    \item Sending to a full queue can block until space is available
    \item Receiving from an empty queue can block until data is available
\end{itemize}
\end{definition}

\begin{lstlisting}[language=C, caption=FreeRTOS queue for IMU data]
// Create queue holding 5 IMU samples
QueueHandle_t imuQueue = xQueueCreate(5, sizeof(ImuData_t));

// Producer (sensor task)
void SensorTask(void *pvParameters) {
    ImuData_t data;
    for (;;) {
        ReadImuHardware(&data);
        xQueueSend(imuQueue, &data, portMAX_DELAY);
    }
}

// Consumer (fusion task)
void FusionTask(void *pvParameters) {
    ImuData_t data;
    for (;;) {
        xQueueReceive(imuQueue, &data, portMAX_DELAY);
        UpdateOrientationEstimate(&data);
    }
}
\end{lstlisting}

\textbf{Advantages of queues}:
\begin{itemize}
    \item Data is copied into the queue, avoiding shared memory issues
    \item Built-in blocking semantics for producer-consumer patterns
    \item Natural buffering handles timing variations between tasks
\end{itemize}

\section{Synchronization Patterns}

\subsection{Pattern 1: Signaling (Event Notification)}

One task signals an event; another task waits for it.

\begin{lstlisting}[language=C, caption=Signaling pattern: ISR to task]
SemaphoreHandle_t imuDataReady;  // Initialize to 0

// ISR: Signal that data is ready
void IMU_IRQHandler(void) {
    ClearInterruptFlag();
    BaseType_t woken = pdFALSE;
    xSemaphoreGiveFromISR(imuDataReady, &woken);
    portYIELD_FROM_ISR(woken);
}

// Task: Wait for data
void SensorTask(void *pvParameters) {
    for (;;) {
        xSemaphoreTake(imuDataReady, portMAX_DELAY);
        // ISR has signaled - data is ready
        ProcessImuData();
    }
}
\end{lstlisting}

\textbf{Quadrotor application}: The IMU generates a data-ready interrupt at a fixed rate (e.g., 1 kHz). The ISR signals a semaphore, waking the sensor processing task. This ensures the task runs exactly when new data is available.

\subsection{Pattern 2: Mutual Exclusion (Protecting Shared Data)}

Multiple tasks access shared data; only one can access at a time.

\begin{lstlisting}[language=C, caption=Mutex pattern: protecting shared state]
SemaphoreHandle_t stateMutex;  // Initialize as mutex
StateEstimate_t sharedState;   // Shared data

// Writer task (sensor fusion)
void FusionTask(void *pvParameters) {
    StateEstimate_t localState;
    for (;;) {
        // Compute new state estimate (no lock needed)
        ComputeStateEstimate(&localState);

        // Update shared state (lock required)
        xSemaphoreTake(stateMutex, portMAX_DELAY);
        sharedState = localState;  // Copy entire struct
        xSemaphoreGive(stateMutex);
    }
}

// Reader task (control)
void ControlTask(void *pvParameters) {
    StateEstimate_t localCopy;
    for (;;) {
        // Read shared state (lock required)
        xSemaphoreTake(stateMutex, portMAX_DELAY);
        localCopy = sharedState;  // Copy to local
        xSemaphoreGive(stateMutex);

        // Use local copy (no lock needed)
        ComputeControl(&localCopy);

        vTaskDelayUntil(...);
    }
}
\end{lstlisting}

\begin{keyidea}[title=Minimize Lock Hold Time]
Hold the mutex only while accessing shared data. Do computation before acquiring and after releasing the lock. This reduces the time other tasks are blocked and improves system responsiveness.
\end{keyidea}

\subsection{Pattern 3: Rendezvous (Barrier Synchronization)}

Two tasks must wait for each other before proceeding.

\begin{lstlisting}[language=C, caption=Rendezvous pattern]
SemaphoreHandle_t taskAReady;  // Initialize to 0
SemaphoreHandle_t taskBReady;  // Initialize to 0

void TaskA(void *pvParameters) {
    for (;;) {
        DoWorkA_Part1();

        // Signal that A is ready, wait for B
        xSemaphoreGive(taskAReady);
        xSemaphoreTake(taskBReady, portMAX_DELAY);

        // Both tasks have reached this point
        DoWorkA_Part2();
    }
}

void TaskB(void *pvParameters) {
    for (;;) {
        DoWorkB_Part1();

        // Signal that B is ready, wait for A
        xSemaphoreGive(taskBReady);
        xSemaphoreTake(taskAReady, portMAX_DELAY);

        // Both tasks have reached this point
        DoWorkB_Part2();
    }
}
\end{lstlisting}

\textbf{Quadrotor application}: Ensuring attitude control and position control tasks synchronize at specific points, or coordinating sensor calibration across multiple sensor tasks.

\begin{warningbox}[title=Deadlock in Rendezvous]
The order of \texttt{give} and \texttt{take} matters! If both tasks did \texttt{take} before \texttt{give}, they would both block forever waiting for each other. Always signal first, then wait.
\end{warningbox}


\chapter{Deadlocks and Priority Inversion}
\index{deadlock}

\section{Deadlocks}

\begin{definition}[Deadlock]
A \textbf{deadlock}\index{deadlock|textbf} is a situation where two or more tasks are blocked forever, each waiting for a resource held by another task in the cycle.
\end{definition}

\subsection{Deadlock Example}

\begin{lstlisting}[language=C, caption=Deadlock scenario]
SemaphoreHandle_t mutexA, mutexB;

void Task1(void *pvParameters) {
    for (;;) {
        xSemaphoreTake(mutexA, portMAX_DELAY);  // Acquire A
        vTaskDelay(1);  // Some work...
        xSemaphoreTake(mutexB, portMAX_DELAY);  // Try to acquire B

        // Critical section using both A and B

        xSemaphoreGive(mutexB);
        xSemaphoreGive(mutexA);
    }
}

void Task2(void *pvParameters) {
    for (;;) {
        xSemaphoreTake(mutexB, portMAX_DELAY);  // Acquire B
        vTaskDelay(1);  // Some work...
        xSemaphoreTake(mutexA, portMAX_DELAY);  // Try to acquire A

        // Critical section using both A and B

        xSemaphoreGive(mutexA);
        xSemaphoreGive(mutexB);
    }
}
\end{lstlisting}

If Task1 acquires mutexA and Task2 acquires mutexB, then:
\begin{itemize}
    \item Task1 blocks waiting for mutexB (held by Task2)
    \item Task2 blocks waiting for mutexA (held by Task1)
    \item Neither can proceed---deadlock!
\end{itemize}

\subsection{Necessary Conditions for Deadlock}

A deadlock can occur if and only if \textbf{all four} of the following conditions hold simultaneously:

\begin{enumerate}
    \item \textbf{Mutual Exclusion}: At least one resource must be held in a non-sharable mode
    \item \textbf{Hold and Wait}: A task holds at least one resource while waiting for additional resources
    \item \textbf{No Preemption}: Resources cannot be forcibly taken from a task; they must be released voluntarily
    \item \textbf{Circular Wait}: A cycle of tasks exists where each task waits for a resource held by the next task in the cycle
\end{enumerate}

\textbf{Intuition}: These conditions describe the ``perfect storm'' for deadlock. Breaking any one condition prevents deadlock.

\subsection{Deadlock Prevention Strategies}

\textbf{Strategy 1: Resource Ordering (Break Circular Wait)}

Assign a global ordering to all resources: $R_1 < R_2 < \ldots < R_n$. Every task must acquire resources in increasing order.

\begin{lstlisting}[language=C, caption=Resource ordering prevents deadlock]
// Convention: Always acquire mutexA before mutexB

void Task1(void *pvParameters) {
    xSemaphoreTake(mutexA, portMAX_DELAY);  // First A
    xSemaphoreTake(mutexB, portMAX_DELAY);  // Then B
    // ...
    xSemaphoreGive(mutexB);
    xSemaphoreGive(mutexA);
}

void Task2(void *pvParameters) {
    xSemaphoreTake(mutexA, portMAX_DELAY);  // First A (same order!)
    xSemaphoreTake(mutexB, portMAX_DELAY);  // Then B
    // ...
    xSemaphoreGive(mutexB);
    xSemaphoreGive(mutexA);
}
\end{lstlisting}

Circular wait is now impossible: if Task1 holds A and wants B, Task2 cannot hold B while wanting A (it would have to acquire A first).

\textbf{Strategy 2: Timeout (Break Hold and Wait)}

Use timeouts when acquiring resources. If acquisition fails, release all held resources and retry.

\begin{lstlisting}[language=C, caption=Timeout-based deadlock avoidance]
bool AcquireBothResources(void) {
    if (xSemaphoreTake(mutexA, pdMS_TO_TICKS(10)) == pdTRUE) {
        if (xSemaphoreTake(mutexB, pdMS_TO_TICKS(10)) == pdTRUE) {
            return true;  // Got both
        }
        xSemaphoreGive(mutexA);  // Release A, retry later
    }
    return false;
}
\end{lstlisting}

\textbf{Strategy 3: Single Lock (Break Multiple Resources)}

Use a single mutex to protect all related shared data, eliminating the need for multiple locks.

\begin{keyidea}[title=Practical Deadlock Prevention for Quadrotors]
For flight controllers, the simplest approach is often:
\begin{enumerate}
    \item Minimize the number of mutexes (one per major subsystem)
    \item If multiple mutexes are needed, document and enforce acquisition order
    \item Use queues instead of shared memory where possible (no locks needed)
    \item Keep critical sections short to reduce contention
\end{enumerate}
\end{keyidea}

\section{Priority Inversion}

\begin{definition}[Priority Inversion]
\textbf{Priority inversion} occurs when a high-priority task is blocked waiting for a resource held by a low-priority task, while medium-priority tasks (that don't need the resource) run instead of the low-priority task.
\end{definition}

This effectively ``inverts'' the priority order: the high-priority task waits for medium-priority tasks to complete, even though those medium-priority tasks have no direct relationship to the blocking.

\subsection{Priority Inversion Scenario}

Consider three tasks with priorities High $>$ Medium $>$ Low, and a shared resource protected by a mutex:

\begin{enumerate}
    \item \textbf{Time 0}: Task Low starts executing and acquires the mutex
    \item \textbf{Time 1}: Task High becomes ready, preempts Low, and tries to acquire the mutex---blocks (Low holds it)
    \item \textbf{Time 2}: Task Medium becomes ready; since High is blocked and Medium $>$ Low, Medium runs
    \item \textbf{Time 3--100}: Medium continues running (perhaps for a long time)
    \item \textbf{Time 101}: Medium finishes; Low resumes and eventually releases mutex
    \item \textbf{Time 102}: High finally acquires mutex and runs
\end{enumerate}

\textbf{The problem}: High was delayed not just by Low (which legitimately held the resource) but by Medium (which had nothing to do with the resource). This delay is \textbf{unbounded}---any number of medium-priority tasks could run while High waits.

\subsection{The Mars Pathfinder Incident}

\begin{warningbox}[title=Real-World Priority Inversion: Mars Pathfinder (1997)]
The Mars Pathfinder spacecraft landed on Mars on July 4, 1997. A few days into the mission, the spacecraft began experiencing system resets, losing data each time.

\textbf{The cause}: A low-priority meteorological data collection task held a mutex for a shared information bus. A high-priority bus management task needed this mutex. Medium-priority communication tasks kept preempting the low-priority task, preventing it from releasing the mutex, causing the high-priority task to miss deadlines and triggering watchdog resets.

\textbf{The fix}: Engineers at JPL diagnosed the problem and uploaded a patch enabling priority inheritance on the mutex---from 180 million kilometers away!

This incident demonstrated that priority inversion is not just a theoretical concern but can affect real mission-critical systems.
\end{warningbox}

\subsection{Priority Inheritance Protocol}

\begin{definition}[Priority Inheritance]
Under the \textbf{Priority Inheritance Protocol}, when a high-priority task blocks on a mutex held by a low-priority task, the low-priority task temporarily inherits the priority of the blocked high-priority task until it releases the mutex.
\end{definition}

\textbf{How it works}:
\begin{enumerate}
    \item Low acquires mutex (priority = Low)
    \item High tries to acquire mutex, blocks
    \item Low's priority is raised to High (inheritance)
    \item Medium becomes ready but cannot preempt (Low now has High priority)
    \item Low completes critical section, releases mutex
    \item Low's priority returns to Low; High acquires mutex and runs
\end{enumerate}

The high-priority task is delayed only by the low-priority task's critical section, not by unrelated medium-priority tasks.

\begin{lstlisting}[language=C, caption=Priority inheritance in FreeRTOS]
// Mutexes in FreeRTOS automatically use priority inheritance
SemaphoreHandle_t mutex = xSemaphoreCreateMutex();

// Binary semaphores do NOT use priority inheritance
SemaphoreHandle_t sem = xSemaphoreCreateBinary();

// ALWAYS use mutex (not binary semaphore) for protecting shared data
\end{lstlisting}

\subsection{Priority Ceiling Protocol}

The \textbf{Priority Ceiling Protocol} is an alternative that provides even stronger guarantees:

\begin{definition}[Priority Ceiling]
Each mutex is assigned a \textbf{ceiling priority} equal to the highest priority of any task that might acquire it. When a task acquires a mutex, its priority is immediately raised to the ceiling priority.
\end{definition}

\textbf{Advantages over priority inheritance}:
\begin{itemize}
    \item Prevents deadlock (tasks cannot block on resources they need if they already hold a higher-ceiling resource)
    \item Bounds the maximum blocking time to one critical section
\end{itemize}

\textbf{Disadvantage}: Requires knowing all tasks that might use each mutex at design time.

\begin{notebox}[title=FreeRTOS Support]
FreeRTOS implements priority inheritance for mutexes but not priority ceiling. For most flight controller applications, priority inheritance is sufficient. If you need priority ceiling, you can implement it manually by calling \texttt{vTaskPrioritySet()} when acquiring/releasing mutexes.
\end{notebox}

\section{Practical Guidelines for Quadrotor Synchronization}

\subsection{Minimizing Synchronization}

The best synchronization is no synchronization. Design strategies to minimize shared data:

\begin{enumerate}
    \item \textbf{Use queues}: Data is copied, not shared. No explicit locking needed.
    \item \textbf{Task-local data}: Each task works with its own copy of data.
    \item \textbf{Single-writer principle}: Only one task writes each piece of shared data.
    \item \textbf{Immutable data}: Configuration data that never changes needs no protection.
\end{enumerate}

\subsection{When Synchronization Is Necessary}

For shared state that must be accessed by multiple tasks:

\begin{enumerate}
    \item Use \textbf{mutexes} (not binary semaphores) for priority inheritance
    \item Keep critical sections \textbf{short}---copy data in, release lock, process locally
    \item Establish \textbf{lock ordering} if multiple mutexes are needed
    \item Consider \textbf{lock-free} techniques for simple data (atomic operations)
\end{enumerate}

\subsection{Lock-Free Techniques for Simple Data}

For simple data types (single integers, pointers), atomic operations can avoid locks entirely:

\begin{lstlisting}[language=C, caption=Atomic operations for simple data]
#include <stdatomic.h>

// Atomic flag for signaling
atomic_flag dataReady = ATOMIC_FLAG_INIT;

// Atomic integer for simple counters
atomic_int packetCount = 0;

// Atomic pointer for lock-free buffer switching
typedef struct { float data[100]; } Buffer_t;
Buffer_t bufferA, bufferB;
atomic_ptr currentBuffer = &bufferA;

// Writer
void Writer(void) {
    Buffer_t *writeBuffer = (currentBuffer == &bufferA) ? &bufferB : &bufferA;
    FillBuffer(writeBuffer);
    atomic_store(&currentBuffer, writeBuffer);  // Atomic pointer swap
}

// Reader
void Reader(void) {
    Buffer_t *readBuffer = atomic_load(&currentBuffer);
    UseBuffer(readBuffer);  // Safe - buffer won't change mid-read
}
\end{lstlisting}

\begin{warningbox}[title=Lock-Free Programming Is Hard]
Lock-free programming is notoriously difficult to get right. Subtle memory ordering issues can cause bugs that are even harder to find than lock-based race conditions. For most flight controller applications, well-designed mutex-based synchronization is safer and sufficient.
\end{warningbox}


\chapter{Advanced Scheduling Theory}

\section{Introduction: Why Scheduling Theory Matters}

In previous chapters, we learned how to create tasks and protect shared data. But a critical question remains: \textbf{will all tasks meet their deadlines?}

Scheduling theory provides mathematical tools to answer this question \emph{before} running the system. This is essential for safety-critical systems like flight controllers---we cannot simply ``try it and see'' whether deadlines are met.

\textbf{The fundamental question}: Given a set of tasks with known execution times, periods, and deadlines, can we guarantee that all deadlines will always be met?

\section{Task Model and Notation}

\subsection{Periodic Task Model}

We model each task $\tau_i$ with the following parameters:

\begin{center}
\begin{tabular}{cl}
\toprule
\textbf{Symbol} & \textbf{Description} \\
\midrule
$C_i$ & Worst-case execution time (WCET) \\
$T_i$ & Period (minimum time between releases) \\
$D_i$ & Relative deadline \\
$P_i$ & Priority (higher number = higher priority) \\
$R_i$ & Worst-case response time \\
$U_i = C_i/T_i$ & Utilization (fraction of CPU time used) \\
\bottomrule
\end{tabular}
\end{center}

\begin{notebox}[title=Assumptions in the Standard Task Model]
The classical scheduling theory makes several simplifying assumptions:
\begin{enumerate}
    \item Tasks are \textbf{periodic} with known, constant periods
    \item \textbf{WCET is known} and constant across all instances
    \item Tasks are \textbf{independent} (no synchronization or communication)
    \item \textbf{Context switch time is negligible}
    \item Tasks \textbf{do not self-suspend} during execution
    \item The system has a \textbf{single processor}
    \item \textbf{Deadlines equal periods}: $D_i = T_i$
\end{enumerate}
Real systems violate some of these assumptions, requiring engineering margin in the analysis.
\end{notebox}

\subsection{Quadrotor Task Set Example}

\begin{example}[Flight Controller Task Set]
\begin{center}
\begin{tabular}{lcccc}
\toprule
\textbf{Task} & $C_i$ (ms) & $T_i$ (ms) & $D_i$ (ms) & $U_i$ \\
\midrule
IMU Processing & 0.1 & 1 & 1 & 10.0\% \\
Attitude Control & 0.3 & 2 & 2 & 15.0\% \\
Sensor Fusion & 0.2 & 2 & 2 & 10.0\% \\
Position Control & 1.0 & 20 & 20 & 5.0\% \\
Communication & 2.0 & 50 & 50 & 4.0\% \\
Logging & 5.0 & 100 & 100 & 5.0\% \\
\midrule
\textbf{Total} & & & & \textbf{49.0\%} \\
\bottomrule
\end{tabular}
\end{center}
\end{example}

\section{CPU Utilization}

\begin{definition}[CPU Utilization]
The \textbf{total CPU utilization} is the fraction of processor time used by all tasks:
\[
U = \sum_{i=1}^{n} U_i = \sum_{i=1}^{n} \frac{C_i}{T_i}
\]
\end{definition}

\textbf{Interpretation}:
\begin{itemize}
    \item $U > 1$ (100\%): \textbf{Definitely not schedulable}---there isn't enough CPU time
    \item $U \leq 1$: \textbf{Might be schedulable}---depends on scheduling algorithm and task parameters
\end{itemize}

\textbf{Intuition}: Utilization measures the ``average'' CPU load. Even if utilization is below 100\%, the system might not be schedulable because tasks can interfere with each other at specific times (e.g., when multiple tasks are released simultaneously).

\section{Scheduling Algorithms}

\subsection{Fixed-Priority Scheduling (FPS)}

\begin{definition}[Fixed-Priority Scheduling]
In \textbf{Fixed-Priority Scheduling}, each task is assigned a static priority at design time. The scheduler always runs the highest-priority ready task. If a higher-priority task becomes ready, it preempts the currently running task.
\end{definition}

Most RTOS (including FreeRTOS) use fixed-priority preemptive scheduling.

\textbf{Key question}: How should we assign priorities?

\subsection{Rate-Monotonic Scheduling (RMS)}

\begin{definition}[Rate-Monotonic Scheduling]
\textbf{Rate-Monotonic Scheduling (RMS)} assigns priorities based on period:
\[
T_i < T_j \Rightarrow P_i > P_j
\]
Shorter period = higher priority.
\end{definition}

\begin{theorem}[RMS Optimality (Liu \& Layland, 1973)]
For periodic tasks with $D_i = T_i$, RMS is \textbf{optimal} among fixed-priority scheduling algorithms. If any fixed-priority assignment can schedule a task set, RMS can too.
\end{theorem}

\textbf{Intuition}: Why does RMS work? Tasks with shorter periods have tighter deadlines (since $D_i = T_i$). By giving them higher priority, they can always preempt slower tasks and meet their deadlines. Slower tasks have more time until their deadlines, so they can tolerate being preempted.

\subsection{Deadline-Monotonic Scheduling (DMS)}

When $D_i \neq T_i$ (deadline can be shorter than period), use Deadline-Monotonic:

\begin{definition}[Deadline-Monotonic Scheduling]
\textbf{Deadline-Monotonic Scheduling (DMS)} assigns priorities based on deadline:
\[
D_i < D_j \Rightarrow P_i > P_j
\]
Shorter deadline = higher priority.
\end{definition}

DMS is optimal for fixed-priority scheduling when $D_i \leq T_i$.

\subsection{Earliest Deadline First (EDF)}

\begin{definition}[Earliest Deadline First]
\textbf{Earliest Deadline First (EDF)} is a dynamic-priority algorithm where the task with the nearest absolute deadline always runs. Priorities are recomputed at each scheduling decision.
\end{definition}

\textbf{Advantages of EDF}:
\begin{itemize}
    \item Can achieve 100\% utilization (compared to ~69\% for RMS)
    \item Optimal among all scheduling algorithms (not just fixed-priority)
\end{itemize}

\textbf{Disadvantages of EDF}:
\begin{itemize}
    \item More complex to implement (priorities change dynamically)
    \item Higher runtime overhead
    \item \textbf{Unpredictable under overload}: When the system is overloaded, EDF can cause a ``domino effect'' where many tasks miss deadlines. With FPS, only low-priority tasks miss deadlines.
\end{itemize}

\begin{keyidea}[title=FPS vs. EDF for Flight Controllers]
Fixed-Priority Scheduling (RMS/DMS) is preferred for flight controllers because:
\begin{enumerate}
    \item Simpler implementation (supported by all RTOS)
    \item Predictable behavior under overload (critical tasks still meet deadlines)
    \item Easier to analyze and debug
\end{enumerate}
The lower utilization bound (69\% vs. 100\%) is rarely a practical limitation---flight controllers typically run well below this bound for safety margin.
\end{keyidea}

\section{Schedulability Analysis}

\subsection{Utilization-Based Test (Liu-Layland Bound)}
\label{sec:liu-layland}

\begin{theorem}[Liu-Layland Utilization Bound]
A set of $n$ periodic tasks with $D_i = T_i$ is \textbf{guaranteed schedulable} under RMS if:
\[
U = \sum_{i=1}^{n} \frac{C_i}{T_i} \leq n(2^{1/n} - 1)
\]
\end{theorem}

\begin{center}
\begin{tabular}{cc}
\toprule
$n$ (tasks) & Utilization Bound \\
\midrule
1 & 100.0\% \\
2 & 82.8\% \\
3 & 78.0\% \\
4 & 75.7\% \\
5 & 74.3\% \\
10 & 71.8\% \\
$\infty$ & 69.3\% \\
\bottomrule
\end{tabular}
\end{center}

\begin{example}[Applying the Utilization Test]
For our flight controller with $U = 49\%$ and $n = 6$ tasks:
\[
U = 0.49 < 6(2^{1/6} - 1) = 0.735 \approx 73.5\%
\]
The task set \textbf{passes} the utilization test and is guaranteed schedulable under RMS.
\end{example}

\begin{notebox}[title=The Test is Sufficient but Not Necessary]
The Liu-Layland bound is a \textbf{sufficient condition}: if the test passes, the system is definitely schedulable. However, systems that fail the test might still be schedulable---the bound is pessimistic.

For exact schedulability analysis, use Response-Time Analysis (next section).
\end{notebox}

\subsubsection{Proof Sketch and Intuition}

Understanding \emph{why} the bound takes the form $n(2^{1/n} - 1)$ provides insight into RMS scheduling behavior.

\textbf{Step 1: Identify the worst case.}
The hardest scheduling scenario is the \emph{critical instant}---when all tasks release simultaneously. At this moment, the lowest-priority task $\tau_n$ (longest period) faces maximum interference: it must wait for all higher-priority tasks to complete before it can run.

\textbf{Step 2: Find the worst-case task set.}
Not all task sets with the same total utilization $U$ are equally hard to schedule. The key insight is that \textbf{harmonic periods are easy, non-harmonic periods are hard}:
\begin{itemize}
    \item If $T_2 = 2T_1$ (harmonic), then $\tau_1$ executes exactly twice per period of $\tau_2$, leaving no wasted gaps
    \item If $T_2 = 1.5T_1$ (non-harmonic), then $\tau_1$ executes twice but leaves an awkward partial gap that may be too small for $\tau_2$
\end{itemize}

Using calculus of variations, one can show that the worst-case period configuration is:
\[
\frac{T_{i+1}}{T_i} = 2^{1/n} \quad \text{for all } i
\]
This creates ``maximally non-harmonic'' periods---each ratio is identical, and collectively the periods are as far as possible from any harmonic relationship.

\textbf{Step 3: Compute the bound at the worst case.}
With this worst-case period structure, task $\tau_n$ \emph{just barely} meets its deadline when the total utilization equals $n(2^{1/n} - 1)$. Any higher utilization causes $\tau_n$ to miss its deadline.

\begin{example}[Two-Task Derivation ($n = 2$)]
Consider two tasks with utilizations $U_1 = C_1/T_1$ and $U_2 = C_2/T_2$, where $T_1 < T_2$.

The worst case occurs when $T_2/T_1 = \sqrt{2} \approx 1.414$.

At the critical instant (both tasks release at $t = 0$):
\begin{enumerate}
    \item $\tau_1$ runs immediately (higher priority), completing at $t = C_1$
    \item $\tau_2$ starts, but before it finishes, $\tau_1$ releases again at $t = T_1$
    \item $\tau_1$ preempts and runs again, completing at approximately $t = T_1 + C_1$
    \item $\tau_2$ resumes and must complete before $t = T_2$
\end{enumerate}

For $\tau_2$ to just meet its deadline: $2C_1 + C_2 = T_2$.

Assuming equal utilizations $U_1 = U_2 = u$ and $T_2 = \sqrt{2} T_1$:
\begin{align*}
2C_1 + C_2 &= T_2 \\
2uT_1 + uT_2 &= T_2 \\
2uT_1 &= T_2(1 - u) = \sqrt{2}T_1(1 - u) \\
u &= \frac{\sqrt{2}}{2 + \sqrt{2}} = \sqrt{2} - 1 \approx 0.414
\end{align*}

Total utilization: $U = 2u = 2(\sqrt{2} - 1) \approx 0.828$, which matches $2(2^{1/2} - 1)$. \qed
\end{example}

\textbf{Why does the bound decrease with $n$?}
Each additional task creates more opportunities for ``bad'' period alignments that waste processor time in gaps too small for any task to use:

\begin{center}
\begin{tabular}{cll}
\toprule
$n$ & Bound & Interpretation \\
\midrule
1 & 100.0\% & Single task: no interference possible \\
2 & 82.8\% & One pairwise interference relationship \\
3 & 78.0\% & Three pairwise relationships \\
$\infty$ & 69.3\% & Maximum potential fragmentation \\
\bottomrule
\end{tabular}
\end{center}

\begin{keyidea}[title=The Harmonic Insight]
If task periods are \textbf{harmonic} (each period divides all longer periods), the utilization bound reaches \textbf{100\%}. The Liu-Layland bound is pessimistic precisely because it must handle the worst-case non-harmonic periods.

\textbf{Practical implication}: When designing a real-time system, choosing harmonic periods (e.g., 1 ms, 2 ms, 4 ms, 8 ms for attitude, position, navigation, logging) allows much higher utilization than the Liu-Layland bound suggests. This is why well-designed flight controllers often achieve 80--90\% utilization despite the 69.3\% theoretical bound.
\end{keyidea}

\begin{example}[Liu-Layland is Not Necessary: 100\% Utilization with Harmonic Periods]
Consider two tasks with harmonic periods:

\begin{center}
\begin{tabular}{cccc}
\toprule
Task & $C_i$ & $T_i$ & $U_i$ \\
\midrule
$\tau_1$ & 1 ms & 2 ms & 50\% \\
$\tau_2$ & 2 ms & 4 ms & 50\% \\
\bottomrule
\end{tabular}
\end{center}

Total utilization: $U = 50\% + 50\% = 100\%$.

Liu-Layland bound for $n = 2$: $2(2^{1/2} - 1) = 82.8\%$.

The task set \textbf{fails} the Liu-Layland test ($100\% > 82.8\%$), yet it is perfectly schedulable:

\begin{center}
\begin{tikzpicture}[xscale=1.2]
    % Time axis
    \draw[->] (0,0) -- (5,0) node[right] {$t$ (ms)};
    \foreach \x in {0,1,2,3,4} {
        \draw (\x,0.1) -- (\x,-0.1) node[below] {\x};
    }

    % Task 1 timeline
    \draw (0,0.8) -- (5,0.8);
    \node[left] at (0,0.8) {$\tau_1$};
    \fill[blue!60] (0,0.6) rectangle (1,1.0);
    \fill[blue!60] (2,0.6) rectangle (3,1.0);
    \draw[red,thick,->] (0,1.1) -- (0,1.0);
    \draw[red,thick,->] (2,1.1) -- (2,1.0);
    \node[above,font=\tiny] at (0,1.1) {release};
    \node[above,font=\tiny] at (2,1.1) {release};

    % Task 2 timeline
    \draw (0,-0.5) -- (5,-0.5);
    \node[left] at (-0.3,-0.5) {$\tau_2$};
    \fill[green!60] (1,-0.7) rectangle (2,-0.3);
    \fill[green!60] (3,-0.7) rectangle (4,-0.3);
    \draw[red,thick,->] (0,-0.9) -- (0,-0.7);
    \node[below,font=\tiny,red] at (0,-0.9) {release};
    \draw[orange,thick] (4,-0.9) -- (4,-0.7);
    \node[below,font=\tiny,orange] at (4,-0.9) {deadline};
\end{tikzpicture}
\end{center}

\textbf{Timeline analysis} (critical instant at $t = 0$):
\begin{enumerate}
    \item $t = 0$: Both tasks release. $\tau_1$ has higher priority (shorter period), runs first.
    \item $t = 1$: $\tau_1$ completes. $\tau_2$ starts executing.
    \item $t = 2$: $\tau_1$ releases again, preempts $\tau_2$. ($\tau_2$ has executed 1 ms so far.)
    \item $t = 3$: $\tau_1$ completes. $\tau_2$ resumes.
    \item $t = 4$: $\tau_2$ completes (total execution: 1 + 1 = 2 ms). Deadline met exactly!
\end{enumerate}

Response time of $\tau_2$: $R_2 = 4$ ms $= D_2$. The task meets its deadline with zero slack.

\textbf{Why does this work?} The periods are harmonic ($T_2 = 2 T_1$), so $\tau_1$'s interference fits perfectly into $\tau_2$'s period. There are no wasted gaps---every millisecond is used productively.

This demonstrates that the Liu-Layland bound is \textbf{sufficient but not necessary}: failing the test does not mean the system is unschedulable.
\end{example}

\subsection{Response-Time Analysis}

Response-Time Analysis~\cite{audsley1993applying} computes the worst-case response time for each task and compares it to the deadline.

\begin{definition}[Response Time]
The \textbf{response time} $R_i$ of a task is the time from release to completion in the worst case.
\end{definition}

A task is schedulable if and only if:
\[
R_i \leq D_i
\]

\subsubsection{Computing Response Time}

The response time consists of:
\begin{itemize}
    \item The task's own execution time $C_i$
    \item Interference from higher-priority tasks $I_i$
\end{itemize}

\[
R_i = C_i + I_i
\]

During $R_i$, each higher-priority task $j$ executes $\lceil R_i / T_j \rceil$ times, each taking up to $C_j$ time:

\[
I_i = \sum_{j \in hp(i)} \left\lceil \frac{R_i}{T_j} \right\rceil C_j
\]

where $hp(i)$ is the set of tasks with higher priority than task $i$.

Combining:
\[
R_i = C_i + \sum_{j \in hp(i)} \left\lceil \frac{R_i}{T_j} \right\rceil C_j
\]

\textbf{Problem}: $R_i$ appears on both sides! This is a fixed-point equation.

\subsubsection{Iterative Solution}

Solve by iteration:

\[
w_i^{(k+1)} = C_i + \sum_{j \in hp(i)} \left\lceil \frac{w_i^{(k)}}{T_j} \right\rceil C_j
\]

Start with $w_i^{(0)} = C_i$ and iterate until convergence ($w_i^{(k+1)} = w_i^{(k)}$) or until $w_i^{(k)} > D_i$ (deadline miss).

\begin{example}[Response-Time Analysis]
Consider three tasks:

\begin{center}
\begin{tabular}{cccc}
\toprule
Task & $C_i$ (ms) & $T_i$ (ms) & Priority \\
\midrule
$\tau_1$ & 3 & 7 & High \\
$\tau_2$ & 3 & 12 & Medium \\
$\tau_3$ & 5 & 20 & Low \\
\bottomrule
\end{tabular}
\end{center}

\textbf{Task 1} (highest priority, no interference):
\[
R_1 = C_1 = 3 \text{ ms} \leq D_1 = 7 \text{ ms} \quad \checkmark
\]

\textbf{Task 2} (interference from Task 1):
\begin{align*}
w_2^{(0)} &= 3 \\
w_2^{(1)} &= 3 + \lceil 3/7 \rceil \cdot 3 = 3 + 1 \cdot 3 = 6 \\
w_2^{(2)} &= 3 + \lceil 6/7 \rceil \cdot 3 = 3 + 1 \cdot 3 = 6 \quad \text{(converged)}
\end{align*}
$R_2 = 6$ ms $\leq D_2 = 12$ ms $\checkmark$

\textbf{Task 3} (interference from Tasks 1 and 2):
\begin{align*}
w_3^{(0)} &= 5 \\
w_3^{(1)} &= 5 + \lceil 5/7 \rceil \cdot 3 + \lceil 5/12 \rceil \cdot 3 = 5 + 3 + 3 = 11 \\
w_3^{(2)} &= 5 + \lceil 11/7 \rceil \cdot 3 + \lceil 11/12 \rceil \cdot 3 = 5 + 6 + 3 = 14 \\
w_3^{(3)} &= 5 + \lceil 14/7 \rceil \cdot 3 + \lceil 14/12 \rceil \cdot 3 = 5 + 6 + 6 = 17 \\
w_3^{(4)} &= 5 + \lceil 17/7 \rceil \cdot 3 + \lceil 17/12 \rceil \cdot 3 = 5 + 9 + 6 = 20 \\
w_3^{(5)} &= 5 + \lceil 20/7 \rceil \cdot 3 + \lceil 20/12 \rceil \cdot 3 = 5 + 9 + 6 = 20 \quad \text{(converged)}
\end{align*}
$R_3 = 20$ ms $\leq D_3 = 20$ ms $\checkmark$ (just meets deadline!)
\end{example}

\begin{keyidea}[title=Response-Time Analysis is Exact]
Unlike the utilization bound, Response-Time Analysis is \textbf{sufficient and necessary}:
\begin{itemize}
    \item If the test passes, the system will meet all deadlines
    \item If the test fails, a deadline will be missed at runtime
\end{itemize}
This makes it the preferred method for verifying flight controller schedulability.
\end{keyidea}

\subsubsection{Proof Sketch: Why Response-Time Analysis Works}

We establish four key results that justify the response-time equation and its exactness.

\textbf{Result 1: Critical Instant Theorem.}
The worst-case response time for task $\tau_i$ occurs when it is released simultaneously with all higher-priority tasks (the \emph{critical instant}).

\emph{Intuition}: Any other release pattern means some higher-priority tasks release later, reducing interference during $\tau_i$'s response window.

\emph{Sketch}: Consider $\tau_i$ released at time $t_0$. If a higher-priority task $\tau_j$ releases at $t_0 - \delta$ (before $\tau_i$), it might complete before $\tau_i$ needs to run. If $\tau_j$ releases at $t_0 + \delta$ (after $\tau_i$), interference is delayed. Both cases reduce or maintain---but never increase---total interference compared to simultaneous release.

\textbf{Result 2: Interference Calculation.}
During response time $R_i$, each higher-priority task $\tau_j$ causes interference:
\[
I_{i,j} = \left\lceil \frac{R_i}{T_j} \right\rceil C_j
\]

\emph{Intuition}: In the interval $[0, R_i]$, task $\tau_j$ releases $\lceil R_i/T_j \rceil$ times, each execution consuming $C_j$ time units.

\emph{Sketch}: At the critical instant, $\tau_j$ first releases at $t = 0$, with subsequent releases at $T_j, 2T_j, 3T_j, \ldots$ The number of releases in $[0, R_i]$ is $\lfloor R_i/T_j \rfloor + 1 = \lceil R_i/T_j \rceil$, where the $+1$ accounts for the release at $t = 0$.

\begin{center}
\begin{tikzpicture}[xscale=0.9]
    % Time axis
    \draw[->] (0,0) -- (8,0) node[right] {$t$};
    \node[below] at (0,-0.1) {$0$};
    \node[below] at (7,-0.1) {$R_i$};
    \draw[dashed] (7,0) -- (7,1.5);

    % Task j releases and executions
    \node[left] at (0,1) {$\tau_j$:};
    \fill[blue!60] (0,0.7) rectangle (1,1.3);
    \fill[blue!60] (2.3,0.7) rectangle (3.3,1.3);
    \fill[blue!60] (4.6,0.7) rectangle (5.6,1.3);

    % Release arrows
    \draw[->,red,thick] (0,1.5) -- (0,1.3);
    \draw[->,red,thick] (2.3,1.5) -- (2.3,1.3);
    \draw[->,red,thick] (4.6,1.5) -- (4.6,1.3);

    % Period markers
    \draw[<->] (0,1.7) -- (2.3,1.7) node[midway,above] {$T_j$};
    \draw[<->] (2.3,1.7) -- (4.6,1.7) node[midway,above] {$T_j$};

    % Task i (fragmented)
    \node[left] at (0,0.2) {$\tau_i$:};
    \fill[green!60] (1,0) rectangle (2.3,0.4);
    \fill[green!60] (3.3,0) rectangle (4.6,0.4);
    \fill[green!60] (5.6,0) rectangle (7,0.4);

    % Annotation
    \node[right,font=\small] at (8,1) {$\lceil R_i/T_j \rceil = 3$ releases};
\end{tikzpicture}
\end{center}

\textbf{Result 3: Fixed-Point Convergence.}
The iteration $w_i^{(k+1)} = C_i + \sum_{j \in hp(i)} \lceil w_i^{(k)}/T_j \rceil C_j$ converges to the response time.

\emph{Sketch}:
\begin{itemize}
    \item The right-hand side $f(w) = C_i + \sum_j \lceil w/T_j \rceil C_j$ is monotonically non-decreasing in $w$
    \item $f(w)$ is a step function, increasing only at multiples of period $T_j$
    \item Starting from $w^{(0)} = C_i$: since $w^{(1)} = f(w^{(0)}) \geq w^{(0)}$, the sequence is non-decreasing
    \item If $w^{(k+1)} = w^{(k)}$, we have found the fixed point $R_i$
    \item If $w^{(k)} > D_i$, the task is unschedulable and iteration terminates
    \item Termination is guaranteed: there are only finitely many step increases in $[C_i, D_i]$
\end{itemize}

\textbf{Result 4: Sufficiency and Necessity.}
Response-time analysis is both sufficient \emph{and} necessary, unlike the Liu-Layland bound.

\emph{Why sufficient}: If $R_i \leq D_i$, the task completes by its deadline even in the worst case (the critical instant).

\emph{Why necessary}: If $R_i > D_i$, then at the critical instant---which can actually occur (e.g., at system startup when all tasks release simultaneously)---the task misses its deadline. This is a real failure scenario, not a conservative bound.

\emph{Contrast with Liu-Layland}: The Liu-Layland bound assumes worst-case \emph{period relationships} (the maximally non-harmonic case) that may not exist in the actual task set. Response-time analysis uses the \emph{actual} task parameters, making it exact.

\begin{notebox}[title=Practical Implication]
For flight controller certification, response-time analysis provides the definitive answer:
\begin{itemize}
    \item \textbf{Pass}: The system \emph{will} meet all deadlines under all conditions
    \item \textbf{Fail}: There \emph{exists} a scenario (the critical instant) where a deadline is missed
\end{itemize}
This exactness is why response-time analysis is preferred over utilization bounds in safety-critical systems, despite requiring more computation.
\end{notebox}

\subsection{Accounting for Blocking}

When tasks share resources (mutexes), low-priority tasks can block high-priority tasks. This adds a \textbf{blocking term} $B_i$ to the response time:

\[
R_i = C_i + B_i + \sum_{j \in hp(i)} \left\lceil \frac{R_i}{T_j} \right\rceil C_j
\]

where $B_i$ is the maximum time task $i$ can be blocked by lower-priority tasks holding mutexes.

With priority inheritance, $B_i$ is bounded by the longest critical section of any lower-priority task that uses a mutex also used by task $i$ or any higher-priority task.

\section{Cyclic Executives}

Before RTOS became common, many real-time systems used \textbf{cyclic executives}:

\begin{definition}[Cyclic Executive]
A \textbf{cyclic executive} is a scheduling approach where the programmer manually constructs a fixed schedule that repeats periodically. Tasks are implemented as procedures called at predetermined times.
\end{definition}

\begin{example}[Cyclic Executive Schedule]
\begin{center}
\begin{tabular}{ccc}
\toprule
Task & Period (ms) & WCET (ms) \\
\midrule
A & 25 & 10 \\
B & 25 & 8 \\
C & 50 & 5 \\
D & 50 & 4 \\
E & 100 & 2 \\
\bottomrule
\end{tabular}
\end{center}

Minor cycle: 25 ms, Major cycle: 100 ms

\begin{center}
\begin{tabular}{c|cccc}
\toprule
Minor cycle & 1 & 2 & 3 & 4 \\
\midrule
Tasks & A, B, C, E & A, B, D & A, B, C & A, B, D \\
\bottomrule
\end{tabular}
\end{center}
\end{example}

\textbf{Advantages}:
\begin{itemize}
    \item Fully deterministic---exact same execution every major cycle
    \item Very low overhead (no scheduler, no context switches within minor cycle)
    \item Easy to verify timing
\end{itemize}

\textbf{Disadvantages}:
\begin{itemize}
    \item \textbf{Cannot handle sporadic events} (external inputs at unpredictable times)
    \item \textbf{Difficult to construct} (finding a valid schedule is NP-hard)
    \item \textbf{Difficult to modify} (adding a task may require redesigning the entire schedule)
    \item \textbf{Large tasks must be split} into pieces that fit in minor cycles
\end{itemize}

\begin{notebox}[title=When to Use Cyclic Executives]
Cyclic executives are still used in highly safety-critical systems (e.g., aircraft fly-by-wire) where determinism is paramount and the task set is fixed. For quadrotor flight controllers with varying requirements and rapid development cycles, RTOS-based scheduling is more practical.
\end{notebox}


\chapter{Latency and Jitter in Control Systems}

\section{Introduction: Why Timing Affects Control}

Control theory typically assumes:
\begin{itemize}
    \item Instantaneous sampling (sensor values captured at exact times)
    \item Instantaneous computation (control law evaluates in zero time)
    \item Instantaneous actuation (outputs applied immediately)
\end{itemize}

Reality is different:
\begin{itemize}
    \item Sampling takes time (ADC conversion, communication)
    \item Computation takes time (running control algorithm)
    \item Actuation takes time (PWM update, motor response)
\end{itemize}

These delays, and variations in these delays, affect control performance.

\section{Timing Definitions}

Consider a periodic control task with period $T$. Job $k$ is released at time $r_k = kT$.

\begin{definition}[Timing Parameters]
For each job $k$ of a control task:
\begin{itemize}
    \item $r_k$: \textbf{Release time} (when task becomes ready)
    \item $s_k$: \textbf{Start time} (when task begins executing)
    \item $f_k$: \textbf{Finish time} (when task completes)
    \item $R_k = f_k - r_k$: \textbf{Response time}
\end{itemize}
\end{definition}

\begin{definition}[Latency]
\begin{itemize}
    \item \textbf{Sampling latency} $L^s_k = s_k - r_k$: Delay from release to when input is sampled
    \item \textbf{Input-output latency} $L^{io}_k = f_k - r_k$: Delay from release to when output is applied
\end{itemize}
\end{definition}

\begin{definition}[Jitter]
\begin{itemize}
    \item \textbf{Sampling jitter} $J^s = \max_k L^s_k - \min_k L^s_k$: Variation in sampling times
    \item \textbf{Input-output jitter} $J^{io} = \max_k L^{io}_k - \min_k L^{io}_k$: Variation in output times
\end{itemize}
\end{definition}

\textbf{Intuition}:
\begin{itemize}
    \item \textbf{Latency} is like a constant time delay---it shifts the control response but doesn't destabilize (within limits).
    \item \textbf{Jitter} is like a randomly varying time delay---it makes the system non-deterministic and can cause oscillations or instability.
\end{itemize}

\section{Effect of Latency on Control}

A control system with sampling period $T$ and computational delay $\tau$ behaves like a continuous system with time delay $\tau + T/2$ (on average).

\textbf{Effect on stability}: Time delay adds negative phase to the loop gain:
\[
\Delta\phi = -\omega_c \tau_{total}
\]
where $\omega_c$ is the crossover frequency.

\begin{example}[Phase Margin Degradation]
Quadrotor attitude loop: $\omega_c = 20$ rad/s, designed phase margin 45°

If computational delay $\tau = 2$ ms:
\[
\Delta\phi = -20 \times 0.002 = -0.04 \text{ rad} = -2.3°
\]

Remaining phase margin: $45° - 2.3° = 42.7°$---still adequate.

If delay increases to $\tau = 10$ ms:
\[
\Delta\phi = -20 \times 0.010 = -0.2 \text{ rad} = -11.5°
\]

Remaining phase margin: $45° - 11.5° = 33.5°$---getting marginal.
\end{example}

\begin{keyidea}[title=Latency Design Rule]
For a control loop with crossover frequency $\omega_c$, total latency should satisfy:
\[
\tau_{total} < \frac{\phi_{margin}}{2 \omega_c}
\]
where $\phi_{margin}$ is the design phase margin in radians.

For a 45° (0.79 rad) phase margin and $\omega_c = 20$ rad/s:
\[
\tau_{total} < \frac{0.79}{2 \times 20} = 20 \text{ ms}
\]
\end{keyidea}

\section{Effect of Jitter on Control}

Jitter is more problematic than constant latency because:

\begin{enumerate}
    \item \textbf{Non-uniform sampling}: The control law assumes uniform $T$, but actual intervals vary
    \item \textbf{Time-varying delay}: Cannot be compensated by fixed controller design
    \item \textbf{Can excite resonances}: Random timing can inject energy at problematic frequencies
\end{enumerate}

\begin{warningbox}[title=Jitter Sensitivity]
As a rule of thumb, jitter should be less than 1\% of the sampling period:
\[
J < 0.01 \times T
\]

For attitude control at 500 Hz ($T = 2$ ms):
\[
J < 20 \text{ }\mu\text{s}
\]

This is achievable with careful RTOS design but requires attention to:
\begin{itemize}
    \item Priority assignment (control task must have highest priority)
    \item Interrupt latency (ISR overhead)
    \item Critical section length (blocking time)
\end{itemize}
\end{warningbox}

\section{Reducing Latency with Subtask Scheduling}

A control algorithm typically has this structure:

\begin{lstlisting}[language=C]
void ControlTask(void) {
    while (1) {
        y = ReadSensor();        // Input
        u = ControlLaw(y, x);    // Calculate output
        WriteActuator(u);        // Output
        x = UpdateState(y, u);   // Update internal state
        WaitNextPeriod();
    }
}
\end{lstlisting}

\textbf{Observation}: The \texttt{UpdateState} doesn't affect the \emph{current} output---it only prepares for the \emph{next} sample. We can defer it!

\subsection{Subtask Decomposition}

Split the control task into two subtasks:

\begin{lstlisting}[language=C, caption=Subtask decomposition]
// High priority: Calculate and Output
void ControlOutputTask(void) {
    while (1) {
        WaitForSensor();         // Triggered by sensor
        y = ReadSensor();
        u = ControlLaw(y, x);    // Use state from previous cycle
        WriteActuator(u);
        SignalUpdateTask();      // Tell update task to run
    }
}

// Lower priority: Update State
void StateUpdateTask(void) {
    while (1) {
        WaitForSignal();         // Wait for output task
        x = UpdateState(y, u);   // Can be preempted
    }
}
\end{lstlisting}

\textbf{Benefits}:
\begin{itemize}
    \item \textbf{Shorter critical path}: Input $\to$ Output latency is minimized
    \item \textbf{State update runs in ``slack time''}: Between output and next sample
    \item \textbf{Other tasks can preempt update}: Doesn't affect I/O latency
\end{itemize}

\subsection{Deadline Assignment for Subtasks}

For a control task with period $T$ and WCET split as:
\begin{itemize}
    \item $C_{CO}$: Calculate Output WCET
    \item $C_{US}$: Update State WCET
\end{itemize}

Assign deadlines:
\begin{itemize}
    \item $D_{CO}$: As short as possible (minimize I/O latency)
    \item $D_{US} = T$: Full period (less critical)
\end{itemize}

\begin{example}[Inverted Pendulum Case Study]
Three inverted pendulums controlled by one CPU:
\begin{center}
\begin{tabular}{lccc}
\toprule
Pendulum & Period & Standard RMS & Subtask Scheduling \\
\midrule
1 (fastest) & 10 ms & 1.5 ms latency & 1.5 ms latency \\
2 & 14.5 ms & 4.5 ms latency & 3.0 ms latency \\
3 (slowest) & 17.5 ms & \textbf{Unstable!} & 4.5 ms latency \\
\bottomrule
\end{tabular}
\end{center}

With standard RMS, the lowest-priority pendulum becomes unstable due to excessive latency. With subtask scheduling, all three are stable because the I/O path is prioritized.
\end{example}

\section{Measuring Timing in Practice}

\subsection{GPIO-Based Measurement}

The simplest way to measure task timing is with GPIO pins and an oscilloscope:

\begin{lstlisting}[language=C, caption=GPIO timing measurement]
#define TIMING_PIN  GPIO_PIN_0

void ControlTask(void *pvParameters) {
    for (;;) {
        GPIO_SetHigh(TIMING_PIN);   // Rising edge: start

        // Control computation
        ReadSensors();
        ComputeControl();
        WriteActuators();

        GPIO_SetLow(TIMING_PIN);    // Falling edge: end

        vTaskDelayUntil(&xLastWakeTime, pdMS_TO_TICKS(2));
    }
}
\end{lstlisting}

\textbf{What to measure}:
\begin{itemize}
    \item \textbf{Pulse width}: Execution time
    \item \textbf{Pulse period}: Actual task period (should be constant)
    \item \textbf{Period variation}: Jitter
\end{itemize}

\subsection{Software Timestamps}

\begin{lstlisting}[language=C, caption=Software jitter measurement]
void ControlTask(void *pvParameters) {
    TickType_t xLastWakeTime = xTaskGetTickCount();
    uint32_t lastTimestamp = 0;

    for (;;) {
        uint32_t now = GetHighResTimer();  // Microsecond timer
        uint32_t period = now - lastTimestamp;
        lastTimestamp = now;

        // Check for jitter
        int32_t jitter = period - EXPECTED_PERIOD_US;
        if (abs(jitter) > MAX_ALLOWED_JITTER_US) {
            LogJitterEvent(jitter);
        }

        // Control computation...

        vTaskDelayUntil(&xLastWakeTime, pdMS_TO_TICKS(2));
    }
}
\end{lstlisting}

\subsection{Trace Tools}

For detailed timing analysis:

\begin{itemize}
    \item \textbf{Segger SystemView}: Free tool for FreeRTOS, shows task execution timeline
    \item \textbf{Tracealyzer}: Commercial tool with advanced analysis features
    \item \textbf{Logic analyzer}: Hardware capture of GPIO timing signals
\end{itemize}


\chapter{Worst-Case Execution Time Estimation}

\section{Introduction: Why WCET Matters}

Schedulability analysis requires knowing the \textbf{worst-case execution time (WCET)} of each task. If we underestimate WCET, the system may miss deadlines even though analysis said it wouldn't.

\textbf{The challenge}: Execution time varies depending on:
\begin{itemize}
    \item Input data (different code paths)
    \item Cache state (hits vs. misses)
    \item Pipeline state (branch prediction)
    \item Interrupt timing
\end{itemize}

Finding the true worst case is difficult---and for complex processors, often impossible to determine exactly.

\section{Why WCET Estimation Is Hard}

\subsection{Program Path Variability}

\begin{lstlisting}[language=C, caption=Code with variable execution time]
void ProcessSensorData(SensorData_t *data) {
    // Path 1: Normal processing
    if (data->status == VALID) {
        ApplyCalibration(data);      // Fast path
    }
    // Path 2: Error handling
    else {
        LogError(data);              // Slower path
        AttemptRecovery(data);       // Even slower
    }

    // Loop with data-dependent iteration count
    for (int i = 0; i < data->numSamples; i++) {
        ProcessSample(&data->samples[i]);
    }
}
\end{lstlisting}

The execution time depends on:
\begin{itemize}
    \item Whether the data is valid
    \item How many samples need processing
    \item Whether recovery succeeds
\end{itemize}

\subsection{Hardware Effects}

\textbf{Cache}: First access to memory is slow (cache miss); subsequent accesses to same region are fast (cache hit). The cache state depends on what ran before.

\textbf{Pipeline}: Modern processors execute multiple instructions simultaneously. Branch mispredictions cause pipeline flushes, adding cycles.

\textbf{Memory wait states}: External memory (Flash, SDRAM) is slower than internal SRAM. Access patterns affect timing.

\begin{notebox}[title=ARM Cortex-M4 Timing Characteristics]
The STM32F4 (Crazyflie main processor) has:
\begin{itemize}
    \item 3-stage pipeline with branch prediction
    \item Instruction and data caches
    \item Flash memory with wait states (up to 7 cycles at 168 MHz)
    \item Single-cycle multiply, multi-cycle divide
\end{itemize}
These features make WCET estimation non-trivial even for a ``simple'' embedded processor.
\end{notebox}

\section{WCET Estimation Methods}

\subsection{Measurement-Based Estimation}

Run the code many times with different inputs and measure execution time.

\begin{lstlisting}[language=C, caption=Measurement-based WCET estimation]
#define NUM_TESTS 10000

uint32_t MeasureWCET(void (*task)(void*), void *params) {
    uint32_t maxTime = 0;

    for (int i = 0; i < NUM_TESTS; i++) {
        // Vary inputs to cover different paths
        GenerateTestInput(params, i);

        uint32_t start = GetCycleCount();
        task(params);
        uint32_t elapsed = GetCycleCount() - start;

        if (elapsed > maxTime) {
            maxTime = elapsed;
        }
    }

    return maxTime;
}
\end{lstlisting}

\textbf{Advantages}:
\begin{itemize}
    \item Simple to implement
    \item Measures real hardware behavior
    \item Catches hardware effects automatically
\end{itemize}

\textbf{Disadvantages}:
\begin{itemize}
    \item May not find the true worst case (input coverage problem)
    \item Results depend on test inputs
    \item Time-consuming for thorough coverage
\end{itemize}

\begin{keyidea}[title=Adding Safety Margin]
Since measurement may not find the true worst case, add a safety margin:
\[
WCET_{design} = \alpha \times WCET_{measured}
\]
Typical values: $\alpha = 1.2$ to $1.5$ (20--50\% margin)

The margin accounts for:
\begin{itemize}
    \item Untested input combinations
    \item Cache state variations
    \item Interrupt interference
\end{itemize}
\end{keyidea}

\subsection{Static Analysis}

Analyze the code without running it to compute a safe upper bound.

\textbf{Steps}:
\begin{enumerate}
    \item Build control flow graph (all possible paths)
    \item Determine loop bounds (maximum iterations)
    \item Model processor timing (cycles per instruction)
    \item Find longest path
\end{enumerate}

\textbf{Tools}:
\begin{itemize}
    \item aiT (AbsInt): Commercial, widely used in aerospace
    \item OTAWA: Open-source, academic
    \item Bound-T: Originally from Tidorum, now discontinued
\end{itemize}

\textbf{Advantages}:
\begin{itemize}
    \item Produces safe upper bound (no underestimation)
    \item Covers all paths automatically
    \item Can handle complex code
\end{itemize}

\textbf{Disadvantages}:
\begin{itemize}
    \item Often overly pessimistic (over-estimates WCET)
    \item Requires processor timing model
    \item May need manual annotations for loop bounds
    \item Expensive tools
\end{itemize}

\subsection{Hybrid Approach}

Combine measurement and analysis:

\begin{enumerate}
    \item Use static analysis to identify worst-case paths
    \item Use measurement to get accurate timing for those paths
    \item Add margin for hardware variability
\end{enumerate}

\section{Practical WCET Estimation for Quadrotors}

\subsection{Strategy for Flight Controller Code}

\begin{enumerate}
    \item \textbf{Profile each task} with representative inputs:
    \begin{lstlisting}[language=C]
    uint32_t start = DWT->CYCCNT;
    RunAttitudeControl();
    uint32_t cycles = DWT->CYCCNT - start;
    float timeMs = cycles / (SystemCoreClock / 1000.0f);
    \end{lstlisting}

    \item \textbf{Test edge cases}: Sensor saturation, maximum rates, error conditions

    \item \textbf{Add 30--50\% margin} for safety

    \item \textbf{Monitor at runtime} to detect if estimates are exceeded:
    \begin{lstlisting}[language=C]
    if (actualTime > WCET_ESTIMATE) {
        LogOverrun(taskId, actualTime);
    }
    \end{lstlisting}
\end{enumerate}

\subsection{Code Design for Predictable Timing}

\begin{keyidea}[title=Writing Timing-Predictable Code]
\begin{enumerate}
    \item \textbf{Avoid data-dependent loops}: Use fixed iteration counts where possible
    \item \textbf{Minimize branching}: Especially in inner loops
    \item \textbf{Use lookup tables}: Faster and more predictable than computation
    \item \textbf{Avoid dynamic memory}: \texttt{malloc}/\texttt{free} have variable timing
    \item \textbf{Bound iterations}: Add maximum iteration limits to all loops
    \item \textbf{Avoid recursion}: Stack depth is hard to bound
\end{enumerate}
\end{keyidea}

\begin{lstlisting}[language=C, caption=Timing-predictable code patterns]
// BAD: Data-dependent loop bound
for (int i = 0; i < sensor.numReadings; i++) { ... }

// GOOD: Fixed bound with early exit
for (int i = 0; i < MAX_READINGS; i++) {
    if (i >= sensor.numReadings) break;
    ...
}

// BAD: Unbounded iteration
while (!converged) {
    iterate();
}

// GOOD: Bounded iteration
for (int iter = 0; iter < MAX_ITERATIONS && !converged; iter++) {
    iterate();
}
\end{lstlisting}


\chapter{Fault Detection and Recovery}

\section{Introduction: Why Fault Handling Matters}

In a safety-critical system like a flight controller, things go wrong:
\begin{itemize}
    \item Tasks may take longer than expected (overrun)
    \item Sensors may fail or produce invalid data
    \item Communication may be lost
    \item Software bugs may cause unexpected behavior
\end{itemize}

A well-designed system detects these faults and recovers gracefully rather than crashing.

\section{Watchdog Timers}

\subsection{Hardware Watchdog}

A \textbf{hardware watchdog} is a countdown timer that resets the processor if not periodically ``fed'' by software.

\begin{lstlisting}[language=C, caption=Hardware watchdog usage]
// Initialize watchdog with 100ms timeout
void InitWatchdog(void) {
    IWDG->KR = 0x5555;  // Enable register access
    IWDG->PR = 4;       // Prescaler
    IWDG->RLR = 1000;   // Reload value (~100ms)
    IWDG->KR = 0xCCCC;  // Start watchdog
}

// Feed watchdog (call from main loop or dedicated task)
void FeedWatchdog(void) {
    IWDG->KR = 0xAAAA;  // Reload counter
}

// If FeedWatchdog() is not called within 100ms, processor resets!
\end{lstlisting}

\textbf{Design pattern}: Feed the watchdog only when the system is healthy:

\begin{lstlisting}[language=C, caption=Watchdog feeding strategy]
void WatchdogTask(void *pvParameters) {
    for (;;) {
        // Check system health
        bool healthy = true;
        healthy &= AttitudeControlRunning();
        healthy &= SensorFusionRunning();
        healthy &= MotorOutputValid();

        if (healthy) {
            FeedWatchdog();
        }
        // If unhealthy, don't feed -> system resets

        vTaskDelay(pdMS_TO_TICKS(20));
    }
}
\end{lstlisting}

\subsection{Software Watchdogs}

\textbf{Software watchdogs} monitor individual tasks without resetting the entire system:

\begin{lstlisting}[language=C, caption=Software watchdog for tasks]
typedef struct {
    TickType_t lastActivity;
    TickType_t timeout;
    bool healthy;
} TaskWatchdog_t;

TaskWatchdog_t taskWatchdogs[NUM_TASKS];

// Called by each task when it runs
void TaskCheckin(int taskId) {
    taskWatchdogs[taskId].lastActivity = xTaskGetTickCount();
    taskWatchdogs[taskId].healthy = true;
}

// Monitoring task
void WatchdogMonitorTask(void *pvParameters) {
    for (;;) {
        TickType_t now = xTaskGetTickCount();

        for (int i = 0; i < NUM_TASKS; i++) {
            TickType_t elapsed = now - taskWatchdogs[i].lastActivity;
            if (elapsed > taskWatchdogs[i].timeout) {
                taskWatchdogs[i].healthy = false;
                HandleTaskTimeout(i);
            }
        }

        vTaskDelay(pdMS_TO_TICKS(10));
    }
}
\end{lstlisting}

\section{Deadline Miss Detection}

\begin{lstlisting}[language=C, caption=Deadline miss detection]
void ControlTask(void *pvParameters) {
    TickType_t xLastWakeTime = xTaskGetTickCount();
    const TickType_t xPeriod = pdMS_TO_TICKS(2);

    for (;;) {
        TickType_t startTime = xTaskGetTickCount();

        // Control computation
        ReadSensors();
        ComputeControl();
        WriteActuators();

        // Check deadline
        TickType_t endTime = xTaskGetTickCount();
        TickType_t expectedEnd = xLastWakeTime + xPeriod;

        if (endTime > expectedEnd) {
            // Deadline missed!
            g_deadlineMissCount++;
            LogDeadlineMiss(TASK_CONTROL, endTime - expectedEnd);

            if (g_consecutiveMisses++ > MAX_CONSECUTIVE_MISSES) {
                TriggerFailsafe();
            }
        } else {
            g_consecutiveMisses = 0;
        }

        vTaskDelayUntil(&xLastWakeTime, xPeriod);
    }
}
\end{lstlisting}

\section{Failsafe Behavior}

\subsection{Failsafe Hierarchy}

\begin{center}
\begin{tabular}{lll}
\toprule
\textbf{Fault} & \textbf{Detection} & \textbf{Response} \\
\midrule
Single deadline miss & Runtime monitoring & Log, continue \\
Multiple deadline misses & Consecutive count & Reduce performance \\
Sensor failure & Validation checks & Use backup/estimate \\
Communication loss & Timeout & Return to home \\
Critical failure & Watchdog & Controlled descent \\
Catastrophic failure & Hardware watchdog & System reset \\
\bottomrule
\end{tabular}
\end{center}

\subsection{Graceful Degradation}

\begin{lstlisting}[language=C, caption=Graceful degradation example]
typedef enum {
    MODE_FULL_AUTONOMOUS,   // All systems nominal
    MODE_REDUCED_AUTONOMY,  // Some systems degraded
    MODE_MANUAL_ONLY,       // Autopilot disabled
    MODE_FAILSAFE_DESCENT,  // Controlled descent
    MODE_EMERGENCY_STOP     // Motors off
} FlightMode_t;

void HandleFault(FaultType_t fault) {
    switch (fault) {
        case FAULT_GPS_LOST:
            if (currentMode == MODE_FULL_AUTONOMOUS) {
                currentMode = MODE_REDUCED_AUTONOMY;
                DisablePositionControl();
                // Can still fly manually with attitude control
            }
            break;

        case FAULT_IMU_FAILURE:
            // Critical - cannot maintain attitude
            currentMode = MODE_EMERGENCY_STOP;
            DisarmMotors();
            break;

        case FAULT_RADIO_LOST:
            currentMode = MODE_FAILSAFE_DESCENT;
            InitiateControlledDescent();
            break;
    }
}
\end{lstlisting}

\section{Practical Recommendations}

\begin{keyidea}[title=Fault Handling Best Practices]
\begin{enumerate}
    \item \textbf{Always use a hardware watchdog}: Last line of defense against software lockup
    \item \textbf{Monitor task health}: Each critical task should check in regularly
    \item \textbf{Log faults}: Record timing anomalies for post-flight analysis
    \item \textbf{Define failsafe hierarchy}: Know what happens for each type of fault
    \item \textbf{Test failure modes}: Deliberately inject faults during development
    \item \textbf{Prefer degradation over shutdown}: Keep flying if at all possible
\end{enumerate}
\end{keyidea}


\chapter{Case Study: Complete Quadrotor Scheduler Design}

\section{System Requirements}

Design a flight controller scheduler with the following requirements:

\begin{center}
\begin{tabular}{lcc}
\toprule
\textbf{Function} & \textbf{Rate} & \textbf{Deadline} \\
\midrule
IMU sampling & 1000 Hz & Hard \\
Attitude control & 500 Hz & Hard \\
Sensor fusion & 500 Hz & Hard \\
Position control & 50 Hz & Firm \\
State estimation (EKF) & 50 Hz & Firm \\
Radio communication & 100 Hz & Soft \\
Telemetry logging & 10 Hz & Soft \\
Battery monitoring & 1 Hz & Soft \\
\bottomrule
\end{tabular}
\end{center}

\section{Task Design}

\subsection{Step 1: Measure WCET}

Profile each function on target hardware:

\begin{center}
\begin{tabular}{lcc}
\toprule
\textbf{Task} & \textbf{Measured Max} & \textbf{WCET (with 30\% margin)} \\
\midrule
IMU sampling & 50 $\mu$s & 65 $\mu$s \\
Attitude control & 180 $\mu$s & 235 $\mu$s \\
Sensor fusion & 150 $\mu$s & 195 $\mu$s \\
Position control & 400 $\mu$s & 520 $\mu$s \\
State estimation & 800 $\mu$s & 1040 $\mu$s \\
Radio communication & 500 $\mu$s & 650 $\mu$s \\
Telemetry logging & 2000 $\mu$s & 2600 $\mu$s \\
Battery monitoring & 100 $\mu$s & 130 $\mu$s \\
\bottomrule
\end{tabular}
\end{center}

\subsection{Step 2: Assign Priorities (RMS)}

\begin{center}
\begin{tabular}{lccc}
\toprule
\textbf{Task} & \textbf{Period} & \textbf{RMS Priority} & \textbf{FreeRTOS Priority} \\
\midrule
IMU sampling & 1 ms & 1 (highest) & 7 \\
Attitude control & 2 ms & 2 & 6 \\
Sensor fusion & 2 ms & 2 & 6 \\
Radio communication & 10 ms & 3 & 5 \\
Position control & 20 ms & 4 & 4 \\
State estimation & 20 ms & 4 & 4 \\
Telemetry logging & 100 ms & 5 & 3 \\
Battery monitoring & 1000 ms & 6 (lowest) & 2 \\
\bottomrule
\end{tabular}
\end{center}

\subsection{Step 3: Utilization Analysis}

\begin{center}
\begin{tabular}{lccc}
\toprule
\textbf{Task} & $C_i$ ($\mu$s) & $T_i$ ($\mu$s) & $U_i$ \\
\midrule
IMU sampling & 65 & 1000 & 6.5\% \\
Attitude control & 235 & 2000 & 11.75\% \\
Sensor fusion & 195 & 2000 & 9.75\% \\
Radio communication & 650 & 10000 & 6.5\% \\
Position control & 520 & 20000 & 2.6\% \\
State estimation & 1040 & 20000 & 5.2\% \\
Telemetry logging & 2600 & 100000 & 2.6\% \\
Battery monitoring & 130 & 1000000 & 0.013\% \\
\midrule
\textbf{Total} & & & \textbf{44.9\%} \\
\bottomrule
\end{tabular}
\end{center}

Liu-Layland bound for 8 tasks: $8(2^{1/8} - 1) = 72.4\%$

Since $44.9\% < 72.4\%$, the task set is \textbf{guaranteed schedulable}.

\subsection{Step 4: Response-Time Analysis}

For the lowest-priority task (Battery monitoring), we verify:

Since it has very long period (1000 ms) and all higher-priority tasks have low utilization, the response time is bounded by approximately:
\[
R_{battery} \approx C_{battery} + \sum_{higher} \frac{R_{battery}}{T_j} C_j
\]

Given the low total utilization, this converges well within the 1000 ms deadline.

\section{Synchronization Design}

\begin{lstlisting}[language=C, caption=Synchronization structure]
// Mutexes for shared state
SemaphoreHandle_t g_orientationMutex;  // Protects orientation estimate
SemaphoreHandle_t g_positionMutex;     // Protects position estimate
SemaphoreHandle_t g_setpointMutex;     // Protects control setpoints

// Queues for data flow
QueueHandle_t g_imuQueue;              // IMU -> Sensor Fusion
QueueHandle_t g_commandQueue;          // Radio -> Position Control
QueueHandle_t g_logQueue;              // All -> Logging

// Semaphores for event signaling
SemaphoreHandle_t g_imuDataReady;      // ISR -> IMU task
\end{lstlisting}

\section{Complete Implementation}

\begin{lstlisting}[language=C, caption=Complete flight controller initialization]
void FlightController_Init(void) {
    // Create synchronization primitives
    g_orientationMutex = xSemaphoreCreateMutex();
    g_positionMutex = xSemaphoreCreateMutex();
    g_setpointMutex = xSemaphoreCreateMutex();

    g_imuQueue = xQueueCreate(10, sizeof(ImuData_t));
    g_commandQueue = xQueueCreate(5, sizeof(Command_t));
    g_logQueue = xQueueCreate(50, sizeof(LogEntry_t));

    g_imuDataReady = xSemaphoreCreateBinary();

    // Create tasks (highest priority first)
    xTaskCreate(ImuTask,           "IMU",     256,  NULL, 7, NULL);
    xTaskCreate(AttitudeCtrlTask,  "AttCtrl", 512,  NULL, 6, NULL);
    xTaskCreate(SensorFusionTask,  "SensFus", 512,  NULL, 6, NULL);
    xTaskCreate(RadioTask,         "Radio",   256,  NULL, 5, NULL);
    xTaskCreate(PositionCtrlTask,  "PosCtrl", 512,  NULL, 4, NULL);
    xTaskCreate(StateEstTask,      "StateEst",1024, NULL, 4, NULL);
    xTaskCreate(LoggingTask,       "Log",     512,  NULL, 3, NULL);
    xTaskCreate(BatteryTask,       "Battery", 128,  NULL, 2, NULL);
    xTaskCreate(WatchdogTask,      "WDT",     128,  NULL, 1, NULL);

    // Initialize hardware watchdog
    InitHardwareWatchdog(200);  // 200ms timeout

    // Start scheduler
    vTaskStartScheduler();
}
\end{lstlisting}

%======================================================================
\chapter{Power Management for Flight Controllers}
%======================================================================

Power management is critical for battery-powered quadrotors. Effective power management extends flight time, prevents unexpected shutdowns, and ensures safe operation throughout the discharge cycle.

\section{Why Power Management Matters}

A quadrotor's flight time is fundamentally limited by battery capacity:

\[
t_{\text{flight}} = \frac{E_{\text{battery}}}{\overline{P}_{\text{total}}}
\]

where $E_{\text{battery}}$ is the battery energy (Wh) and $\overline{P}_{\text{total}}$ is the average power consumption. For a typical micro-quadrotor:

\begin{center}
\begin{tabular}{lcc}
\toprule
\textbf{Component} & \textbf{Power} & \textbf{Percentage} \\
\midrule
Motors (hover) & 3.5 W & 85--90\% \\
MCU & 0.2 W & 5\% \\
Sensors & 0.05 W & 1\% \\
Radio & 0.1 W & 2--3\% \\
Other (LED, etc.) & 0.05 W & 1--2\% \\
\bottomrule
\end{tabular}
\end{center}

While motors dominate power consumption, the flight controller's power management strategy affects safety and reliability.

\section{Battery Monitoring}

\subsection{Voltage Monitoring}

LiPo batteries have a characteristic discharge curve:

%----------------------------------------------------------------------
% FIGURE: LiPo Discharge Curve
%----------------------------------------------------------------------
% Description: Plot showing cell voltage vs. state of charge.
%
% X-axis: State of charge (0-100%)
% Y-axis: Cell voltage (3.0V - 4.2V)
%
% Show: Characteristic curve with:
%   - Full charge at 4.2V
%   - Nominal region around 3.7V
%   - Knee at ~3.5V where voltage drops rapidly
%   - Minimum safe voltage at 3.0V
%
% Mark danger zone below 3.0V.
% Dimensions: Column width, ~5cm height.
%----------------------------------------------------------------------

\begin{lstlisting}[language=C, caption=Battery voltage monitoring]
#define BATTERY_CELLS        1       // Single cell LiPo
#define CELL_FULL_VOLTAGE    4.2f    // Fully charged
#define CELL_NOMINAL_VOLTAGE 3.7f    // Nominal voltage
#define CELL_LOW_VOLTAGE     3.5f    // Low warning threshold
#define CELL_CRITICAL_VOLTAGE 3.3f   // Critical - land immediately
#define CELL_CUTOFF_VOLTAGE  3.0f    // Cutoff - damage risk

typedef struct {
    float voltage;           // Measured voltage [V]
    float current;           // Measured current [A] (if available)
    float capacity_mAh;      // Remaining capacity estimate
    uint8_t percentage;      // State of charge (0-100%)
    BatteryState_t state;    // Normal, Low, Critical, Cutoff
} BatteryStatus_t;

// ADC reading to voltage conversion
// Assumes voltage divider: Vbat -> R1 -> ADC -> R2 -> GND
float ADC_ToVoltage(uint16_t adc_reading)
{
    const float ADC_REF = 3.3f;       // ADC reference voltage
    const float ADC_MAX = 4095.0f;    // 12-bit ADC
    const float DIVIDER_RATIO = 2.0f; // R1 = R2, so ratio = 2

    float v_adc = (adc_reading / ADC_MAX) * ADC_REF;
    return v_adc * DIVIDER_RATIO;
}

// Apply low-pass filter to reduce noise
float BatteryVoltageFiltered(float new_reading)
{
    static float voltage_filtered = 4.2f;
    const float alpha = 0.1f;  // Filter coefficient

    voltage_filtered = alpha * new_reading + (1.0f - alpha) * voltage_filtered;
    return voltage_filtered;
}

// Estimate state of charge from voltage (open-circuit approximation)
// Note: This is approximate - accurate SOC requires current integration
uint8_t VoltageToSOC(float cell_voltage)
{
    // Piecewise linear approximation of LiPo discharge curve
    if (cell_voltage >= 4.2f) return 100;
    if (cell_voltage >= 4.0f) return 80 + (cell_voltage - 4.0f) * 100;
    if (cell_voltage >= 3.8f) return 40 + (cell_voltage - 3.8f) * 200;
    if (cell_voltage >= 3.6f) return 15 + (cell_voltage - 3.6f) * 125;
    if (cell_voltage >= 3.0f) return (cell_voltage - 3.0f) * 25;
    return 0;
}
\end{lstlisting}

\subsection{Current Monitoring and Coulomb Counting}

For accurate state-of-charge estimation, integrate current over time:

\begin{lstlisting}[language=C, caption=Coulomb counting for SOC estimation]
typedef struct {
    float capacity_mAh;      // Nominal battery capacity
    float consumed_mAh;      // Integrated consumption
    float current_filtered;  // Filtered current reading
    uint32_t last_update_ms; // Last integration time
} CoulombCounter_t;

void CoulombCounter_Init(CoulombCounter_t *cc, float capacity_mAh)
{
    cc->capacity_mAh = capacity_mAh;
    cc->consumed_mAh = 0.0f;
    cc->current_filtered = 0.0f;
    cc->last_update_ms = HAL_GetTick();
}

void CoulombCounter_Update(CoulombCounter_t *cc, float current_A)
{
    uint32_t now = HAL_GetTick();
    float dt_hours = (now - cc->last_update_ms) / 3600000.0f;

    // Low-pass filter on current
    const float alpha = 0.2f;
    cc->current_filtered = alpha * current_A + (1.0f - alpha) * cc->current_filtered;

    // Integrate: mAh = A * h * 1000
    cc->consumed_mAh += cc->current_filtered * dt_hours * 1000.0f;

    cc->last_update_ms = now;
}

uint8_t CoulombCounter_GetSOC(CoulombCounter_t *cc)
{
    float remaining = cc->capacity_mAh - cc->consumed_mAh;
    if (remaining < 0) remaining = 0;
    return (uint8_t)(100.0f * remaining / cc->capacity_mAh);
}

// Combine voltage and coulomb counting for robust estimate
uint8_t Battery_GetSOC(float voltage, CoulombCounter_t *cc)
{
    uint8_t soc_voltage = VoltageToSOC(voltage / BATTERY_CELLS);
    uint8_t soc_coulomb = CoulombCounter_GetSOC(cc);

    // Weighted average: trust coulomb counting more when current is stable
    // Trust voltage more at full charge and when resting
    return (uint8_t)(0.3f * soc_voltage + 0.7f * soc_coulomb);
}
\end{lstlisting}

\subsection{Battery State Machine}

See Listing~\ref{lst:battery-state} (Appendix~\ref{app:code-listings}) for a complete battery state machine implementation with hysteresis, state transitions (Normal $\to$ Low $\to$ Critical $\to$ Cutoff), and appropriate actions for each state.

\section{Low-Power Modes}

ARM Cortex-M processors offer several power-saving modes:

\begin{center}
\begin{tabular}{lccl}
\toprule
\textbf{Mode} & \textbf{Power} & \textbf{Wake Time} & \textbf{Use Case} \\
\midrule
Run & 100\% & -- & Active flight \\
Sleep & 30--50\% & $<$1 $\mu$s & Idle between control loops \\
Stop & 1--5\% & 10--100 $\mu$s & Ground idle, waiting for commands \\
Standby & 0.01\% & $>$1 ms & Long-term storage \\
\bottomrule
\end{tabular}
\end{center}

\subsection{Sleep Mode During Idle}

\begin{lstlisting}[language=C, caption=Using sleep mode in FreeRTOS idle hook]
// FreeRTOS idle hook - called when no tasks are ready
void vApplicationIdleHook(void)
{
    // Enter sleep mode until next interrupt
    // (timer, DMA complete, external event)
    __WFI();  // Wait For Interrupt instruction
}

// Configure for optimal sleep mode behavior
void LowPower_Init(void)
{
    // Keep clocks running for fast wake-up
    // Disable unused peripherals

    // Disable debug during sleep (saves power but prevents SWD)
    #ifndef DEBUG
    DBGMCU->CR &= ~(DBGMCU_CR_DBG_SLEEP);
    #endif
}
\end{lstlisting}

\subsection{Ground Idle Mode}

When the quadrotor is on the ground and not armed, reduce power consumption:

\begin{lstlisting}[language=C, caption=Ground idle power management]
typedef enum {
    POWER_MODE_FLIGHT,      // Full power, all systems active
    POWER_MODE_GROUND_IDLE, // Reduced power, waiting for arm
    POWER_MODE_SLEEP        // Minimal power, require wake event
} PowerMode_t;

void Power_SetMode(PowerMode_t mode)
{
    switch (mode) {
        case POWER_MODE_FLIGHT:
            // Full sensor rates
            IMU_SetRate(1000);
            // Full control loop rates
            ControlLoop_SetEnabled(true);
            // Radio in active mode
            Radio_SetPower(RADIO_POWER_FULL);
            break;

        case POWER_MODE_GROUND_IDLE:
            // Reduced sensor rates (just for monitoring)
            IMU_SetRate(100);
            // Disable control loops
            ControlLoop_SetEnabled(false);
            // Radio in low-power receive
            Radio_SetPower(RADIO_POWER_LOW);
            // Reduce CPU clock if supported
            SystemClock_SetMode(SYSCLK_REDUCED);
            break;

        case POWER_MODE_SLEEP:
            // Disable all peripherals
            IMU_SetRate(0);
            ControlLoop_SetEnabled(false);
            Radio_Sleep();
            // Configure wake source
            EXTI_ConfigureWake(GPIO_PIN_RADIO_INT);
            // Enter stop mode
            HAL_PWR_EnterSTOPMode(PWR_LOWPOWERREGULATOR_ON,
                                   PWR_STOPENTRY_WFI);
            break;
    }
}

// State machine for power mode transitions
void Power_UpdateMode(void)
{
    static uint32_t ground_idle_start = 0;

    if (FlightState_IsArmed()) {
        Power_SetMode(POWER_MODE_FLIGHT);
        ground_idle_start = 0;
    }
    else if (FlightState_IsOnGround()) {
        if (ground_idle_start == 0) {
            ground_idle_start = HAL_GetTick();
        }

        uint32_t idle_time = HAL_GetTick() - ground_idle_start;

        // After 30 seconds on ground, enter ground idle
        if (idle_time > 30000) {
            Power_SetMode(POWER_MODE_GROUND_IDLE);
        }

        // After 5 minutes, enter sleep
        if (idle_time > 300000) {
            Power_SetMode(POWER_MODE_SLEEP);
        }
    }
}
\end{lstlisting}

\section{Voltage Compensation}

Battery voltage drops during flight, affecting motor performance. Compensate to maintain consistent thrust:

\begin{lstlisting}[language=C, caption=Voltage-compensated motor output]
// Motor command compensation for battery voltage
// Assumes: actual_thrust = k * V^2 * duty^2
// To maintain constant thrust as V drops, increase duty

float Motor_VoltageCompensate(float command, float battery_voltage)
{
    // Reference voltage (fully charged)
    const float V_REF = 4.2f * BATTERY_CELLS;

    // Compensation: command_out = command * (V_ref / V_actual)
    // Clamp to prevent over-driving at low voltage
    float ratio = V_REF / battery_voltage;

    if (ratio > 1.5f) ratio = 1.5f;  // Max 50% boost
    if (ratio < 1.0f) ratio = 1.0f;  // No reduction

    return command * ratio;
}

// Apply in motor output function
void Motor_SetThrust(uint8_t motor, float thrust)
{
    float battery_v = Battery_GetVoltage();
    float compensated = Motor_VoltageCompensate(thrust, battery_v);

    // Convert to PWM duty cycle
    uint16_t pwm = (uint16_t)(compensated * PWM_MAX);
    if (pwm > PWM_MAX) pwm = PWM_MAX;

    PWM_SetDuty(motor, pwm);
}
\end{lstlisting}

\begin{warningbox}[title=Voltage Compensation Limits]
Voltage compensation cannot overcome physics. As battery voltage drops:
\begin{itemize}
    \item Maximum available thrust decreases regardless of compensation
    \item Compensation increases current draw, accelerating discharge
    \item At critically low voltage, motors may not produce enough thrust for hover
\end{itemize}
Always trigger auto-landing before battery reaches critical levels.
\end{warningbox}

\section{Power-Aware Task Design}

\subsection{Task Consolidation}

Reduce context-switching overhead by consolidating related tasks:

\begin{lstlisting}[language=C, caption=Power-efficient task design]
// Bad: Separate tasks with frequent wake-ups
void BadDesign_Task1(void *params) {
    for (;;) {
        DoSmallWork1();
        vTaskDelay(pdMS_TO_TICKS(1));
    }
}

void BadDesign_Task2(void *params) {
    for (;;) {
        DoSmallWork2();
        vTaskDelay(pdMS_TO_TICKS(1));
    }
}

// Good: Consolidated task, longer sleep periods
void GoodDesign_ConsolidatedTask(void *params) {
    for (;;) {
        DoSmallWork1();
        DoSmallWork2();
        vTaskDelay(pdMS_TO_TICKS(2));  // Longer sleep = more time in low-power
    }
}
\end{lstlisting}

\subsection{Event-Driven vs Polling}

Event-driven design saves power by sleeping until events occur:

\begin{lstlisting}[language=C, caption=Event-driven vs polling comparison]
// Polling: Wastes power checking for data
void PollingTask(void *params) {
    for (;;) {
        if (Radio_HasData()) {
            ProcessRadioData();
        }
        vTaskDelay(pdMS_TO_TICKS(1));  // Wakes up 1000 times/second
    }
}

// Event-driven: Sleeps until data arrives
void EventDrivenTask(void *params) {
    for (;;) {
        // Block until interrupt signals data ready
        xSemaphoreTake(xRadioDataReady, portMAX_DELAY);
        ProcessRadioData();
        // Task sleeps in low-power until next interrupt
    }
}
\end{lstlisting}

\section{Power Budget Analysis}

\begin{lstlisting}[language=Matlab, caption=Flight time estimation]
% Power budget analysis for flight time estimation

% Battery parameters
battery_capacity_mAh = 250;    % Crazyflie battery
battery_voltage_nom = 3.7;     % Nominal cell voltage
battery_energy_Wh = battery_capacity_mAh * battery_voltage_nom / 1000;

% Power consumers
hover_power_W = 3.5;           % Motors at hover
mcu_power_W = 0.15;            % STM32 active
sensors_power_W = 0.05;        % IMU, barometer, etc.
radio_power_W = 0.1;           % Radio transceiver
misc_power_W = 0.05;           % LED, voltage regulators

total_hover_power_W = hover_power_W + mcu_power_W + sensors_power_W + ...
                       radio_power_W + misc_power_W;

% Flight time estimate
flight_time_hours = battery_energy_Wh / total_hover_power_W;
flight_time_minutes = flight_time_hours * 60;

fprintf('Battery energy: %.2f Wh\n', battery_energy_Wh);
fprintf('Total hover power: %.2f W\n', total_hover_power_W);
fprintf('Estimated flight time: %.1f minutes\n', flight_time_minutes);

% Sensitivity analysis
power_range = linspace(3, 6, 50);  % Power range [W]
flight_times = battery_energy_Wh ./ power_range * 60;

figure;
plot(power_range, flight_times, 'b-', 'LineWidth', 2);
xlabel('Total Power (W)');
ylabel('Flight Time (minutes)');
title('Flight Time vs Power Consumption');
grid on;
\end{lstlisting}

\begin{keyidea}[title=Power Management Best Practices]
\begin{enumerate}
    \item \textbf{Monitor continuously}: Track voltage and current throughout flight
    \item \textbf{Warn early}: Give pilot warning before critical levels
    \item \textbf{Auto-land safely}: Trigger automated landing at critical battery
    \item \textbf{Compensate voltage sag}: Maintain consistent thrust response
    \item \textbf{Use low-power modes}: Sleep when not actively flying
    \item \textbf{Event-driven design}: Avoid polling where possible
    \item \textbf{Budget power}: Understand where power goes
\end{enumerate}
\end{keyidea}

%======================================================================
\chapter{Distributed Systems and Multi-Robot Communication}
%======================================================================

Modern cyber-physical systems increasingly involve multiple robots working together. A swarm of drones performing search and rescue, a fleet of delivery vehicles coordinating routes, or multiple robots collaborating in a warehouse all require distributed communication and coordination. This chapter introduces the fundamentals of distributed systems for robotics, focusing on ROS~2, DDS, and their integration with embedded systems running FreeRTOS.

\section{Why Distributed Systems for Robotics?}

\subsection{From Single Robot to Swarm}

A single quadrotor is already a complex cyber-physical system. When multiple quadrotors must work together, new challenges emerge:

\begin{center}
\begin{tabular}{p{3.5cm}p{4cm}p{4.5cm}}
\toprule
\textbf{Challenge} & \textbf{Single Robot} & \textbf{Multi-Robot System} \\
\midrule
State estimation & Local sensors only & Share sensor data, collaborative localization \\
Control & Self-contained & Coordinate actions, avoid collisions \\
Decision making & Autonomous & Distributed consensus, task allocation \\
Communication & Ground station only & Robot-to-robot, mesh networks \\
Fault tolerance & Single point of failure & Graceful degradation, redundancy \\
\bottomrule
\end{tabular}
\end{center}

\subsection{Use Cases for Multi-Robot Systems}

\begin{itemize}
    \item \textbf{Search and rescue}: Multiple drones cover large areas quickly, share findings
    \item \textbf{Precision agriculture}: Swarm monitors crops, coordinates spraying
    \item \textbf{Infrastructure inspection}: Fleet inspects bridges, power lines, pipelines
    \item \textbf{Delivery}: Coordinated last-mile delivery with dynamic routing
    \item \textbf{Entertainment}: Synchronized drone light shows
    \item \textbf{Research}: Distributed sensing, environmental monitoring
\end{itemize}

\subsection{Communication Requirements}

Multi-robot systems have stringent communication requirements:

\begin{center}
\begin{tabular}{lccl}
\toprule
\textbf{Data Type} & \textbf{Rate} & \textbf{Latency} & \textbf{Reliability} \\
\midrule
Position/velocity & 10--50 Hz & $<$50 ms & High \\
Sensor data (compressed) & 1--10 Hz & $<$200 ms & Medium \\
Commands & Event-driven & $<$20 ms & Very high \\
Heartbeat/status & 1--5 Hz & $<$500 ms & High \\
Mission updates & Event-driven & $<$1 s & Very high \\
Video stream & 15--30 Hz & $<$500 ms & Low \\
\bottomrule
\end{tabular}
\end{center}

\begin{keyidea}[title=The Fundamental Trade-off]
In wireless networks, you cannot simultaneously optimize for:
\begin{itemize}
    \item Low latency
    \item High bandwidth
    \item High reliability
    \item Long range
    \item Low power consumption
\end{itemize}
System design requires prioritizing based on application requirements.
\end{keyidea}

\section{Network Topologies for Robot Swarms}

\subsection{Star Topology}

All robots communicate through a central node (ground station or lead robot):

%----------------------------------------------------------------------
% FIGURE: Star Topology
%----------------------------------------------------------------------
% Description: Central node with multiple drones connected radially.
%
% Show: Ground station in center, 4-6 drones around it, each with
%   bidirectional arrow to center. No direct drone-to-drone links.
%
% Label: "Ground Station" at center, "Drone 1", "Drone 2", etc.
% Dimensions: Half column width, ~4cm height.
%----------------------------------------------------------------------

\textbf{Advantages}:
\begin{itemize}
    \item Simple routing---all traffic goes through center
    \item Easy to manage and monitor
    \item Central node can coordinate all decisions
\end{itemize}

\textbf{Disadvantages}:
\begin{itemize}
    \item Single point of failure
    \item Central node becomes bottleneck
    \item Range limited by weakest link to center
\end{itemize}

\subsection{Mesh Topology}

Robots communicate directly with nearby neighbors, forwarding messages as needed:

%----------------------------------------------------------------------
% FIGURE: Mesh Topology
%----------------------------------------------------------------------
% Description: Multiple drones with interconnected links.
%
% Show: 6 drones arranged in rough hexagon, each connected to 2-3
%   neighbors. Some connections dashed (weak), some solid (strong).
%
% Dimensions: Half column width, ~4cm height.
%----------------------------------------------------------------------

\textbf{Advantages}:
\begin{itemize}
    \item No single point of failure
    \item Self-healing---routes adapt when nodes leave
    \item Extended range through multi-hop
\end{itemize}

\textbf{Disadvantages}:
\begin{itemize}
    \item Complex routing protocols
    \item Higher latency for multi-hop paths
    \item More difficult to coordinate globally
\end{itemize}

\subsection{Peer-to-Peer with Discovery}

All nodes are equal; they discover each other and communicate directly when in range:

\textbf{Advantages}:
\begin{itemize}
    \item Fully decentralized
    \item Scales well
    \item Natural fit for publish-subscribe middleware
\end{itemize}

\textbf{Disadvantages}:
\begin{itemize}
    \item Discovery overhead
    \item Requires careful QoS configuration
    \item Challenging for real-time guarantees
\end{itemize}

\section{The Publish-Subscribe Communication Model}

Traditional client-server communication requires knowing who to talk to. In dynamic multi-robot systems, this is impractical---robots join and leave, and the number of interested parties varies.

\subsection{Publish-Subscribe Fundamentals}

In publish-subscribe (pub-sub), communication is organized around \textbf{topics}:

\begin{itemize}
    \item \textbf{Publishers} send messages to a topic (without knowing who receives)
    \item \textbf{Subscribers} receive messages from a topic (without knowing who sends)
    \item \textbf{Middleware} handles discovery, routing, and delivery
\end{itemize}

%----------------------------------------------------------------------
% FIGURE: Publish-Subscribe Model
%----------------------------------------------------------------------
% Description: Diagram showing decoupled publishers and subscribers.
%
% Left side: Two "Publisher" boxes with arrows pointing to center
% Center: Cloud/box labeled "Topic: /drone/position"
% Right side: Three "Subscriber" boxes with arrows from center
%
% Show that publishers don't know about subscribers and vice versa.
% Dimensions: Column width, ~4cm height.
%----------------------------------------------------------------------

\begin{lstlisting}[language=C, caption=Conceptual publish-subscribe pattern]
// Publisher (Drone 1)
void PublishPosition(void) {
    PositionMsg msg;
    msg.drone_id = 1;
    msg.x = current_x;
    msg.y = current_y;
    msg.z = current_z;
    msg.timestamp = GetTime();

    // Publish to topic - middleware handles delivery
    Publish("/swarm/positions", &msg);
}

// Subscriber (Drone 2) - receives positions from ALL drones
void OnPositionReceived(PositionMsg *msg) {
    if (msg->drone_id != MY_ID) {
        UpdateNeighborPosition(msg->drone_id, msg->x, msg->y, msg->z);
        CheckCollisionRisk();
    }
}
\end{lstlisting}

\subsection{Benefits for Multi-Robot Systems}

\begin{enumerate}
    \item \textbf{Decoupling}: Publishers and subscribers are independent---adding a new drone requires no changes to existing code
    \item \textbf{Scalability}: One publisher can serve many subscribers efficiently
    \item \textbf{Flexibility}: Subscribers can filter by topic, not by source
    \item \textbf{Dynamic membership}: Robots can join/leave without reconfiguration
\end{enumerate}

\subsection{Topics and Message Types}

Topics are named channels with typed messages:

\begin{lstlisting}[language=C, caption=Example topic structure for drone swarm]
// Topic naming convention: /<domain>/<entity>/<data_type>

// Swarm-wide topics (all drones publish and subscribe)
"/swarm/positions"      // Position of each drone
"/swarm/status"         // Health/battery status
"/swarm/detections"     // Objects detected by any drone

// Individual drone topics (each drone has its own namespace)
"/drone_01/cmd_vel"     // Velocity commands for drone 1
"/drone_01/imu"         // IMU data from drone 1
"/drone_01/camera"      // Camera feed from drone 1

// Ground station topics
"/ground/mission"       // Mission commands
"/ground/geofence"      // No-fly zone updates
\end{lstlisting}

\section{Data Distribution Service (DDS)}

DDS is an open standard for real-time publish-subscribe middleware, widely used in aerospace, defense, and robotics.

\subsection{What is DDS?}

The Data Distribution Service (DDS) is a middleware standard defined by the Object Management Group (OMG). It provides:

\begin{itemize}
    \item \textbf{Data-centric} communication: Focus on data, not connections
    \item \textbf{Automatic discovery}: Participants find each other without configuration
    \item \textbf{Quality of Service}: Fine-grained control over reliability, latency, etc.
    \item \textbf{Peer-to-peer}: No central broker required
    \item \textbf{Real-time capable}: Designed for time-critical systems
\end{itemize}

\subsection{DDS Concepts}

\begin{center}
\begin{tabular}{lp{8cm}}
\toprule
\textbf{Concept} & \textbf{Description} \\
\midrule
Domain & Logical partition of the network; participants in same domain can communicate \\
DomainParticipant & Entry point to DDS; represents an application in the domain \\
Topic & Named data type; defines what data is shared \\
DataWriter & Publishes data to a topic \\
DataReader & Subscribes to data from a topic \\
Publisher & Groups DataWriters; manages publication policies \\
Subscriber & Groups DataReaders; manages subscription policies \\
\bottomrule
\end{tabular}
\end{center}

\subsection{Quality of Service (QoS) Policies}

DDS provides extensive QoS policies to match communication characteristics to application needs:

\begin{lstlisting}[language=C, caption=Key DDS QoS policies]
// Reliability: Best-effort vs. Reliable delivery
typedef enum {
    BEST_EFFORT,    // May lose messages (lower latency)
    RELIABLE        // Guarantees delivery (higher latency)
} ReliabilityKind;

// Durability: What happens to late joiners?
typedef enum {
    VOLATILE,           // Late joiners miss old data
    TRANSIENT_LOCAL,    // Writer keeps history for late joiners
    TRANSIENT,          // Separate service keeps history
    PERSISTENT          // Data survives system restart
} DurabilityKind;

// History: How much data to keep?
typedef struct {
    HistoryKind kind;   // KEEP_LAST or KEEP_ALL
    int32_t depth;      // Number of samples (for KEEP_LAST)
} HistoryQos;

// Deadline: Expected update rate
typedef struct {
    Duration period;    // Maximum time between updates
} DeadlineQos;

// Liveliness: Is the writer still alive?
typedef struct {
    LivelinessKind kind;  // AUTOMATIC, MANUAL_BY_PARTICIPANT, MANUAL_BY_TOPIC
    Duration lease;       // How long before considered dead
} LivelinessQos;
\end{lstlisting}

\begin{example}[QoS for Position Sharing]
For sharing drone positions in a swarm:
\begin{itemize}
    \item \textbf{Reliability}: Best-effort (position updates are frequent; losing one is acceptable)
    \item \textbf{History}: Keep last 1 (only current position matters)
    \item \textbf{Deadline}: 100 ms (expect updates at least 10 Hz)
    \item \textbf{Liveliness}: Automatic, 500 ms lease (detect if drone disconnects)
\end{itemize}
\end{example}

\subsection{DDS Discovery}

DDS uses a two-phase discovery protocol:

\begin{enumerate}
    \item \textbf{SPDP} (Simple Participant Discovery Protocol): Participants announce themselves via multicast; others learn who is in the domain
    \item \textbf{SEDP} (Simple Endpoint Discovery Protocol): Participants exchange information about their DataWriters and DataReaders; matching topics are connected
\end{enumerate}

\begin{lstlisting}[language=C, caption=Simplified discovery flow]
// Phase 1: Participant Discovery (SPDP)
// Drone 1 announces: "I am participant P1 at IP 192.168.1.10"
// Drone 2 announces: "I am participant P2 at IP 192.168.1.11"

// Phase 2: Endpoint Discovery (SEDP)
// P1 announces: "I have DataWriter for topic '/swarm/positions'"
// P2 announces: "I have DataReader for topic '/swarm/positions'"

// Middleware matches compatible endpoints automatically
// P1's writer connects to P2's reader (and any other matching readers)
\end{lstlisting}

\begin{notebox}[title=Discovery in Practice]
DDS discovery is automatic but can be slow on constrained networks. For drone swarms:
\begin{itemize}
    \item Use unicast peer lists instead of multicast when possible
    \item Configure discovery periods based on expected join/leave rates
    \item Consider static discovery for known, fixed configurations
\end{itemize}
\end{notebox}

\section{ROS 2: Robot Operating System}

ROS~2 (Robot Operating System 2) is a middleware framework for robotics that builds on DDS for communication.

\subsection{What is ROS 2?}

Despite its name, ROS is not an operating system but a \textbf{middleware framework} providing:

\begin{itemize}
    \item \textbf{Communication infrastructure}: Topics, services, actions (built on DDS)
    \item \textbf{Tools}: Visualization (RViz), logging, debugging
    \item \textbf{Libraries}: Common robotics algorithms, data types
    \item \textbf{Build system}: Colcon, CMake-based
    \item \textbf{Package ecosystem}: Thousands of reusable packages
\end{itemize}

\subsection{ROS 2 vs ROS 1}

ROS~2 was redesigned from scratch to address limitations of ROS~1:

\begin{center}
\begin{tabular}{lll}
\toprule
\textbf{Feature} & \textbf{ROS 1} & \textbf{ROS 2} \\
\midrule
Middleware & Custom (TCPROS/UDPROS) & DDS (standard) \\
Discovery & Central rosmaster & Peer-to-peer (DDS) \\
Real-time & Not designed for & Supported \\
Multi-robot & Difficult & Native support \\
Security & None built-in & DDS Security \\
Embedded & Not supported & micro-ROS \\
Platforms & Linux only & Linux, Windows, macOS, RTOS \\
\bottomrule
\end{tabular}
\end{center}

\subsection{ROS 2 Core Concepts}

\textbf{Nodes}: The basic computational unit in ROS~2. Each node is a process that performs a specific function.

\begin{lstlisting}[language=Python, caption=ROS 2 node example (Python)]
import rclpy
from rclpy.node import Node
from geometry_msgs.msg import PoseStamped

class DronePositionPublisher(Node):
    def __init__(self):
        super().__init__('drone_position_publisher')

        # Create publisher for position
        self.publisher = self.create_publisher(
            PoseStamped,
            '/drone/position',
            10  # QoS queue depth
        )

        # Create timer for periodic publishing
        self.timer = self.create_timer(0.1, self.publish_position)  # 10 Hz

    def publish_position(self):
        msg = PoseStamped()
        msg.header.stamp = self.get_clock().now().to_msg()
        msg.header.frame_id = 'world'
        msg.pose.position.x = self.get_x()
        msg.pose.position.y = self.get_y()
        msg.pose.position.z = self.get_z()
        self.publisher.publish(msg)

def main():
    rclpy.init()
    node = DronePositionPublisher()
    rclpy.spin(node)  # Process callbacks until shutdown
    node.destroy_node()
    rclpy.shutdown()
\end{lstlisting}

\textbf{Topics}: Named buses for publish-subscribe communication (same as DDS topics).

\textbf{Services}: Synchronous request-response communication for occasional operations:

\begin{lstlisting}[language=Python, caption=ROS 2 service example]
# Service definition (Arm.srv)
# ---
# bool success
# string message

from drone_interfaces.srv import Arm

class DroneController(Node):
    def __init__(self):
        super().__init__('drone_controller')

        # Create service
        self.arm_service = self.create_service(
            Arm,
            '/drone/arm',
            self.arm_callback
        )

    def arm_callback(self, request, response):
        if self.battery_ok() and self.sensors_ok():
            self.arm_motors()
            response.success = True
            response.message = "Motors armed"
        else:
            response.success = False
            response.message = "Pre-arm checks failed"
        return response
\end{lstlisting}

\textbf{Actions}: Asynchronous, long-running operations with feedback:

\begin{lstlisting}[language=Python, caption=ROS 2 action example]
# Used for tasks like "fly to waypoint" that take time
# Client sends goal, server provides feedback during execution, then result

from nav2_msgs.action import NavigateToPose

class WaypointNavigator(Node):
    def __init__(self):
        super().__init__('waypoint_navigator')

        self.action_server = ActionServer(
            self,
            NavigateToPose,
            'navigate_to_pose',
            self.navigate_callback
        )

    async def navigate_callback(self, goal_handle):
        target = goal_handle.request.pose

        while not self.at_target(target):
            # Compute and execute control
            self.fly_towards(target)

            # Publish feedback
            feedback = NavigateToPose.Feedback()
            feedback.distance_remaining = self.distance_to(target)
            goal_handle.publish_feedback(feedback)

            await asyncio.sleep(0.1)

        goal_handle.succeed()
        result = NavigateToPose.Result()
        return result
\end{lstlisting}

\section{ROS 2 and DDS Integration}

\subsection{The RMW Abstraction Layer}

ROS~2 doesn't implement DDS directly. Instead, it uses an abstraction layer called \textbf{RMW} (ROS Middleware Interface):

%----------------------------------------------------------------------
% FIGURE: ROS 2 Architecture Layers
%----------------------------------------------------------------------
% Description: Layered diagram showing ROS 2 stack.
%
% Top: "ROS 2 Application (Nodes)"
% Middle-upper: "ROS 2 Client Library (rclcpp, rclpy)"
% Middle: "RMW Interface"
% Middle-lower: "RMW Implementation (rmw_fastrtps, rmw_cyclonedds)"
% Bottom: "DDS Implementation (Fast DDS, Cyclone DDS, Connext)"
%
% Dimensions: Half column width, ~5cm height.
%----------------------------------------------------------------------

\begin{lstlisting}[language=bash, caption=Selecting DDS implementation at runtime]
# List available RMW implementations
$ ros2 doctor --report | grep middleware

# Set RMW implementation via environment variable
$ export RMW_IMPLEMENTATION=rmw_cyclonedds_cpp
$ ros2 run my_package my_node

# Or use Fast DDS (default in many ROS 2 distributions)
$ export RMW_IMPLEMENTATION=rmw_fastrtps_cpp
\end{lstlisting}

\subsection{Mapping ROS 2 to DDS}

\begin{center}
\begin{tabular}{ll}
\toprule
\textbf{ROS 2 Concept} & \textbf{DDS Concept} \\
\midrule
Node & DomainParticipant (typically) \\
Topic & Topic \\
Publisher & Publisher + DataWriter \\
Subscription & Subscriber + DataReader \\
Message type & IDL-defined data type \\
QoS profile & QoS policies \\
\bottomrule
\end{tabular}
\end{center}

\subsection{QoS Profiles in ROS 2}

ROS~2 provides pre-defined QoS profiles for common use cases:

\begin{lstlisting}[language=Python, caption=ROS 2 QoS profiles]
from rclpy.qos import QoSProfile, ReliabilityPolicy, HistoryPolicy, DurabilityPolicy

# Sensor data: high rate, some loss acceptable
sensor_qos = QoSProfile(
    reliability=ReliabilityPolicy.BEST_EFFORT,
    history=HistoryPolicy.KEEP_LAST,
    depth=5,
    durability=DurabilityPolicy.VOLATILE
)

# Commands: must be delivered reliably
command_qos = QoSProfile(
    reliability=ReliabilityPolicy.RELIABLE,
    history=HistoryPolicy.KEEP_LAST,
    depth=10,
    durability=DurabilityPolicy.TRANSIENT_LOCAL
)

# Create publisher with specific QoS
self.position_pub = self.create_publisher(
    PoseStamped,
    '/drone/position',
    sensor_qos
)

self.command_sub = self.create_subscription(
    Twist,
    '/drone/cmd_vel',
    self.cmd_callback,
    command_qos
)
\end{lstlisting}

\begin{warningbox}[title=QoS Compatibility]
Publishers and subscribers must have compatible QoS settings. Common issues:
\begin{itemize}
    \item Reliable publisher + best-effort subscriber: Works (subscriber may lose data)
    \item Best-effort publisher + reliable subscriber: \textbf{Incompatible!} No connection.
    \item Transient-local publisher + volatile subscriber: Works (subscriber misses old data)
    \item Volatile publisher + transient-local subscriber: \textbf{Incompatible!}
\end{itemize}
\end{warningbox}

\section{ROS 2 Network Topology for Multi-Robot Systems}

\subsection{Domain IDs}

DDS domains partition the network. In ROS~2, the \texttt{ROS\_DOMAIN\_ID} environment variable controls which domain a node joins:

\begin{lstlisting}[language=bash, caption=Using domain IDs]
# Default domain (ID 0) - all nodes see each other
$ ros2 run drone_pkg drone_node

# Separate domain for testing
$ ROS_DOMAIN_ID=42 ros2 run drone_pkg drone_node

# Different swarms on same network
$ ROS_DOMAIN_ID=1 ros2 run drone_pkg drone_node  # Swarm 1
$ ROS_DOMAIN_ID=2 ros2 run drone_pkg drone_node  # Swarm 2
\end{lstlisting}

\subsection{Namespacing for Multi-Robot}

Use namespaces to distinguish between robots:

\begin{lstlisting}[language=Python, caption=Namespaced nodes for multi-robot]
# Launch file spawning multiple drones
from launch import LaunchDescription
from launch_ros.actions import Node

def generate_launch_description():
    drones = []

    for i in range(4):
        drone_node = Node(
            package='drone_controller',
            executable='controller',
            namespace=f'drone_{i}',  # /drone_0, /drone_1, etc.
            parameters=[{'drone_id': i}],
            remappings=[
                # Local topics stay in namespace
                ('imu', 'imu'),
                ('cmd_vel', 'cmd_vel'),
                # Global topics go to shared namespace
                ('/swarm/positions', '/swarm/positions'),
            ]
        )
        drones.append(drone_node)

    return LaunchDescription(drones)
\end{lstlisting}

\begin{lstlisting}[language=bash, caption=Resulting topic structure]
# Per-drone topics (namespaced)
/drone_0/imu
/drone_0/cmd_vel
/drone_0/position
/drone_1/imu
/drone_1/cmd_vel
/drone_1/position
...

# Swarm-wide topics (shared)
/swarm/positions      # All drones publish here
/swarm/formation_cmd  # Commands to entire swarm
/ground/mission       # Mission from ground station
\end{lstlisting}

\subsection{Discovery Configuration}

For large swarms or unreliable networks, configure discovery carefully:

\begin{lstlisting}[language=xml, caption=Fast DDS discovery configuration (fastdds.xml)]
<?xml version="1.0" encoding="UTF-8"?>
<dds xmlns="http://www.eprosima.com/XMLSchemas/fastRTPS_Profiles">
    <profiles>
        <participant profile_name="drone_participant" is_default_profile="true">
            <rtps>
                <builtin>
                    <discovery_config>
                        <!-- Increase discovery period for battery savings -->
                        <leaseDuration>
                            <sec>10</sec>
                        </leaseDuration>

                        <!-- Use unicast for specific peers (more reliable than multicast) -->
                        <initialPeersList>
                            <locator>
                                <udpv4>
                                    <address>192.168.1.1</address>  <!-- Ground station -->
                                </udpv4>
                            </locator>
                            <locator>
                                <udpv4>
                                    <address>192.168.1.10</address>  <!-- Drone 1 -->
                                </udpv4>
                            </locator>
                        </initialPeersList>
                    </discovery_config>
                </builtin>
            </rtps>
        </participant>
    </profiles>
</dds>
\end{lstlisting}

\section{micro-ROS: ROS 2 on Microcontrollers}

Standard ROS~2 requires a full operating system (Linux, Windows). For embedded flight controllers running FreeRTOS, \textbf{micro-ROS} provides ROS~2 compatibility.

\subsection{What is micro-ROS?}

micro-ROS is the official ROS~2 implementation for microcontrollers:

\begin{center}
\begin{tabular}{ll}
\toprule
\textbf{Feature} & \textbf{Description} \\
\midrule
Target hardware & ARM Cortex-M, ESP32, other MCUs \\
Supported RTOS & FreeRTOS, Zephyr, NuttX \\
Memory footprint & $\sim$100 KB Flash, $\sim$30 KB RAM (minimum) \\
DDS implementation & Micro XRCE-DDS (client-agent model) \\
ROS 2 compatibility & Nodes, topics, services, parameters \\
\bottomrule
\end{tabular}
\end{center}

\subsection{Client-Agent Architecture}

Unlike full ROS~2, micro-ROS uses a \textbf{client-agent} model to reduce resource requirements:

%----------------------------------------------------------------------
% FIGURE: micro-ROS Architecture
%----------------------------------------------------------------------
% Description: Diagram showing client-agent communication.
%
% Left: "Microcontroller (FreeRTOS)" box containing:
%   - "micro-ROS Client"
%   - "Flight Controller App"
%
% Center: "Serial/UDP/TCP" bidirectional arrow
%
% Right: "Linux Computer" box containing:
%   - "micro-ROS Agent"
%   - "DDS"
%   - Connection to "ROS 2 Network"
%
% Dimensions: Column width, ~4cm height.
%----------------------------------------------------------------------

\begin{itemize}
    \item \textbf{Client} (on MCU): Lightweight, handles local pub/sub
    \item \textbf{Agent} (on Linux): Bridges to full DDS network
    \item \textbf{Transport}: Serial (UART), UDP, TCP, or custom
\end{itemize}

\begin{lstlisting}[language=bash, caption=Running the micro-ROS agent]
# Install micro-ROS agent
$ sudo apt install ros-humble-micro-ros-agent

# Run agent for serial connection
$ ros2 run micro_ros_agent micro_ros_agent serial --dev /dev/ttyUSB0

# Run agent for UDP connection
$ ros2 run micro_ros_agent micro_ros_agent udp4 --port 8888
\end{lstlisting}

\subsection{micro-ROS on FreeRTOS}

Integration with FreeRTOS requires proper task setup:

See Listing~\ref{lst:microros-freertos} (Appendix~\ref{app:code-listings}) for a complete micro-ROS integration with FreeRTOS, including agent connection handling, position publisher setup, timer callbacks, and proper task priority configuration.

\subsection{Subscribing to Commands}

\begin{lstlisting}[language=C, caption=Receiving commands via micro-ROS]
#include <geometry_msgs/msg/twist.h>

rcl_subscription_t cmd_subscription;
geometry_msgs__msg__Twist cmd_msg;

// Callback when command received
void cmd_callback(const void *msgin) {
    const geometry_msgs__msg__Twist *msg = (const geometry_msgs__msg__Twist *)msgin;

    // Apply velocity command to flight controller
    SetVelocityCommand(msg->linear.x, msg->linear.y, msg->linear.z,
                       msg->angular.z);
}

// In initialization:
rclc_subscription_init_default(
    &cmd_subscription,
    &node,
    ROSIDL_GET_MSG_TYPE_SUPPORT(geometry_msgs, msg, Twist),
    "/drone/cmd_vel"
);

rclc_executor_add_subscription(&executor, &cmd_subscription,
                               &cmd_msg, &cmd_callback, ON_NEW_DATA);
\end{lstlisting}

\subsection{Resource Considerations}

\begin{warningbox}[title=micro-ROS Resource Usage]
micro-ROS requires careful resource management:
\begin{itemize}
    \item \textbf{Memory}: Each publisher/subscriber uses $\sim$1--2 KB RAM
    \item \textbf{Stack}: micro-ROS task needs 4--8 KB stack
    \item \textbf{Timing}: Executor spin should not block control loops
    \item \textbf{Priority}: Run micro-ROS at lower priority than flight-critical tasks
    \item \textbf{Agent dependency}: No communication without agent connection
\end{itemize}
\end{warningbox}

\section{Swarm Coordination Fundamentals}

\subsection{Formation Control}

Formation control maintains desired geometric relationships between drones:

\begin{lstlisting}[language=Python, caption=Simple formation control]
import numpy as np

class FormationController:
    def __init__(self, drone_id, formation_offsets):
        """
        formation_offsets: dict mapping drone_id to [dx, dy, dz] offset
                          from formation center
        """
        self.drone_id = drone_id
        self.my_offset = np.array(formation_offsets[drone_id])

    def compute_target(self, formation_center, formation_yaw):
        """Compute this drone's target position given formation center."""
        # Rotate offset by formation heading
        c, s = np.cos(formation_yaw), np.sin(formation_yaw)
        R = np.array([[c, -s, 0], [s, c, 0], [0, 0, 1]])

        rotated_offset = R @ self.my_offset
        target = formation_center + rotated_offset

        return target

# Example: Diamond formation
offsets = {
    0: [0, 2, 0],    # Drone 0: front
    1: [-1.5, 0, 0], # Drone 1: left
    2: [1.5, 0, 0],  # Drone 2: right
    3: [0, -2, 0],   # Drone 3: back
}
\end{lstlisting}

\subsection{Leader-Follower Architecture}

One drone (leader) navigates; others (followers) maintain relative positions:

\begin{lstlisting}[language=Python, caption=Leader-follower implementation]
class LeaderNode(Node):
    def __init__(self):
        super().__init__('leader')

        # Publish leader's position for followers
        self.position_pub = self.create_publisher(
            PoseStamped, '/leader/position', 10
        )

        # Navigate based on mission waypoints
        self.waypoint_sub = self.create_subscription(
            PoseStamped, '/mission/waypoint', self.waypoint_callback, 10
        )

class FollowerNode(Node):
    def __init__(self, offset):
        super().__init__('follower')
        self.offset = offset

        # Subscribe to leader position
        self.leader_sub = self.create_subscription(
            PoseStamped, '/leader/position', self.leader_callback, 10
        )

    def leader_callback(self, msg):
        # Compute target as offset from leader
        target_x = msg.pose.position.x + self.offset[0]
        target_y = msg.pose.position.y + self.offset[1]
        target_z = msg.pose.position.z + self.offset[2]

        self.fly_to(target_x, target_y, target_z)
\end{lstlisting}

\subsection{Collision Avoidance}

When sharing airspace, drones must avoid each other:

\begin{lstlisting}[language=C, caption=Simple collision avoidance]
#define SAFETY_RADIUS 1.0f  // meters
#define AVOIDANCE_GAIN 2.0f

typedef struct {
    uint8_t id;
    float x, y, z;
    float vx, vy, vz;
} NeighborState_t;

NeighborState_t neighbors[MAX_NEIGHBORS];
int num_neighbors;

void ComputeAvoidanceVelocity(float *avoidance_vx, float *avoidance_vy, float *avoidance_vz) {
    *avoidance_vx = 0;
    *avoidance_vy = 0;
    *avoidance_vz = 0;

    for (int i = 0; i < num_neighbors; i++) {
        // Vector from neighbor to self
        float dx = my_x - neighbors[i].x;
        float dy = my_y - neighbors[i].y;
        float dz = my_z - neighbors[i].z;

        float distance = sqrtf(dx*dx + dy*dy + dz*dz);

        if (distance < SAFETY_RADIUS && distance > 0.01f) {
            // Repulsion force: stronger when closer
            float strength = AVOIDANCE_GAIN * (SAFETY_RADIUS - distance) / distance;

            *avoidance_vx += strength * dx;
            *avoidance_vy += strength * dy;
            *avoidance_vz += strength * dz;
        }
    }
}

void VelocityController(float target_vx, float target_vy, float target_vz) {
    float avoid_vx, avoid_vy, avoid_vz;
    ComputeAvoidanceVelocity(&avoid_vx, &avoid_vy, &avoid_vz);

    // Combine target velocity with avoidance
    float cmd_vx = target_vx + avoid_vx;
    float cmd_vy = target_vy + avoid_vy;
    float cmd_vz = target_vz + avoid_vz;

    ExecuteVelocityCommand(cmd_vx, cmd_vy, cmd_vz);
}
\end{lstlisting}

\section{Complete Example: Multi-Drone Position Sharing}

This example shows a complete swarm setup where drones share positions.

\subsection{System Architecture}

%----------------------------------------------------------------------
% FIGURE: Swarm System Architecture
%----------------------------------------------------------------------
% Description: Complete system diagram.
%
% Show:
%   - 3 drones, each with: STM32 + FreeRTOS + micro-ROS client
%   - Raspberry Pi companion computers running micro-ROS agent
%   - Wireless network connecting all
%   - Ground station with ROS 2 and visualization
%
% Dimensions: Column width, ~6cm height.
%----------------------------------------------------------------------

\begin{lstlisting}[language=C, caption=Complete micro-ROS swarm node (on STM32)]
#include "FreeRTOS.h"
#include "task.h"
#include <rcl/rcl.h>
#include <rclc/rclc.h>
#include <geometry_msgs/msg/pose_stamped.h>

#define SWARM_SIZE 4
#define MY_DRONE_ID 0

// Shared state (protected by mutex)
typedef struct {
    float x, y, z;
    uint32_t timestamp;
    bool valid;
} DronePosition_t;

DronePosition_t swarm_positions[SWARM_SIZE];
SemaphoreHandle_t swarm_mutex;

// ROS entities
rcl_node_t node;
rcl_publisher_t my_position_pub;
rcl_subscription_t swarm_position_sub;
geometry_msgs__msg__PoseStamped swarm_msg;

// Callback: position received from another drone
void swarm_position_callback(const void *msgin) {
    const geometry_msgs__msg__PoseStamped *msg =
        (const geometry_msgs__msg__PoseStamped *)msgin;

    // Extract drone ID from frame_id (e.g., "drone_1")
    int drone_id = -1;
    sscanf(msg->header.frame_id.data, "drone_%d", &drone_id);

    if (drone_id >= 0 && drone_id < SWARM_SIZE && drone_id != MY_DRONE_ID) {
        xSemaphoreTake(swarm_mutex, portMAX_DELAY);
        swarm_positions[drone_id].x = msg->pose.position.x;
        swarm_positions[drone_id].y = msg->pose.position.y;
        swarm_positions[drone_id].z = msg->pose.position.z;
        swarm_positions[drone_id].timestamp = HAL_GetTick();
        swarm_positions[drone_id].valid = true;
        xSemaphoreGive(swarm_mutex);
    }
}

// Timer: publish my position
void publish_timer_callback(rcl_timer_t *timer, int64_t last_call_time) {
    geometry_msgs__msg__PoseStamped msg;

    // Get current position from state estimator
    msg.pose.position.x = StateEstimator_GetX();
    msg.pose.position.y = StateEstimator_GetY();
    msg.pose.position.z = StateEstimator_GetZ();

    // Set frame_id to identify this drone
    static char frame_id[16];
    snprintf(frame_id, sizeof(frame_id), "drone_%d", MY_DRONE_ID);
    msg.header.frame_id.data = frame_id;
    msg.header.frame_id.size = strlen(frame_id);

    rcl_publish(&my_position_pub, &msg, NULL);
}

void MicroRosTask(void *argument) {
    rcl_allocator_t allocator = rcl_get_default_allocator();
    rclc_support_t support;
    rclc_executor_t executor;

    // Wait for agent
    while (rmw_uros_ping_agent(100, 1) != RMW_RET_OK) {
        vTaskDelay(pdMS_TO_TICKS(500));
    }

    // Initialize
    rclc_support_init(&support, 0, NULL, &allocator);

    char node_name[32];
    snprintf(node_name, sizeof(node_name), "drone_%d_fc", MY_DRONE_ID);
    rclc_node_init_default(&node, node_name, "", &support);

    // Publisher: my position
    rclc_publisher_init_best_effort(
        &my_position_pub, &node,
        ROSIDL_GET_MSG_TYPE_SUPPORT(geometry_msgs, msg, PoseStamped),
        "/swarm/positions"
    );

    // Subscription: all positions (including mine, but we filter)
    rclc_subscription_init_best_effort(
        &swarm_position_sub, &node,
        ROSIDL_GET_MSG_TYPE_SUPPORT(geometry_msgs, msg, PoseStamped),
        "/swarm/positions"
    );

    // Timer: 20 Hz position publish
    rcl_timer_t publish_timer;
    rclc_timer_init_default(&publish_timer, &support,
                            RCL_MS_TO_NS(50), publish_timer_callback);

    // Executor with timer and subscription
    rclc_executor_init(&executor, &support.context, 2, &allocator);
    rclc_executor_add_timer(&executor, &publish_timer);
    rclc_executor_add_subscription(&executor, &swarm_position_sub,
                                   &swarm_msg, swarm_position_callback,
                                   ON_NEW_DATA);

    // Main loop
    for (;;) {
        rclc_executor_spin_some(&executor, RCL_MS_TO_NS(10));
        vTaskDelay(pdMS_TO_TICKS(10));
    }
}

// Control task uses swarm positions for collision avoidance
void PositionControlTask(void *argument) {
    for (;;) {
        xSemaphoreTake(xControlSemaphore, portMAX_DELAY);

        // Get target from mission
        float target_x, target_y, target_z;
        Mission_GetTarget(&target_x, &target_y, &target_z);

        // Compute collision avoidance from swarm positions
        float avoid_x = 0, avoid_y = 0, avoid_z = 0;
        xSemaphoreTake(swarm_mutex, portMAX_DELAY);
        for (int i = 0; i < SWARM_SIZE; i++) {
            if (i != MY_DRONE_ID && swarm_positions[i].valid) {
                // Check if data is fresh (< 500ms old)
                if (HAL_GetTick() - swarm_positions[i].timestamp < 500) {
                    float dx = StateEstimator_GetX() - swarm_positions[i].x;
                    float dy = StateEstimator_GetY() - swarm_positions[i].y;
                    float dz = StateEstimator_GetZ() - swarm_positions[i].z;
                    float dist = sqrtf(dx*dx + dy*dy + dz*dz);

                    if (dist < SAFETY_RADIUS && dist > 0.1f) {
                        float strength = 2.0f * (SAFETY_RADIUS - dist) / dist;
                        avoid_x += strength * dx;
                        avoid_y += strength * dy;
                        avoid_z += strength * dz;
                    }
                }
            }
        }
        xSemaphoreGive(swarm_mutex);

        // Run position controller with avoidance offset
        PositionController_Run(target_x + avoid_x,
                               target_y + avoid_y,
                               target_z + avoid_z);
    }
}

int main(void) {
    HAL_Init();
    SystemClock_Config();

    // Create mutex for shared swarm state
    swarm_mutex = xSemaphoreCreateMutex();

    // Initialize swarm positions as invalid
    for (int i = 0; i < SWARM_SIZE; i++) {
        swarm_positions[i].valid = false;
    }

    // Create tasks
    xTaskCreate(MicroRosTask, "MicroROS", 4096, NULL, 3, NULL);
    xTaskCreate(AttitudeControlTask, "AttCtrl", 512, NULL, 6, NULL);
    xTaskCreate(PositionControlTask, "PosCtrl", 512, NULL, 5, NULL);
    xTaskCreate(SensorTask, "Sensor", 256, NULL, 7, NULL);

    vTaskStartScheduler();
    for (;;);
}
\end{lstlisting}

\begin{keyidea}[title=Multi-Robot Communication Summary]
\begin{enumerate}
    \item \textbf{DDS} provides robust, real-time publish-subscribe middleware
    \item \textbf{ROS 2} builds on DDS to offer a complete robotics framework
    \item \textbf{micro-ROS} enables ROS 2 on FreeRTOS microcontrollers via client-agent model
    \item \textbf{QoS policies} must be configured for swarm communication needs
    \item \textbf{Namespacing} and domain IDs organize multi-robot topics
    \item \textbf{Priority separation}: Communication tasks must not interfere with flight-critical control
\end{enumerate}
\end{keyidea}

\section{Summary}

This module covered real-time embedded systems for quadrotor flight controllers:

\begin{enumerate}
    \item \textbf{RTOS Fundamentals}
    \begin{itemize}
        \item Multi-rate control requires task scheduling
        \item Timing precision affects control stability
        \item RTOS provides predictable, preemptive scheduling
    \end{itemize}

    \item \textbf{Concurrency hazards}: Race conditions, data races, and how to prevent them

    \item \textbf{Synchronization primitives}: Semaphores, mutexes, queues---when and how to use each

    \item \textbf{Deadlocks and priority inversion}: How to prevent and handle these problems

    \item \textbf{Scheduling theory}: RMS, EDF, utilization bounds, response-time analysis

    \item \textbf{Timing analysis}: Latency, jitter, and their effects on control performance

    \item \textbf{WCET estimation}: Measurement-based and static analysis approaches

    \item \textbf{Fault handling}: Watchdogs, deadline monitoring, failsafe behavior
\end{enumerate}

\begin{keyidea}[title=Key Takeaway]
A flight controller is a real-time system where timing correctness is as important as functional correctness. Proper scheduling ensures that critical control loops meet their deadlines, while proper synchronization ensures that shared data remains consistent. Together, these techniques make the difference between a quadrotor that flies reliably and one that crashes unpredictably.
\end{keyidea}


% ===== MODULE 4 =====
\part{Module 4: Testing and Verification}
%   - CPS Testing Fundamentals
%   - Testing Levels (MIL/SIL/PIL/HIL/VIL)
%   - Signal Temporal Logic and Robustness
%   - Falsification-Based Testing
%   - Code Coverage for Safety-Critical Systems
%   - Debugging Embedded Systems
%======================================================================
% MODULE 4: Testing and Verification of Cyber-Physical Systems
%
% Learning Objectives:
% After completing this module, students will be able to:
% - Explain why testing CPS is fundamentally different from testing discrete software
% - Choose appropriate testing levels (MIL, SIL, HIL) for different development phases
% - Write formal requirements using Signal Temporal Logic
% - Compute robustness values for STL specifications
% - Set up and run falsification experiments using optimization
% - Apply debugging techniques for embedded flight controllers
%======================================================================

\chapter{Introduction to CPS Testing}
\index{testing}

\section{Why Testing Cyber-Physical Systems is Hard}

Testing a cyber-physical system like a quadrotor flight controller is fundamentally different from testing traditional software. To understand why, consider the differences:

\textbf{Traditional software testing}: The program takes discrete inputs and produces discrete outputs. You can enumerate test cases, check outputs against expected values, and achieve meaningful coverage of the input space.

\textbf{CPS testing}: The controller interacts with continuous physical dynamics. Inputs are continuous signals (sensor readings over time), outputs are continuous signals (motor commands over time), and the behavior depends on differential equations that evolve in continuous time.

\begin{keyidea}[title=The Continuous State Space Problem]
A quadrotor's state includes position (3D), velocity (3D), orientation (4D quaternion), and angular velocity (3D)---at minimum 13 continuous variables. Each variable takes values from a continuous range. The space of possible trajectories is \textbf{uncountably infinite}.

Unlike discrete software where you might test all branches, you cannot test all possible flight trajectories. Testing must be strategic, focusing on scenarios most likely to reveal problems.
\end{keyidea}

\subsection{What Makes Quadrotor Testing Challenging}

\begin{enumerate}
    \item \textbf{Nonlinear dynamics}: Small changes in initial conditions can lead to dramatically different trajectories. A controller that works perfectly in one scenario may fail in a slightly different one.

    \item \textbf{Real-time constraints}: The controller must respond within milliseconds. A correct algorithm that runs too slowly is as bad as an incorrect one.

    \item \textbf{Environmental uncertainty}: Wind gusts, sensor noise, and battery voltage variations create disturbances that are impossible to predict exactly.

    \item \textbf{Coupled subsystems}: Attitude control affects position control; sensor fusion affects both. Bugs may only appear when subsystems interact.

    \item \textbf{Safety criticality}: A software crash on your laptop is annoying. A flight controller crash means a falling quadrotor---potential injury, property damage, or worse.
\end{enumerate}

\subsection{The Role of Models in Testing}

Since we cannot test exhaustively, we rely on \textbf{models} to:
\begin{itemize}
    \item Simulate the physical plant (quadrotor dynamics) without risking hardware
    \item Generate test scenarios systematically
    \item Specify requirements formally so they can be checked automatically
    \item Predict behavior in conditions we haven't explicitly tested
\end{itemize}

\begin{notebox}
The quality of testing is limited by the quality of models. A test that passes in simulation only means the \textit{model} of the quadrotor behaves correctly. Whether the \textit{real} quadrotor behaves the same depends on model fidelity---how accurately the model captures reality.
\end{notebox}

\section{Verification vs. Validation vs. Testing}

These terms are often confused. In the context of CPS development:

\begin{definition}[Verification]
\index{verification}
\textbf{Verification}\index{verification|textbf} asks: ``Did we build the system right?'' It checks that the implementation matches the specification. Formal verification uses mathematical proofs; testing-based verification uses systematic test cases.
\end{definition}

\begin{definition}[Validation]
\index{validation}
\textbf{Validation}\index{validation|textbf} asks: ``Did we build the right system?'' It checks that the specification captures what users actually need. A controller might perfectly meet its specification yet still be useless if the specification was wrong.
\end{definition}

\begin{definition}[Testing]
\textbf{Testing} executes the system with specific inputs and checks outputs against expected behavior. Testing can find bugs (show presence of errors) but cannot prove their absence---there's always another test case you haven't tried.
\end{definition}

\section{The Fundamental Limitation}

\begin{warningbox}
For hybrid systems (combining discrete software states with continuous physical dynamics), the problem of determining whether a given state is reachable is \textbf{undecidable}.

This means: No algorithm exists that can always determine, in finite time, whether a specification is satisfied for all possible inputs and initial conditions.
\end{warningbox}

This undecidability result has profound implications:
\begin{itemize}
    \item We cannot build a ``verify'' button that guarantees safety for arbitrary CPS
    \item Formal verification methods work only for restricted system classes
    \item For general CPS, we must rely on testing, which can never be complete
    \item Safety-critical systems require defense in depth: multiple complementary methods
\end{itemize}

Despite this limitation, we can still achieve high confidence through:
\begin{enumerate}
    \item \textbf{Formal methods} for critical subsystems that fit within decidable classes
    \item \textbf{Systematic testing} guided by formal specifications
    \item \textbf{Runtime monitoring} to detect violations during operation
    \item \textbf{Fail-safe mechanisms} to handle unexpected situations
\end{enumerate}

%======================================================================
\chapter{Development Process and Testing Levels}
%======================================================================

\section{The V-Model for CPS Development}
\index{V-model}

The V-model~\cite{forsberg2005visualizing}\index{V-model|textbf} is a development framework that maps each design phase to a corresponding testing phase:

\begin{center}
\begin{tikzpicture}[scale=0.9, every node/.style={font=\small}]
    % Left side - development (descending from top-left to center-bottom)
    \node[draw, rectangle, minimum width=3cm, minimum height=0.8cm] (req) at (0, 3) {Requirements};
    \node[draw, rectangle, minimum width=3cm, minimum height=0.8cm] (arch) at (2, 2) {Architecture};
    \node[draw, rectangle, minimum width=3cm, minimum height=0.8cm] (detail) at (4, 1) {Detailed Design};
    \node[draw, rectangle, minimum width=3cm, minimum height=0.8cm] (impl) at (6, 0) {Implementation};

    % Right side - testing (ascending from center-bottom to top-right)
    \node[draw, rectangle, minimum width=3cm, minimum height=0.8cm] (unit) at (8, 1) {Unit Testing};
    \node[draw, rectangle, minimum width=3cm, minimum height=0.8cm] (integ) at (10, 2) {Integration Testing};
    \node[draw, rectangle, minimum width=3cm, minimum height=0.8cm] (sys) at (12, 3) {System Testing};
    \node[draw, rectangle, minimum width=3cm, minimum height=0.8cm] (accept) at (14, 4) {Acceptance Testing};

    % Arrows down left
    \draw[-{Stealth}] (req.south east) -- (arch.north west);
    \draw[-{Stealth}] (arch.south east) -- (detail.north west);
    \draw[-{Stealth}] (detail.south east) -- (impl.north west);

    % Arrows up right
    \draw[-{Stealth}] (impl.north east) -- (unit.south west);
    \draw[-{Stealth}] (unit.north east) -- (integ.south west);
    \draw[-{Stealth}] (integ.north east) -- (sys.south west);
    \draw[-{Stealth}] (sys.north east) -- (accept.south west);

    % Horizontal arrows (traceability)
    \draw[dashed, -{Stealth}] (req.east) -- (accept.west) node[midway, above] {validates};
    \draw[dashed, -{Stealth}] (arch.east) -- (sys.west) node[midway, above] {verifies};
    \draw[dashed, -{Stealth}] (detail.east) -- (integ.west) node[midway, above] {verifies};
\end{tikzpicture}
\end{center}

For a quadrotor flight controller:
\begin{itemize}
    \item \textbf{Requirements}: ``The quadrotor shall maintain hover within 10 cm for 30 seconds''
    \item \textbf{Architecture}: Cascaded control (outer position loop, inner attitude loop)
    \item \textbf{Detailed Design}: PID gains, filter coefficients, task priorities
    \item \textbf{Implementation}: C code for FreeRTOS tasks
    \item \textbf{Unit Testing}: Test each function (e.g., quaternion multiplication)
    \item \textbf{Integration Testing}: Test subsystems together (sensor fusion + attitude control)
    \item \textbf{System Testing}: Test complete flight controller in simulation
    \item \textbf{Acceptance Testing}: Test on real hardware in flight
\end{itemize}

\section{Testing Levels for CPS}

CPS development uses specialized testing levels that progressively increase realism:

\subsection{Model-in-the-Loop (MIL) Testing}
\index{MIL (Model-in-the-Loop)}

In MIL testing\index{MIL (Model-in-the-Loop)|textbf}, both the controller and plant are \textbf{models} running in a simulation environment (e.g., MATLAB/Simulink).

\begin{center}
\begin{tikzpicture}[node distance=2cm]
    \node[draw, rectangle, minimum width=2.5cm, minimum height=1cm, fill=blue!10] (ctrl) {Controller Model};
    \node[draw, rectangle, minimum width=2.5cm, minimum height=1cm, fill=green!10, right=of ctrl] (plant) {Plant Model};

    \draw[-{Stealth}] (ctrl.east) -- (plant.west) node[midway, above] {$u$};
    \draw[-{Stealth}] (plant.south) -- ++(0, -0.5) -| (ctrl.south) node[near start, right] {$y$};

    \node[draw, dashed, fit=(ctrl)(plant), inner sep=10pt, label=above:Simulation Environment] {};
\end{tikzpicture}
\end{center}

\textbf{Advantages}:
\begin{itemize}
    \item Fast execution (faster than real-time possible)
    \item Complete observability of all internal states
    \item Easy to run thousands of test scenarios
    \item No risk to hardware
    \item Reproducible results
\end{itemize}

\textbf{Limitations}:
\begin{itemize}
    \item Controller model may not match actual implementation
    \item Plant model may not capture all physical effects
    \item No real-time constraints tested
\end{itemize}

\textbf{Use for quadrotors}: Early algorithm development, control design iteration, extensive parameter sweeps.

\subsection{Software-in-the-Loop (SIL) Testing}
\index{SIL (Software-in-the-Loop)}

In SIL testing\index{SIL (Software-in-the-Loop)|textbf}, the \textbf{actual controller code} runs on a development computer, interfacing with a plant model.

\begin{center}
\begin{tikzpicture}[node distance=2cm]
    \node[draw, rectangle, minimum width=2.5cm, minimum height=1cm, fill=blue!30] (ctrl) {Controller Code};
    \node[draw, rectangle, minimum width=2.5cm, minimum height=1cm, fill=green!10, right=of ctrl] (plant) {Plant Model};

    \draw[-{Stealth}] (ctrl.east) -- (plant.west) node[midway, above] {$u$};
    \draw[-{Stealth}] (plant.south) -- ++(0, -0.5) -| (ctrl.south) node[near start, right] {$y$};

    \node[draw, dashed, fit=(ctrl)(plant), inner sep=10pt, label=above:Development Computer] {};
\end{tikzpicture}
\end{center}

\textbf{Advantages}:
\begin{itemize}
    \item Tests actual implementation (same C code as embedded target)
    \item Can detect implementation bugs (overflow, precision loss)
    \item Still fast and reproducible
    \item Can use debugging tools (breakpoints, memory inspection)
\end{itemize}

\textbf{Limitations}:
\begin{itemize}
    \item Different processor architecture than target (x86 vs ARM)
    \item No real-time scheduling tested
    \item No hardware peripherals (I2C, SPI, PWM)
\end{itemize}

\textbf{Use for quadrotors}: Testing control algorithms after porting to C, verifying numerical correctness, debugging logic errors.

\subsection{Processor-in-the-Loop (PIL) Testing}

In PIL testing, the controller code runs on the \textbf{actual target processor}, but the plant is still simulated on a host computer.

\begin{center}
\begin{tikzpicture}[node distance=2.5cm]
    \node[draw, rectangle, minimum width=2.5cm, minimum height=1cm, fill=blue!50] (ctrl) {Controller Code};
    \node[draw, rectangle, minimum width=2.5cm, minimum height=1cm, fill=green!10, right=of ctrl] (plant) {Plant Model};

    \draw[-{Stealth}] (ctrl.east) -- (plant.west) node[midway, above] {$u$ (serial)};
    \draw[-{Stealth}] (plant.south) -- ++(0, -0.5) -| (ctrl.south) node[near start, right] {$y$ (serial)};

    \node[draw, dashed, fit=(ctrl), inner sep=5pt, label=below:Embedded Target] {};
    \node[draw, dashed, fit=(plant), inner sep=5pt, label=below:Host Computer] {};
\end{tikzpicture}
\end{center}

\textbf{Advantages}:
\begin{itemize}
    \item Tests on actual target processor (correct instruction set, FPU behavior)
    \item Reveals architecture-specific bugs (endianness, alignment)
    \item Can measure actual execution times
\end{itemize}

\textbf{Limitations}:
\begin{itemize}
    \item Communication overhead slows testing
    \item Still no real hardware peripherals
    \item Real-time behavior may be affected by communication
\end{itemize}

\subsection{Hardware-in-the-Loop (HIL) Testing}
\index{HIL (Hardware-in-the-Loop)}

In HIL testing\index{HIL (Hardware-in-the-Loop)|textbf}, the complete embedded system (controller + hardware interfaces) runs in real-time, but the plant is simulated by specialized real-time hardware.

\begin{center}
\begin{tikzpicture}[node distance=2.5cm]
    \node[draw, rectangle, minimum width=2.8cm, minimum height=1.2cm, fill=blue!50] (ctrl) {Flight Controller};
    \node[draw, rectangle, minimum width=2.8cm, minimum height=1.2cm, fill=green!30, right=of ctrl] (plant) {Real-Time Simulator};

    \draw[-{Stealth}] (ctrl.east) -- (plant.west) node[midway, above] {PWM signals};
    \draw[-{Stealth}] (plant.south) -- ++(0, -0.7) -| (ctrl.south) node[near start, right] {Sensor signals};

    \node[draw, dashed, fit=(ctrl), inner sep=5pt, label=below:Actual Hardware] {};
    \node[draw, dashed, fit=(plant), inner sep=5pt, label=below:Real-Time Target] {};
\end{tikzpicture}
\end{center}

\textbf{Advantages}:
\begin{itemize}
    \item Tests complete system including hardware interfaces
    \item Real-time behavior tested accurately
    \item Can test failure modes (sensor disconnect, actuator saturation)
    \item Safe testing of dangerous scenarios
\end{itemize}

\textbf{Limitations}:
\begin{itemize}
    \item Expensive real-time simulation hardware required
    \item Slower test execution (real-time only)
    \item Plant model fidelity still limits realism
    \item Cannot test all physical effects (vibration, temperature)
\end{itemize}

\textbf{Use for quadrotors}: Testing complete flight controller with actual sensor interfaces, testing failsafe behavior, regression testing before flight.

\subsection{Vehicle-in-the-Loop (VIL) Testing}

In VIL testing, the actual vehicle operates in a controlled environment, possibly with some simulated elements.

\textbf{Examples}:
\begin{itemize}
    \item Quadrotor on a test stand (constrained motion, safe if control fails)
    \item Flight in a netted arena with motion capture
    \item Outdoor flight with geofencing
\end{itemize}

\textbf{Advantages}:
\begin{itemize}
    \item Tests real physics, real sensors, real actuators
    \item Reveals effects not captured in simulation
    \item Final validation before deployment
\end{itemize}

\textbf{Limitations}:
\begin{itemize}
    \item Risk to hardware (and potentially people)
    \item Slow and expensive
    \item Not reproducible (environmental variations)
    \item Limited ability to test extreme scenarios
\end{itemize}

\section{Choosing the Right Testing Level}

\begin{keyidea}[title=The Testing Pyramid]
Most testing should happen at lower levels (MIL, SIL) where it's fast and cheap. Higher levels (HIL, VIL) are used selectively for integration and final validation.

\begin{center}
\begin{tikzpicture}
    % Wider pyramid to accommodate text labels
    \fill[blue!10] (-1,0) -- (5,0) -- (2,1) -- cycle;
    \fill[blue!20] (-0.25,1) -- (4.25,1) -- (2,2) -- cycle;
    \fill[blue!30] (0.5,2) -- (3.5,2) -- (2,3) -- cycle;
    \fill[blue!40] (1.25,3) -- (2.75,3) -- (2,4) -- cycle;

    \node[font=\footnotesize] at (2, 0.3) {MIL (1000s of tests)};
    \node[font=\footnotesize] at (2, 1.3) {SIL (100s of tests)};
    \node[font=\scriptsize] at (2, 2.3) {HIL (10s of tests)};
    \node[font=\scriptsize] at (2, 3.3) {VIL};
\end{tikzpicture}
\end{center}
\end{keyidea}

\begin{center}
\begin{tabular}{lcccc}
\toprule
\textbf{Aspect} & \textbf{MIL} & \textbf{SIL} & \textbf{HIL} & \textbf{VIL} \\
\midrule
Execution speed & 100$\times$ RT & 10$\times$ RT & 1$\times$ RT & 1$\times$ RT \\
Tests per day & 10,000+ & 1,000+ & 100 & 10 \\
Hardware needed & None & None & RT simulator & Vehicle \\
Risk & None & None & Low & Medium-High \\
Realism & Low & Medium & High & Highest \\
Debugging ease & Easy & Easy & Medium & Hard \\
\bottomrule
\end{tabular}
\end{center}

%======================================================================
\chapter{Requirements for Flight Controllers}
%======================================================================

\section{From Informal to Formal Requirements}

Requirements typically start as informal natural language statements from stakeholders:
\begin{itemize}
    \item ``The drone should fly stably''
    \item ``It should respond quickly to commands''
    \item ``It must not fly outside the designated area''
    \item ``Battery should last at least 5 minutes''
\end{itemize}

These informal requirements must be refined into \textbf{formal, testable specifications}. The process involves:
\begin{enumerate}
    \item \textbf{Clarification}: What exactly does ``stably'' mean? What is ``quickly''?
    \item \textbf{Quantification}: Add numerical bounds and tolerances
    \item \textbf{Formalization}: Express in a language that can be checked automatically
\end{enumerate}

\begin{example}[Refining a Requirement]
\textbf{Informal}: ``The drone should fly stably''

\textbf{Clarified}: ``The attitude angles should not exceed safe limits''

\textbf{Quantified}: ``Roll and pitch angles shall remain within $\pm30°$ during normal flight''

\textbf{Formalized} (in Signal Temporal Logic):
\[
\Box_{[0,T]}\left( |\phi| < 30° \land |\theta| < 30° \right)
\]
where $\phi$ is roll, $\theta$ is pitch, and $T$ is the flight duration.
\end{example}

\section{Categories of Flight Controller Requirements}

\subsection{Safety Requirements}

Safety requirements define conditions that must \textbf{never} be violated:

\begin{center}
\begin{tabular}{p{4cm}p{7cm}}
\toprule
\textbf{Requirement} & \textbf{Formalization} \\
\midrule
Attitude limits & $\Box_{[0,T]}(|\phi| < \phi_{max} \land |\theta| < \theta_{max})$ \\
Altitude ceiling & $\Box_{[0,T]}(z < z_{max})$ \\
Geofencing & $\Box_{[0,T]}(x_{min} < x < x_{max} \land y_{min} < y < y_{max})$ \\
Motor limits & $\Box_{[0,T]}(0 \leq \omega_i \leq \omega_{max})$ for all motors \\
Angular rate limits & $\Box_{[0,T]}(|\dot{\phi}| < \dot{\phi}_{max})$ \\
\bottomrule
\end{tabular}
\end{center}

\subsection{Performance Requirements}

Performance requirements specify how well the system should respond:

\begin{center}
\begin{tabular}{p{4cm}p{7cm}}
\toprule
\textbf{Requirement} & \textbf{Formalization} \\
\midrule
Settling time & $\Diamond_{[0, t_s]}(|e| < \epsilon) \land \Box_{[t_s, T]}(|e| < \epsilon)$ \\
Overshoot & $\Box_{[0,T]}(y < y_{ref} \cdot (1 + OS_{max}))$ \\
Steady-state error & $\Box_{[t_s, T]}(|e| < e_{ss})$ \\
Rise time & $\Diamond_{[0, t_r]}(y > 0.9 \cdot y_{ref})$ \\
\bottomrule
\end{tabular}
\end{center}

\begin{example}[Settling Time Requirement]
``After a step command, the position error shall be less than 5 cm within 2 seconds and remain below 5 cm thereafter.''

\[
\Diamond_{[0, 2]}\left( \|p - p_{ref}\| < 0.05 \right) \land \Box_{[2, T]}\left( \|p - p_{ref}\| < 0.05 \right)
\]

This requires the error to \textit{eventually} (within 2 seconds) become small, and then \textit{always} remain small.
\end{example}

\subsection{Robustness Requirements}

Robustness requirements specify that the system should work under disturbances:

\begin{itemize}
    \item ``The controller shall maintain hover within 20 cm under wind gusts up to 5 m/s''
    \item ``The system shall remain stable with up to 10\% mass uncertainty''
    \item ``Position accuracy shall be maintained with sensor noise up to $\sigma = 0.1$ m''
\end{itemize}

These are tested by applying disturbances during simulation and checking that safety/performance requirements still hold.

\subsection{Timing Requirements}

For real-time systems, timing is part of correctness:

\begin{itemize}
    \item ``Control loop shall execute at 500 Hz with jitter less than 100 $\mu$s''
    \item ``Sensor data shall be processed within 1 ms of arrival''
    \item ``Failsafe shall activate within 50 ms of communication loss''
\end{itemize}

\section{Requirements for the Crazyflie Project}

For the course project, consider requirements such as:

\textbf{Attitude Control}:
\begin{enumerate}
    \item Roll/pitch angles shall remain within $\pm 45°$ during all maneuvers
    \item Given a step change in attitude reference, the attitude error shall settle to within $2°$ in less than 0.5 seconds
    \item Attitude control shall remain stable with up to 20\% variation in moment of inertia
\end{enumerate}

\textbf{Position Control}:
\begin{enumerate}
    \item Position shall remain within a 2 m $\times$ 2 m $\times$ 2 m box at all times
    \item Hover position error shall be less than 10 cm in steady state
    \item After a position step command, the quadrotor shall reach within 15 cm of target in less than 3 seconds
\end{enumerate}

\textbf{Failsafe}:
\begin{enumerate}
    \item If communication is lost for more than 500 ms, motors shall be cut
    \item If attitude exceeds $60°$, motors shall be cut
    \item If battery voltage drops below 3.0 V per cell, the quadrotor shall land
\end{enumerate}

%======================================================================
\chapter{Signal Temporal Logic (STL)}
\index{STL (Signal Temporal Logic)}
%======================================================================

\section{Why Temporal Logic for CPS}

Traditional testing checks: ``Is the output correct at this instant?''

For CPS, we need to check properties over \textbf{time}: ``Does the signal stay below the limit for the entire interval?'' ``Does it eventually reach the target?'' ``Does event A always precede event B?''

\textbf{Signal Temporal Logic (STL)}~\cite{maler2004monitoring}\index{STL (Signal Temporal Logic)|textbf} provides a formal language to express such temporal properties over continuous-time signals.

\section{Predicates Over Continuous Signals}

The building blocks of STL are \textbf{predicates}---conditions on signal values at a particular time:

\begin{definition}[Predicate]
A predicate $\mu$ is a condition of the form $f(x(t)) \sim c$ where:
\begin{itemize}
    \item $x(t)$ is the signal (e.g., position, velocity, attitude)
    \item $f$ is a function of the signal
    \item $\sim$ is a comparison operator ($<$, $\leq$, $>$, $\geq$, $=$)
    \item $c$ is a constant threshold
\end{itemize}
\end{definition}

\begin{example}[Predicates for Quadrotor Signals]
\begin{align*}
\mu_1 &: z < 10 && \text{``altitude is below 10 meters''} \\
\mu_2 &: \|p - p_{ref}\| < 0.1 && \text{``position error is less than 10 cm''} \\
\mu_3 &: |\phi| < 30° && \text{``roll angle magnitude is less than 30 degrees''} \\
\mu_4 &: \omega_1 > 1000 && \text{``motor 1 speed exceeds 1000 RPM''} \\
\mu_5 &: v_z > 0 && \text{``quadrotor is ascending''}
\end{align*}
\end{example}

A predicate $\mu$ is either \texttt{True} or \texttt{False} at each time instant $t$, depending on the signal value $x(t)$.

\section{Boolean Operators}

Predicates can be combined using standard Boolean operators:

\begin{center}
\begin{tabular}{lll}
\toprule
\textbf{Operator} & \textbf{Symbol} & \textbf{Meaning} \\
\midrule
Negation & $\neg \mu$ & NOT $\mu$ \\
Conjunction & $\mu_1 \land \mu_2$ & $\mu_1$ AND $\mu_2$ \\
Disjunction & $\mu_1 \lor \mu_2$ & $\mu_1$ OR $\mu_2$ \\
Implication & $\mu_1 \Rightarrow \mu_2$ & IF $\mu_1$ THEN $\mu_2$ (equivalent to $\neg\mu_1 \lor \mu_2$) \\
\bottomrule
\end{tabular}
\end{center}

\begin{example}[Combined Predicates]
``The quadrotor is in a safe attitude'':
\[
\text{safe\_attitude} = (|\phi| < 45°) \land (|\theta| < 45°)
\]

``Either hovering or moving slowly'':
\[
(\|v\| < 0.1) \lor (\|v\| < 1.0 \land z < 2)
\]
\end{example}

\section{The Always Operator ($\Box$)}

\begin{definition}[Always / Globally]
$\Box_{[a,b]} \varphi$ means: formula $\varphi$ holds at \textbf{every} time instant in the interval $[t+a, t+b]$, where $t$ is the current time.
\end{definition}

\begin{center}
\begin{tikzpicture}
    \draw[->] (0,0) -- (10,0) node[right] {time};
    \draw[thick] (0,0) -- (0,0.1) node[above] {$t$};
    \draw[thick] (2,0) -- (2,0.1) node[above] {$t+a$};
    \draw[thick] (7,0) -- (7,0.1) node[above] {$t+b$};

    \draw[very thick, blue] (2,0) -- (7,0);
    \node[blue] at (4.5, 0.5) {$\varphi$ must hold everywhere};

    \fill[blue!20] (2,-0.3) rectangle (7,-0.1);
\end{tikzpicture}
\end{center}

\begin{example}[Always Operator]
``Altitude stays below 10 m for the entire flight (0 to T seconds)'':
\[
\Box_{[0,T]}(z < 10)
\]

``For the next 5 seconds, attitude remains safe'':
\[
\Box_{[0,5]}(|\phi| < 45° \land |\theta| < 45°)
\]

``From time 2s to 10s, position error is small'':
\[
\Box_{[2,10]}(\|p - p_{ref}\| < 0.1)
\]
\end{example}

\textbf{Intuition}: $\Box_{[a,b]}$ is like a \texttt{for all} quantifier over time. The formula must be true at every moment in the interval.

\section{The Eventually Operator ($\Diamond$)}

\begin{definition}[Eventually / Finally]
$\Diamond_{[a,b]} \varphi$ means: formula $\varphi$ holds at \textbf{some} time instant in the interval $[t+a, t+b]$.
\end{definition}

\begin{center}
\begin{tikzpicture}
    \draw[->] (0,0) -- (10,0) node[right] {time};
    \draw[thick] (0,0) -- (0,0.1) node[above] {$t$};
    \draw[thick] (2,0) -- (2,0.1) node[above] {$t+a$};
    \draw[thick] (7,0) -- (7,0.1) node[above] {$t+b$};

    \draw[dashed, blue] (2,0) -- (7,0);
    \fill[blue] (5,0) circle (3pt);
    \node[blue] at (5, 0.5) {$\varphi$ must hold at least once};

    \fill[blue!20] (2,-0.3) rectangle (7,-0.1);
\end{tikzpicture}
\end{center}

\begin{example}[Eventually Operator]
``The quadrotor reaches the target within 5 seconds'':
\[
\Diamond_{[0,5]}(\|p - p_{ref}\| < 0.1)
\]

``At some point between 2s and 4s, the quadrotor is descending'':
\[
\Diamond_{[2,4]}(v_z < 0)
\]

``Eventually (within the mission time), the quadrotor lands'':
\[
\Diamond_{[0,T]}(z < 0.1 \land \|v\| < 0.1)
\]
\end{example}

\textbf{Intuition}: $\Diamond_{[a,b]}$ is like an \texttt{exists} quantifier over time. The formula needs to be true at just one moment in the interval.

\textbf{Relationship}: $\Diamond_{[a,b]} \varphi \equiv \neg \Box_{[a,b]} \neg\varphi$ (eventually $\varphi$ means it's not always not-$\varphi$).

\section{The Until Operator ($\mathcal{U}$)}

\begin{definition}[Until]
$\varphi_1 \mathcal{U}_{[a,b]} \varphi_2$ means: $\varphi_1$ holds continuously until $\varphi_2$ becomes true, and $\varphi_2$ becomes true at some time in $[t+a, t+b]$.
\end{definition}

\begin{center}
\begin{tikzpicture}
    \draw[->] (0,0) -- (10,0) node[right] {time};
    \draw[thick] (0,0) -- (0,0.1) node[above] {$t$};
    \draw[thick] (2,0) -- (2,0.1) node[above] {$t+a$};
    \draw[thick] (7,0) -- (7,0.1) node[above] {$t+b$};

    \draw[very thick, red] (0,0) -- (5,0);
    \node[red] at (2.5, 0.5) {$\varphi_1$ holds};

    \fill[blue] (5,0) circle (3pt);
    \node[blue] at (5, 0.5) {$\varphi_2$ becomes true};

    \fill[blue!20] (2,-0.3) rectangle (7,-0.1);
\end{tikzpicture}
\end{center}

\begin{example}[Until Operator]
``The quadrotor maintains altitude until it receives a land command'':
\[
(|z - z_{hover}| < 0.1) \mathcal{U}_{[0,T]} (\text{land\_cmd} = 1)
\]

``Attitude remains level until the waypoint is reached'':
\[
(|\phi| < 5° \land |\theta| < 5°) \mathcal{U}_{[0,10]} (\|p - p_{wp}\| < 0.2)
\]
\end{example}

\section{Combining Operators}

Complex requirements often require nested temporal operators:

\begin{example}[Settling Time Specification]
``After a step command at $t=0$, the error shall eventually become small and then always remain small'':
\[
\Diamond_{[0,2]} \Box_{[0,\infty)} (\|e\| < 0.05)
\]

This means: at some time within 2 seconds, a point is reached after which the error is always small.
\end{example}

\begin{example}[Response to Disturbance]
``If a wind gust occurs, the position error may temporarily increase but shall return to normal within 3 seconds'':
\[
\Box_{[0,T]} \left( (\text{gust} > 2) \Rightarrow \Diamond_{[0,3]}(\|e_p\| < 0.1) \right)
\]

Whenever a gust exceeds 2 m/s, the system must recover within 3 seconds.
\end{example}

\begin{example}[Safety with Recovery]
``Attitude angles shall always remain safe, and if they exceed 20°, they must return below 10° within 1 second'':
\[
\Box_{[0,T]}(|\phi| < 45°) \land \Box_{[0,T]}\left( (|\phi| > 20°) \Rightarrow \Diamond_{[0,1]}(|\phi| < 10°) \right)
\]
\end{example}

\section{Common STL Specification Patterns}

\begin{center}
\begin{tabular}{p{3.5cm}p{4cm}p{4.5cm}}
\toprule
\textbf{Pattern} & \textbf{Informal} & \textbf{STL} \\
\midrule
Invariance & Always safe & $\Box_{[0,T]} \varphi$ \\
Reachability & Eventually reach goal & $\Diamond_{[0,T]} \varphi$ \\
Stability & Reach and stay & $\Diamond_{[0,t_s]} \Box_{[0,\infty)} \varphi$ \\
Bounded response & If $\varphi_1$ then $\varphi_2$ within $\tau$ & $\Box_{[0,T]}(\varphi_1 \Rightarrow \Diamond_{[0,\tau]}\varphi_2)$ \\
Recurrence & Keep returning to $\varphi$ & $\Box_{[0,T]} \Diamond_{[0,\tau]} \varphi$ \\
\bottomrule
\end{tabular}
\end{center}

%======================================================================
\chapter{Quantitative Semantics: Robustness}
\index{robustness}
%======================================================================

\section{From Boolean to Quantitative}

STL formulas are either satisfied or violated---a Boolean outcome. However, for testing and optimization, we need more information:
\begin{itemize}
    \item \textbf{How much margin} do we have before violation?
    \item Is a trajectory that barely satisfies the spec as good as one with large margin?
    \item How can we guide optimization to find violations?
\end{itemize}

\textbf{Robustness}~\cite{donze2010robust,fainekos2009robustness} extends STL with quantitative semantics: instead of just True/False, we compute a real number indicating how strongly the formula is satisfied or violated.

\begin{keyidea}[title=Robustness Semantics]
The robustness value $\rho(\varphi, x, t)$ of formula $\varphi$ for signal $x$ at time $t$ is:
\begin{itemize}
    \item \textbf{Positive} if $\varphi$ is satisfied (larger = more margin)
    \item \textbf{Negative} if $\varphi$ is violated (more negative = worse violation)
    \item \textbf{Zero} at the boundary between satisfaction and violation
\end{itemize}

Sign of robustness determines satisfaction: $\rho > 0 \Leftrightarrow$ formula satisfied.
\end{keyidea}

\section{Robustness of Predicates}

For a predicate $\mu : f(x) < c$, the robustness is simply the distance to the threshold:
\[
\rho(\mu, x, t) = c - f(x(t))
\]

\begin{example}[Predicate Robustness]
For $\mu : z < 10$ (altitude below 10 m):
\begin{itemize}
    \item If $z(t) = 7$: $\rho = 10 - 7 = 3$ (3 meters of margin)
    \item If $z(t) = 9.5$: $\rho = 10 - 9.5 = 0.5$ (barely satisfied)
    \item If $z(t) = 12$: $\rho = 10 - 12 = -2$ (violated by 2 meters)
\end{itemize}
\end{example}

For predicates with $>$: $\rho(f(x) > c, x, t) = f(x(t)) - c$.

\section{Robustness of Boolean Operators}

The robustness of compound formulas is computed recursively:

\begin{align}
\rho(\neg\varphi, x, t) &= -\rho(\varphi, x, t) \\
\rho(\varphi_1 \land \varphi_2, x, t) &= \min(\rho(\varphi_1, x, t), \rho(\varphi_2, x, t)) \\
\rho(\varphi_1 \lor \varphi_2, x, t) &= \max(\rho(\varphi_1, x, t), \rho(\varphi_2, x, t))
\end{align}

\textbf{Intuition}:
\begin{itemize}
    \item \textbf{Negation}: Flips the sign (what was margin becomes violation)
    \item \textbf{AND}: The conjunction is only as robust as its weakest part (minimum)
    \item \textbf{OR}: The disjunction is as robust as its strongest part (maximum)
\end{itemize}

\begin{example}[Compound Robustness]
For $\varphi = (z < 10) \land (|\phi| < 45°)$:

If $z = 8$ and $|\phi| = 30°$:
\begin{itemize}
    \item $\rho(z < 10) = 10 - 8 = 2$
    \item $\rho(|\phi| < 45°) = 45 - 30 = 15$
    \item $\rho(\varphi) = \min(2, 15) = 2$
\end{itemize}

The overall robustness is limited by altitude (the weaker constraint).
\end{example}

\section{Robustness of Temporal Operators}

\begin{align}
\rho(\Box_{[a,b]}\varphi, x, t) &= \min_{\tau \in [t+a, t+b]} \rho(\varphi, x, \tau) \\
\rho(\Diamond_{[a,b]}\varphi, x, t) &= \max_{\tau \in [t+a, t+b]} \rho(\varphi, x, \tau)
\end{align}

\textbf{Intuition}:
\begin{itemize}
    \item \textbf{Always}: Must hold at every instant, so robustness is the minimum over time
    \item \textbf{Eventually}: Must hold at some instant, so robustness is the maximum over time
\end{itemize}

\begin{example}[Temporal Robustness]
Consider $\varphi = \Box_{[0,5]}(z < 10)$ with altitude trajectory:
\[
z(t) = 5 + 2\sin(t)
\]

The altitude oscillates between 3 and 7. At each instant:
\[
\rho(z < 10, z, t) = 10 - z(t) = 5 - 2\sin(t)
\]

The minimum occurs when $\sin(t) = 1$: $\rho_{min} = 5 - 2 = 3$.

Therefore: $\rho(\Box_{[0,5]}(z < 10)) = 3$.

The trajectory satisfies the spec with 3 meters of margin (the closest approach to the limit).
\end{example}

\section{Geometric Interpretation}

\begin{center}
\begin{tikzpicture}
    \begin{axis}[
        width=10cm, height=6cm,
        xlabel={Time (s)}, ylabel={Altitude (m)},
        xmin=0, xmax=10, ymin=0, ymax=15,
        legend pos=north east
    ]
    % Threshold
    \addplot[red, thick, dashed, domain=0:10] {10};
    \addlegendentry{Limit ($z=10$)}

    % Trajectory 1 - good margin
    \addplot[blue, thick, domain=0:10, samples=100] {5 + 2*sin(deg(x))};
    \addlegendentry{Trajectory A ($\rho=3$)}

    % Trajectory 2 - close to limit
    \addplot[orange, thick, domain=0:10, samples=100] {8 + 1.5*sin(deg(x))};
    \addlegendentry{Trajectory B ($\rho=0.5$)}

    % Trajectory 3 - violation
    \addplot[purple, thick, domain=0:10, samples=100] {9 + 3*sin(deg(x))};
    \addlegendentry{Trajectory C ($\rho=-2$)}

    \end{axis}
\end{tikzpicture}
\end{center}

The robustness measures the \textbf{vertical distance} between the trajectory and the constraint boundary at the closest point:
\begin{itemize}
    \item Trajectory A: Always well below limit, robustness = +3
    \item Trajectory B: Comes close to limit, robustness = +0.5
    \item Trajectory C: Exceeds limit, robustness = -2
\end{itemize}

\section{Why Robustness Enables Optimization}

Traditional Boolean satisfaction provides no gradient information:
\begin{itemize}
    \item Trajectory that violates by 0.01 m: \texttt{False}
    \item Trajectory that violates by 10 m: \texttt{False}
\end{itemize}

Robustness provides a \textbf{continuous objective function}:
\begin{itemize}
    \item Trajectory that violates by 0.01 m: $\rho = -0.01$
    \item Trajectory that violates by 10 m: $\rho = -10$
\end{itemize}

This allows optimization algorithms to:
\begin{enumerate}
    \item Distinguish between ``almost satisfies'' and ``badly violates''
    \item Follow the gradient toward violations (for falsification)
    \item Quantify how much design margin exists
\end{enumerate}

%======================================================================
\chapter{Falsification: Testing as Optimization}
\index{falsification}
%======================================================================

Falsification-based testing uses optimization to systematically search for specification violations. For foundational work on this approach, see Donzé~\cite{donze2010breach} and Annpureddy et al.~\cite{annpureddy2011s}.

\section{The Falsification Problem}

\begin{definition}[Falsification]
\index{falsification|textbf}
Given:
\begin{enumerate}
    \item A \textbf{system model} $M$ (plant + controller)
    \item A \textbf{specification} $\varphi$ (in STL)
    \item A \textbf{set of test inputs} $\mathcal{K}$ (disturbances, initial conditions, parameters)
\end{enumerate}

\textbf{Find}: An input $k \in \mathcal{K}$ such that the resulting trajectory $x = M(k)$ violates the specification, i.e., $\rho(\varphi, x, 0) < 0$.
\end{definition}

Falsification is formulated as an \textbf{optimization problem}:
\[
\min_{k \in \mathcal{K}} \rho(\varphi, M(k), 0)
\]

If the minimum is negative, we have found a counterexample (falsified the spec). If extensive search fails to find negative robustness, we gain confidence (but not proof) that the spec is satisfied.

\begin{keyidea}[title=Testing as Optimization]
Instead of random testing or manual test case design, falsification uses optimization to \textbf{systematically search} for violations. The robustness function guides the search toward the most problematic inputs.
\end{keyidea}

\section{Test Input Parameterization}

The test inputs $k$ define the ``scenario'' being tested. For a quadrotor, inputs might include:

\textbf{Initial conditions}:
\begin{itemize}
    \item Initial position offset: $p_0 \in [-0.5, 0.5]^3$ m
    \item Initial velocity: $v_0 \in [-1, 1]^3$ m/s
    \item Initial attitude error: $\phi_0, \theta_0 \in [-10°, 10°]$
\end{itemize}

\textbf{Disturbance signals} (e.g., wind):
\begin{itemize}
    \item Time-varying signals are parameterized using \textbf{control points}
    \item Interpolation between control points creates smooth signals
\end{itemize}

\begin{example}[Wind Gust Parameterization]
A wind gust signal $w(t)$ over 10 seconds can be parameterized by 5 control points:
\[
k = [w_0, w_1, w_2, w_3, w_4, t_1, t_2, t_3, t_4]
\]
where $w_i$ is the wind speed at time $t_i$, with linear interpolation between points.

\begin{center}
\begin{tikzpicture}
    \begin{axis}[
        width=8cm, height=4cm,
        xlabel={Time (s)}, ylabel={Wind (m/s)},
        xmin=0, xmax=10, ymin=0, ymax=6
    ]
    \addplot[blue, thick, mark=*] coordinates {
        (0, 1) (2, 2) (4, 5) (7, 3) (10, 1)
    };
    \node at (axis cs:4,5.5) {\small control point};
    \end{axis}
\end{tikzpicture}
\end{center}

The optimizer searches over the 9-dimensional parameter space to find wind patterns that cause violations.
\end{example}

\textbf{Reference signals}:
\begin{itemize}
    \item Waypoint positions
    \item Trajectory aggressiveness
    \item Command timing
\end{itemize}

\section{The Optimization Landscape}

The function $f(k) = \rho(\varphi, M(k), 0)$ defines a landscape over the parameter space:

\begin{itemize}
    \item \textbf{Valleys} (negative regions) represent violations
    \item \textbf{Hills} (positive regions) represent satisfaction
    \item \textbf{Zero crossings} are the boundaries
\end{itemize}

\begin{warningbox}
The robustness landscape is typically:
\begin{itemize}
    \item \textbf{Non-convex}: Multiple local minima
    \item \textbf{Non-smooth}: Discontinuities from mode switches, saturation
    \item \textbf{High-dimensional}: Many parameters
    \item \textbf{Expensive to evaluate}: Each point requires a full simulation
\end{itemize}

Gradient-based optimization often fails. We need gradient-free global optimization methods.
\end{warningbox}

\section{Gradient-Free Optimization Methods}

\subsection{CMA-ES (Covariance Matrix Adaptation Evolution Strategy)}

CMA-ES maintains a population of candidate solutions and adapts the search distribution based on which candidates perform well.

\textbf{Key idea}: Learn the shape of the objective function by tracking correlations between parameters in successful candidates.

\textbf{Strengths}: Good for non-convex, high-dimensional problems; self-adapting step size.

\subsection{Simulated Annealing}

Inspired by metallurgical annealing, this method accepts worse solutions with decreasing probability as the ``temperature'' cools.

\textbf{Key idea}: Early exploration accepts bad moves to escape local minima; later exploitation focuses on refinement.

\subsection{Nelder-Mead (Simplex Method)}

Maintains a simplex (triangle in 2D, tetrahedron in 3D) and iteratively moves it toward the minimum through reflection, expansion, and contraction operations.

\textbf{Strengths}: Simple, no hyperparameters, works well in low dimensions.

\textbf{Weaknesses}: Can get stuck in local minima; slow in high dimensions.

\subsection{Bayesian Optimization}

Builds a probabilistic model (Gaussian Process) of the objective function and uses it to decide where to sample next.

\textbf{Key idea}: Balance \textbf{exploitation} (sample where model predicts low values) with \textbf{exploration} (sample where model is uncertain).

\textbf{Strengths}: Sample-efficient (finds minima with fewer simulations); provides uncertainty estimates.

\textbf{Weaknesses}: Computational overhead for large datasets; GP scales poorly with dimensions.

\begin{center}
\begin{tabular}{lccc}
\toprule
\textbf{Method} & \textbf{Sample Efficiency} & \textbf{Scalability} & \textbf{Local Minima} \\
\midrule
CMA-ES & Medium & Good & Good escape \\
Simulated Annealing & Low & Good & Good escape \\
Nelder-Mead & Medium & Poor & Often stuck \\
Bayesian Opt. & High & Poor & Good escape \\
\bottomrule
\end{tabular}
\end{center}

\section{The Breach Toolbox}

\textbf{Breach} is a MATLAB toolbox for falsification of cyber-physical systems:

\begin{lstlisting}[language=Matlab, caption=Breach workflow overview]
% 1. Load Simulink model
B = BreachSimulinkSystem('quadrotor_model');

% 2. Define input parameters (wind gust)
B.SetParamRanges({'wind_x', 'wind_y'}, [-5, 5; -5, 5]);

% 3. Define STL specification
STL_spec = 'alw_[0,10] (abs(roll) < 0.5)';

% 4. Create falsification problem
falsif_pb = FalsificationProblem(B, STL_spec);

% 5. Set optimizer
falsif_pb.setup_solver('cmaes');

% 6. Run falsification
falsif_pb.solve();

% 7. Check result
if falsif_pb.obj_best < 0
    disp('Specification FALSIFIED!');
    % Analyze counterexample
    BrFalsified = falsif_pb.GetBrSet_Logged();
    BrFalsified.PlotSignals({'roll', 'wind_x'});
end
\end{lstlisting}

\section{Worked Example: Finding Dangerous Wind Conditions}

\textbf{Scenario}: Test if the attitude controller can maintain safe roll angles under wind disturbances.

\textbf{Specification}: ``Roll angle shall stay within $\pm30°$ for 10 seconds''
\[
\varphi = \Box_{[0,10]}(|\phi| < 30°)
\]

\textbf{Test inputs}: Wind velocity components $(w_x, w_y)$ parameterized as piecewise-constant with 3 segments:
\[
k = [w_{x,1}, w_{x,2}, w_{x,3}, w_{y,1}, w_{y,2}, w_{y,3}] \in [-8, 8]^6 \text{ m/s}
\]

\textbf{Falsification process}:

\begin{enumerate}
    \item \textbf{Initial samples}: Evaluate robustness for random wind patterns
    \item \textbf{Iteration}: Optimizer proposes new wind patterns based on previous results
    \item \textbf{Convergence}: After 50 simulations, optimizer finds:
    \[
    k^* = [6.2, -7.1, 5.8, 4.5, 6.9, -3.2]
    \]
    with robustness $\rho = -8.3°$
    \item \textbf{Counterexample}: This wind pattern causes roll to reach $38.3°$, violating the spec
\end{enumerate}

\textbf{Analysis}: The counterexample reveals that rapidly changing cross-winds create roll oscillations that exceed limits. This insight guides controller improvement (e.g., increase roll damping, add wind feedforward).

\section{From Counterexample to Fix}

When falsification finds a violation:

\begin{enumerate}
    \item \textbf{Analyze the counterexample}:
    \begin{itemize}
        \item Plot all relevant signals
        \item Identify the moment and cause of violation
        \item Check if the scenario is realistic
    \end{itemize}

    \item \textbf{Determine root cause}:
    \begin{itemize}
        \item Is it a controller tuning issue?
        \item Is it a fundamental limitation?
        \item Is the specification too strict?
    \end{itemize}

    \item \textbf{Fix and re-test}:
    \begin{itemize}
        \item Modify controller or spec as appropriate
        \item Re-run falsification
        \item Verify the specific counterexample no longer causes violation
    \end{itemize}
\end{enumerate}

%======================================================================
\chapter{Code Coverage for Safety-Critical Systems}
\index{code coverage}
%======================================================================

\section{Why Code Coverage Matters}

\textbf{Code coverage}\index{code coverage|textbf} measures what fraction of the source code is executed during testing. It answers: ``Have we tested all the code?''

For safety-critical systems, untested code is a risk:
\begin{itemize}
    \item Untested code may contain bugs
    \item Untested code may behave unexpectedly in corner cases
    \item Certification standards require coverage evidence
\end{itemize}

\section{Coverage Criteria}

\subsection{Statement Coverage}

Every statement in the code is executed at least once.

\begin{lstlisting}[language=C, caption=Statement coverage example]
float compute_thrust(float error, float error_rate) {
    float thrust = 0.0f;                    // Statement 1
    thrust = Kp * error;                    // Statement 2
    thrust += Kd * error_rate;              // Statement 3
    if (thrust > MAX_THRUST) {              // Statement 4
        thrust = MAX_THRUST;                // Statement 5 <- needs test that saturates
    }
    return thrust;                          // Statement 6
}
\end{lstlisting}

To achieve 100\% statement coverage, you need at least one test where \texttt{thrust > MAX\_THRUST}.

\subsection{Branch Coverage}

Every branch (if/else, switch cases) is taken at least once.

\begin{lstlisting}[language=C, caption=Branch coverage example]
if (altitude > ceiling) {       // Branch 1: true and false
    descend();
} else if (altitude < floor) {  // Branch 2: true and false
    ascend();
} else {                        // Branch 3: else case
    hover();
}
\end{lstlisting}

Branch coverage requires tests for: (1) altitude $>$ ceiling, (2) altitude $<$ floor, (3) floor $\leq$ altitude $\leq$ ceiling.

\subsection{Condition Coverage}

Every Boolean sub-expression evaluates to both true and false.

\begin{lstlisting}[language=C, caption=Condition coverage example]
if (armed && (altitude > 0.5f || override)) {
    // Conditions: armed, (altitude > 0.5f), override
    // Each must be true and false in some test
}
\end{lstlisting}

\subsection{MC/DC (Modified Condition/Decision Coverage)}
\index{MC/DC (Modified Condition/Decision Coverage)}

\textbf{MC/DC}~\cite{chilenski1994applicability}\index{MC/DC (Modified Condition/Decision Coverage)|textbf} requires that each condition independently affects the decision outcome.

\begin{definition}[MC/DC]
For MC/DC coverage, demonstrate that:
\begin{enumerate}
    \item Each condition has been true and false
    \item Each condition independently affects the decision
    \item ``Independently'' means: changing that condition alone changes the outcome
\end{enumerate}
\end{definition}

\begin{example}[MC/DC for \texttt{if (A \&\& (B || C))}]

The expression has 3 conditions: A, B, C. Full truth table has $2^3 = 8$ rows. MC/DC can be achieved with 4 tests:

\begin{center}
\begin{tabular}{cccc|c|l}
\toprule
\textbf{Test} & \textbf{A} & \textbf{B} & \textbf{C} & \textbf{Result} & \textbf{Shows independence of} \\
\midrule
1 & T & T & F & T & \\
2 & F & T & F & F & A (compare with test 1) \\
3 & T & F & F & F & B (compare with test 1) \\
4 & T & F & T & T & C (compare with test 3) \\
\bottomrule
\end{tabular}
\end{center}

\begin{itemize}
    \item Tests 1 \& 2: Only A differs, result changes $\Rightarrow$ A is independent
    \item Tests 1 \& 3: Only B differs, result changes $\Rightarrow$ B is independent
    \item Tests 3 \& 4: Only C differs, result changes $\Rightarrow$ C is independent
\end{itemize}
\end{example}

\section{Coverage Requirements in Standards}

Safety-critical software must meet coverage requirements specified in industry standards~\cite{do178c,iso26262}.

\begin{center}
\begin{tabular}{lll}
\toprule
\textbf{Standard} & \textbf{Domain} & \textbf{Coverage Requirement} \\
\midrule
DO-178C Level A & Avionics (catastrophic) & MC/DC \\
DO-178C Level B & Avionics (hazardous) & Decision coverage \\
ISO 26262 ASIL D & Automotive (highest) & MC/DC \\
ISO 26262 ASIL B & Automotive (medium) & Branch coverage \\
IEC 61508 SIL 4 & Industrial (highest) & MC/DC \\
\bottomrule
\end{tabular}
\end{center}

\section{Limitations of Coverage for CPS}

\begin{warningbox}
High code coverage does \textbf{not} guarantee correctness for CPS:
\begin{itemize}
    \item Coverage measures code execution, not input space coverage
    \item 100\% code coverage with one input trajectory leaves most behaviors untested
    \item The continuous input space is infinite---no finite test set covers it
    \item Coverage doesn't check timing, concurrency, or real-time behavior
\end{itemize}

Use coverage as a \textbf{necessary but not sufficient} criterion. Combine with specification-based testing (STL falsification) for thorough validation.
\end{warningbox}

%======================================================================
\chapter{Unit Testing for Embedded Systems}
\index{unit testing}
%======================================================================

Unit testing\index{unit testing|textbf} is the foundation of software quality. For flight controllers, unit tests verify that individual functions work correctly before integration. This chapter introduces unit testing frameworks suitable for embedded C code.

\section{Why Unit Test Flight Controller Code?}

\begin{itemize}
    \item \textbf{Catch bugs early}: Fix errors before they propagate to system-level testing
    \item \textbf{Enable refactoring}: Tests provide confidence that changes don't break existing functionality
    \item \textbf{Document behavior}: Tests serve as executable specifications
    \item \textbf{Isolate problems}: When a test fails, you know exactly which function has the bug
    \item \textbf{Support CI/CD}: Automated tests run on every code change
\end{itemize}

\section{The Unity Test Framework}

\textbf{Unity} is a lightweight unit testing framework designed for embedded C:

\begin{itemize}
    \item Single header file (\texttt{unity.h})
    \item No dynamic memory allocation
    \item Small footprint (suitable for constrained targets)
    \item Rich assertion macros
    \item Portable across compilers and platforms
\end{itemize}

\subsection{Installation}

\begin{lstlisting}[language=bash, caption=Getting Unity]
# Clone Unity repository
git clone https://github.com/ThrowTheSwitch/Unity.git

# Copy to your project
cp Unity/src/unity.c Unity/src/unity.h Unity/src/unity_internals.h \
   your_project/test/
\end{lstlisting}

\subsection{Basic Test Structure}

\begin{lstlisting}[language=C, caption=Basic Unity test file structure]
// test_quaternion.c - Unit tests for quaternion operations

#include "unity.h"
#include "quaternion.h"  // Code under test

// Called before each test
void setUp(void) {
    // Initialize test fixtures
}

// Called after each test
void tearDown(void) {
    // Clean up
}

// Test: Quaternion multiplication identity
void test_quaternion_multiply_identity(void) {
    Quaternion_t q = {1.0f, 0.0f, 0.0f, 0.0f};  // Identity
    Quaternion_t p = {0.707f, 0.707f, 0.0f, 0.0f};

    Quaternion_t result = QuaternionMultiply(q, p);

    // Result should equal p (identity * p = p)
    TEST_ASSERT_FLOAT_WITHIN(1e-5f, p.w, result.w);
    TEST_ASSERT_FLOAT_WITHIN(1e-5f, p.x, result.x);
    TEST_ASSERT_FLOAT_WITHIN(1e-5f, p.y, result.y);
    TEST_ASSERT_FLOAT_WITHIN(1e-5f, p.z, result.z);
}

// Test: Quaternion normalization
void test_quaternion_normalize(void) {
    Quaternion_t q = {2.0f, 0.0f, 0.0f, 0.0f};  // Not normalized

    Quaternion_t result = QuaternionNormalize(q);

    // Magnitude should be 1
    float mag = sqrtf(result.w*result.w + result.x*result.x +
                      result.y*result.y + result.z*result.z);
    TEST_ASSERT_FLOAT_WITHIN(1e-6f, 1.0f, mag);
}

// Test: Quaternion to Euler conversion
void test_quaternion_to_euler_level(void) {
    Quaternion_t q = {1.0f, 0.0f, 0.0f, 0.0f};  // Identity = level

    float roll, pitch, yaw;
    QuaternionToEuler(q, &roll, &pitch, &yaw);

    TEST_ASSERT_FLOAT_WITHIN(1e-5f, 0.0f, roll);
    TEST_ASSERT_FLOAT_WITHIN(1e-5f, 0.0f, pitch);
    TEST_ASSERT_FLOAT_WITHIN(1e-5f, 0.0f, yaw);
}

// Main test runner
int main(void) {
    UNITY_BEGIN();

    RUN_TEST(test_quaternion_multiply_identity);
    RUN_TEST(test_quaternion_normalize);
    RUN_TEST(test_quaternion_to_euler_level);

    return UNITY_END();
}
\end{lstlisting}

\subsection{Common Unity Assertions}

\begin{center}
\begin{tabular}{p{6cm}p{6.5cm}}
\toprule
\textbf{Assertion} & \textbf{Purpose} \\
\midrule
\texttt{TEST\_ASSERT\_TRUE(cond)} & Check condition is true \\
\texttt{TEST\_ASSERT\_FALSE(cond)} & Check condition is false \\
\texttt{TEST\_ASSERT\_EQUAL(exp, act)} & Integer equality \\
\texttt{TEST\_ASSERT\_FLOAT\_WITHIN(tol, exp, act)} & Float equality with tolerance \\
\texttt{TEST\_ASSERT\_EQUAL\_FLOAT\_ARRAY(exp, act, n)} & Array equality \\
\texttt{TEST\_ASSERT\_NULL(ptr)} & Pointer is NULL \\
\texttt{TEST\_ASSERT\_NOT\_NULL(ptr)} & Pointer is not NULL \\
\texttt{TEST\_FAIL\_MESSAGE("msg")} & Force test failure \\
\bottomrule
\end{tabular}
\end{center}

\section{Testing Flight Controller Functions}

\subsection{Testing PID Controller}

See Listing~\ref{lst:test-pid} (Appendix~\ref{app:code-listings}) for complete PID controller unit tests covering proportional, integral, derivative terms, saturation, and anti-windup behavior.

\subsection{Testing Sensor Processing}

See Listing~\ref{lst:test-filters} (Appendix~\ref{app:code-listings}) for complete sensor filtering unit tests, including lowpass filter initialization, steady-state convergence, high-frequency attenuation, and complementary filter behavior.

\section{Mocking Hardware Dependencies}

Flight controller code often depends on hardware (sensors, timers, I/O). For unit testing, we \textbf{mock} these dependencies.

\begin{lstlisting}[language=C, caption=Mocking sensor reads]
// mock_sensors.h - Mock sensor interface for testing

#ifndef MOCK_SENSORS_H
#define MOCK_SENSORS_H

// Mock data storage
typedef struct {
    float gyro[3];
    float accel[3];
    int call_count;
} MockSensorData_t;

extern MockSensorData_t g_mockSensor;

// Set up mock return values
void MockSensor_SetGyro(float gx, float gy, float gz);
void MockSensor_SetAccel(float ax, float ay, float az);
void MockSensor_Reset(void);

#endif

// mock_sensors.c
#include "mock_sensors.h"

MockSensorData_t g_mockSensor = {0};

void MockSensor_SetGyro(float gx, float gy, float gz) {
    g_mockSensor.gyro[0] = gx;
    g_mockSensor.gyro[1] = gy;
    g_mockSensor.gyro[2] = gz;
}

void MockSensor_SetAccel(float ax, float ay, float az) {
    g_mockSensor.accel[0] = ax;
    g_mockSensor.accel[1] = ay;
    g_mockSensor.accel[2] = az;
}

void MockSensor_Reset(void) {
    g_mockSensor.call_count = 0;
}

// sensors.c (production code with conditional compilation)
#ifdef UNIT_TEST
#include "mock_sensors.h"
void ReadGyro(float *data) {
    data[0] = g_mockSensor.gyro[0];
    data[1] = g_mockSensor.gyro[1];
    data[2] = g_mockSensor.gyro[2];
    g_mockSensor.call_count++;
}
#else
void ReadGyro(float *data) {
    // Real hardware read
    IMU_GetGyro(data);
}
#endif
\end{lstlisting}

\begin{lstlisting}[language=C, caption=Using mocks in tests]
// test_attitude_estimator.c

#include "unity.h"
#include "attitude_estimator.h"
#include "mock_sensors.h"

void setUp(void) {
    MockSensor_Reset();
    AttitudeEstimator_Init();
}

void test_level_attitude_from_accelerometer(void) {
    // Mock level accelerometer reading (gravity along Z)
    MockSensor_SetAccel(0.0f, 0.0f, 9.81f);
    MockSensor_SetGyro(0.0f, 0.0f, 0.0f);

    // Run estimator
    AttitudeEstimator_Update(0.002f);
    Attitude_t att = AttitudeEstimator_GetAttitude();

    // Should report level attitude
    TEST_ASSERT_FLOAT_WITHIN(0.01f, 0.0f, att.roll);
    TEST_ASSERT_FLOAT_WITHIN(0.01f, 0.0f, att.pitch);
}

void test_pitched_attitude_from_accelerometer(void) {
    // Mock pitched forward (X up relative to gravity)
    MockSensor_SetAccel(9.81f * 0.5f, 0.0f, 9.81f * 0.866f);  // 30 deg pitch
    MockSensor_SetGyro(0.0f, 0.0f, 0.0f);

    // Let filter settle
    for (int i = 0; i < 1000; i++) {
        AttitudeEstimator_Update(0.002f);
    }

    Attitude_t att = AttitudeEstimator_GetAttitude();

    TEST_ASSERT_FLOAT_WITHIN(0.05f, 0.523f, att.pitch);  // ~30 degrees in rad
}
\end{lstlisting}

\section{Running Tests}

\subsection{Native (Host) Testing}

Compile and run tests on your development machine:

\begin{lstlisting}[language=bash, caption=Building and running tests natively]
# Compile tests
gcc -DUNIT_TEST -I../src -Iunity \
    test_quaternion.c ../src/quaternion.c unity/unity.c \
    -o test_quaternion -lm

# Run tests
./test_quaternion

# Expected output:
# test_quaternion.c:25:test_quaternion_multiply_identity:PASS
# test_quaternion.c:35:test_quaternion_normalize:PASS
# test_quaternion.c:45:test_quaternion_to_euler_level:PASS
#
# -----------------------
# 3 Tests 0 Failures 0 Ignored
# OK
\end{lstlisting}

\subsection{On-Target Testing}

For testing on embedded hardware, link tests into firmware:

\begin{lstlisting}[language=C, caption=Running tests on embedded target]
// test_runner_embedded.c - Run on Crazyflie

#include "unity.h"
#include "FreeRTOS.h"
#include "task.h"
#include "debug.h"

// Import test functions
extern void test_quaternion_multiply_identity(void);
extern void test_quaternion_normalize(void);
extern void test_pid_proportional_only(void);

void TestTask(void *pvParameters) {
    // Wait for system to stabilize
    vTaskDelay(pdMS_TO_TICKS(1000));

    DEBUG_PRINT("Starting unit tests...\n");

    UNITY_BEGIN();

    RUN_TEST(test_quaternion_multiply_identity);
    RUN_TEST(test_quaternion_normalize);
    RUN_TEST(test_pid_proportional_only);

    int failures = UNITY_END();

    DEBUG_PRINT("Tests complete: %d failures\n", failures);

    // Signal completion via LED
    if (failures == 0) {
        SetLED(GREEN);
    } else {
        SetLED(RED);
    }

    vTaskDelete(NULL);
}
\end{lstlisting}

\section{Test Organization Best Practices}

\begin{keyidea}[title=Unit Testing Guidelines]
\begin{enumerate}
    \item \textbf{One test file per module}: \texttt{test\_pid.c} tests \texttt{pid.c}
    \item \textbf{Test names describe behavior}: \texttt{test\_pid\_saturates\_at\_limit}
    \item \textbf{Independent tests}: Each test should pass regardless of order
    \item \textbf{Fast tests}: Unit tests should run in milliseconds
    \item \textbf{Test edge cases}: Zero, negative, maximum values, NaN
    \item \textbf{Use setup/teardown}: Initialize state before each test
    \item \textbf{Test failure paths}: Verify error handling works correctly
\end{enumerate}
\end{keyidea}

\begin{center}
\begin{tabular}{ll}
\toprule
\textbf{Directory Structure} & \textbf{Purpose} \\
\midrule
\texttt{src/} & Production source code \\
\texttt{src/pid.c} & PID controller implementation \\
\texttt{test/} & Test files and framework \\
\texttt{test/unity/} & Unity framework files \\
\texttt{test/mocks/} & Mock implementations \\
\texttt{test/test\_pid.c} & Tests for PID module \\
\texttt{test/test\_main.c} & Test runner \\
\bottomrule
\end{tabular}
\end{center}

%======================================================================
\chapter{Continuous Integration for Embedded Systems}
%======================================================================

Continuous Integration (CI) automates the process of building, testing, and validating code changes. For embedded flight controllers, CI helps catch bugs early and ensures that every change is tested consistently.

\section{Why CI Matters for Flight Controllers}

Manual testing of embedded software is error-prone and time-consuming:

\begin{itemize}
    \item \textbf{Consistency}: CI runs the same tests the same way every time
    \item \textbf{Early feedback}: Bugs are caught within minutes of introduction
    \item \textbf{Documentation}: CI history shows what was tested and when
    \item \textbf{Collaboration}: Team members can trust that merged code works
    \item \textbf{Release confidence}: Every release candidate passes the same criteria
\end{itemize}

\begin{keyidea}[title=The CI Philosophy]
Every commit should be:
\begin{enumerate}
    \item Automatically built
    \item Automatically tested
    \item Validated against quality gates
\end{enumerate}
before it can be merged into the main branch.
\end{keyidea}

\section{CI Pipeline Architecture}

A typical CI pipeline for embedded flight controller code:

\begin{center}
\begin{tabular}{clp{7cm}}
\toprule
\textbf{Stage} & \textbf{Name} & \textbf{Actions} \\
\midrule
1 & Build & Cross-compile for target (ARM), compile for host (x86) \\
2 & Static Analysis & Run linters, check coding standards, analyze warnings \\
3 & Unit Tests & Run all unit tests on host \\
4 & Integration Tests & Test module interactions \\
5 & MIL Simulation & Run automated flight scenarios in simulation \\
6 & Coverage Report & Generate and check code coverage \\
7 & Documentation & Build docs, check for broken links \\
8 & Artifact Storage & Store binary, coverage reports, test results \\
\bottomrule
\end{tabular}
\end{center}

\section{GitHub Actions Configuration}

GitHub Actions is a popular CI platform that's free for open-source projects. Here's a complete workflow for a flight controller project:

See Listing~\ref{lst:ci-yaml} (Appendix~\ref{app:code-listings}) for a complete GitHub Actions CI workflow including ARM cross-compilation, unit testing, static analysis, code coverage with threshold enforcement, and MATLAB simulation tests.

\section{Cross-Compilation in CI}

Flight controller code targets ARM Cortex-M processors. The CI pipeline must cross-compile:

\begin{lstlisting}[language=cmake, caption=cmake/arm-none-eabi.cmake]
# ARM Cortex-M cross-compilation toolchain file

set(CMAKE_SYSTEM_NAME Generic)
set(CMAKE_SYSTEM_PROCESSOR arm)

# Specify the cross compiler
set(CMAKE_C_COMPILER arm-none-eabi-gcc)
set(CMAKE_CXX_COMPILER arm-none-eabi-g++)
set(CMAKE_ASM_COMPILER arm-none-eabi-gcc)

# Compiler flags for Cortex-M4F (STM32F4)
set(CPU_FLAGS "-mcpu=cortex-m4 -mthumb -mfpu=fpv4-sp-d16 -mfloat-abi=hard")
set(CMAKE_C_FLAGS "${CPU_FLAGS} -Wall -Wextra -Os -ffunction-sections -fdata-sections")
set(CMAKE_EXE_LINKER_FLAGS "-Wl,--gc-sections -specs=nosys.specs")

# Don't try to run test executables (they're for ARM)
set(CMAKE_TRY_COMPILE_TARGET_TYPE STATIC_LIBRARY)
\end{lstlisting}

\begin{lstlisting}[language=cmake, caption=CMakeLists.txt with dual-target support]
cmake_minimum_required(VERSION 3.16)
project(FlightController C)

option(BUILD_TESTS "Build unit tests" OFF)
option(ENABLE_COVERAGE "Enable code coverage" OFF)

# Flight controller source files
set(FC_SOURCES
    src/pid.c
    src/attitude_controller.c
    src/sensor_fusion.c
    src/motor_mixer.c
    src/failsafe.c
)

if(CMAKE_CROSSCOMPILING)
    # Build for target hardware (ARM)
    add_executable(flight_controller
        ${FC_SOURCES}
        src/main.c
        src/hal_stm32.c
    )
    target_link_libraries(flight_controller PRIVATE m)

else()
    # Build for host (testing)
    if(BUILD_TESTS)
        enable_testing()

        if(ENABLE_COVERAGE)
            set(CMAKE_C_FLAGS "${CMAKE_C_FLAGS} --coverage")
            set(CMAKE_EXE_LINKER_FLAGS "${CMAKE_EXE_LINKER_FLAGS} --coverage")
        endif()

        # Build test executable
        add_executable(run_tests
            ${FC_SOURCES}
            test/unity/unity.c
            test/test_main.c
            test/test_pid.c
            test/test_attitude.c
            test/test_sensor_fusion.c
            test/mocks/mock_hal.c
        )
        target_include_directories(run_tests PRIVATE
            src/
            test/
            test/unity/
            test/mocks/
        )
        target_compile_definitions(run_tests PRIVATE
            UNIT_TEST
            MOCK_HAL
        )
        target_link_libraries(run_tests PRIVATE m)

        add_test(NAME unit_tests COMMAND run_tests)
    endif()
endif()
\end{lstlisting}

\section{Static Analysis Integration}

Static analysis catches bugs without running code:

\begin{lstlisting}[language=yaml, caption=Static analysis with multiple tools]
static-analysis:
  runs-on: ubuntu-latest

  steps:
  - uses: actions/checkout@v4

  # Cppcheck: general static analysis
  - name: Cppcheck
    run: |
      cppcheck --enable=all \
               --suppress=missingIncludeSystem \
               --error-exitcode=1 \
               --xml --xml-version=2 \
               src/ 2> cppcheck-results.xml

  # Clang-tidy: more advanced analysis
  - name: Clang-tidy
    run: |
      clang-tidy src/*.c -- -I include/

  # MISRA compliance check (requires commercial tool or custom rules)
  - name: MISRA Check
    run: |
      # Using cppcheck MISRA addon as example
      cppcheck --addon=misra --suppress=unusedFunction src/

  # Check for common embedded bugs
  - name: Check for dangerous patterns
    run: |
      # Check for malloc in safety-critical code
      ! grep -r "malloc\|calloc\|realloc" src/*.c || \
        (echo "ERROR: Dynamic allocation found in flight-critical code" && exit 1)

      # Check for floating division that might cause NaN
      # (simplified check - real analysis needs AST)
      ! grep -E "/ *0\.0" src/*.c || \
        (echo "WARNING: Potential division by zero" && exit 1)
\end{lstlisting}

\begin{notebox}[title=Common Static Analysis Rules for Flight Controllers]
Configure static analyzers to catch:
\begin{itemize}
    \item \textbf{Uninitialized variables}: Can cause random behavior
    \item \textbf{Array bounds}: Buffer overflows crash or corrupt data
    \item \textbf{Null pointer dereference}: Common cause of hard faults
    \item \textbf{Dead code}: May indicate logic errors
    \item \textbf{Integer overflow}: Silent corruption of calculations
    \item \textbf{Unused return values}: Missing error checks
\end{itemize}
\end{notebox}

\section{Automated Simulation Testing}

CI can run MATLAB/Simulink simulations to catch integration issues:

\begin{lstlisting}[language=Matlab, caption=tests/test\_attitude\_control.m]
function tests = test_attitude_control
    tests = functiontests(localfunctions);
end

function test_step_response_settles(testCase)
    % Load model
    model = 'attitude_control_sim';
    load_system(model);

    % Configure step input
    set_param([model '/Roll_Ref'], 'Value', '0.1');  % 0.1 rad step

    % Run simulation
    simOut = sim(model, 'StopTime', '2.0');

    % Extract results
    roll = simOut.logsout.get('roll').Values.Data;
    time = simOut.logsout.get('roll').Values.Time;

    % Check settling time (2% criterion)
    final_value = roll(end);
    settled_idx = find(abs(roll - final_value) < 0.02 * 0.1, 1, 'first');
    settling_time = time(settled_idx);

    verifyLessThan(testCase, settling_time, 0.5, ...
        'Settling time exceeds 500ms');
end

function test_no_overshoot_exceeds_limit(testCase)
    model = 'attitude_control_sim';
    load_system(model);

    set_param([model '/Roll_Ref'], 'Value', '0.2');
    simOut = sim(model, 'StopTime', '2.0');

    roll = simOut.logsout.get('roll').Values.Data;

    % Check overshoot < 20%
    max_roll = max(roll);
    overshoot = (max_roll - 0.2) / 0.2 * 100;

    verifyLessThan(testCase, overshoot, 20, ...
        'Overshoot exceeds 20%');
end

function test_wind_disturbance_rejection(testCase)
    model = 'attitude_control_sim';
    load_system(model);

    set_param([model '/Roll_Ref'], 'Value', '0');
    set_param([model '/Wind_Disturbance'], 'Value', '0.5');  % 0.5 N*m

    simOut = sim(model, 'StopTime', '3.0');
    roll = simOut.logsout.get('roll').Values.Data;

    % Steady-state error should be small
    steady_state_error = abs(roll(end));
    verifyLessThan(testCase, steady_state_error, 0.01, ...
        'Steady-state error too large under wind disturbance');
end
\end{lstlisting}

\section{Branch Protection and Quality Gates}

Configure repository settings to enforce quality:

\begin{lstlisting}[language=yaml, caption=Branch protection configuration (GitHub)]
# In repository Settings > Branches > Branch protection rules

Branch name pattern: main

Required status checks:
  - build-and-test
  - static-analysis
  - coverage

Additional settings:
  - Require branches to be up to date before merging
  - Require review from code owners
  - Dismiss stale pull request approvals when new commits are pushed
\end{lstlisting}

\textbf{Quality gates example}:
\begin{itemize}
    \item All unit tests must pass
    \item Code coverage must be $\geq$ 80\%
    \item No cppcheck errors (warnings may be allowed)
    \item Code must compile without warnings (\texttt{-Werror})
    \item Formatting must match project style
\end{itemize}

\section{Artifact Management}

CI should preserve build outputs for debugging and deployment:

\begin{lstlisting}[language=yaml, caption=Artifact storage configuration]
- name: Build release firmware
  run: |
    mkdir build-release && cd build-release
    cmake -DCMAKE_BUILD_TYPE=Release \
          -DCMAKE_TOOLCHAIN_FILE=../cmake/arm-none-eabi.cmake ..
    make

    # Generate binary and hex files for flashing
    arm-none-eabi-objcopy -O binary flight_controller flight_controller.bin
    arm-none-eabi-objcopy -O ihex flight_controller flight_controller.hex

    # Generate size report
    arm-none-eabi-size flight_controller > size_report.txt

- name: Upload firmware artifacts
  uses: actions/upload-artifact@v4
  with:
    name: firmware-${{ github.sha }}
    path: |
      build-release/flight_controller.bin
      build-release/flight_controller.hex
      build-release/size_report.txt

- name: Check flash size limit
  run: |
    TEXT_SIZE=$(arm-none-eabi-size build-release/flight_controller | tail -1 | awk '{print $1}')
    DATA_SIZE=$(arm-none-eabi-size build-release/flight_controller | tail -1 | awk '{print $2}')
    TOTAL=$((TEXT_SIZE + DATA_SIZE))

    # STM32F405 has 1MB flash, leave margin
    MAX_SIZE=900000

    if [ $TOTAL -gt $MAX_SIZE ]; then
      echo "ERROR: Firmware size $TOTAL exceeds limit $MAX_SIZE"
      exit 1
    fi
    echo "Firmware size: $TOTAL bytes (limit: $MAX_SIZE)"
\end{lstlisting}

\section{Hardware-in-the-Loop CI}

For critical projects, CI can include HIL testing with self-hosted runners:

\begin{lstlisting}[language=yaml, caption=HIL testing with self-hosted runner]
hil-test:
  runs-on: [self-hosted, hil-bench]
  needs: build-and-test  # Only run after unit tests pass

  steps:
  - name: Checkout code
    uses: actions/checkout@v4

  - name: Download firmware artifact
    uses: actions/download-artifact@v4
    with:
      name: firmware-${{ github.sha }}

  - name: Flash firmware to test board
    run: |
      st-flash write flight_controller.bin 0x8000000

  - name: Wait for boot
    run: sleep 2

  - name: Run HIL test sequence
    timeout-minutes: 10
    run: |
      python3 hil_tests/run_hil_tests.py \
        --port /dev/ttyUSB0 \
        --output results/

  - name: Upload HIL results
    uses: actions/upload-artifact@v4
    if: always()
    with:
      name: hil-results
      path: results/
\end{lstlisting}

\begin{warningbox}[title=Self-Hosted Runner Security]
Self-hosted runners with hardware access require careful security:
\begin{itemize}
    \item Only allow trusted repositories
    \item Use dedicated service accounts
    \item Isolate the HIL network
    \item Log all hardware access
\end{itemize}
For open-source projects, HIL testing is typically triggered manually by maintainers.
\end{warningbox}

\section{Notifications and Monitoring}

Configure notifications for pipeline failures:

\begin{lstlisting}[language=yaml, caption=Slack notification on failure]
notify-failure:
  runs-on: ubuntu-latest
  needs: [build-and-test, static-analysis, coverage]
  if: failure()

  steps:
  - name: Notify Slack
    uses: slackapi/slack-github-action@v1
    with:
      channel-id: 'flight-controller-ci'
      slack-message: |
        CI Failed for ${{ github.repository }}
        Commit: ${{ github.sha }}
        Author: ${{ github.actor }}
        Branch: ${{ github.ref_name }}
        <${{ github.server_url }}/${{ github.repository }}/actions/runs/${{ github.run_id }}|View Run>
    env:
      SLACK_BOT_TOKEN: ${{ secrets.SLACK_BOT_TOKEN }}
\end{lstlisting}

\begin{keyidea}[title=CI Best Practices for Flight Controllers]
\begin{enumerate}
    \item \textbf{Fast feedback}: Keep total pipeline time under 15 minutes
    \item \textbf{Fail fast}: Run quick checks (build, lint) before slow tests
    \item \textbf{Reproducibility}: Pin tool versions, use containers
    \item \textbf{Visibility}: Show status badges in README
    \item \textbf{Incremental adoption}: Start with build + unit tests, add more later
    \item \textbf{Test the tests}: Occasionally verify that tests can fail
\end{enumerate}
\end{keyidea}

% CI status badge example for README.md
% ![CI Status](https://github.com/user/repo/actions/workflows/ci.yml/badge.svg)

%======================================================================
\chapter{Debugging Embedded Flight Controllers}
%======================================================================

\section{Challenges of Embedded Debugging}

Debugging a flight controller is fundamentally different from debugging desktop software:

\begin{center}
\begin{tabular}{p{4cm}p{7.5cm}}
\toprule
\textbf{Challenge} & \textbf{Implication} \\
\midrule
No display & Cannot use \texttt{printf} to console during flight \\
Real-time constraints & Cannot pause execution (quadrotor would fall) \\
Limited memory & Cannot store large debug logs onboard \\
Concurrent tasks & Bugs may depend on timing, hard to reproduce \\
Physical coupling & Software bugs cause physical crashes \\
Remote operation & Cannot attach debugger during flight \\
\bottomrule
\end{tabular}
\end{center}

\section{Debugging Techniques}

\subsection{Ground Testing with Printf}

Before flight, test as much as possible on the ground with the quadrotor constrained:

\begin{lstlisting}[language=C, caption=Debug output for ground testing]
void AttitudeControlTask(void *pvParameters) {
    for (;;) {
        // Read sensors
        Quaternion_t q = GetOrientation();

        #ifdef DEBUG_ATTITUDE
        printf("q: %.3f %.3f %.3f %.3f\n", q.w, q.x, q.y, q.z);
        printf("roll: %.1f pitch: %.1f yaw: %.1f\n",
               RAD2DEG(roll), RAD2DEG(pitch), RAD2DEG(yaw));
        #endif

        // Compute control...
    }
}
\end{lstlisting}

Compile with \texttt{-DDEBUG\_ATTITUDE} for ground testing, without for flight.

\subsection{Data Logging and Telemetry}

Log essential data to onboard storage or transmit via radio:

\begin{lstlisting}[language=C, caption=Efficient data logging structure]
typedef struct {
    uint32_t timestamp_ms;
    int16_t accel[3];      // Raw accelerometer (scaled)
    int16_t gyro[3];       // Raw gyroscope (scaled)
    int16_t attitude[3];   // Roll, pitch, yaw (scaled degrees * 100)
    int16_t motors[4];     // Motor commands (PWM)
    uint8_t flags;         // Status flags
} LogEntry_t;              // 26 bytes per entry

// At 100 Hz, 1 minute = 6000 entries = 156 KB
\end{lstlisting}

\textbf{Key considerations}:
\begin{itemize}
    \item Use fixed-point or scaled integers to reduce size
    \item Log at lower rate than control loop if memory limited
    \item Include timestamps for synchronization
    \item Transmit subset in real-time, log full set for post-analysis
\end{itemize}

\subsection{Post-Flight Analysis}

After a flight (or crash), download logged data and analyze:

\begin{enumerate}
    \item \textbf{Plot key signals}: Attitude, position, motor commands over time
    \item \textbf{Identify anomalies}: Spikes, oscillations, saturation
    \item \textbf{Correlate events}: What happened just before the failure?
    \item \textbf{Check timing}: Were tasks running at expected rates?
    \item \textbf{Compare to simulation}: Does logged behavior match expected?
\end{enumerate}

\subsection{Hardware Debuggers (JTAG/SWD)}

For low-level debugging, use a hardware debugger:

\begin{itemize}
    \item \textbf{JTAG/SWD}: Standard debug interfaces on ARM Cortex-M
    \item \textbf{Capabilities}: Set breakpoints, inspect memory, single-step code
    \item \textbf{Tools}: ST-Link, J-Link, OpenOCD with GDB
\end{itemize}

\begin{lstlisting}[language=bash, caption=GDB debugging session with ST-Link]
# Start OpenOCD server
openocd -f interface/stlink.cfg -f target/stm32f4x.cfg

# In another terminal, connect GDB
arm-none-eabi-gdb firmware.elf
(gdb) target remote localhost:3333
(gdb) break AttitudeControlTask
(gdb) continue
# Execution stops at breakpoint
(gdb) print current_attitude
(gdb) next
\end{lstlisting}

\begin{warningbox}
Hardware debuggers \textbf{stop execution}. Use only for ground testing or post-mortem analysis (when quadrotor has crashed and is safe). Never use breakpoints during flight!
\end{warningbox}

\subsection{LED and Audio Indicators}

Simple but effective for real-time status:

\begin{lstlisting}[language=C, caption=LED status indicators]
void UpdateStatusLED(void) {
    if (system_state == STATE_ERROR) {
        LED_SetColor(RED);
        LED_Blink(FAST);
    } else if (battery_low) {
        LED_SetColor(YELLOW);
        LED_Blink(SLOW);
    } else if (armed) {
        LED_SetColor(GREEN);
        LED_Solid();
    } else {
        LED_SetColor(BLUE);
        LED_Solid();
    }
}
\end{lstlisting}

\section{Common Failure Modes in Quadrotors}

\begin{center}
\begin{tabular}{p{3.5cm}p{4cm}p{4.5cm}}
\toprule
\textbf{Symptom} & \textbf{Possible Cause} & \textbf{Debugging Approach} \\
\midrule
Uncontrolled spin & Yaw controller issue, motor direction & Check motor order, yaw gains \\
Flip on takeoff & Attitude reference wrong, IMU orientation & Verify sensor orientation, initial attitude \\
Oscillation & Gains too high, sensor noise & Reduce gains, check sensor data \\
Drift & Sensor bias, calibration & Check accelerometer offsets \\
Sudden drop & Motor saturation, battery & Log motor commands, voltage \\
Erratic behavior & Race condition, stack overflow & Check task priorities, stack usage \\
\bottomrule
\end{tabular}
\end{center}

\section{Defensive Programming Practices}

Prevent bugs through careful coding:

\begin{lstlisting}[language=C, caption=Defensive programming examples]
// 1. Assert preconditions (disabled in release build)
void SetMotorSpeed(uint8_t motor_id, float speed) {
    assert(motor_id < NUM_MOTORS);
    assert(speed >= 0.0f && speed <= 1.0f);
    // ...
}

// 2. Saturate outputs to safe limits
float thrust_cmd = ComputeThrust();
thrust_cmd = CLAMP(thrust_cmd, 0.0f, MAX_THRUST);

// 3. Check for NaN/Inf (can happen with bad sensor data)
if (isnan(attitude.roll) || isinf(attitude.roll)) {
    EnterFailsafe("Invalid attitude");
    return;
}

// 4. Watchdog for control loop
void ControlTask(void) {
    for (;;) {
        FeedWatchdog();  // Must be called every iteration
        // Control code...
        vTaskDelayUntil(&lastWakeTime, CONTROL_PERIOD);
    }
}

// 5. Stack overflow detection (FreeRTOS)
void vApplicationStackOverflowHook(TaskHandle_t task, char *name) {
    // Log error and enter safe mode
    LogError("Stack overflow in task: %s", name);
    EmergencyLand();
}
\end{lstlisting}

%======================================================================
\chapter{Case Study: Testing Crazyflie Attitude Control}
%======================================================================

This case study walks through a complete testing workflow for a quadrotor attitude controller.

\section{System Description}

\textbf{Plant}: Crazyflie 2.1 quadrotor
\begin{itemize}
    \item Mass: 27 g
    \item Moment of inertia: $I_{xx} = I_{yy} \approx 1.4 \times 10^{-5}$ kg$\cdot$m$^2$
    \item Control rate: 500 Hz
\end{itemize}

\textbf{Controller}: Cascaded PID
\begin{itemize}
    \item Outer loop: Attitude error $\rightarrow$ desired angular rate
    \item Inner loop: Angular rate error $\rightarrow$ motor torque
\end{itemize}

\textbf{Objective}: Verify the attitude controller meets requirements before flight testing.

\section{Step 1: Define Requirements}

\textbf{R1 - Safety}: Roll and pitch shall remain within $\pm 45°$ at all times.
\[
\varphi_1 = \Box_{[0,T]}(|\phi| < 45° \land |\theta| < 45°)
\]

\textbf{R2 - Settling time}: Given a 20° step in roll reference, the roll error shall be less than 2° within 0.3 seconds.
\[
\varphi_2 = \Box_{[0,T]}\left( (\phi_{ref}' = 20°) \Rightarrow \Diamond_{[0,0.3]}(|\phi - \phi_{ref}| < 2°) \right)
\]

\textbf{R3 - Overshoot}: Roll shall not exceed the reference by more than 25\%.
\[
\varphi_3 = \Box_{[0,T]}\left( \phi < \phi_{ref} \cdot 1.25 + 5° \right)
\]

\textbf{R4 - Disturbance rejection}: Under torque disturbances up to 0.001 Nm, attitude error shall remain below 10°.
\[
\varphi_4 = \Box_{[0,T]}\left( (|\tau_d| < 0.001) \Rightarrow (|\phi - \phi_{ref}| < 10°) \right)
\]

\section{Step 2: Build Simulation Model}

Create a Simulink model with:
\begin{enumerate}
    \item Quadrotor dynamics (rigid body + motor dynamics)
    \item PID attitude controller (matching C implementation)
    \item Sensor models (IMU with noise)
    \item Disturbance inputs (external torques)
\end{enumerate}

\section{Step 3: MIL Testing}

Run initial tests in pure simulation:

\begin{lstlisting}[language=Matlab, caption=MIL test script]
% Load model
model = 'crazyflie_attitude_sim';
load_system(model);

% Test nominal step response
sim_out = sim(model, 'StopTime', '2');
roll = sim_out.roll;
roll_ref = sim_out.roll_ref;

% Check settling time manually
settling_idx = find(abs(roll - roll_ref(end)) < 2, 1);
settling_time = sim_out.tout(settling_idx);
fprintf('Settling time: %.3f s\n', settling_time);

% Check overshoot
overshoot = (max(roll) - roll_ref(end)) / roll_ref(end) * 100;
fprintf('Overshoot: %.1f%%\n', overshoot);
\end{lstlisting}

Results:
\begin{itemize}
    \item Settling time: 0.21 s (meets R2)
    \item Overshoot: 18\% (meets R3)
\end{itemize}

\section{Step 4: Falsification with Breach}

Set up systematic falsification to search for violations:

\begin{lstlisting}[language=Matlab, caption=Breach falsification setup]
% Initialize Breach with Simulink model
B = BreachSimulinkSystem('crazyflie_attitude_sim');

% Define test input ranges
% Disturbance torque (Nm)
B.SetParamRanges('tau_dist_x', [-0.002, 0.002]);
B.SetParamRanges('tau_dist_y', [-0.002, 0.002]);

% Initial attitude (degrees)
B.SetParamRanges('phi_init', [-10, 10]);
B.SetParamRanges('theta_init', [-10, 10]);

% Reference step parameters
B.SetParamRanges('phi_ref_step', [10, 30]);  % Step size

% Define STL specifications
STL_R1 = 'alw_[0,2] (abs(roll_deg) < 45 and abs(pitch_deg) < 45)';
STL_R2 = 'alw_[0,2] ((step_trigger > 0) => ev_[0,0.3] (abs(roll_error) < 2))';
STL_R4 = 'alw_[0,2] ((abs(tau_dist_x) < 0.001) => (abs(roll_error) < 10))';

% Create falsification problem for R1
falsif_R1 = FalsificationProblem(B, STL_R1);
falsif_R1.max_obj_eval = 100;
falsif_R1.setup_solver('cmaes');

% Run falsification
fprintf('Falsifying R1 (safety)...\n');
falsif_R1.solve();

if falsif_R1.obj_best < 0
    fprintf('R1 FALSIFIED with robustness %.2f\n', falsif_R1.obj_best);
    % Get and plot counterexample
    BrCE = falsif_R1.GetBrSet_Logged();
    figure;
    BrCE.PlotSignals({'roll_deg', 'pitch_deg', 'tau_dist_x'});
else
    fprintf('R1 not falsified (robustness %.2f)\n', falsif_R1.obj_best);
end
\end{lstlisting}

\section{Step 5: Analyze Results}

After running falsification for all requirements:

\begin{center}
\begin{tabular}{lccc}
\toprule
\textbf{Requirement} & \textbf{Evaluations} & \textbf{Min Robustness} & \textbf{Status} \\
\midrule
R1 (Safety) & 100 & +12.3° & Not falsified \\
R2 (Settling) & 100 & +0.08 s & Not falsified \\
R3 (Overshoot) & 100 & +3.2° & Not falsified \\
R4 (Disturbance) & 100 & -2.1° & \textbf{Falsified!} \\
\bottomrule
\end{tabular}
\end{center}

\textbf{Finding}: R4 was falsified. A specific combination of disturbance torque and initial conditions causes the roll error to exceed 10° momentarily.

\section{Step 6: Debug and Fix}

Analyze the counterexample for R4:

\begin{itemize}
    \item Disturbance: $\tau_x = 0.00095$ Nm (just under the 0.001 Nm limit)
    \item Initial roll: $\phi_0 = -8°$
    \item Peak error: 12.1° (violates the 10° limit)
\end{itemize}

\textbf{Root cause}: When the disturbance starts while the quadrotor is already tilted, the combined effect exceeds the error limit.

\textbf{Fix options}:
\begin{enumerate}
    \item Increase integral gain to reject disturbances faster
    \item Relax the requirement to allow 15° error
    \item Add feedforward disturbance rejection
\end{enumerate}

After increasing $K_i$ by 20\%:
\begin{itemize}
    \item Re-run falsification: R4 no longer falsified
    \item Check other requirements: Still satisfied
    \item Minimum robustness for R4: +1.8° (satisfies with margin)
\end{itemize}

\section{Step 7: Proceed to HIL and Flight Testing}

With all requirements satisfied in simulation:
\begin{enumerate}
    \item Flash updated firmware to Crazyflie
    \item Perform constrained ground tests (props off, then on with restraint)
    \item Conduct test flights in netted arena
    \item Compare logged flight data to simulation predictions
\end{enumerate}

\section{Lessons Learned}

\begin{keyidea}[title=Key Takeaways from the Case Study]
\begin{enumerate}
    \item \textbf{Formal requirements} force precise thinking about what ``working'' means
    \item \textbf{Falsification} found a bug that random testing missed
    \item The counterexample provided \textbf{actionable insight} for fixing the controller
    \item \textbf{Multiple requirements} must be checked---fixing one might break another
    \item Simulation testing is \textbf{necessary but not sufficient}---flight testing is still required
\end{enumerate}
\end{keyidea}

%======================================================================
\section{Conclusions}
%======================================================================

\begin{itemize}
    \item Testing CPS is fundamentally harder than testing discrete software due to the continuous state space and real-time constraints.

    \item \textbf{Model-based testing} at multiple levels (MIL, SIL, HIL) provides scalable verification with increasing realism.

    \item \textbf{Signal Temporal Logic} provides a formal language to express requirements over continuous-time signals.

    \item \textbf{Robustness semantics} turn Boolean satisfaction into a continuous metric, enabling optimization-based testing.

    \item \textbf{Falsification} systematically searches for violations, finding corner cases that random testing misses.

    \item Falsification can only show \textbf{presence of errors}, not their absence---it complements but doesn't replace other verification methods.

    \item \textbf{Code coverage} criteria like MC/DC are required by safety standards but don't guarantee correctness for CPS.

    \item \textbf{Debugging embedded systems} requires specialized techniques: data logging, post-flight analysis, and defensive programming.

    \item Testing and verification methods \textbf{complement each other}: formal methods for critical subsystems, falsification for system-level properties, coverage for code quality, and flight testing for final validation.
\end{itemize}


% ===== APPENDICES =====
\part{Appendices}

% Appendix: C Programming for Python Programmers
%======================================================================
% APPENDIX: C Programming for Python Programmers
%
% Target audience: Students familiar with Python who need to write
% embedded C code for microcontrollers and FreeRTOS applications
%======================================================================

\appendix
\chapter{C Programming for Python Programmers}

\section{Introduction: Why C for Embedded Systems}

If you're coming from Python, you might wonder why embedded systems use C instead of a more modern, user-friendly language. The answer lies in the fundamental differences between desktop/server computing and embedded systems:

\begin{itemize}
    \item \textbf{Memory constraints}: The Crazyflie's STM32 has 192 KB of RAM. Python's runtime alone requires megabytes.
    \item \textbf{Predictable timing}: C compiles to machine code with deterministic execution time. Python's garbage collector can pause execution unpredictably.
    \item \textbf{Direct hardware access}: C can read/write specific memory addresses (hardware registers) directly.
    \item \textbf{No runtime overhead}: C has no interpreter, no virtual machine, no garbage collector.
\end{itemize}

\begin{keyidea}[title=Compiled vs. Interpreted]
Python is \textbf{interpreted}: your code is read and executed line-by-line by the Python interpreter at runtime. C is \textbf{compiled}: your code is translated to machine instructions \textit{before} execution. The compiled binary runs directly on the processor with no intermediate layer.
\end{keyidea}

The trade-off is clear: C requires more careful programming (you manage memory manually, declare types explicitly, and handle errors yourself), but it gives you complete control over exactly what the processor does and when.

%======================================================================
\section{Variables and Type Declarations}
%======================================================================

\subsection{Python's Dynamic Typing vs. C's Static Typing}

In Python, variables don't have types---values do. You can assign any value to any variable:

\begin{center}
\begin{tabular}{p{0.45\textwidth}|p{0.45\textwidth}}
\textbf{Python} & \textbf{C} \\
\hline
\begin{lstlisting}[language=Python, numbers=none]
x = 42          # x holds an integer
x = 3.14        # now x holds a float
x = "hello"     # now x holds a string
\end{lstlisting}
&
\begin{lstlisting}[language=C, numbers=none]
int x = 42;       // x can ONLY hold integers
x = 3.14;         // truncates to 3!
x = "hello";      // compile error
\end{lstlisting}
\end{tabular}
\end{center}

In C, every variable must be \textbf{declared} with a specific type before use. The type determines:
\begin{enumerate}
    \item How many bytes of memory to allocate
    \item How to interpret the bit pattern stored there
    \item What operations are valid
\end{enumerate}

\subsection{Basic Variable Declaration}

\begin{center}
\begin{tabular}{p{0.45\textwidth}|p{0.45\textwidth}}
\textbf{Python} & \textbf{C} \\
\hline
\begin{lstlisting}[language=Python, numbers=none]
# Variables created on first assignment
count = 0
temperature = 25.5
is_active = True
\end{lstlisting}
&
\begin{lstlisting}[language=C, numbers=none]
// Must declare type before use
int count = 0;
float temperature = 25.5f;
bool is_active = true;  // needs stdbool.h
\end{lstlisting}
\end{tabular}
\end{center}

\begin{notebox}
In C, the \texttt{f} suffix on \texttt{25.5f} indicates a \texttt{float} literal. Without it, \texttt{25.5} is a \texttt{double} (64-bit), which would be implicitly converted to \texttt{float} (32-bit).
\end{notebox}

\subsection{Why Types Matter for Embedded Systems}

On a desktop computer, you rarely think about whether an integer takes 4 or 8 bytes. On an embedded system with 192 KB of RAM, every byte counts:

\begin{lstlisting}[language=C, caption=Memory-conscious variable declarations]
// Sensor data buffer: 1000 samples
// Using int (4 bytes each): 4000 bytes
int sensor_buffer_wasteful[1000];

// Using int16_t (2 bytes each): 2000 bytes - half the memory!
int16_t sensor_buffer_efficient[1000];

// For flags and small counters, use the smallest type that fits
uint8_t motor_id;        // 0-255 is enough for motor index (1 byte)
uint16_t sample_count;   // 0-65535 samples (2 bytes)
\end{lstlisting}

%======================================================================
\section{Data Types for Embedded Programming}
%======================================================================

\subsection{Standard C Types}

C provides several built-in types, but their sizes can vary between platforms:

\begin{center}
\begin{tabular}{lll}
\toprule
\textbf{Type} & \textbf{Typical Size} & \textbf{Description} \\
\midrule
\texttt{char} & 1 byte & Character or small integer \\
\texttt{short} & 2 bytes & Short integer \\
\texttt{int} & 4 bytes & Standard integer \\
\texttt{long} & 4 or 8 bytes & Long integer (platform-dependent!) \\
\texttt{float} & 4 bytes & Single-precision floating point \\
\texttt{double} & 8 bytes & Double-precision floating point \\
\bottomrule
\end{tabular}
\end{center}

\begin{warningbox}
The sizes of \texttt{int} and \texttt{long} are \textbf{platform-dependent}. Code that assumes \texttt{int} is 4 bytes may break on different processors. This is why embedded programmers use fixed-width types.
\end{warningbox}

\subsection{Fixed-Width Integer Types (stdint.h)}

For embedded programming, always use the fixed-width types from \texttt{<stdint.h>}:

\begin{center}
\begin{tabular}{llll}
\toprule
\textbf{Type} & \textbf{Size} & \textbf{Range} & \textbf{Use Case} \\
\midrule
\texttt{uint8\_t} & 1 byte & 0 to 255 & GPIO pins, small counters \\
\texttt{int8\_t} & 1 byte & -128 to 127 & Small signed values \\
\texttt{uint16\_t} & 2 bytes & 0 to 65,535 & ADC readings, PWM values \\
\texttt{int16\_t} & 2 bytes & -32,768 to 32,767 & Accelerometer raw data \\
\texttt{uint32\_t} & 4 bytes & 0 to 4,294,967,295 & Timestamps, addresses \\
\texttt{int32\_t} & 4 bytes & $\pm$2.1 billion & General computation \\
\texttt{uint64\_t} & 8 bytes & 0 to $\sim 1.8 \times 10^{19}$ & High-resolution timestamps \\
\texttt{int64\_t} & 8 bytes & $\pm 9.2 \times 10^{18}$ & Large calculations \\
\bottomrule
\end{tabular}
\end{center}

\begin{lstlisting}[language=C, caption=Using fixed-width types for sensor data]
#include <stdint.h>

// IMU sensor readings (raw 16-bit signed values from hardware)
typedef struct {
    int16_t accel_x, accel_y, accel_z;  // Accelerometer
    int16_t gyro_x, gyro_y, gyro_z;     // Gyroscope
} ImuRawData_t;  // Total: 12 bytes

// Motor PWM commands (0-65535 duty cycle)
typedef struct {
    uint16_t motor1, motor2, motor3, motor4;
} MotorCommands_t;  // Total: 8 bytes

// System timestamp (milliseconds since boot)
uint32_t system_time_ms;  // Overflows after 49 days
\end{lstlisting}

\subsection{The sizeof Operator}

C provides the \texttt{sizeof} operator to determine type sizes at compile time:

\begin{center}
\begin{tabular}{p{0.45\textwidth}|p{0.45\textwidth}}
\textbf{Python} & \textbf{C} \\
\hline
\begin{lstlisting}[language=Python, numbers=none]
import sys
x = 42
print(sys.getsizeof(x))  # 28 bytes!
# Python integers have overhead
\end{lstlisting}
&
\begin{lstlisting}[language=C, numbers=none]
#include <stdio.h>
int32_t x = 42;
printf("%zu\n", sizeof(x));  // 4 bytes
printf("%zu\n", sizeof(int32_t)); // 4
\end{lstlisting}
\end{tabular}
\end{center}

A Python integer object includes type information, reference count, and other overhead. A C \texttt{int32\_t} is \textit{exactly} 4 bytes---just the value, nothing else.

%======================================================================
\section{Numeric Operations and Precision}
%======================================================================

\subsection{Integer Division: A Critical Difference}

One of the most common sources of bugs when moving from Python to C is integer division:

\begin{center}
\begin{tabular}{p{0.45\textwidth}|p{0.45\textwidth}}
\textbf{Python 3} & \textbf{C} \\
\hline
\begin{lstlisting}[language=Python, numbers=none]
result = 7 / 2
print(result)  # 3.5 (float)

result = 7 // 2
print(result)  # 3 (integer division)
\end{lstlisting}
&
\begin{lstlisting}[language=C, numbers=none]
int result = 7 / 2;
printf("%d\n", result);  // 3 (truncated!)

float result2 = 7 / 2;
printf("%f\n", result2); // 3.0 (still truncated!)
\end{lstlisting}
\end{tabular}
\end{center}

\begin{warningbox}
In C, when both operands are integers, division produces an integer result (truncated toward zero). This happens \textbf{before} any assignment, so assigning to a float doesn't help---the precision is already lost!
\end{warningbox}

\subsection{Mixing Integers and Floats Safely}

To get floating-point division, at least one operand must be a float:

\begin{lstlisting}[language=C, caption=Correct ways to get floating-point division]
int a = 7, b = 2;

// WRONG: integer division happens first
float wrong = a / b;           // wrong = 3.0

// CORRECT: cast one operand to float
float right1 = (float)a / b;   // right1 = 3.5
float right2 = a / (float)b;   // right2 = 3.5
float right3 = (float)a / (float)b;  // right3 = 3.5

// CORRECT: use floating-point literal
float right4 = a / 2.0f;       // right4 = 3.5
\end{lstlisting}

\subsection{Type Casting}

C allows explicit type conversion (casting):

\begin{lstlisting}[language=C, caption=Explicit type casting]
// Converting float to int (truncates toward zero)
float temperature = 25.7f;
int temp_int = (int)temperature;  // temp_int = 25

// Converting int to float
int count = 100;
float percentage = (float)count / 1000;  // percentage = 0.1

// DANGER: narrowing conversions can lose data
int32_t big_value = 100000;
int16_t small = (int16_t)big_value;  // small = -31072 (overflow!)

uint32_t positive = 42;
int32_t signed_val = (int32_t)positive;  // OK if value fits
\end{lstlisting}

\subsection{Integer Overflow and Underflow}

Unlike Python, C integers have fixed sizes and can overflow:

\begin{center}
\begin{tabular}{p{0.45\textwidth}|p{0.45\textwidth}}
\textbf{Python} & \textbf{C} \\
\hline
\begin{lstlisting}[language=Python, numbers=none]
x = 255
x = x + 1
print(x)  # 256 (arbitrary precision)

x = 2 ** 100
print(x)  # Works! Very large number
\end{lstlisting}
&
\begin{lstlisting}[language=C, numbers=none]
uint8_t x = 255;
x = x + 1;
printf("%d\n", x);  // 0 (wrapped around!)

int32_t y = 2147483647;  // Max value
y = y + 1;
printf("%d\n", y);  // -2147483648 (overflow!)
\end{lstlisting}
\end{tabular}
\end{center}

\begin{keyidea}[title=Overflow Behavior]
Unsigned integer overflow is well-defined in C: values wrap around (modulo $2^n$). Signed integer overflow is \textbf{undefined behavior}---the compiler can do anything, including optimizing away checks you thought would catch it!
\end{keyidea}

\subsection{Safe Arithmetic for Control Systems}

In flight controller code, overflow can be catastrophic:

\begin{lstlisting}[language=C, caption=Safe arithmetic patterns]
// BAD: Can overflow before comparison
int16_t sensor_value = 30000;
if (sensor_value + 10000 > 32767) {  // Overflow happens first!
    // This branch may never execute
}

// GOOD: Check before arithmetic
int16_t sensor_value = 30000;
if (sensor_value > 32767 - 10000) {  // Safe check
    // Handle overflow case
}

// GOOD: Use larger type for intermediate calculations
int16_t a = 30000, b = 20000;
int32_t product = (int32_t)a * b;  // Won't overflow
int16_t result = (product > 32767) ? 32767 : (int16_t)product;  // Saturate
\end{lstlisting}

%======================================================================
\section{Pointers and Memory Addresses}
%======================================================================

Pointers are C's most powerful and most confusing feature. Coming from Python, you've never had to think about where values are stored in memory. In C, you often need to work with memory addresses directly.

\subsection{What Is a Pointer?}

A \textbf{pointer} is a variable that holds a memory address. Instead of storing a value like \texttt{42}, it stores the \textit{location} where \texttt{42} is stored.

\begin{lstlisting}[language=C, caption=Pointer basics]
int x = 42;       // x is a regular variable holding 42
int *p = &x;      // p is a pointer holding the ADDRESS of x

printf("Value of x: %d\n", x);       // 42
printf("Address of x: %p\n", &x);    // 0x7ffd1234 (some memory address)
printf("Value of p: %p\n", p);       // 0x7ffd1234 (same address)
printf("Value at p: %d\n", *p);      // 42 (dereferencing)
\end{lstlisting}

\textbf{Key operators}:
\begin{itemize}
    \item \texttt{\&x} --- ``address of x'' --- gives you a pointer to \texttt{x}
    \item \texttt{*p} --- ``dereference p'' --- gives you the value that \texttt{p} points to
\end{itemize}

\subsection{Pointer Declaration Syntax}

\begin{lstlisting}[language=C, caption=Declaring pointers]
int *p;        // p is a pointer to an int
float *fp;     // fp is a pointer to a float
char *str;     // str is a pointer to a char (commonly used for strings)

// The * is part of the declaration, not the name
int *p1, *p2;  // Both are pointers
int *p3, x;    // p3 is a pointer, x is a regular int (confusing!)

// Clearer style: one declaration per line
int *p4;
int *p5;
int y;
\end{lstlisting}

\subsection{Why Pointers Matter for Embedded Systems}

\textbf{1. Hardware registers are at fixed memory addresses:}
\begin{lstlisting}[language=C, caption=Accessing hardware registers via pointers]
// STM32 GPIO register at fixed address
#define GPIOA_ODR (*(volatile uint32_t*)0x40020014)

// Set pin 5 high
GPIOA_ODR |= (1 << 5);
\end{lstlisting}

\textbf{2. Functions can modify caller's variables:}
\begin{lstlisting}[language=C, caption=Using pointers to modify variables]
// Without pointers: can't modify caller's variable
void increment_wrong(int x) {
    x = x + 1;  // Modifies local copy, original unchanged
}

// With pointers: can modify caller's variable
void increment_right(int *x) {
    *x = *x + 1;  // Modifies value at address x points to
}

int main(void) {
    int value = 10;
    increment_wrong(value);
    printf("%d\n", value);  // Still 10

    increment_right(&value);
    printf("%d\n", value);  // Now 11
}
\end{lstlisting}

\textbf{3. Efficient passing of large structures:}
\begin{lstlisting}[language=C, caption=Passing structures by pointer]
typedef struct {
    float position[3];
    float velocity[3];
    float orientation[4];
    float angular_velocity[3];
} QuadrotorState_t;  // 52 bytes

// BAD: Copies 52 bytes every call
void process_state_slow(QuadrotorState_t state) { ... }

// GOOD: Passes only 4 bytes (the pointer)
void process_state_fast(QuadrotorState_t *state) { ... }

// BETTER: const prevents accidental modification
void process_state_safe(const QuadrotorState_t *state) { ... }
\end{lstlisting}

\subsection{NULL Pointers}

A pointer that doesn't point to valid memory should be set to \texttt{NULL}:

\begin{lstlisting}[language=C, caption=NULL pointer safety]
int *p = NULL;  // p points to nothing

// Always check before dereferencing
if (p != NULL) {
    int value = *p;  // Safe
} else {
    // Handle error
}

// Dereferencing NULL is undefined behavior (usually crashes)
int bad = *p;  // CRASH or unpredictable behavior
\end{lstlisting}

%======================================================================
\section{Memory: Stack vs. Heap}
%======================================================================

In Python, you never think about where objects are stored---Python handles it automatically. In C, you must understand the two regions of memory where variables can live: the \textbf{stack} and the \textbf{heap}.

\subsection{Stack Memory}

The \textbf{stack} is used for:
\begin{itemize}
    \item Local variables inside functions
    \item Function parameters
    \item Return addresses
\end{itemize}

\begin{lstlisting}[language=C, caption=Stack allocation (automatic)]
void process_sensor_data(void) {
    // These variables are allocated on the stack
    int16_t raw_reading;           // 2 bytes on stack
    float calibrated_value;        // 4 bytes on stack
    float buffer[100];             // 400 bytes on stack

    // ... use the variables ...

}  // Variables automatically deallocated when function returns
\end{lstlisting}

\textbf{Stack characteristics}:
\begin{itemize}
    \item \textbf{Fast}: Allocation is just moving a pointer
    \item \textbf{Automatic}: Variables are deallocated when function returns
    \item \textbf{Limited size}: Typically 1-8 KB on embedded systems
    \item \textbf{LIFO}: Last allocated, first deallocated
\end{itemize}

\begin{warningbox}
Stack overflow is a serious risk in embedded systems. If you declare large arrays on the stack, you can overflow into other memory regions, causing mysterious crashes. The FreeRTOS task stacks are typically only 1-4 KB!
\end{warningbox}

\subsection{Heap Memory}

The \textbf{heap} is used for dynamically allocated memory that persists beyond a single function call.

\begin{lstlisting}[language=C, caption=Heap allocation with malloc and calloc]
#include <stdlib.h>

void example(void) {
    // malloc: allocate uninitialized memory
    // Returns pointer to allocated block, or NULL on failure
    int *data = (int*)malloc(100 * sizeof(int));  // 400 bytes

    if (data == NULL) {
        // Allocation failed - handle error!
        return;
    }

    // WARNING: malloc memory is UNINITIALIZED (contains garbage)
    data[0] = 42;  // Must initialize before reading

    // calloc: allocate AND initialize to zero
    float *buffer = (float*)calloc(100, sizeof(float));  // 400 bytes, all zeros

    // ... use the memory ...

    // MUST free when done, or memory is leaked
    free(data);
    free(buffer);

    // Good practice: set pointer to NULL after freeing
    data = NULL;
    buffer = NULL;
}
\end{lstlisting}

\begin{center}
\begin{tabular}{p{0.45\textwidth}|p{0.45\textwidth}}
\textbf{Python} & \textbf{C} \\
\hline
\begin{lstlisting}[language=Python, numbers=none]
# Python manages memory automatically
data = [0] * 100  # Allocate list

# Use the list...

# No need to free - garbage collector
# handles it when no references remain
\end{lstlisting}
&
\begin{lstlisting}[language=C, numbers=none]
// C requires manual memory management
int *data = calloc(100, sizeof(int));
if (data == NULL) { /* handle error */ }

// Use the array...

// MUST free or memory leaks forever
free(data);
\end{lstlisting}
\end{tabular}
\end{center}

\subsection{malloc vs. calloc}

\begin{center}
\begin{tabular}{lll}
\toprule
\textbf{Function} & \textbf{Syntax} & \textbf{Initialization} \\
\midrule
\texttt{malloc} & \texttt{malloc(size\_in\_bytes)} & Uninitialized (garbage) \\
\texttt{calloc} & \texttt{calloc(count, element\_size)} & Zero-initialized \\
\bottomrule
\end{tabular}
\end{center}

\begin{lstlisting}[language=C, caption=malloc vs calloc]
// These allocate the same amount of memory:
int *a = (int*)malloc(100 * sizeof(int));  // 400 bytes, garbage values
int *b = (int*)calloc(100, sizeof(int));   // 400 bytes, all zeros

// For sensor buffers, calloc is often safer:
float *sensor_history = (float*)calloc(HISTORY_SIZE, sizeof(float));
// All values start at 0.0, which is a sensible default
\end{lstlisting}

\subsection{Memory Leaks and Dangling Pointers}

\textbf{Memory leak}: Allocated memory that is never freed:

\begin{lstlisting}[language=C, caption=Memory leak example]
void memory_leak(void) {
    int *data = (int*)malloc(1000 * sizeof(int));
    // ... use data ...

    // Oops, forgot to free!
    // This memory is now lost until system reboot
}

void call_many_times(void) {
    for (int i = 0; i < 10000; i++) {
        memory_leak();  // Leaks 4KB each call -> 40MB leaked!
    }
}
\end{lstlisting}

\textbf{Dangling pointer}: Pointer to memory that has been freed:

\begin{lstlisting}[language=C, caption=Dangling pointer example]
int *create_data(void) {
    int *data = (int*)malloc(100 * sizeof(int));
    data[0] = 42;
    return data;  // OK: returning heap pointer
}

int *dangling_example(void) {
    int local_array[100];
    local_array[0] = 42;
    return local_array;  // DANGER: returning pointer to stack!
}  // local_array is deallocated here, pointer is now dangling

void double_free_example(void) {
    int *data = (int*)malloc(100 * sizeof(int));
    free(data);
    free(data);  // DANGER: double free - undefined behavior!
}
\end{lstlisting}

\subsection{Why Embedded Systems Often Avoid Heap}

Many embedded systems, including safety-critical flight controllers, avoid dynamic memory allocation:

\begin{itemize}
    \item \textbf{Fragmentation}: After many alloc/free cycles, heap becomes fragmented
    \item \textbf{Non-deterministic timing}: malloc may take variable time
    \item \textbf{Failure modes}: What if malloc returns NULL during flight?
    \item \textbf{Certification}: Safety standards often prohibit dynamic allocation
\end{itemize}

\begin{keyidea}[title=Static Allocation in Embedded Systems]
Embedded flight controllers typically allocate all memory statically at compile time. If the code compiles and links, you know it will fit in memory. FreeRTOS supports static allocation where you provide memory buffers for tasks, queues, and semaphores.
\end{keyidea}

%======================================================================
\section{Functions in C}
%======================================================================

\subsection{Function Declaration and Definition}

\begin{center}
\begin{tabular}{p{0.45\textwidth}|p{0.45\textwidth}}
\textbf{Python} & \textbf{C} \\
\hline
\begin{lstlisting}[language=Python, numbers=none]
def add(a, b):
    """Add two numbers."""
    return a + b

# Can call immediately
result = add(3, 4)
\end{lstlisting}
&
\begin{lstlisting}[language=C, numbers=none]
// Declaration (prototype) - optional but recommended
int add(int a, int b);

// Definition (implementation)
int add(int a, int b) {
    return a + b;
}

// Call
int result = add(3, 4);
\end{lstlisting}
\end{tabular}
\end{center}

In C, functions must be declared before use. The declaration (prototype) specifies:
\begin{itemize}
    \item Return type (\texttt{int}, \texttt{void}, \texttt{float}, etc.)
    \item Function name
    \item Parameter types
\end{itemize}

\subsection{Pass by Value (Default)}

C passes arguments \textbf{by value}: the function receives a copy of the argument.

\begin{lstlisting}[language=C, caption=Pass by value - function cannot modify original]
void try_to_modify(int x) {
    x = 100;  // Modifies local copy only
    printf("Inside function: %d\n", x);  // 100
}

int main(void) {
    int value = 42;
    try_to_modify(value);
    printf("After function: %d\n", value);  // Still 42!
    return 0;
}
\end{lstlisting}

\subsection{Pass by Pointer (Simulating Pass by Reference)}

To modify the caller's variable, pass a pointer:

\begin{center}
\begin{tabular}{p{0.45\textwidth}|p{0.45\textwidth}}
\textbf{Python} & \textbf{C} \\
\hline
\begin{lstlisting}[language=Python, numbers=none]
# Python passes object references
def modify_list(lst):
    lst.append(4)  # Modifies original

data = [1, 2, 3]
modify_list(data)
print(data)  # [1, 2, 3, 4]
\end{lstlisting}
&
\begin{lstlisting}[language=C, numbers=none]
// C needs explicit pointers
void modify_value(int *ptr) {
    *ptr = 100;  // Modifies original
}

int main(void) {
    int data = 42;
    modify_value(&data);  // Pass address
    printf("%d\n", data); // 100
}
\end{lstlisting}
\end{tabular}
\end{center}

\subsection{Returning Multiple Values}

Python can return tuples; C cannot. Use pointers for output parameters:

\begin{center}
\begin{tabular}{p{0.45\textwidth}|p{0.45\textwidth}}
\textbf{Python} & \textbf{C} \\
\hline
\begin{lstlisting}[language=Python, numbers=none]
def compute_stats(data):
    return min(data), max(data), sum(data)/len(data)

lo, hi, avg = compute_stats([1,2,3,4,5])
\end{lstlisting}
&
\begin{lstlisting}[language=C, numbers=none]
void compute_stats(int *data, int len,
                   int *min_out, int *max_out,
                   float *avg_out) {
    *min_out = data[0];
    *max_out = data[0];
    int sum = 0;
    for (int i = 0; i < len; i++) {
        if (data[i] < *min_out) *min_out = data[i];
        if (data[i] > *max_out) *max_out = data[i];
        sum += data[i];
    }
    *avg_out = (float)sum / len;
}

// Usage:
int arr[] = {1, 2, 3, 4, 5};
int lo, hi;
float avg;
compute_stats(arr, 5, &lo, &hi, &avg);
\end{lstlisting}
\end{tabular}
\end{center}

\subsection{Void Functions}

Functions that don't return a value use \texttt{void}:

\begin{lstlisting}[language=C, caption=Void functions]
void print_status(int motor_id, float speed) {
    printf("Motor %d: %.2f RPM\n", motor_id, speed);
    // No return statement needed (or use: return;)
}

// Functions with no parameters should use void explicitly
void initialize_system(void) {  // void means "no parameters"
    // Setup code...
}

// Note: empty parentheses means "unspecified parameters" in C (legacy)
void ambiguous();     // Avoid this
void clear(void);     // Prefer this
\end{lstlisting}

%======================================================================
\section{Arrays}
%======================================================================

\subsection{Stack-Allocated Arrays (Fixed Size)}

\begin{center}
\begin{tabular}{p{0.45\textwidth}|p{0.45\textwidth}}
\textbf{Python} & \textbf{C} \\
\hline
\begin{lstlisting}[language=Python, numbers=none]
# Lists are dynamic
data = [0.0] * 10  # 10 floats
data.append(1.0)   # Can grow
print(len(data))   # 11
\end{lstlisting}
&
\begin{lstlisting}[language=C, numbers=none]
// Arrays are fixed size
float data[10];  // 10 floats on stack
// data[10] = 1.0;  // OUT OF BOUNDS!
// No len() - you must track size yourself
#define DATA_SIZE 10
\end{lstlisting}
\end{tabular}
\end{center}

\begin{lstlisting}[language=C, caption=Stack arrays in embedded code]
#define FILTER_TAPS 32

void apply_filter(float *input, float *output, int length) {
    // Coefficients stored on stack (fixed at compile time)
    static const float coefficients[FILTER_TAPS] = {
        0.01f, 0.02f, 0.03f, /* ... */ 0.01f
    };

    // Temporary buffer on stack
    float temp[FILTER_TAPS];

    // Process data...
}
\end{lstlisting}

\begin{warningbox}
C arrays have \textbf{no bounds checking}. Accessing \texttt{data[10]} when the array has only 10 elements (indices 0-9) is undefined behavior---it may read/write random memory, corrupt data, or crash. Always track array sizes carefully!
\end{warningbox}

\subsection{Heap-Allocated Arrays (Dynamic Size)}

\begin{lstlisting}[language=C, caption=Heap-allocated arrays]
// Size determined at runtime
int num_sensors = get_sensor_count();

// Allocate array on heap
float *readings = (float*)calloc(num_sensors, sizeof(float));
if (readings == NULL) {
    // Handle allocation failure
    return ERROR_NO_MEMORY;
}

// Use like a regular array
for (int i = 0; i < num_sensors; i++) {
    readings[i] = read_sensor(i);
}

// Must free when done
free(readings);
readings = NULL;
\end{lstlisting}

\subsection{Arrays and Pointers}

In C, arrays and pointers are closely related. When you pass an array to a function, it ``decays'' to a pointer:

\begin{lstlisting}[language=C, caption=Array decay to pointer]
void process_data(float *data, int size) {
    // data is a pointer, not an array
    // sizeof(data) gives pointer size (4 or 8), NOT array size!
    for (int i = 0; i < size; i++) {
        data[i] *= 2.0f;
    }
}

int main(void) {
    float values[100];

    // These two calls are equivalent:
    process_data(values, 100);       // Array decays to pointer
    process_data(&values[0], 100);   // Explicitly pass address of first element

    // CRITICAL: Always pass the size separately!
    // The function has no way to know the array size
    return 0;
}
\end{lstlisting}

\subsection{Multi-Dimensional Arrays}

For sensor data matrices, IMU calibration, rotation matrices, etc.:

\begin{lstlisting}[language=C, caption=Multi-dimensional arrays]
// 3x3 rotation matrix (stack allocated)
float rotation[3][3] = {
    {1.0f, 0.0f, 0.0f},
    {0.0f, 1.0f, 0.0f},
    {0.0f, 0.0f, 1.0f}
};

// Access elements
float r00 = rotation[0][0];  // Row 0, Column 0

// Passing to functions - must specify all but first dimension
void apply_rotation(float R[3][3], float v[3], float result[3]) {
    for (int i = 0; i < 3; i++) {
        result[i] = 0.0f;
        for (int j = 0; j < 3; j++) {
            result[i] += R[i][j] * v[j];
        }
    }
}

// Alternative: use 1D array with manual indexing
float matrix[9];  // 3x3 stored as 1D
#define MAT(r, c) matrix[(r) * 3 + (c)]
MAT(1, 2) = 0.5f;  // Set row 1, column 2
\end{lstlisting}

%======================================================================
\section{Function Pointers}
%======================================================================

A \textbf{function pointer} stores the address of a function, allowing functions to be passed as arguments, stored in data structures, and called indirectly. This is essential for understanding how FreeRTOS creates tasks.

\subsection{Basic Syntax}

\begin{lstlisting}[language=C, caption=Function pointer declaration and use]
// A regular function
int add(int a, int b) {
    return a + b;
}

int subtract(int a, int b) {
    return a - b;
}

int main(void) {
    // Declare a function pointer
    // int (*operation)(int, int) means:
    // - operation is a pointer (*)
    // - to a function that takes two ints
    // - and returns an int
    int (*operation)(int, int);

    // Assign function address (& is optional for functions)
    operation = add;        // or: operation = &add;

    // Call through the pointer
    int result = operation(5, 3);  // result = 8

    // Change to different function
    operation = subtract;
    result = operation(5, 3);      // result = 2

    return 0;
}
\end{lstlisting}

\subsection{Function Pointers as Parameters}

\begin{center}
\begin{tabular}{p{0.45\textwidth}|p{0.45\textwidth}}
\textbf{Python} & \textbf{C} \\
\hline
\begin{lstlisting}[language=Python, numbers=none]
def apply_operation(func, a, b):
    return func(a, b)

def multiply(x, y):
    return x * y

result = apply_operation(multiply, 4, 5)
# result = 20
\end{lstlisting}
&
\begin{lstlisting}[language=C, numbers=none]
int apply_operation(int (*func)(int, int),
                    int a, int b) {
    return func(a, b);
}

int multiply(int x, int y) {
    return x * y;
}

int result = apply_operation(multiply, 4, 5);
// result = 20
\end{lstlisting}
\end{tabular}
\end{center}

\subsection{Typedef for Cleaner Syntax}

Function pointer syntax is notoriously confusing. Use \texttt{typedef} for clarity:

\begin{lstlisting}[language=C, caption=Using typedef with function pointers]
// Without typedef - hard to read
int (*operation)(int, int);
void register_callback(int (*callback)(int, int));

// With typedef - much clearer
typedef int (*BinaryOp)(int, int);

BinaryOp operation;
void register_callback(BinaryOp callback);
\end{lstlisting}

\subsection{FreeRTOS Task Creation}

FreeRTOS uses function pointers to specify what code each task should run:

\begin{lstlisting}[language=C, caption=FreeRTOS xTaskCreate uses function pointers]
// FreeRTOS task function signature
// Must return void and take void* parameter
typedef void (*TaskFunction_t)(void *pvParameters);

// Your task function must match this signature
void SensorTask(void *pvParameters) {
    // Task code - runs in infinite loop
    for (;;) {
        // Read sensors
        vTaskDelay(pdMS_TO_TICKS(10));
    }
}

void ControlTask(void *pvParameters) {
    for (;;) {
        // Compute control
        vTaskDelay(pdMS_TO_TICKS(5));
    }
}

int main(void) {
    // xTaskCreate takes a function pointer as first argument
    // This tells FreeRTOS which function to run for each task

    xTaskCreate(
        SensorTask,       // Function pointer - which code to run
        "Sensor",         // Task name (for debugging)
        256,              // Stack size (words)
        NULL,             // Parameter passed to task (pvParameters)
        2,                // Priority
        NULL              // Task handle (optional output)
    );

    xTaskCreate(
        ControlTask,      // Different function for this task
        "Control",
        256,
        NULL,
        3,                // Higher priority
        NULL
    );

    vTaskStartScheduler();
    return 0;
}
\end{lstlisting}

\begin{keyidea}[title=Why Function Pointers in FreeRTOS]
The RTOS scheduler needs to know \textit{where} each task's code is located in memory. By passing a function pointer to \texttt{xTaskCreate}, you tell the scheduler: ``When it's this task's turn to run, jump to this memory address and start executing.'' The scheduler saves and restores the CPU state and stack for each task, then uses the function pointer to resume execution.
\end{keyidea}

\subsection{Callback Pattern}

Function pointers enable the callback pattern, common in embedded systems:

\begin{lstlisting}[language=C, caption=Callback pattern for sensor data]
// Define callback type
typedef void (*SensorCallback)(float value, uint32_t timestamp);

// Module stores the callback
static SensorCallback user_callback = NULL;

// User registers their callback
void register_sensor_callback(SensorCallback cb) {
    user_callback = cb;
}

// Internal: called when new data arrives
void sensor_interrupt_handler(void) {
    float value = read_sensor_hardware();
    uint32_t time = get_timestamp();

    // Invoke user's callback if registered
    if (user_callback != NULL) {
        user_callback(value, time);
    }
}

// User code
void my_handler(float value, uint32_t timestamp) {
    printf("Sensor: %.2f at time %lu\n", value, timestamp);
}

int main(void) {
    register_sensor_callback(my_handler);
    // Now my_handler will be called whenever sensor data arrives
}
\end{lstlisting}

%======================================================================
\section{Common Pitfalls for Python Programmers}
%======================================================================

\subsection{No Automatic Bounds Checking}

\begin{lstlisting}[language=C, caption=Buffer overflow - silent corruption]
char buffer[10];
buffer[15] = 'X';  // UNDEFINED BEHAVIOR - writes beyond array
// No exception, no error message - just corrupts memory
// May work in testing, crash in production

// Python equivalent would raise IndexError
\end{lstlisting}

\textbf{Solution}: Always validate indices, use constants for array sizes, consider using static analysis tools.

\subsection{Uninitialized Variables}

\begin{lstlisting}[language=C, caption=Uninitialized variables contain garbage]
int x;           // Contains garbage - whatever was in memory
if (x > 0) {     // Undefined behavior!
    // May or may not execute, unpredictably
}

float data[100]; // All 100 values are garbage
float sum = 0;
for (int i = 0; i < 100; i++) {
    sum += data[i];  // Adding garbage!
}

// SOLUTION: Always initialize
int y = 0;
float values[100] = {0};  // All zeros
\end{lstlisting}

\subsection{String Handling}

C strings are arrays of characters terminated by a null byte (\texttt{'\textbackslash 0'}):

\begin{center}
\begin{tabular}{p{0.45\textwidth}|p{0.45\textwidth}}
\textbf{Python} & \textbf{C} \\
\hline
\begin{lstlisting}[language=Python, numbers=none]
s = "Hello"
t = s + " World"  # Easy concatenation
print(len(s))     # 5
\end{lstlisting}
&
\begin{lstlisting}[language=C, numbers=none]
char s[] = "Hello";  // Actually 6 bytes: H e l l o \0
char t[20];
strcpy(t, s);        // Copy s to t
strcat(t, " World"); // Append - DANGER if t too small!
printf("%zu\n", strlen(s));  // 5 (not counting \0)
\end{lstlisting}
\end{tabular}
\end{center}

\begin{warningbox}
Buffer overflows from string operations are a major source of security vulnerabilities. Always use safe functions like \texttt{strncpy} and \texttt{snprintf} that take a size limit, and ensure your buffers are large enough.
\end{warningbox}

\subsection{Assignment in Conditionals}

\begin{lstlisting}[language=C, caption=Dangerous typo: = vs ==]
int status = get_status();

// WRONG - assigns 0 to status, then evaluates as false
if (status = 0) {
    // Never executed!
}

// CORRECT - compares status to 0
if (status == 0) {
    // Executed when status is zero
}

// Defensive style: put constant first (compiler warns on assignment)
if (0 == status) {
    // Can't accidentally assign to 0
}
\end{lstlisting}

\subsection{Integer Division Truncation}

\begin{lstlisting}[language=C, caption=Integer division surprise]
// Calculate percentage
int completed = 3;
int total = 4;

float percent = completed / total * 100;  // percent = 0.0 !
// 3/4 = 0 (integer division), then 0*100 = 0

float correct = (float)completed / total * 100;  // percent = 75.0
\end{lstlisting}

\subsection{Scope and Lifetime Confusion}

\begin{lstlisting}[language=C, caption=Returning pointer to local variable]
int* create_array(void) {
    int arr[10];           // Local array on stack
    arr[0] = 42;
    return arr;            // DANGER: returning pointer to stack memory
}  // arr is deallocated here!

int* p = create_array();   // p points to deallocated memory
printf("%d\n", p[0]);      // Undefined behavior!

// SOLUTION: Allocate on heap
int* create_array_safe(void) {
    int *arr = (int*)calloc(10, sizeof(int));
    if (arr != NULL) {
        arr[0] = 42;
    }
    return arr;  // Caller must free this!
}
\end{lstlisting}

\subsection{Summary: Key Differences from Python}

\begin{center}
\begin{tabular}{lll}
\toprule
\textbf{Aspect} & \textbf{Python} & \textbf{C} \\
\midrule
Typing & Dynamic & Static \\
Memory & Garbage collected & Manual (malloc/free) \\
Bounds checking & Yes (IndexError) & No (undefined behavior) \\
Integer overflow & Arbitrary precision & Wraps or undefined \\
Division & Returns float by default & Integer if both operands int \\
Strings & Unicode objects & Null-terminated char arrays \\
Error handling & Exceptions & Return codes, errno \\
\bottomrule
\end{tabular}
\end{center}

\begin{keyidea}[title=The C Philosophy]
C trusts the programmer. It doesn't check array bounds, doesn't initialize variables, doesn't manage memory---because all of these cost CPU cycles. For a flight controller running at 1000 Hz, every microsecond matters. C gives you complete control, but with that power comes responsibility for correctness.
\end{keyidea}


% Appendix: Embedded Number Representation
%======================================================================
% APPENDIX: Embedded Number Representation
%
% This appendix covers how numbers are represented in embedded processors
% that run quadrotor flight controllers.
%======================================================================

\chapter{Embedded Number Representation}

\section{Introduction: Why Number Representation Matters}

When we write mathematical equations for control systems, we implicitly assume real numbers with infinite precision. But computers represent numbers with finite bits, introducing several complications:

\begin{itemize}
    \item \textbf{Limited range}: Numbers can only be so large or small before overflow or underflow
    \item \textbf{Limited precision}: Most real numbers cannot be represented exactly
    \item \textbf{Special values}: Operations can produce infinity, negative zero, or ``not a number''
    \item \textbf{Performance tradeoffs}: Different representations have different computational costs
\end{itemize}

\textbf{Why this matters for quadrotors}: A control loop running at 500 Hz performs millions of arithmetic operations per hour. Subtle numerical issues can accumulate:
\begin{itemize}
    \item Integrator windup from accumulated rounding errors
    \item NaN propagation that disables the entire controller
    \item Unexpected behavior at extreme attitudes (gimbal lock at $\pm 90°$ pitch)
    \item Different results between simulation and embedded code
\end{itemize}

Understanding number representation helps you anticipate and prevent these issues.

\begin{notebox}[title=Context for This Material]
This appendix covers how numbers are represented in the embedded processors that run quadrotor flight controllers. The focus is on practical implications for control system implementation, not a comprehensive treatment of computer arithmetic.
\end{notebox}

\section{The Crazyflie Hardware Platform}

The Crazyflie 2.x uses two processors:

\textbf{Main processor (STM32F405)}:
\begin{itemize}
    \item ARM Cortex-M4 core at 168 MHz
    \item \textbf{Hardware floating-point unit (FPU)}: Single-precision (32-bit) IEEE 754
    \item 196 KB RAM, 1 MB Flash
    \item Runs FreeRTOS with control algorithms
\end{itemize}

\textbf{Radio processor (nRF51822)}:
\begin{itemize}
    \item ARM Cortex-M0 core at 16 MHz
    \item \textbf{No FPU}: All floating-point is emulated in software (slow!)
    \item 16 KB RAM, 256 KB Flash
    \item Handles wireless communication
\end{itemize}

\begin{keyidea}[title=Why This Matters]
The main processor can do floating-point math quickly (hardware FPU). The radio processor cannot. If you accidentally run floating-point code on the wrong processor, performance collapses.
\end{keyidea}

\section{Integer Representation}

\subsection{Unsigned Integers}

An $N$-bit unsigned integer represents values $0$ to $2^N - 1$:

\begin{center}
\begin{tabular}{lcc}
\toprule
\textbf{Type} & \textbf{Bits} & \textbf{Range} \\
\midrule
\texttt{uint8\_t} & 8 & 0 to 255 \\
\texttt{uint16\_t} & 16 & 0 to 65,535 \\
\texttt{uint32\_t} & 32 & 0 to 4,294,967,295 \\
\bottomrule
\end{tabular}
\end{center}

Binary representation of 11 in 8 bits:
\[
11 = 0 \cdot 2^7 + 0 \cdot 2^6 + 0 \cdot 2^5 + 0 \cdot 2^4 + 1 \cdot 2^3 + 0 \cdot 2^2 + 1 \cdot 2^1 + 1 \cdot 2^0
\]
\[
\texttt{00001011}_2 = 11_{10}
\]

\subsection{Signed Integers: Two's Complement}

Negative numbers use \textbf{two's complement} representation:

\begin{definition}[Two's Complement]
To negate a number: invert all bits and add 1.
\end{definition}

\begin{example}
Represent $-5$ in 8 bits:
\begin{enumerate}
    \item Start with $+5$: \texttt{00000101}
    \item Invert bits: \texttt{11111010}
    \item Add 1: \texttt{11111011}
\end{enumerate}
So $-5$ is represented as \texttt{11111011}.
\end{example}

\textbf{Why two's complement?}
\begin{itemize}
    \item Addition and subtraction use the same hardware circuit
    \item Only one representation of zero
    \item The most significant bit indicates sign (1 = negative)
\end{itemize}

Range of $N$-bit signed integer: $-2^{N-1}$ to $2^{N-1} - 1$

\begin{center}
\begin{tabular}{lcc}
\toprule
\textbf{Type} & \textbf{Bits} & \textbf{Range} \\
\midrule
\texttt{int8\_t} & 8 & $-128$ to $127$ \\
\texttt{int16\_t} & 16 & $-32,768$ to $32,767$ \\
\texttt{int32\_t} & 32 & $-2,147,483,648$ to $2,147,483,647$ \\
\bottomrule
\end{tabular}
\end{center}

\section{Floating-Point Representation (IEEE 754)}

Floating-point representation is the standard way to represent real numbers in computers. The IEEE 754 standard, established in 1985 and revised in 2008, ensures consistent behavior across different processors and programming languages.

\textbf{Intuition}: Floating-point is like scientific notation. Instead of writing $0.000000123$, we write $1.23 \times 10^{-7}$. The ``significand'' (1.23) provides precision, and the ``exponent'' ($-7$) provides range. Binary floating-point works the same way, using powers of 2 instead of powers of 10.

\subsection{Structure}

A floating-point number has three fields:

\begin{center}
\begin{tabular}{|c|c|c|}
\hline
Sign (1 bit) & Exponent ($E$ bits) & Fraction ($F$ bits) \\
\hline
\end{tabular}
\end{center}

\begin{center}
\begin{tabular}{lccc}
\toprule
\textbf{Precision} & \textbf{Total bits} & \textbf{Exponent} & \textbf{Fraction} \\
\midrule
Single (\texttt{float}) & 32 & 8 & 23 \\
Double (\texttt{double}) & 64 & 11 & 52 \\
\bottomrule
\end{tabular}
\end{center}

\subsection{Value Calculation}

For a normalized number:
\[
\text{Value} = (-1)^{\text{sign}} \times 2^{\text{exponent} - \text{bias}} \times (1 + \text{fraction})
\]

where:
\begin{itemize}
    \item Bias = 127 for single precision, 1023 for double
    \item Fraction is interpreted as $0.b_1b_2b_3\ldots$ in binary
    \item The leading 1 is implicit (not stored)
\end{itemize}

\begin{example}
Decode \texttt{0 10000010 10100000000000000000000}:
\begin{itemize}
    \item Sign = 0 (positive)
    \item Exponent = \texttt{10000010}$_2$ = 130, so actual exponent = $130 - 127 = 3$
    \item Fraction = \texttt{.101}$_2$ = $0.5 + 0.125 = 0.625$
    \item Value = $1 \times 2^3 \times (1 + 0.625) = 8 \times 1.625 = 13.0$
\end{itemize}
\end{example}

\subsection{Special Values}

\begin{center}
\begin{tabular}{lll}
\toprule
\textbf{Value} & \textbf{Exponent} & \textbf{Fraction} \\
\midrule
Zero ($\pm 0$) & All 0s & All 0s \\
Infinity ($\pm \infty$) & All 1s & All 0s \\
NaN (Not a Number) & All 1s & Non-zero \\
Denormalized & All 0s & Non-zero \\
\bottomrule
\end{tabular}
\end{center}

\textbf{Operations with special values}:
\begin{itemize}
    \item $x / 0 = \pm\infty$ (for $x \neq 0$)
    \item $0 / 0 = $ NaN
    \item $\infty - \infty = $ NaN
    \item Any operation with NaN $=$ NaN
\end{itemize}

\begin{warningbox}[title=NaN Propagation]
If a NaN appears anywhere in your computation, it propagates to all dependent outputs. In a control system, this means:
\begin{itemize}
    \item One bad sensor reading can produce NaN
    \item NaN propagates through the controller
    \item Motor commands become NaN
    \item Quadrotor crashes
\end{itemize}
Always validate sensor inputs and check for NaN in critical paths!
\end{warningbox}

\subsection{Precision Limitations}

\begin{notebox}[title=Not All Numbers Are Representable]
Many decimal numbers have no exact binary floating-point representation.

\textbf{Example}: $0.1_{10}$ in binary is $0.0001100110011\ldots$ (repeating). The stored value is approximately $0.100000001490116$.

For most control applications, this error is negligible. But be aware when:
\begin{itemize}
    \item Comparing floating-point numbers for equality
    \item Accumulating many small values
    \item Working near the limits of precision
\end{itemize}
\end{notebox}

\section{Fixed-Point Arithmetic}

\subsection{When to Use Fixed-Point}

Fixed-point is useful when:
\begin{itemize}
    \item No hardware FPU available
    \item Deterministic timing required
    \item Memory is very limited
\end{itemize}

\subsection{Q-Format Notation}

Q$m.n$ format: $m$ integer bits, $n$ fractional bits.

\begin{example}
Q3.4 format (7 bits total, signed):
\begin{itemize}
    \item Range: $-8$ to $7.9375$
    \item Resolution: $2^{-4} = 0.0625$
\end{itemize}

To convert $3.25$ to Q3.4:
\[
3.25 \times 2^4 = 52 = \texttt{0110100}_2
\]

To convert back:
\[
52 / 2^4 = 3.25
\]
\end{example}

\subsection{Fixed-Point Operations}

\textbf{Addition/Subtraction}: Same format, direct operation.

\textbf{Multiplication}: Result has $2n$ fractional bits; must shift right by $n$.

\begin{lstlisting}[language=C, caption=Fixed-point multiply in Q15]
int16_t fixed_mul_q15(int16_t a, int16_t b) {
    int32_t result = (int32_t)a * (int32_t)b;
    return (int16_t)(result >> 15);
}
\end{lstlisting}

\begin{keyidea}[title=Modern Recommendation]
For the Crazyflie (Cortex-M4 with FPU), use \texttt{float} for all control computations. Fixed-point is only needed on processors without FPU or for extremely timing-critical code.
\end{keyidea}

\section{Summary}

Understanding number representation helps write robust embedded code:

\begin{itemize}
    \item \textbf{Integers}: Two's complement for signed; watch for overflow
    \item \textbf{Floating-point}: IEEE 754; watch for NaN, infinity, precision limits
    \item \textbf{Fixed-point}: Use when no FPU; requires careful scaling
    \item \textbf{Practical advice}: Use \texttt{float} on Cortex-M4; validate inputs; check for NaN
\end{itemize}


% Appendix: ARM Architecture and Assembly Language
%======================================================================
% APPENDIX: ARM Architecture and Assembly Language
%
% Target audience: Students who have completed the C appendix and want
% to understand what happens "under the hood" on the STM32 processor
%======================================================================

\chapter{ARM Architecture and Assembly Language}
\label{app:arm-assembly}
\index{ARM architecture|(}
\index{assembly language|(}

\section{Introduction: Why Understand Assembly?}

You will rarely write assembly language directly. Modern compilers generate efficient machine code, and writing assembly by hand is error-prone and time-consuming. So why learn it?

\begin{itemize}
    \item \textbf{Debugging crashes}: When your quadrotor crashes (the software kind), the fault handler gives you register values and a program counter. Understanding assembly lets you trace back to the C code that caused the fault.

    \item \textbf{Understanding timing}: When you need to know exactly how long a piece of code takes, you need to count instructions and their cycle costs.

    \item \textbf{Optimizing critical loops}: The attitude control inner loop runs 1000 times per second. Understanding what the compiler generates helps you write C that compiles to efficient assembly.

    \item \textbf{Hardware interaction}: Startup code, interrupt handlers, and context switches often require assembly.
\end{itemize}

\begin{keyidea}[title=Reading vs. Writing Assembly]
The goal is to \textbf{read} assembly, not write it. When you see a crash dump or disassembly listing, you should be able to understand what the code is doing and trace it back to your C source.
\end{keyidea}

\subsection{The Compilation Pipeline}

Your C code goes through several transformations before running on the processor:

\begin{center}
\begin{tikzpicture}[node distance=2cm, auto,
    block/.style={rectangle, draw, minimum width=2.5cm, minimum height=1cm, align=center}]

    \node[block] (c) {C Source\\(\texttt{.c})};
    \node[block, right=of c] (asm) {Assembly\\(\texttt{.s})};
    \node[block, right=of asm] (obj) {Object\\(\texttt{.o})};
    \node[block, right=of obj] (elf) {Executable\\(\texttt{.elf})};

    \draw[->] (c) -- node[above] {\small compile} (asm);
    \draw[->] (asm) -- node[above] {\small assemble} (obj);
    \draw[->] (obj) -- node[above] {\small link} (elf);
\end{tikzpicture}
\end{center}

\begin{enumerate}
    \item \textbf{Compilation}: The compiler (\texttt{arm-none-eabi-gcc}) translates C to assembly
    \item \textbf{Assembly}: The assembler converts assembly to machine code (object file)
    \item \textbf{Linking}: The linker combines object files and resolves addresses
\end{enumerate}

You can examine the assembly output at any stage:
\begin{lstlisting}[language=bash]
# Generate assembly from C (human-readable)
arm-none-eabi-gcc -S -O2 controller.c -o controller.s

# Disassemble object file
arm-none-eabi-objdump -d controller.o

# Disassemble with source intermixed
arm-none-eabi-objdump -d -S firmware.elf > firmware.lst
\end{lstlisting}

%======================================================================
\section{The STM32F4 and ARM Cortex-M4 Architecture}
%======================================================================

The Crazyflie 2.x uses the \textbf{STM32F405RG}\index{STM32F405} microcontroller, which contains an ARM Cortex-M4\index{ARM Cortex-M4} processor core.

\subsection{STM32F405 Overview}

\begin{center}
\begin{tabular}{ll}
\toprule
\textbf{Feature} & \textbf{Specification} \\
\midrule
Core & ARM Cortex-M4F (with FPU) \\
Clock speed & Up to 168 MHz \\
Flash memory & 1 MB (program storage) \\
SRAM & 192 KB (runtime data) \\
FPU & Single-precision hardware floating point \\
Instruction set & Thumb-2 (16/32-bit mixed) \\
Pipeline & 3-stage (Fetch, Decode, Execute) \\
Interrupts & NVIC with 82 maskable channels \\
\bottomrule
\end{tabular}
\end{center}

\subsection{Memory Map}

The Cortex-M4 has a 32-bit address space (4 GB), but only portions are used:

\begin{center}
\begin{tikzpicture}[scale=0.8]
    % Memory regions
    \draw[fill=blue!20] (0,0) rectangle (4,1.5);
    \node at (2,0.75) {Flash (Code)};
    \node[right] at (4.2,0.75) {\texttt{0x0800\_0000}};

    \draw[fill=green!20] (0,2) rectangle (4,3.2);
    \node at (2,2.6) {SRAM (Data)};
    \node[right] at (4.2,2.6) {\texttt{0x2000\_0000}};

    \draw[fill=yellow!20] (0,3.7) rectangle (4,5.2);
    \node at (2,4.45) {Peripherals};
    \node[right] at (4.2,4.45) {\texttt{0x4000\_0000}};

    \draw[fill=red!20] (0,5.7) rectangle (4,6.7);
    \node at (2,6.2) {System (NVIC)};
    \node[right] at (4.2,6.2) {\texttt{0xE000\_0000}};

    % Size annotations
    \node[left] at (-0.2,0.75) {1 MB};
    \node[left] at (-0.2,2.6) {192 KB};
\end{tikzpicture}
\end{center}

\begin{itemize}
    \item \textbf{Flash} (\texttt{0x0800\_0000}): Your compiled code lives here. Read-only during execution.
    \item \textbf{SRAM} (\texttt{0x2000\_0000}): Variables, stack, and heap. Read-write.
    \item \textbf{Peripherals} (\texttt{0x4000\_0000}): Timer registers, GPIO, SPI, I2C, etc.
    \item \textbf{System} (\texttt{0xE000\_0000}): NVIC (interrupt controller), SysTick timer, debug registers.
\end{itemize}

\subsection{Cortex-M4 Features for Real-Time Control}

\textbf{Hardware Floating-Point Unit (FPU)}\index{FPU (Floating-Point Unit)}:
\begin{itemize}
    \item Single-precision (32-bit) operations in hardware
    \item Most operations complete in 1 cycle
    \item Division and square root take 14 cycles
    \item Without FPU, floating-point would require ~100 cycles per operation
\end{itemize}

\textbf{Nested Vectored Interrupt Controller (NVIC)}\index{NVIC (Nested Vectored Interrupt Controller)}:
\begin{itemize}
    \item Handles sensor data ready, timer events, communication
    \item 12-cycle interrupt latency (deterministic)
    \item Priority levels allow critical interrupts to preempt less critical ones
\end{itemize}

\textbf{SysTick Timer}:
\begin{itemize}
    \item 24-bit countdown timer
    \item Used by FreeRTOS for task scheduling
    \item Typically configured for 1 ms ticks
\end{itemize}

%======================================================================
\section{ARM Registers}
\index{registers!ARM}
%======================================================================

The Cortex-M4 has 16 general-purpose registers (R0--R15) plus special registers.

\subsection{General-Purpose Registers}

\begin{center}
\begin{tabular}{llp{7cm}}
\toprule
\textbf{Register} & \textbf{Alias} & \textbf{Purpose} \\
\midrule
R0--R3 & --- & Arguments and return value. R0 holds return value. Caller-saved (may be overwritten by called functions). \\
R4--R11 & --- & Local variables. Callee-saved (function must preserve these). \\
R12 & IP & Intra-procedure scratch register. \\
R13 & SP & Stack Pointer. Points to top of stack. \\
R14 & LR & Link Register. Holds return address after BL instruction. \\
R15 & PC & Program Counter. Address of next instruction. \\
\bottomrule
\end{tabular}
\end{center}

\begin{keyidea}[title=Caller-Saved vs. Callee-Saved]
\begin{itemize}
    \item \textbf{Caller-saved} (R0--R3, R12): If you need these values after a function call, save them yourself before calling.
    \item \textbf{Callee-saved} (R4--R11): Functions must restore these before returning. If a function uses R4, it must save and restore it.
\end{itemize}
This convention allows efficient function calls---only registers that are actually used need to be saved.
\end{keyidea}

\subsection{Special Registers}

\begin{center}
\begin{tabular}{lp{9cm}}
\toprule
\textbf{Register} & \textbf{Purpose} \\
\midrule
xPSR & Program Status Register. Contains condition flags (N, Z, C, V), interrupt status, and execution state. \\
PRIMASK & Interrupt mask. Setting bit 0 disables all interrupts (except NMI and HardFault). \\
BASEPRI & Base priority mask. Disables interrupts below a certain priority level. \\
CONTROL & Controls stack pointer selection and privilege level. \\
\bottomrule
\end{tabular}
\end{center}

\subsection{Condition Flags}

The condition flags in xPSR are set by comparison and arithmetic instructions:

\begin{center}
\begin{tabular}{clp{7cm}}
\toprule
\textbf{Flag} & \textbf{Name} & \textbf{Meaning} \\
\midrule
N & Negative & Result was negative (bit 31 = 1) \\
Z & Zero & Result was zero \\
C & Carry & Unsigned overflow occurred \\
V & Overflow & Signed overflow occurred \\
\bottomrule
\end{tabular}
\end{center}

\begin{example}[Flag Setting]
\begin{lstlisting}[language=arm]
    MOV   R0, #5
    MOV   R1, #3
    CMP   R0, R1      @ Computes R0 - R1, sets flags
                      @ Result = 2: N=0 (positive), Z=0 (not zero)

    SUBS  R2, R1, R0  @ R2 = R1 - R0 = 3 - 5 = -2
                      @ N=1 (negative), Z=0, C=0 (borrow)
\end{lstlisting}
\end{example}

\subsection{FPU Registers}

The Cortex-M4F has 32 single-precision floating-point registers:

\begin{center}
\begin{tabular}{llp{7cm}}
\toprule
\textbf{Registers} & \textbf{Convention} & \textbf{Purpose} \\
\midrule
S0--S15 & Caller-saved & Arguments, return value (S0), scratch \\
S16--S31 & Callee-saved & Must be preserved across function calls \\
FPSCR & --- & FPU Status and Control Register (flags, rounding mode) \\
\bottomrule
\end{tabular}
\end{center}

Floating-point arguments are passed in S0, S1, S2, ... and the return value is in S0.

%======================================================================
\section{Memory Model and Addressing}
%======================================================================

\subsection{Byte Ordering: Little-Endian}

ARM Cortex-M uses \textbf{little-endian} byte ordering: the least significant byte is stored at the lowest address.

\begin{center}
\begin{tabular}{c|c|c|c|c}
\multicolumn{5}{c}{32-bit value \texttt{0x12345678} at address \texttt{0x2000\_0000}} \\
\hline
Address & +0 & +1 & +2 & +3 \\
\hline
Content & \texttt{78} & \texttt{56} & \texttt{34} & \texttt{12} \\
\hline
\end{tabular}
\end{center}

\subsection{Alignment}

Memory accesses should be \textbf{aligned} to their natural boundaries:
\begin{itemize}
    \item 32-bit (word): Address divisible by 4
    \item 16-bit (halfword): Address divisible by 2
    \item 8-bit (byte): Any address
\end{itemize}

\begin{warningbox}[title=Alignment Faults]
Unaligned 32-bit access (e.g., loading a word from address \texttt{0x2000\_0001}) causes a \textbf{Usage Fault} on Cortex-M4. This is a common cause of crashes when casting pointers incorrectly:
\begin{lstlisting}[language=C]
uint8_t buffer[10];
uint32_t* ptr = (uint32_t*)&buffer[1];  // Unaligned!
uint32_t value = *ptr;  // CRASH: UsageFault
\end{lstlisting}
\end{warningbox}

\subsection{Load and Store Instructions}

ARM is a \textbf{load-store architecture}: arithmetic operates only on registers. Data must be loaded from memory into registers, processed, then stored back.

\begin{center}
\begin{tabular}{llp{6cm}}
\toprule
\textbf{Instruction} & \textbf{Example} & \textbf{Meaning} \\
\midrule
LDR & \texttt{LDR R0, [R1]} & Load word: R0 = mem[R1] \\
LDRH & \texttt{LDRH R0, [R1]} & Load halfword (16-bit, zero-extend) \\
LDRB & \texttt{LDRB R0, [R1]} & Load byte (8-bit, zero-extend) \\
LDRSB & \texttt{LDRSB R0, [R1]} & Load byte (sign-extend) \\
STR & \texttt{STR R0, [R1]} & Store word: mem[R1] = R0 \\
STRH & \texttt{STRH R0, [R1]} & Store halfword \\
STRB & \texttt{STRB R0, [R1]} & Store byte \\
\bottomrule
\end{tabular}
\end{center}

\subsection{Addressing Modes}

\begin{center}
\begin{tabular}{lll}
\toprule
\textbf{Mode} & \textbf{Syntax} & \textbf{Meaning} \\
\midrule
Register & \texttt{[R1]} & Address in R1 \\
Immediate offset & \texttt{[R1, \#8]} & Address = R1 + 8 \\
Register offset & \texttt{[R1, R2]} & Address = R1 + R2 \\
Scaled offset & \texttt{[R1, R2, LSL \#2]} & Address = R1 + (R2 << 2) \\
Pre-indexed & \texttt{[R1, \#8]!} & R1 += 8, then load from R1 \\
Post-indexed & \texttt{[R1], \#8} & Load from R1, then R1 += 8 \\
\bottomrule
\end{tabular}
\end{center}

\begin{example}[Accessing Array Elements]
\begin{center}
\begin{tabular}{p{0.45\textwidth}|p{0.45\textwidth}}
\textbf{C Code} & \textbf{Assembly} \\
\hline
\begin{lstlisting}[language=C, numbers=none]
int arr[10];
int x = arr[3];
// arr is at R0, x goes to R1
\end{lstlisting}
&
\begin{lstlisting}[language=arm, numbers=none]
@ arr in R0, index 3
@ offset = 3 * 4 = 12 bytes
LDR R1, [R0, #12]
\end{lstlisting}
\end{tabular}
\end{center}
\end{example}

\begin{example}[Accessing Struct Fields]
\begin{center}
\begin{tabular}{p{0.45\textwidth}|p{0.45\textwidth}}
\textbf{C Code} & \textbf{Assembly} \\
\hline
\begin{lstlisting}[language=C, numbers=none]
typedef struct {
    float x;  // offset 0
    float y;  // offset 4
    float z;  // offset 8
} Vector3;

Vector3* v;
float y_val = v->y;
\end{lstlisting}
&
\begin{lstlisting}[language=arm, numbers=none]
@ v (pointer) in R0
@ y is at offset 4
LDR R1, [R0, #4]
@ or for float:
VLDR S0, [R0, #4]
\end{lstlisting}
\end{tabular}
\end{center}
\end{example}

%======================================================================
\section{ARM Instruction Set Basics}
%======================================================================

The Cortex-M4 uses the \textbf{Thumb-2}\index{Thumb-2 instruction set} instruction set, which mixes 16-bit and 32-bit instructions for code density and performance.

\subsection{Data Movement}

\begin{center}
\begin{tabular}{llp{6cm}}
\toprule
\textbf{Instruction} & \textbf{Example} & \textbf{Operation} \\
\midrule
MOV & \texttt{MOV R0, R1} & R0 = R1 \\
MOV & \texttt{MOV R0, \#42} & R0 = 42 \\
MVN & \texttt{MVN R0, R1} & R0 = \textasciitilde R1 (bitwise NOT) \\
MOVW & \texttt{MOVW R0, \#0x1234} & R0[15:0] = 0x1234 (low halfword) \\
MOVT & \texttt{MOVT R0, \#0x5678} & R0[31:16] = 0x5678 (high halfword) \\
\bottomrule
\end{tabular}
\end{center}

\textbf{Loading large constants}: ARM immediate values are limited. To load a 32-bit constant:
\begin{lstlisting}[language=arm]
    @ Load 0x12345678 into R0
    MOVW  R0, #0x5678      @ R0 = 0x00005678
    MOVT  R0, #0x1234      @ R0 = 0x12345678

    @ Or load from literal pool
    LDR   R0, =0x12345678  @ Assembler places constant in memory
\end{lstlisting}

\subsection{Arithmetic Operations}

\begin{center}
\begin{tabular}{llp{5.5cm}}
\toprule
\textbf{Instruction} & \textbf{Example} & \textbf{Operation} \\
\midrule
ADD & \texttt{ADD R0, R1, R2} & R0 = R1 + R2 \\
ADD & \texttt{ADD R0, R1, \#5} & R0 = R1 + 5 \\
ADDS & \texttt{ADDS R0, R1, R2} & R0 = R1 + R2, update flags \\
ADC & \texttt{ADC R0, R1, R2} & R0 = R1 + R2 + Carry \\
SUB & \texttt{SUB R0, R1, R2} & R0 = R1 - R2 \\
RSB & \texttt{RSB R0, R1, \#0} & R0 = 0 - R1 (reverse subtract) \\
MUL & \texttt{MUL R0, R1, R2} & R0 = R1 * R2 (low 32 bits) \\
MLA & \texttt{MLA R0, R1, R2, R3} & R0 = R1 * R2 + R3 \\
SDIV & \texttt{SDIV R0, R1, R2} & R0 = R1 / R2 (signed) \\
UDIV & \texttt{UDIV R0, R1, R2} & R0 = R1 / R2 (unsigned) \\
\bottomrule
\end{tabular}
\end{center}

\begin{notebox}[title=The S Suffix]
Adding \texttt{S} to an instruction (e.g., \texttt{ADDS}, \texttt{SUBS}) causes it to update the condition flags. Without \texttt{S}, flags are unchanged. This allows you to do arithmetic without affecting a pending comparison.
\end{notebox}

\subsection{Logical and Shift Operations}

\begin{center}
\begin{tabular}{llp{5.5cm}}
\toprule
\textbf{Instruction} & \textbf{Example} & \textbf{Operation} \\
\midrule
AND & \texttt{AND R0, R1, R2} & R0 = R1 \& R2 \\
ORR & \texttt{ORR R0, R1, R2} & R0 = R1 | R2 \\
EOR & \texttt{EOR R0, R1, R2} & R0 = R1 \^{} R2 (XOR) \\
BIC & \texttt{BIC R0, R1, R2} & R0 = R1 \& \textasciitilde R2 (bit clear) \\
LSL & \texttt{LSL R0, R1, \#2} & R0 = R1 << 2 \\
LSR & \texttt{LSR R0, R1, \#2} & R0 = R1 >> 2 (logical, zero-fill) \\
ASR & \texttt{ASR R0, R1, \#2} & R0 = R1 >> 2 (arithmetic, sign-extend) \\
ROR & \texttt{ROR R0, R1, \#4} & R0 = rotate R1 right by 4 \\
\bottomrule
\end{tabular}
\end{center}

\textbf{Flexible second operand}: Many instructions allow a shifted register as the second operand:
\begin{lstlisting}[language=arm]
    ADD R0, R1, R2, LSL #2    @ R0 = R1 + (R2 << 2) = R1 + R2*4
    SUB R0, R1, R2, ASR #1    @ R0 = R1 - (R2 >> 1) = R1 - R2/2
\end{lstlisting}

\subsection{Comparison and Branching}

\begin{center}
\begin{tabular}{llp{5.5cm}}
\toprule
\textbf{Instruction} & \textbf{Example} & \textbf{Operation} \\
\midrule
CMP & \texttt{CMP R0, R1} & Set flags based on R0 - R1 \\
CMN & \texttt{CMN R0, R1} & Set flags based on R0 + R1 \\
TST & \texttt{TST R0, R1} & Set flags based on R0 \& R1 \\
TEQ & \texttt{TEQ R0, R1} & Set flags based on R0 \^{} R1 \\
\bottomrule
\end{tabular}
\end{center}

\begin{center}
\begin{tabular}{llp{5.5cm}}
\toprule
\textbf{Instruction} & \textbf{Example} & \textbf{Condition} \\
\midrule
B & \texttt{B label} & Always branch \\
BEQ & \texttt{BEQ label} & Branch if Z=1 (equal) \\
BNE & \texttt{BNE label} & Branch if Z=0 (not equal) \\
BLT & \texttt{BLT label} & Branch if N!=V (signed less than) \\
BLE & \texttt{BLE label} & Branch if Z=1 or N!=V (signed $\leq$) \\
BGT & \texttt{BGT label} & Branch if Z=0 and N=V (signed $>$) \\
BGE & \texttt{BGE label} & Branch if N=V (signed $\geq$) \\
BLO & \texttt{BLO label} & Branch if C=0 (unsigned $<$) \\
BHI & \texttt{BHI label} & Branch if C=1 and Z=0 (unsigned $>$) \\
BL & \texttt{BL function} & Branch with Link (function call) \\
BX & \texttt{BX R14} & Branch to address in register \\
BLX & \texttt{BLX R0} & Branch with Link to address in register \\
\bottomrule
\end{tabular}
\end{center}

\subsection{Floating-Point Instructions}

The Cortex-M4F FPU provides single-precision floating-point operations:

\begin{center}
\begin{tabular}{llp{5cm}}
\toprule
\textbf{Instruction} & \textbf{Example} & \textbf{Operation} \\
\midrule
VLDR & \texttt{VLDR S0, [R0]} & Load float: S0 = mem[R0] \\
VSTR & \texttt{VSTR S0, [R0]} & Store float: mem[R0] = S0 \\
VMOV & \texttt{VMOV S0, R1} & Move integer to float register \\
VMOV & \texttt{VMOV R0, S1} & Move float register to integer \\
VADD.F32 & \texttt{VADD.F32 S0, S1, S2} & S0 = S1 + S2 \\
VSUB.F32 & \texttt{VSUB.F32 S0, S1, S2} & S0 = S1 - S2 \\
VMUL.F32 & \texttt{VMUL.F32 S0, S1, S2} & S0 = S1 * S2 \\
VDIV.F32 & \texttt{VDIV.F32 S0, S1, S2} & S0 = S1 / S2 \\
VNEG.F32 & \texttt{VNEG.F32 S0, S1} & S0 = -S1 \\
VABS.F32 & \texttt{VABS.F32 S0, S1} & S0 = |S1| \\
VSQRT.F32 & \texttt{VSQRT.F32 S0, S1} & S0 = $\sqrt{\text{S1}}$ \\
VMLA.F32 & \texttt{VMLA.F32 S0, S1, S2} & S0 = S0 + S1 * S2 \\
VMLS.F32 & \texttt{VMLS.F32 S0, S1, S2} & S0 = S0 - S1 * S2 \\
VCMP.F32 & \texttt{VCMP.F32 S0, S1} & Compare S0 with S1 \\
VCVT & \texttt{VCVT.F32.S32 S0, S0} & Convert int to float \\
\bottomrule
\end{tabular}
\end{center}

\begin{notebox}[title=VMLA: Multiply-Accumulate]
The \texttt{VMLA.F32} instruction computes $S0 = S0 + S1 \times S2$ in a single cycle. This is heavily used in control loops and matrix operations. The compiler will use it automatically for expressions like \texttt{sum += a * b}.
\end{notebox}

%======================================================================
\section{C to Assembly: Side-by-Side Examples}
%======================================================================

This section shows C code alongside the assembly the compiler generates, building from simple to complex.

\subsection{Simple Variable Assignment}

\begin{center}
\begin{tabular}{p{0.45\textwidth}|p{0.45\textwidth}}
\textbf{C Code} & \textbf{Assembly} \\
\hline
\begin{lstlisting}[language=C, numbers=none]
int x = 42;
int y = x + 10;
int z = y * 3;
\end{lstlisting}
&
\begin{lstlisting}[language=arm, numbers=none]
    MOV   R0, #42       @ x = 42
    ADD   R1, R0, #10   @ y = x + 10 = 52
    ADD   R2, R1, R1, LSL #1
                        @ z = y + y*2 = y*3
\end{lstlisting}
\end{tabular}
\end{center}

\textbf{Note}: The compiler optimized \texttt{y * 3} into \texttt{y + y*2} using a shifted add, avoiding the slower MUL instruction.

\subsection{If-Else Statement}

\begin{center}
\begin{tabular}{p{0.45\textwidth}|p{0.45\textwidth}}
\textbf{C Code} & \textbf{Assembly} \\
\hline
\begin{lstlisting}[language=C, numbers=none]
int abs_val(int x) {
    if (x < 0)
        return -x;
    else
        return x;
}
\end{lstlisting}
&
\begin{lstlisting}[language=arm, numbers=none]
abs_val:
    CMP   R0, #0      @ Compare x with 0
    BGE   .L_pos      @ If x >= 0, skip
    RSB   R0, R0, #0  @ R0 = 0 - R0
.L_pos:
    BX    LR          @ Return R0
\end{lstlisting}
\end{tabular}
\end{center}

\textbf{Instruction breakdown}:
\begin{itemize}
    \item \texttt{CMP R0, \#0}: Computes $R0 - 0$ and sets flags (doesn't store result)
    \item \texttt{BGE}: Branch if Greater or Equal (signed), checks N=V
    \item \texttt{RSB}: Reverse Subtract, computes $0 - R0$ (negation)
    \item \texttt{BX LR}: Branch to address in Link Register (return)
\end{itemize}

\subsection{For Loop with Array}

\begin{center}
\begin{tabular}{p{0.45\textwidth}|p{0.45\textwidth}}
\textbf{C Code} & \textbf{Assembly} \\
\hline
\begin{lstlisting}[language=C, numbers=none]
int sum_array(int* arr, int n) {
    int sum = 0;
    for (int i = 0; i < n; i++) {
        sum += arr[i];
    }
    return sum;
}
\end{lstlisting}
&
\begin{lstlisting}[language=arm, numbers=none]
sum_array:
    @ R0 = arr, R1 = n
    MOV   R2, #0      @ sum = 0
    MOV   R3, #0      @ i = 0
.loop:
    CMP   R3, R1      @ i < n?
    BGE   .done       @ Exit if i >= n
    LDR   R12, [R0, R3, LSL #2]
                      @ R12 = arr[i]
    ADD   R2, R2, R12 @ sum += arr[i]
    ADD   R3, R3, #1  @ i++
    B     .loop
.done:
    MOV   R0, R2      @ Return sum
    BX    LR
\end{lstlisting}
\end{tabular}
\end{center}

\textbf{Key instruction}: \texttt{LDR R12, [R0, R3, LSL \#2]}
\begin{itemize}
    \item Base address: R0 (arr)
    \item Offset: R3 << 2 = i * 4 (each int is 4 bytes)
    \item Loads \texttt{arr[i]} into R12
\end{itemize}

\subsection{Floating-Point Computation}

\begin{center}
\begin{tabular}{p{0.45\textwidth}|p{0.45\textwidth}}
\textbf{C Code} & \textbf{Assembly} \\
\hline
\begin{lstlisting}[language=C, numbers=none]
float compute_torque(
    float error,
    float rate,
    float Kp,
    float Kd)
{
    return Kp * error + Kd * rate;
}
\end{lstlisting}
&
\begin{lstlisting}[language=arm, numbers=none]
compute_torque:
    @ S0=error, S1=rate
    @ S2=Kp, S3=Kd
    VMUL.F32  S0, S2, S0
              @ S0 = Kp * error
    VMLA.F32  S0, S3, S1
              @ S0 += Kd * rate
    BX        LR
              @ Return S0
\end{lstlisting}
\end{tabular}
\end{center}

\textbf{Efficiency}: Only 3 instructions! The \texttt{VMLA} (multiply-accumulate) combines multiply and add.

\subsection{Struct Access}

\begin{center}
\begin{tabular}{p{0.45\textwidth}|p{0.45\textwidth}}
\textbf{C Code} & \textbf{Assembly} \\
\hline
\begin{lstlisting}[language=C, numbers=none]
typedef struct {
    float x;  // offset 0
    float y;  // offset 4
    float z;  // offset 8
} Vector3;

float dot(Vector3* a, Vector3* b) {
    return a->x * b->x
         + a->y * b->y
         + a->z * b->z;
}
\end{lstlisting}
&
\begin{lstlisting}[language=arm, numbers=none]
dot:
    @ R0 = a, R1 = b
    VLDR  S0, [R0, #0]   @ a->x
    VLDR  S1, [R1, #0]   @ b->x
    VMUL.F32 S0, S0, S1  @ x*x

    VLDR  S2, [R0, #4]   @ a->y
    VLDR  S3, [R1, #4]   @ b->y
    VMLA.F32 S0, S2, S3  @ += y*y

    VLDR  S2, [R0, #8]   @ a->z
    VLDR  S3, [R1, #8]   @ b->z
    VMLA.F32 S0, S2, S3  @ += z*z

    BX    LR             @ Return S0
\end{lstlisting}
\end{tabular}
\end{center}

\subsection{Function Call}

\begin{center}
\begin{tabular}{p{0.45\textwidth}|p{0.45\textwidth}}
\textbf{C Code} & \textbf{Assembly} \\
\hline
\begin{lstlisting}[language=C, numbers=none]
int square(int x) {
    return x * x;
}

int sum_of_squares(int a, int b) {
    return square(a) + square(b);
}
\end{lstlisting}
&
\begin{lstlisting}[language=arm, numbers=none]
square:
    MUL   R0, R0, R0
    BX    LR

sum_of_squares:
    PUSH  {R4, R5, LR}
    MOV   R4, R0      @ Save a
    MOV   R5, R1      @ Save b

    @ Call square(a)
    BL    square      @ R0 = square(a)
    MOV   R4, R0      @ Save result

    @ Call square(b)
    MOV   R0, R5      @ Arg = b
    BL    square      @ R0 = square(b)

    ADD   R0, R4, R0  @ Return a^2 + b^2
    POP   {R4, R5, PC}
\end{lstlisting}
\end{tabular}
\end{center}

\textbf{Key points}:
\begin{itemize}
    \item \texttt{PUSH \{R4, R5, LR\}}: Save callee-saved registers and return address
    \item \texttt{BL square}: Branch with Link---saves return address in LR, jumps to square
    \item \texttt{POP \{R4, R5, PC\}}: Restore registers; loading into PC returns from function
\end{itemize}

%======================================================================
\section{The Stack and Function Calls}
%======================================================================

\subsection{Stack Basics}

The stack is a region of memory used for:
\begin{itemize}
    \item Saving register values across function calls
    \item Storing local variables that don't fit in registers
    \item Passing arguments (when more than 4)
\end{itemize}

\textbf{Stack properties}:
\begin{itemize}
    \item Grows \textbf{downward} (toward lower addresses)
    \item SP (R13) points to the last item pushed
    \item Must remain 8-byte aligned at function calls (AAPCS requirement)
\end{itemize}

\begin{center}
\begin{tikzpicture}[scale=0.9]
    % Stack frame
    \draw (0,0) rectangle (4,5);

    % Divisions
    \draw (0,4) -- (4,4);
    \draw (0,3) -- (4,3);
    \draw (0,2) -- (4,2);
    \draw (0,1) -- (4,1);

    % Labels
    \node at (2,4.5) {Previous frame};
    \node at (2,3.5) {Saved LR};
    \node at (2,2.5) {Saved R4--R7};
    \node at (2,1.5) {Local variables};
    \node at (2,0.5) {(unused)};

    % SP pointer
    \draw[->] (5,1) -- (4.1,1);
    \node[right] at (5,1) {SP};

    % Addresses
    \node[left] at (0,4.5) {High};
    \node[left] at (0,0.5) {Low};

    % Growth arrow
    \draw[->] (5.5,4) -- (5.5,1);
    \node[right] at (5.5,2.5) {Growth};
\end{tikzpicture}
\end{center}

\subsection{PUSH and POP}

\begin{lstlisting}[language=arm]
    @ Save registers to stack
    PUSH  {R4, R5, R6, LR}    @ SP -= 16, store R4,R5,R6,LR

    @ ... function body ...

    @ Restore registers from stack
    POP   {R4, R5, R6, PC}    @ Load R4,R5,R6,PC; SP += 16
                               @ Loading PC = return
\end{lstlisting}

\textbf{PUSH} decrements SP and stores registers (lowest-numbered register at lowest address).

\textbf{POP} loads registers and increments SP. Popping into PC causes a return.

\subsection{ARM Calling Convention (AAPCS)}

The ARM Architecture Procedure Call Standard defines how functions communicate:

\textbf{Arguments}:
\begin{itemize}
    \item First 4 integer/pointer arguments: R0, R1, R2, R3
    \item Additional arguments: pushed on stack (right to left)
    \item Float arguments: S0, S1, S2, ... (up to S15)
\end{itemize}

\textbf{Return values}:
\begin{itemize}
    \item Integer/pointer: R0 (64-bit uses R0:R1)
    \item Float: S0
\end{itemize}

\textbf{Register preservation}:
\begin{itemize}
    \item Caller-saved (scratch): R0--R3, R12, S0--S15
    \item Callee-saved (preserved): R4--R11, S16--S31, SP
    \item LR is overwritten by BL, so callee must save if it calls other functions
\end{itemize}

\subsection{Stack Frame Example}

\begin{center}
\begin{tabular}{p{0.42\textwidth}|p{0.50\textwidth}}
\textbf{C Code} & \textbf{Assembly} \\
\hline
\begin{lstlisting}[language=C, numbers=none]
int compute(int a, int b, int c) {
    int temp1 = a + b;
    int temp2 = b + c;
    int result = temp1 * temp2;
    return result;
}
\end{lstlisting}
&
\begin{lstlisting}[language=arm, numbers=none]
compute:
    @ Args: R0=a, R1=b, R2=c
    @ No need to save regs (no calls)

    ADD   R3, R0, R1   @ temp1 = a + b
    ADD   R0, R1, R2   @ temp2 = b + c
    MUL   R0, R3, R0   @ result = temp1*temp2
    BX    LR           @ Return result
\end{lstlisting}
\end{tabular}
\end{center}

No stack frame needed! The compiler kept everything in registers.

\subsection{Function with Local Array}

\begin{center}
\begin{tabular}{p{0.45\textwidth}|p{0.47\textwidth}}
\textbf{C Code} & \textbf{Assembly} \\
\hline
\begin{lstlisting}[language=C, numbers=none]
int process(int n) {
    int buffer[4];
    for (int i = 0; i < 4; i++) {
        buffer[i] = n + i;
    }
    return buffer[0] + buffer[3];
}
\end{lstlisting}
&
\begin{lstlisting}[language=arm, numbers=none]
process:
    SUB   SP, SP, #16  @ Allocate 16 bytes
    @ buffer at [SP+0..SP+12]

    MOV   R1, #0       @ i = 0
.loop:
    ADD   R2, R0, R1   @ n + i
    STR   R2, [SP, R1, LSL #2]
                       @ buffer[i] = n+i
    ADD   R1, R1, #1
    CMP   R1, #4
    BLT   .loop

    LDR   R0, [SP, #0]  @ buffer[0]
    LDR   R1, [SP, #12] @ buffer[3]
    ADD   R0, R0, R1

    ADD   SP, SP, #16  @ Deallocate
    BX    LR
\end{lstlisting}
\end{tabular}
\end{center}

\subsection{Heap Allocation}

\begin{center}
\begin{tabular}{p{0.45\textwidth}|p{0.47\textwidth}}
\textbf{C Code} & \textbf{Assembly} \\
\hline
\begin{lstlisting}[language=C, numbers=none]
#include <stdlib.h>

int* create_array(int size) {
    int* arr = malloc(size * sizeof(int));
    if (arr != NULL) {
        arr[0] = 42;
    }
    return arr;
}
\end{lstlisting}
&
\begin{lstlisting}[language=arm, numbers=none]
create_array:
    PUSH  {R4, LR}
    LSL   R0, R0, #2   @ size * 4
    BL    malloc       @ Call malloc
    MOV   R4, R0       @ Save pointer

    CMP   R0, #0       @ NULL check
    BEQ   .done
    MOV   R1, #42
    STR   R1, [R0, #0] @ arr[0] = 42

.done:
    MOV   R0, R4       @ Return pointer
    POP   {R4, PC}
\end{lstlisting}
\end{tabular}
\end{center}

\textbf{Heap vs. Stack}:
\begin{itemize}
    \item Stack: Automatic allocation, fixed size, fast
    \item Heap: Dynamic allocation via \texttt{malloc}, variable size, slower
    \item Stack variables deallocated when function returns
    \item Heap variables persist until \texttt{free()} is called
\end{itemize}

%======================================================================
\section{Debugging with Assembly}
%======================================================================

\subsection{Reading a Crash Dump}

When a hard fault occurs, the fault handler can print register values:

\begin{lstlisting}[language=bash]
*** HARD FAULT ***
R0 =0x00000000  R1 =0x20001A3C  R2 =0x00000005  R3 =0xDEADBEEF
R4 =0x200019F0  R5 =0x08004521  R6 =0x00000000  R7 =0x20001A40
R8 =0x00000000  R9 =0x00000000  R10=0x00000000  R11=0x00000000
R12=0x08002345  SP =0x20001A20  LR =0x08003457  PC =0x08002348
xPSR=0x21000000  CFSR=0x00008200
\end{lstlisting}

\textbf{Key values to examine}:
\begin{itemize}
    \item \textbf{PC}: Address where fault occurred---look up in disassembly
    \item \textbf{LR}: Return address---where did we come from?
    \item \textbf{SP}: Stack pointer---is it valid (in SRAM range)?
    \item \textbf{CFSR}: Configurable Fault Status Register---why did we fault?
\end{itemize}

\subsection{Common Fault Patterns}

\begin{center}
\begin{tabular}{lp{8cm}}
\toprule
\textbf{Symptom} & \textbf{Likely Cause} \\
\midrule
PC = 0x00000000 & Called through NULL function pointer \\
PC in 0x20xxxxxx & Jumped into data memory (stack corruption) \\
SP outside SRAM & Stack overflow \\
LR = 0xFFFFFFF9 & Returning from exception (normal) \\
CFSR bit 15 (INVSTATE) & BX to invalid address (bit 0 must be 1 for Thumb) \\
CFSR bit 25 (UNALIGNED) & Unaligned memory access \\
\bottomrule
\end{tabular}
\end{center}

\subsection{Using objdump}

Generate a listing with source code intermixed:

\begin{lstlisting}[language=bash]
arm-none-eabi-objdump -d -S -l firmware.elf > firmware.lst
\end{lstlisting}

\textbf{Flags}:
\begin{itemize}
    \item \texttt{-d}: Disassemble code sections
    \item \texttt{-S}: Intermix source code (requires -g during compilation)
    \item \texttt{-l}: Show line numbers
\end{itemize}

\begin{example}[Finding a Crash Location]
Given PC = \texttt{0x08002348}, search the listing:

\begin{lstlisting}
08002340 <attitude_control>:
attitude_control():
/src/controller.c:42
    float error = setpoint - current;
 8002340:  ee31 0a40   vsub.f32 s0, s2, s0
/src/controller.c:43
    float output = Kp * error;
 8002344:  ee20 0a80   vmul.f32 s0, s0, s1
 8002348:  ed80 0a00   vstr     s0, [r0]    <-- CRASH HERE
\end{lstlisting}

The crash is at a \texttt{vstr} (store float). R0 = 0 means we're storing to address 0---a NULL pointer dereference on line 43 of controller.c.
\end{example}

\subsection{GDB Commands for Assembly}

\begin{lstlisting}[language=bash]
# Connect to target
arm-none-eabi-gdb firmware.elf
(gdb) target remote :3333

# Disassemble current function
(gdb) disassemble

# Show next 10 instructions
(gdb) x/10i $pc

# Show all registers
(gdb) info registers
(gdb) info registers s0 s1 s2    # FPU registers

# Single-step one instruction
(gdb) stepi

# Step over function call
(gdb) nexti

# Set breakpoint at address
(gdb) break *0x08002348

# Examine memory
(gdb) x/4xw 0x20001000   # 4 words in hex
(gdb) x/s 0x20001000     # as string
\end{lstlisting}

%======================================================================
\section{Performance Considerations}
%======================================================================

\subsection{Instruction Cycle Counts}

Most Cortex-M4 instructions execute in 1 cycle, but some take longer:

\begin{center}
\begin{tabular}{llp{5cm}}
\toprule
\textbf{Operation} & \textbf{Cycles} & \textbf{Notes} \\
\midrule
MOV, ADD, SUB, AND, ORR & 1 & Register operations \\
MUL & 1 & 32-bit multiply \\
SDIV, UDIV & 2--12 & Division (operand-dependent) \\
LDR (from SRAM) & 1--2 & May stall on bus contention \\
LDR (from Flash) & 1--5 & Flash wait states (depends on speed) \\
VADD.F32, VMUL.F32 & 1 & FPU operations \\
VMLA.F32 & 1 & Multiply-accumulate \\
VDIV.F32 & 14 & Floating-point division \\
VSQRT.F32 & 14 & Square root \\
B (taken) & 1--3 & Pipeline flush \\
BL (function call) & 1--3 & Plus function prologue overhead \\
\bottomrule
\end{tabular}
\end{center}

\begin{keyidea}[title=Avoid Division and Square Root]
Division and square root are 14$\times$ slower than multiplication. In inner loops:
\begin{itemize}
    \item Replace \texttt{x / constant} with \texttt{x * (1/constant)}
    \item Precompute reciprocals outside the loop
    \item Use \texttt{rsqrt} approximations if precision allows
\end{itemize}
\end{keyidea}

\subsection{Pipeline Effects}

The Cortex-M4 has a 3-stage pipeline (Fetch, Decode, Execute). Branch instructions can cause a \textbf{pipeline flush}---the fetched/decoded instructions are discarded.

\textbf{Branch prediction}: Cortex-M4 has limited branch prediction. Conditional branches that are usually taken have lower cost than unpredictable branches.

\textbf{Loop optimization}: Small loops that fit in the prefetch buffer execute efficiently.

\subsection{Memory Access Patterns}

\begin{itemize}
    \item \textbf{Sequential access}: Fast---prefetch works well
    \item \textbf{Strided access}: Slower---prefetch less effective
    \item \textbf{Random access}: Slowest---no prefetch benefit
\end{itemize}

For array processing, access elements sequentially when possible:
\begin{lstlisting}[language=C]
// Good: Sequential access
for (int i = 0; i < N; i++) {
    sum += arr[i];
}

// Bad: Strided access (if avoidable)
for (int i = 0; i < N; i += 4) {
    sum += arr[i];
}
\end{lstlisting}

\subsection{Compiler Optimization}

The compiler applies many optimizations automatically:

\begin{itemize}
    \item \textbf{Strength reduction}: \texttt{x * 4} becomes \texttt{x << 2}
    \item \textbf{Common subexpression elimination}: Compute shared values once
    \item \textbf{Loop unrolling}: Reduce branch overhead by doing multiple iterations per loop cycle
    \item \textbf{Inlining}: Replace function call with function body
    \item \textbf{Register allocation}: Keep frequently-used variables in registers
\end{itemize}

\textbf{Optimization flags}:
\begin{itemize}
    \item \texttt{-O0}: No optimization (for debugging)
    \item \texttt{-O2}: Standard optimization (recommended)
    \item \texttt{-O3}: Aggressive optimization (may increase code size)
    \item \texttt{-Os}: Optimize for size
\end{itemize}

\begin{example}[Compiler-Optimized PID Loop]
\begin{lstlisting}[language=C]
float pid_update(PID* pid, float error, float dt) {
    pid->integral += error * dt;
    float derivative = (error - pid->prev_error) / dt;
    pid->prev_error = error;
    return pid->Kp * error + pid->Ki * pid->integral + pid->Kd * derivative;
}
\end{lstlisting}

With \texttt{-O2}, the compiler will:
\begin{enumerate}
    \item Keep \texttt{error} and \texttt{dt} in FPU registers
    \item Use VMLA for the multiply-accumulate operations
    \item Precompute \texttt{1/dt} if it detects \texttt{dt} is constant
    \item Potentially inline this function at call sites
\end{enumerate}
\end{example}

\section{Chapter Summary}

This appendix introduced ARM Cortex-M4 architecture and assembly language:

\begin{itemize}
    \item \textbf{Architecture}: The STM32F405 contains a Cortex-M4F core running at 168 MHz with hardware FPU, 1 MB Flash, and 192 KB SRAM.

    \item \textbf{Registers}: 16 general-purpose registers (R0--R15), with R13 (SP), R14 (LR), and R15 (PC) having special roles. The FPU adds 32 single-precision registers (S0--S31).

    \item \textbf{Instruction set}: ARM Thumb-2 provides efficient 16/32-bit mixed instructions. Key categories include data movement (MOV, LDR, STR), arithmetic (ADD, SUB, MUL), logic (AND, ORR), comparison (CMP), and branching (B, BL, BX).

    \item \textbf{Function calls}: The AAPCS defines argument passing (R0--R3), return values (R0), and register preservation conventions that enable interoperable code.

    \item \textbf{Debugging}: Understanding assembly is essential for interpreting crash dumps, using GDB effectively, and tracing bugs to source code.

    \item \textbf{Performance}: Most instructions take 1 cycle, but division (2--12 cycles) and FPU division/sqrt (14 cycles) are slow. The compiler handles most optimization automatically.
\end{itemize}

\begin{keyidea}[title=The Essential Skill]
You don't need to write assembly, but you do need to read it. When your quadrotor crashes, the CPU gives you a PC value. Your job is to trace that back through the disassembly to your C code and understand what went wrong.
\end{keyidea}
\index{ARM architecture|)}
\index{assembly language|)}


% Appendix: Simulink Code Generation
\chapter{Simulink Code Generation}
\label{app:simulink-codegen}
\index{Simulink!code generation|(}
\index{code generation!Simulink|(}

Model-based development promises that we can design, simulate, and verify control systems using graphical models, then automatically generate production code for embedded targets. This appendix explains how Simulink's code generation works and what the generated code looks like, helping you understand and integrate auto-generated code into your embedded projects.

\section{Introduction: From Model to Embedded Code}

Throughout this course, you have developed Simulink models for quadrotor control---attitude controllers, position estimators, and sensor fusion algorithms. These models run in MATLAB's simulation environment, where numerical integration happens on your laptop with essentially unlimited computational resources. But the ultimate goal is to run these controllers on the Crazyflie's STM32 microcontroller, executing in real-time with strict timing constraints.

\subsection{Why Code Generation?}

You could manually translate your Simulink model into C code, but this approach has significant drawbacks:

\begin{itemize}
    \item \textbf{Error-prone}: Manual translation introduces bugs, especially for complex algorithms
    \item \textbf{Time-consuming}: Re-implementing after every model change is tedious
    \item \textbf{Traceability lost}: No automatic link between model and code
    \item \textbf{Verification burden}: Must re-verify code matches model behavior
\end{itemize}

Automatic code generation addresses these issues by producing C code directly from your Simulink model. The generated code is:

\begin{itemize}
    \item \textbf{Correct by construction}: Mathematically equivalent to the model
    \item \textbf{Traceable}: Every line of code maps to a model element
    \item \textbf{Consistent}: Regenerated automatically when the model changes
    \item \textbf{Verifiable}: Same code used in simulation and on target
\end{itemize}

\begin{keyidea}
Code generation transforms your verified simulation model into deployable embedded software, preserving the design intent and enabling systematic verification through the development process.
\end{keyidea}

\subsection{The V-Model and Code Generation}

Code generation fits into the V-model development process that structures safety-critical system development:

\begin{center}
\begin{tikzpicture}[scale=0.9, every node/.style={font=\small}]
    % Left side - design
    \node[draw, rectangle, minimum width=3cm, minimum height=0.8cm] (req) at (0,4) {Requirements};
    \node[draw, rectangle, minimum width=3cm, minimum height=0.8cm] (arch) at (-1.5,3) {Architecture};
    \node[draw, rectangle, minimum width=3cm, minimum height=0.8cm] (design) at (-3,2) {Detailed Design};
    \node[draw, rectangle, minimum width=3cm, minimum height=0.8cm, fill=blue!10] (model) at (-4.5,1) {Simulink Model};
    \node[draw, rectangle, minimum width=3cm, minimum height=0.8cm, fill=green!10] (code) at (-4.5,0) {Generated Code};

    % Right side - verification
    \node[draw, rectangle, minimum width=3cm, minimum height=0.8cm] (systest) at (1.5,3) {System Test};
    \node[draw, rectangle, minimum width=3cm, minimum height=0.8cm] (inttest) at (3,2) {Integration Test};
    \node[draw, rectangle, minimum width=3cm, minimum height=0.8cm] (unittest) at (4.5,1) {Unit Test};

    % Arrows
    \draw[-{Stealth}] (req) -- (arch);
    \draw[-{Stealth}] (arch) -- (design);
    \draw[-{Stealth}] (design) -- (model);
    \draw[-{Stealth}, thick, blue!60] (model) -- node[right] {Code Gen} (code);

    \draw[dashed, -{Stealth}] (req) -- (systest);
    \draw[dashed, -{Stealth}] (arch) -- (inttest);
    \draw[dashed, -{Stealth}] (design) -- (unittest);

    % Bottom
    \draw[-{Stealth}] (code) -- ++(2,0) -- ++(0,0.5);
    \draw[-{Stealth}] (4.5,0.5) -- (unittest);
\end{tikzpicture}
\end{center}

The Simulink model serves as both the detailed design specification and the source for implementation. Code generation (shown in blue) automatically produces the implementation, while the verification activities on the right side confirm that requirements are met.

\subsection{Simulink Coder vs Embedded Coder}

MathWorks offers two code generation products:

\begin{itemize}
    \item \textbf{Simulink Coder}\index{Simulink Coder}: Generates portable ANSI C/C++ code suitable for rapid prototyping and simulation acceleration. The generated code may use dynamic memory allocation and floating-point arithmetic.

    \item \textbf{Embedded Coder}\index{Embedded Coder}: Extends Simulink Coder with optimizations for production embedded systems. Generates more efficient code with options for fixed-point arithmetic\index{fixed-point arithmetic}, static memory allocation, and compliance with coding standards like MISRA C\index{MISRA C}.
\end{itemize}

For the Crazyflie project, Embedded Coder is preferred because it produces code optimized for resource-constrained microcontrollers and supports the STM32's hardware floating-point unit.

\section{The Code Generation Workflow}

Understanding the code generation workflow helps you configure the process correctly and troubleshoot issues when they arise.

\subsection{From Model to Binary}

The complete workflow from Simulink model to running firmware involves several stages:

\begin{center}
\begin{tikzpicture}[node distance=1.8cm, every node/.style={font=\small}]
    \node[draw, rectangle, rounded corners, fill=blue!10, minimum width=2.5cm, minimum height=1cm, align=center] (model) {Simulink\\Model};
    \node[draw, rectangle, rounded corners, fill=yellow!10, minimum width=2.5cm, minimum height=1cm, right of=model, xshift=1.5cm, align=center] (tlc) {Target\\Language\\Compiler};
    \node[draw, rectangle, rounded corners, fill=green!10, minimum width=2.5cm, minimum height=1cm, right of=tlc, xshift=1.5cm, align=center] (ccode) {Generated\\C Code};
    \node[draw, rectangle, rounded corners, fill=orange!10, minimum width=2.5cm, minimum height=1cm, right of=ccode, xshift=1.5cm, align=center] (compiler) {Cross\\Compiler};
    \node[draw, rectangle, rounded corners, fill=red!10, minimum width=2.5cm, minimum height=1cm, right of=compiler, xshift=1.5cm,align=center] (binary) {Binary\\(.elf)};

    \draw[-{Stealth}, thick] (model) -- (tlc);
    \draw[-{Stealth}, thick] (tlc) -- (ccode);
    \draw[-{Stealth}, thick] (ccode) -- (compiler);
    \draw[-{Stealth}, thick] (compiler) -- (binary);

    \node[below of=model, yshift=0.5cm, font=\scriptsize] {.slx file};
    \node[below of=ccode, yshift=0.5cm, font=\scriptsize] {.c, .h files};
    \node[below of=binary, yshift=0.5cm, font=\scriptsize] {ARM executable};
\end{tikzpicture}
\end{center}

\begin{enumerate}
    \item \textbf{Model Analysis}: Simulink analyzes block connections, sample times, and data types
    \item \textbf{Code Generation}: The Target Language Compiler (TLC)\index{Target Language Compiler (TLC)} produces C source files
    \item \textbf{Compilation}: A cross-compiler (arm-none-eabi-gcc) compiles for the target
    \item \textbf{Linking}: Object files link with runtime libraries and startup code
    \item \textbf{Deployment}: The binary is flashed to the microcontroller
\end{enumerate}

\subsection{Solver Selection for Embedded Targets}

One of the most important configuration choices is the solver. Simulink offers two categories:

\begin{itemize}
    \item \textbf{Variable-step solvers} (ode45, ode23, etc.): Adapt step size based on error estimates. Excellent for simulation accuracy but unsuitable for real-time execution because execution time varies unpredictably.

    \item \textbf{Fixed-step solvers}\index{fixed-step solver} (ode4, ode1, discrete): Take constant-size steps. Required for real-time execution because they guarantee predictable timing.
\end{itemize}

\begin{warningbox}
For embedded targets, you \textbf{must} use a fixed-step solver. Variable-step solvers cannot guarantee real-time deadlines and will cause code generation to fail or produce unsuitable code.
\end{warningbox}

Common fixed-step solver choices:

\begin{center}
\begin{tabular}{lll}
\toprule
\textbf{Solver} & \textbf{Method} & \textbf{Use Case} \\
\midrule
\texttt{ode1} & Euler & Simple, fast, less accurate \\
\texttt{ode2} & Heun & Better accuracy than Euler \\
\texttt{ode4} & Runge-Kutta 4 & Good accuracy, more computation \\
\texttt{discrete} & No integration & Purely discrete-time models \\
\bottomrule
\end{tabular}
\end{center}

For the Crazyflie's attitude controller running at 500\,Hz, the \texttt{discrete} solver is typically used since the controller is designed in discrete-time. For continuous-time plant models in simulation, \texttt{ode4} provides a good accuracy-computation tradeoff.

\subsection{Sample Time Configuration}

Every block in Simulink has an associated sample time that determines when it executes:

\begin{itemize}
    \item \textbf{Continuous} ($T_s = 0$): Updates at every integration step
    \item \textbf{Discrete} ($T_s > 0$): Updates at fixed intervals (e.g., $T_s = 0.002$ for 500\,Hz)
    \item \textbf{Inherited} ($T_s = -1$): Takes sample time from connected blocks
\end{itemize}

For code generation, all blocks must resolve to discrete sample times. The fundamental sample time becomes the base rate for the generated code's execution.

\begin{lstlisting}[language=C, caption={Sample time determines step function timing}]
/* Generated code executes model_step() at the fundamental sample time */
void main_loop(void) {
    while (1) {
        wait_for_timer_interrupt();  /* Fires every 2 ms (500 Hz) */
        model_step();                 /* Execute one step of the model */
    }
}
\end{lstlisting}

\section{Structure of Generated Code}

Understanding the structure of generated code is essential for integrating it with your embedded application and debugging issues that arise.

\subsection{File Organization}

Code generation produces several files, each with a specific purpose:

\begin{center}
\begin{tikzpicture}[every node/.style={font=\small\ttfamily}]
    \node[draw, rectangle, fill=blue!10, minimum width=4cm, minimum height=0.7cm] (modelc) at (0,0) {attitude\_ctrl.c};
    \node[draw, rectangle, fill=blue!10, minimum width=4cm, minimum height=0.7cm] (modelh) at (0,-0.9) {attitude\_ctrl.h};
    \node[draw, rectangle, fill=green!10, minimum width=4cm, minimum height=0.7cm] (types) at (0,-1.8) {attitude\_ctrl\_types.h};
    \node[draw, rectangle, fill=green!10, minimum width=4cm, minimum height=0.7cm] (private) at (0,-2.7) {attitude\_ctrl\_private.h};
    \node[draw, rectangle, fill=yellow!10, minimum width=4cm, minimum height=0.7cm] (rtw) at (0,-3.6) {rtwtypes.h};

    \node[right of=modelc, xshift=4cm, font=\small\rmfamily, align=left] {Main implementation\\(step, initialize functions)};
    \node[right of=modelh, xshift=4cm, font=\small\rmfamily, align=left] {Public interface\\(function prototypes, data)};
    \node[right of=types, xshift=4cm, font=\small\rmfamily, align=left] {Model-specific types\\(structs, enums)};
    \node[right of=private, xshift=4cm, font=\small\rmfamily, align=left] {Internal definitions\\(not for external use)};
    \node[right of=rtw, xshift=4cm, font=\small\rmfamily, align=left] {Runtime type definitions\\(int32\_T, real32\_T, etc.)};
\end{tikzpicture}
\end{center}

For a model named \texttt{attitude\_ctrl}, you typically get:

\begin{itemize}
    \item \texttt{attitude\_ctrl.c} --- Main source file with algorithm implementation
    \item \texttt{attitude\_ctrl.h} --- Header with public function prototypes and data declarations
    \item \texttt{attitude\_ctrl\_types.h} --- Type definitions used by the model
    \item \texttt{attitude\_ctrl\_private.h} --- Internal macros and definitions
    \item \texttt{rtwtypes.h} --- Portable type definitions (e.g., \texttt{real32\_T} for \texttt{float})
\end{itemize}

\subsection{Entry Point Functions}

Generated code provides three main entry points that your application must call:

\begin{lstlisting}[language=C, caption={The three entry point functions}]
/* Call once at system startup */
void attitude_ctrl_initialize(void);

/* Call once per sample period (e.g., every 2 ms) */
void attitude_ctrl_step(void);

/* Call at shutdown (often empty for embedded systems) */
void attitude_ctrl_terminate(void);
\end{lstlisting}

\paragraph{Initialize Function}
The initialize function sets up initial conditions for all states and parameters:

\begin{lstlisting}[language=C, caption={Typical initialize function structure}]
void attitude_ctrl_initialize(void)
{
    /* Initialize states to their initial values */
    attitude_ctrl_DW.Integrator_DSTATE = 0.0F;
    attitude_ctrl_DW.Filter_DSTATE = 0.0F;

    /* Initialize outputs */
    attitude_ctrl_Y.motor_cmd[0] = 0.0F;
    attitude_ctrl_Y.motor_cmd[1] = 0.0F;
    attitude_ctrl_Y.motor_cmd[2] = 0.0F;
    attitude_ctrl_Y.motor_cmd[3] = 0.0F;
}
\end{lstlisting}

\paragraph{Step Function}
The step function executes one iteration of the control algorithm. This is where the actual computation happens:

\begin{lstlisting}[language=C, caption={Step function structure (simplified)}]
void attitude_ctrl_step(void)
{
    /* Read inputs (populated by calling code) */
    real32_T roll_error = attitude_ctrl_U.roll_setpoint
                        - attitude_ctrl_U.roll_measured;

    /* Compute control law */
    real32_T P_term = attitude_ctrl_P.Kp * roll_error;
    real32_T I_term = attitude_ctrl_DW.Integrator_DSTATE;
    real32_T D_term = attitude_ctrl_P.Kd *
        (roll_error - attitude_ctrl_DW.error_prev);

    /* Update states for next iteration */
    attitude_ctrl_DW.Integrator_DSTATE +=
        attitude_ctrl_P.Ki * roll_error * 0.002F;
    attitude_ctrl_DW.error_prev = roll_error;

    /* Write outputs */
    attitude_ctrl_Y.torque_cmd = P_term + I_term + D_term;
}
\end{lstlisting}

\paragraph{Terminate Function}
For embedded systems running continuously, the terminate function is typically empty or omitted. It exists for completeness and desktop applications that need cleanup.

\subsection{Data Structures}

Generated code organizes data into several global structures. Understanding these structures is key to interfacing with generated code:

\begin{center}
\begin{tikzpicture}[every node/.style={font=\small},
    box/.style={draw, rectangle, minimum width=3.5cm, minimum height=1.2cm, align=center}]

    % External inputs
    \node[box, fill=green!20] (U) at (-4, 2) {\texttt{model\_U}\\External Inputs};

    % Block outputs
    \node[box, fill=blue!20] (B) at (0, 2) {\texttt{model\_B}\\Block Outputs};

    % External outputs
    \node[box, fill=red!20] (Y) at (4, 2) {\texttt{model\_Y}\\External Outputs};

    % States
    \node[box, fill=yellow!20] (DW) at (-2, 0) {\texttt{model\_DW}\\Discrete States};
    \node[box, fill=orange!20] (X) at (2, 0) {\texttt{model\_X}\\Continuous States};

    % Parameters
    \node[box, fill=purple!20] (P) at (0, -2) {\texttt{model\_P}\\Parameters};

    % Arrows showing data flow
    \draw[-{Stealth}, thick] (U) -- (B);
    \draw[-{Stealth}, thick] (B) -- (Y);
    \draw[-{Stealth}, thick] (DW) -- (B);
    \draw[-{Stealth}, thick] (X) -- (B);
    \draw[-{Stealth}, thick] (P) -- (B);
    \draw[-{Stealth}, dashed] (B) to[bend left] (DW);
    \draw[-{Stealth}, dashed] (B) to[bend right] (X);

    \node[below of=P, yshift=0.3cm, font=\footnotesize\itshape] {(Tunable gains, thresholds, etc.)};
\end{tikzpicture}
\end{center}

\paragraph{External Inputs (\texttt{model\_U})}
Structure containing all input signals to the model. Your application populates this before calling the step function:

\begin{lstlisting}[language=C]
typedef struct {
    real32_T roll_setpoint;      /* Desired roll angle [rad] */
    real32_T pitch_setpoint;     /* Desired pitch angle [rad] */
    real32_T yaw_rate_setpoint;  /* Desired yaw rate [rad/s] */
    real32_T roll_measured;      /* Measured roll from estimator */
    real32_T pitch_measured;     /* Measured pitch from estimator */
    real32_T yaw_rate_measured;  /* Measured yaw rate from gyro */
} ExtU_attitude_ctrl_T;

extern ExtU_attitude_ctrl_T attitude_ctrl_U;
\end{lstlisting}

\paragraph{External Outputs (\texttt{model\_Y})}
Structure containing all output signals from the model. Your application reads this after the step function returns:

\begin{lstlisting}[language=C]
typedef struct {
    real32_T motor_cmd[4];  /* Motor commands [0-65535] */
} ExtY_attitude_ctrl_T;

extern ExtY_attitude_ctrl_T attitude_ctrl_Y;
\end{lstlisting}

\paragraph{Discrete Work States (\texttt{model\_DW})}
Structure containing discrete-time states that persist between step function calls:

\begin{lstlisting}[language=C]
typedef struct {
    real32_T Integrator_DSTATE;    /* Integrator state */
    real32_T Filter_DSTATE;        /* Derivative filter state */
    real32_T UnitDelay_DSTATE;     /* Unit delay state */
} DW_attitude_ctrl_T;

extern DW_attitude_ctrl_T attitude_ctrl_DW;
\end{lstlisting}

\paragraph{Continuous States (\texttt{model\_X})}
For models with continuous-time dynamics (using integrators with continuous sample time), states are stored separately:

\begin{lstlisting}[language=C]
typedef struct {
    real32_T Integrator_CSTATE;  /* Continuous integrator state */
} X_attitude_ctrl_T;

extern X_attitude_ctrl_T attitude_ctrl_X;
\end{lstlisting}

\paragraph{Parameters (\texttt{model\_P})}
Structure containing tunable parameters. These can be modified at runtime for calibration:

\begin{lstlisting}[language=C]
typedef struct {
    real32_T Kp_roll;    /* Proportional gain for roll */
    real32_T Ki_roll;    /* Integral gain for roll */
    real32_T Kd_roll;    /* Derivative gain for roll */
    real32_T Kp_pitch;   /* Proportional gain for pitch */
    /* ... more parameters ... */
} P_attitude_ctrl_T;

extern P_attitude_ctrl_T attitude_ctrl_P;
\end{lstlisting}

\begin{notebox}
The parameter structure allows runtime tuning without recompiling. During development, you can adjust gains via a debugger or communication link, then update the model with final values.
\end{notebox}

\section{Block-to-Code Mapping}

Understanding how Simulink blocks translate to C code helps you predict generated code behavior and optimize your models.

\subsection{Gain Block}

A Gain block multiplies its input by a constant:

\begin{minipage}[t]{0.45\textwidth}
\textbf{Simulink Block:}
\begin{center}
\begin{tikzpicture}
    \node[draw, trapezium, trapezium left angle=70, trapezium right angle=110,
          minimum width=1.5cm, minimum height=1cm] (gain) {Kp};
    \draw[-{Stealth}] (-1.5,0) -- (gain.west) node[midway, above] {error};
    \draw[-{Stealth}] (gain.east) -- (2,0) node[midway, above] {P\_term};
\end{tikzpicture}
\end{center}
\end{minipage}
\hfill
\begin{minipage}[t]{0.5\textwidth}
\textbf{Generated Code:}
\begin{lstlisting}[language=C]
/* Gain: '<S1>/Kp' */
rtb_P_term = ctrl_P.Kp * rtb_error;
\end{lstlisting}
\end{minipage}

\subsection{Sum Block}

A Sum block adds or subtracts signals:

\begin{minipage}[t]{0.45\textwidth}
\textbf{Simulink Block:}
\begin{center}
\begin{tikzpicture}
    \node[draw, circle, minimum size=0.8cm] (sum) {};
    \node at (sum.center) {$+$};
    \draw[-{Stealth}] (-1.5,0.3) -- (sum.west) node[midway, above] {setpoint};
    \draw[-{Stealth}] (-1.5,-0.3) -- (sum.west);
    \node at (-1.2,-0.5) {\small $-$};
    \node at (-1.8,-0.3) {measured};
    \draw[-{Stealth}] (sum.east) -- (1.5,0) node[midway, above] {error};
\end{tikzpicture}
\end{center}
\end{minipage}
\hfill
\begin{minipage}[t]{0.5\textwidth}
\textbf{Generated Code:}
\begin{lstlisting}[language=C]
/* Sum: '<S1>/Sum' */
rtb_error = ctrl_U.setpoint
          - ctrl_U.measured;
\end{lstlisting}
\end{minipage}

\subsection{Discrete Integrator}

A discrete integrator accumulates its input over time, implementing $y[k] = y[k-1] + T_s \cdot u[k]$:

\begin{minipage}[t]{0.45\textwidth}
\textbf{Simulink Block:}
\begin{center}
\begin{tikzpicture}
    \node[draw, rectangle, minimum width=2cm, minimum height=1cm] (int) {$\frac{T_s}{z-1}$};
    \draw[-{Stealth}] (-2,0) -- (int.west) node[midway, above] {error};
    \draw[-{Stealth}] (int.east) -- (2,0) node[midway, above] {I\_term};
\end{tikzpicture}
\end{center}
\end{minipage}
\hfill
\begin{minipage}[t]{0.5\textwidth}
\textbf{Generated Code:}
\begin{lstlisting}[language=C]
/* DiscreteIntegrator: '<S1>/I' */
rtb_I_term = ctrl_DW.I_DSTATE;

/* Update for DiscreteIntegrator */
ctrl_DW.I_DSTATE += 0.002F * rtb_error;
\end{lstlisting}
\end{minipage}

\vspace{0.5em}

Note how the output uses the \emph{previous} state value, and the state update happens separately. This ensures correct discrete-time semantics.

\subsection{Discrete Transfer Function}

A discrete transfer function implements a difference equation. For example, a first-order low-pass filter:

$$H(z) = \frac{b_0 + b_1 z^{-1}}{1 + a_1 z^{-1}}$$

\begin{lstlisting}[language=C, caption={Generated code for discrete transfer function}]
/* DiscreteTransferFcn: '<S1>/LowPass' */
{
    real32_T denAccum;
    real32_T numAccum;

    /* Denominator: 1 + a1*z^(-1) */
    denAccum = ctrl_U.input - ctrl_P.LowPass_DenCoef[1]
             * ctrl_DW.LowPass_states;

    /* Numerator: b0 + b1*z^(-1) */
    numAccum = ctrl_P.LowPass_NumCoef[0] * denAccum
             + ctrl_P.LowPass_NumCoef[1] * ctrl_DW.LowPass_states;

    /* Update state */
    ctrl_DW.LowPass_states = denAccum;

    rtb_filtered = numAccum;
}
\end{lstlisting}

\subsection{Saturation Block}

A Saturation block limits its input to a specified range:

\begin{minipage}[t]{0.45\textwidth}
\textbf{Simulink Block:}
\begin{center}
\begin{tikzpicture}
    \node[draw, rectangle, minimum width=1.8cm, minimum height=1cm] (sat) {};
    \draw[thick] (-0.5,-0.3) -- (-0.2,-0.3) -- (0.2,0.3) -- (0.5,0.3);
    \draw[-{Stealth}] (-2,0) -- (sat.west) node[midway, above] {cmd};
    \draw[-{Stealth}] (sat.east) -- (2,0) node[midway, above] {cmd\_sat};
\end{tikzpicture}
\end{center}
\end{minipage}
\hfill
\begin{minipage}[t]{0.5\textwidth}
\textbf{Generated Code:}
\begin{lstlisting}[language=C]
/* Saturation: '<S1>/Sat' */
if (rtb_cmd > ctrl_P.Sat_UpperLimit) {
    rtb_cmd_sat = ctrl_P.Sat_UpperLimit;
} else if (rtb_cmd < ctrl_P.Sat_LowerLimit) {
    rtb_cmd_sat = ctrl_P.Sat_LowerLimit;
} else {
    rtb_cmd_sat = rtb_cmd;
}
\end{lstlisting}
\end{minipage}

\subsection{Stateflow Chart}

Stateflow charts generate state machine code using switch-case structures:

\begin{lstlisting}[language=C, caption={Generated code for a flight mode state machine}]
void FlightMode_step(void)
{
    /* Chart: '<Root>/FlightMode' */
    switch (FlightMode_DW.is_c1_FlightMode) {
      case IN_Idle:
        if (FlightMode_U.arm_cmd) {
            FlightMode_DW.is_c1_FlightMode = IN_Armed;
            FlightMode_Y.motors_enabled = false;
        }
        break;

      case IN_Armed:
        if (FlightMode_U.takeoff_cmd) {
            FlightMode_DW.is_c1_FlightMode = IN_Flying;
            FlightMode_Y.motors_enabled = true;
        } else if (!FlightMode_U.arm_cmd) {
            FlightMode_DW.is_c1_FlightMode = IN_Idle;
        }
        break;

      case IN_Flying:
        if (FlightMode_U.land_cmd) {
            FlightMode_DW.is_c1_FlightMode = IN_Landing;
        }
        break;

      case IN_Landing:
        if (FlightMode_U.landed) {
            FlightMode_DW.is_c1_FlightMode = IN_Armed;
            FlightMode_Y.motors_enabled = false;
        }
        break;
    }
}
\end{lstlisting}

\begin{keyidea}
Stateflow generates deterministic, efficient state machine code using enumerated states and switch-case logic. This is more maintainable than hand-coding complex state transitions.
\end{keyidea}

\section{Model Configuration for Embedded Systems}

Proper configuration is essential for generating efficient, correct embedded code.

\subsection{Hardware Implementation Settings}

The Hardware Implementation pane specifies target processor characteristics:

\begin{center}
\begin{tabular}{ll}
\toprule
\textbf{Setting} & \textbf{Value for STM32F4} \\
\midrule
Device vendor & ARM Compatible \\
Device type & ARM Cortex-M \\
Byte ordering & Little Endian \\
Signed integer division rounds to & Zero \\
Native word size & 32 bits \\
\texttt{char} size & 8 bits \\
\texttt{short} size & 16 bits \\
\texttt{int} size & 32 bits \\
\texttt{long} size & 32 bits \\
\texttt{float} size & 32 bits \\
\texttt{double} size & 64 bits \\
\bottomrule
\end{tabular}
\end{center}

These settings affect code generation choices like data type selection and arithmetic operations.

\subsection{Code Generation Options}

Key options in the Code Generation pane:

\paragraph{System Target File}
Specifies the target configuration. For Embedded Coder:
\begin{itemize}
    \item \texttt{ert.tlc} --- Generic Embedded Real-Time target
    \item Custom targets for specific hardware (e.g., STM32)
\end{itemize}

\paragraph{Code Interface}
\begin{itemize}
    \item \textbf{Nonreusable function} --- Single instance, global data (simpler)
    \item \textbf{Reusable function} --- Multiple instances, passed data (more flexible)
\end{itemize}

For a single quadrotor, nonreusable functions are simpler. Reusable functions are useful when you need multiple instances of the same controller.

\paragraph{Optimization Level}
\begin{itemize}
    \item \textbf{Minimum} --- Most readable code, useful for debugging
    \item \textbf{Maximum} --- Most efficient code, aggressive inlining
\end{itemize}

\paragraph{Comments}
Enable \textbf{Include comments} and \textbf{Simulink block comments} for traceability. Each line of generated code includes a comment identifying the source block:

\begin{lstlisting}[language=C]
/* Gain: '<S1>/Kp' incorporates:
 *  Sum: '<S1>/Sum' */
rtb_P_term = ctrl_P.Kp * (ctrl_U.setpoint - ctrl_U.measured);
\end{lstlisting}

\subsection{Fixed-Point vs Floating-Point}

The STM32F405 has a hardware floating-point unit (FPU), so single-precision floating-point is efficient. However, some applications benefit from fixed-point:

\begin{center}
\begin{tabular}{lll}
\toprule
\textbf{Aspect} & \textbf{Floating-Point} & \textbf{Fixed-Point} \\
\midrule
Dynamic range & Very large & Limited by word size \\
Precision & Relative (6-7 digits) & Absolute (fixed) \\
Hardware support & FPU required & Integer ALU \\
Code portability & Universal & May need tuning \\
Development effort & Lower & Higher \\
\bottomrule
\end{tabular}
\end{center}

For the Crazyflie, single-precision floating-point (\texttt{float}, \texttt{real32\_T}) is recommended because:
\begin{itemize}
    \item The Cortex-M4F has hardware single-precision FPU
    \item Control calculations fit comfortably in 32-bit float range
    \item Development time is significantly reduced
\end{itemize}

\begin{warningbox}
Avoid double-precision (\texttt{double}) on the STM32F4. The FPU only supports single-precision, so double operations fall back to slow software emulation.
\end{warningbox}

\subsection{Memory Allocation}

For safety-critical embedded systems, static memory allocation is preferred:

\begin{itemize}
    \item \textbf{Static allocation}: All memory allocated at compile time
    \begin{itemize}
        \item Predictable memory footprint
        \item No fragmentation
        \item No runtime allocation failures
    \end{itemize}

    \item \textbf{Dynamic allocation}: Memory allocated via \texttt{malloc()}
    \begin{itemize}
        \item Flexible but unpredictable
        \item Risk of fragmentation
        \item Risk of allocation failure
    \end{itemize}
\end{itemize}

Configure Embedded Coder to use static allocation by enabling \textbf{Static code} in the Code Generation $\rightarrow$ Interface settings.

\section{Integration with Hand-Written Code}

Generated code must integrate with your existing firmware---sensor drivers, communication stacks, and RTOS infrastructure.

\subsection{Calling Generated Code from Main}

The basic integration pattern places generated code in your main control loop:

\begin{lstlisting}[language=C, caption={Integrating generated code with FreeRTOS}]
#include "attitude_ctrl.h"
#include "sensors.h"
#include "motors.h"

void attitudeControlTask(void *pvParameters)
{
    TickType_t xLastWakeTime = xTaskGetTickCount();
    const TickType_t xPeriod = pdMS_TO_TICKS(2);  /* 500 Hz */

    /* Initialize the generated code */
    attitude_ctrl_initialize();

    for (;;) {
        /* Wait for next period */
        vTaskDelayUntil(&xLastWakeTime, xPeriod);

        /* Read sensors and populate inputs */
        SensorData_t sensors = readIMU();
        attitude_ctrl_U.roll_measured = sensors.roll;
        attitude_ctrl_U.pitch_measured = sensors.pitch;
        attitude_ctrl_U.yaw_rate_measured = sensors.gyro_z;

        /* Get setpoints from commander */
        attitude_ctrl_U.roll_setpoint = getSetpoint(ROLL);
        attitude_ctrl_U.pitch_setpoint = getSetpoint(PITCH);
        attitude_ctrl_U.yaw_rate_setpoint = getSetpoint(YAW_RATE);

        /* Execute one step of the controller */
        attitude_ctrl_step();

        /* Apply outputs to motors */
        setMotorCommands(attitude_ctrl_Y.motor_cmd);
    }
}
\end{lstlisting}

\subsection{S-Functions for Custom Blocks}

When you need functionality not available in standard Simulink blocks, S-functions let you write custom blocks in C:

\begin{lstlisting}[language=C, caption={S-function wrapper for custom sensor driver}]
/* S-function: ReadIMU */
static void mdlOutputs(SimStruct *S, int_T tid)
{
    real32_T *roll  = (real32_T *)ssGetOutputPortSignal(S, 0);
    real32_T *pitch = (real32_T *)ssGetOutputPortSignal(S, 1);
    real32_T *yaw   = (real32_T *)ssGetOutputPortSignal(S, 2);

    /* Call the actual sensor driver */
    SensorData_t data = readIMU();

    *roll  = data.roll;
    *pitch = data.pitch;
    *yaw   = data.yaw;
}
\end{lstlisting}

During simulation, the S-function can provide simulated sensor data. For code generation, it produces calls to your actual driver functions.

\subsection{Legacy Code Tool}

The Legacy Code Tool automates wrapping existing C code as Simulink blocks:

\begin{lstlisting}[language=matlab, caption={Using Legacy Code Tool in MATLAB}]
% Define the legacy function interface
def = legacy_code('initialize');
def.SFunctionName = 'sfun_motor_mixer';
def.HeaderFiles = {'motor_mixer.h'};
def.SourceFiles = {'motor_mixer.c'};
def.OutputFcnSpec = ...
    'void motor_mix(single u1[4], single y1[4])';

% Generate the S-function
legacy_code('sfcn_cmex_generate', def);
legacy_code('compile', def);
legacy_code('slblock_generate', def);
\end{lstlisting}

This creates a Simulink block that calls your existing \texttt{motor\_mix()} function, both in simulation and in generated code.

\subsection{External Mode for Parameter Tuning}

External mode enables live parameter tuning while the generated code runs on target:

\begin{center}
\begin{tikzpicture}[every node/.style={font=\small}]
    \node[draw, rectangle, rounded corners, fill=blue!10, minimum width=3cm, minimum height=1.5cm, align=center] (sim) at (0,0) {Simulink\\(Host PC)};
    \node[draw, rectangle, rounded corners, fill=green!10, minimum width=3cm, minimum height=1.5cm, align=center] (target) at (6,0) {Generated Code\\(STM32)};

    \draw[{Stealth}-{Stealth}, thick] (sim) -- (target)
        node[midway, above] {Serial/USB}
        node[midway, below] {Parameter changes};
\end{tikzpicture}
\end{center}

In External mode:
\begin{enumerate}
    \item Simulink connects to the running target via serial or USB
    \item Changing a parameter in Simulink updates the target's \texttt{model\_P} structure
    \item Signal values can be monitored and logged from the target
\end{enumerate}

This is invaluable for tuning control gains on the actual hardware.

\section{Real-Time Execution Considerations}

Generated code must execute within strict timing constraints. This section connects code generation to the real-time concepts from Module~3.

\subsection{Single-Rate vs Multi-Rate Models}

\paragraph{Single-Rate Models}
All blocks execute at the same rate. Generated code has a single \texttt{model\_step()} function:

\begin{lstlisting}[language=C]
/* Single-rate: everything at 500 Hz */
void control_step(void)
{
    /* All computations here */
}
\end{lstlisting}

\paragraph{Multi-Rate Models}
Blocks execute at different rates. Generated code has rate-specific step functions:

\begin{lstlisting}[language=C]
/* Multi-rate: 500 Hz and 100 Hz */
void control_step0(void)  /* Fast rate: 500 Hz */
{
    /* Attitude control */
}

void control_step1(void)  /* Slow rate: 100 Hz */
{
    /* Position control */
}
\end{lstlisting}

The calling code must invoke each step function at the correct rate:

\begin{lstlisting}[language=C, caption={Scheduling multi-rate generated code}]
static uint32_t tick_count = 0;

void timerISR(void)
{
    tick_count++;

    /* Fast rate: every tick (2 ms = 500 Hz) */
    control_step0();

    /* Slow rate: every 5 ticks (10 ms = 100 Hz) */
    if ((tick_count % 5) == 0) {
        control_step1();
    }
}
\end{lstlisting}

\subsection{Overrun Detection}

If the step function takes longer than the sample period, you have a timing overrun:

\begin{lstlisting}[language=C, caption={Detecting timing overruns}]
volatile uint32_t overrun_count = 0;
volatile bool step_in_progress = false;

void timerISR(void)
{
    if (step_in_progress) {
        /* Previous step didn't complete in time! */
        overrun_count++;
        return;  /* Skip this step or handle error */
    }

    step_in_progress = true;
    control_step();
    step_in_progress = false;
}
\end{lstlisting}

\begin{warningbox}
Timing overruns can cause instability in your control system. Always verify that worst-case execution time is less than the sample period. Use the profiling techniques from Chapter~\ref{ch:rtos} to measure actual execution times.
\end{warningbox}

\subsection{Execution Time Profiling}

Measure generated code execution time to verify timing constraints:

\begin{lstlisting}[language=C, caption={Profiling step function execution time}]
#include "stm32f4xx.h"  /* For DWT cycle counter */

void profile_step(void)
{
    /* Enable cycle counter */
    CoreDebug->DEMCR |= CoreDebug_DEMCR_TRCENA_Msk;
    DWT->CYCCNT = 0;
    DWT->CTRL |= DWT_CTRL_CYCCNTENA_Msk;

    uint32_t start = DWT->CYCCNT;
    control_step();
    uint32_t end = DWT->CYCCNT;

    uint32_t cycles = end - start;
    float microseconds = cycles / 168.0f;  /* 168 MHz clock */

    /* Log or transmit timing data */
    printf("Step time: %.1f us\n", microseconds);
}
\end{lstlisting}

For a 500\,Hz controller, the step function must complete in under 2000\,$\mu$s. In practice, aim for much less (e.g., 500\,$\mu$s) to leave margin for interrupt handling and other tasks.

\section{Verification and Traceability}

Code generation enables systematic verification through the development process.

\subsection{Code-to-Model Traceability}

With comments enabled, every line of generated code traces back to its source block:

\begin{lstlisting}[language=C]
/* Gain: '<S2>/Kp_roll' incorporates:
 *  Inport: '<Root>/roll_error'
 */
ctrl_B.P_term_roll = ctrl_P.Kp_roll * ctrl_U.roll_error;
\end{lstlisting}

The comment \texttt{'<S2>/Kp\_roll'} identifies:
\begin{itemize}
    \item \texttt{S2} --- Subsystem 2 in the model hierarchy
    \item \texttt{Kp\_roll} --- The block name
\end{itemize}

In Simulink, you can navigate from code to model (and vice versa) using the code traceability report.

\subsection{Testing Levels with Generated Code}

Generated code supports multiple testing levels as discussed in Module~4:

\begin{center}
\begin{tabular}{lll}
\toprule
\textbf{Level} & \textbf{Code} & \textbf{Plant} \\
\midrule
MIL & Simulink model & Simulink model \\
SIL & Generated C (on host) & Simulink model \\
PIL & Generated C (on target) & Simulink model \\
HIL & Generated C (on target) & Hardware simulator \\
\bottomrule
\end{tabular}
\end{center}

\paragraph{Software-in-the-Loop (SIL)}
The generated code compiles and runs on the host PC, connected to the Simulink plant model. This verifies the generated code matches the model behavior:

\begin{lstlisting}[language=matlab, caption={Running SIL simulation in MATLAB}]
% Configure for SIL
set_param('attitude_ctrl', 'SimulationMode', 'software-in-the-loop');

% Run simulation
sim('attitude_ctrl');

% Compare outputs
max_error = max(abs(yout_model - yout_sil));
assert(max_error < 1e-6, 'SIL mismatch detected');
\end{lstlisting}

\paragraph{Processor-in-the-Loop (PIL)}
The generated code runs on the actual target processor (STM32), communicating with the Simulink plant model via serial link. This catches issues related to:
\begin{itemize}
    \item Target-specific numerical behavior
    \item Compiler optimizations
    \item Fixed-point overflow (if using fixed-point)
\end{itemize}

\subsection{Code Coverage}

Embedded Coder can generate instrumented code that tracks coverage:

\begin{itemize}
    \item \textbf{Statement coverage}: Which lines executed
    \item \textbf{Decision coverage}: Which branches taken
    \item \textbf{Condition coverage}: Which Boolean sub-expressions evaluated
    \item \textbf{MC/DC}: Modified Condition/Decision Coverage for safety-critical systems
\end{itemize}

Coverage data from PIL tests can satisfy DO-178C or ISO~26262 requirements for safety-critical software.

\section{Practical Example: Attitude Controller}

Let's walk through a complete example of generating code for a PD attitude controller.

\subsection{The Simulink Model}

Consider a simple PD roll controller:

\begin{center}
\begin{tikzpicture}[every node/.style={font=\small}, scale=0.9]
    % Input ports
    \node[draw, rectangle, minimum height=0.6cm] (sp) at (-5, 0.5) {setpoint};
    \node[draw, rectangle, minimum height=0.6cm] (meas) at (-5, -0.5) {measured};

    % Sum
    \node[draw, circle, minimum size=0.6cm] (sum) at (-3, 0) {};
    \node at (-3, 0) {$\Sigma$};

    % Gains
    \node[draw, trapezium, trapezium left angle=70, trapezium right angle=110,
          minimum width=1cm] (kp) at (-0.5, 0.5) {Kp};
    \node[draw, trapezium, trapezium left angle=70, trapezium right angle=110,
          minimum width=1cm] (kd) at (-0.5, -0.5) {Kd};

    % Derivative
    \node[draw, rectangle, minimum width=1.2cm] (deriv) at (-0.5, -1.5) {$\frac{N}{1+N/s}$};

    % Output sum
    \node[draw, circle, minimum size=0.6cm] (outsum) at (2, 0) {};
    \node at (2, 0) {$\Sigma$};

    % Output
    \node[draw, rectangle, minimum height=0.6cm] (out) at (4, 0) {torque};

    % Connections
    \draw[-{Stealth}] (sp) -| (-3.3, 0.3);
    \draw[-{Stealth}] (meas) -| (-3.3, -0.3);
    \draw[-{Stealth}] (sum) -- (-1.5, 0) node[midway, above] {error};
    \draw[-{Stealth}] (-1.5, 0) |- (kp);
    \draw[-{Stealth}] (-1.5, 0) |- (deriv);
    \draw[-{Stealth}] (deriv) -- (kd);
    \draw[-{Stealth}] (kp) -| (1.7, 0.3);
    \draw[-{Stealth}] (kd) -| (1.7, -0.3);
    \draw[-{Stealth}] (outsum) -- (out);

    \node at (-3, -0.5) {\tiny $-$};
\end{tikzpicture}
\end{center}

The model uses a filtered derivative (transfer function $\frac{Ns}{s+N}$) to avoid amplifying noise.

\subsection{Generated File Structure}

After code generation, you get:

\begin{verbatim}
roll_ctrl/
  roll_ctrl.c           # Main implementation
  roll_ctrl.h           # Public interface
  roll_ctrl_types.h     # Data types
  roll_ctrl_private.h   # Internal definitions
  rtwtypes.h            # Runtime types
  roll_ctrl_data.c      # Parameter initialization
\end{verbatim}

\subsection{Key Generated Code}

\paragraph{Data Structures}

\begin{lstlisting}[language=C, caption={Generated data structures}]
/* External inputs */
typedef struct {
    real32_T setpoint;   /* '<Root>/setpoint' */
    real32_T measured;   /* '<Root>/measured' */
} ExtU_roll_ctrl_T;

/* External outputs */
typedef struct {
    real32_T torque;     /* '<Root>/torque' */
} ExtY_roll_ctrl_T;

/* Block states (discrete work) */
typedef struct {
    real32_T Filter_DSTATE;  /* '<S1>/Filter' (derivative filter) */
} DW_roll_ctrl_T;

/* Parameters */
typedef struct {
    real32_T Kp;         /* Variable: Kp */
    real32_T Kd;         /* Variable: Kd */
    real32_T N;          /* Variable: N (filter coefficient) */
} P_roll_ctrl_T;

/* Global instances */
extern ExtU_roll_ctrl_T roll_ctrl_U;
extern ExtY_roll_ctrl_T roll_ctrl_Y;
extern DW_roll_ctrl_T roll_ctrl_DW;
extern P_roll_ctrl_T roll_ctrl_P;
\end{lstlisting}

\paragraph{Initialize Function}

\begin{lstlisting}[language=C, caption={Generated initialize function}]
void roll_ctrl_initialize(void)
{
    /* Initialize states */
    roll_ctrl_DW.Filter_DSTATE = 0.0F;

    /* Initialize outputs */
    roll_ctrl_Y.torque = 0.0F;
}
\end{lstlisting}

\paragraph{Step Function}

\begin{lstlisting}[language=C, caption={Generated step function}]
void roll_ctrl_step(void)
{
    real32_T rtb_error;
    real32_T rtb_FilteredDerivative;

    /* Sum: '<S1>/Sum' */
    rtb_error = roll_ctrl_U.setpoint - roll_ctrl_U.measured;

    /* DiscreteTransferFcn: '<S1>/Filter'
     * Implements filtered derivative: N*s/(s+N) discretized */
    rtb_FilteredDerivative = roll_ctrl_P.N * rtb_error
                           - roll_ctrl_P.N * roll_ctrl_DW.Filter_DSTATE;

    /* Update for DiscreteTransferFcn: '<S1>/Filter' */
    roll_ctrl_DW.Filter_DSTATE = (2.0F - roll_ctrl_P.N * 0.002F)
        / (2.0F + roll_ctrl_P.N * 0.002F) * roll_ctrl_DW.Filter_DSTATE
        + 0.002F / (2.0F + roll_ctrl_P.N * 0.002F)
        * (rtb_error + roll_ctrl_DW.Filter_prev);
    roll_ctrl_DW.Filter_prev = rtb_error;

    /* Sum: '<S1>/Sum1' incorporates:
     *  Gain: '<S1>/Kp'
     *  Gain: '<S1>/Kd' */
    roll_ctrl_Y.torque = roll_ctrl_P.Kp * rtb_error
                       + roll_ctrl_P.Kd * rtb_FilteredDerivative;
}
\end{lstlisting}

\subsection{Integration into Crazyflie Firmware}

\begin{lstlisting}[language=C, caption={Integrating generated controller into Crazyflie}]
/* In stabilizer.c or equivalent */
#include "roll_ctrl.h"

void stabilizerInit(void)
{
    /* Initialize generated controller */
    roll_ctrl_initialize();

    /* Set initial parameters (can be tuned later) */
    roll_ctrl_P.Kp = 3.5f;
    roll_ctrl_P.Kd = 0.1f;
    roll_ctrl_P.N = 100.0f;
}

void stabilizerTask(void *param)
{
    /* ... task setup ... */

    while (1) {
        vTaskDelayUntil(&lastWakeTime, M2T(2));  /* 500 Hz */

        /* Get sensor data */
        sensorsAcquire(&sensorData);
        stateEstimator(&state, &sensorData);

        /* Populate generated code inputs */
        roll_ctrl_U.setpoint = commanderGetRoll();
        roll_ctrl_U.measured = state.attitude.roll;

        /* Execute generated controller */
        roll_ctrl_step();

        /* Use output */
        controlOutput.roll_torque = roll_ctrl_Y.torque;

        /* ... similar for pitch and yaw ... */

        /* Apply to motors via mixer */
        powerDistribution(&controlOutput, motorCommands);
    }
}
\end{lstlisting}

\begin{keyidea}
The generated code becomes a modular component in your firmware. The interface is well-defined (inputs, outputs, parameters), making it easy to swap different controller designs or update the model.
\end{keyidea}

\section{Summary}

This appendix covered the essentials of Simulink code generation:

\begin{itemize}
    \item \textbf{Workflow}: Model $\rightarrow$ TLC $\rightarrow$ C code $\rightarrow$ Cross-compile $\rightarrow$ Binary
    \item \textbf{Configuration}: Fixed-step solver, hardware settings, optimization options
    \item \textbf{Code structure}: Initialize/step/terminate functions, data structures (U, Y, DW, P)
    \item \textbf{Block mapping}: How Simulink blocks become C operations
    \item \textbf{Integration}: Calling generated code from your main loop or RTOS tasks
    \item \textbf{Real-time}: Managing multi-rate execution and detecting overruns
    \item \textbf{Verification}: Traceability, SIL/PIL testing, coverage analysis
\end{itemize}

Code generation bridges the gap between model-based design and embedded implementation. By understanding the generated code structure, you can effectively integrate auto-generated controllers with hand-written firmware, verify correct operation across testing levels, and deploy verified software to your quadrotor.

\begin{notebox}
The examples in this appendix use simplified code for clarity. Actual generated code includes additional features like overflow protection, rate transition blocks for multi-rate systems, and extensive comments for traceability.
\end{notebox}
\index{Simulink!code generation|)}
\index{code generation!Simulink|)}


% Appendix: Requirements Engineering for Embedded Systems
\chapter{Requirements Engineering for Embedded Systems}
\label{app:requirements}
\index{requirements engineering|(}

Requirements engineering is the foundation of successful system development. Before we can build, simulate, or verify a cyber-physical system, we must understand what it should do. This appendix covers the essential practices for collecting, analyzing, documenting, and managing requirements for embedded systems.

\section{Why Requirements Matter}

Studies consistently show that requirements errors are among the most costly defects in system development:

\begin{itemize}
    \item 50--60\% of all software defects can be traced to requirements problems
    \item Requirements errors discovered during testing cost 10--100 times more to fix than those found during requirements analysis
    \item Requirements errors discovered in deployed systems cost 100--1000 times more
\end{itemize}

For cyber-physical systems, the stakes are even higher. A missing or incorrect requirement can lead to:

\begin{itemize}
    \item Safety hazards (incorrect behavior under failure conditions)
    \item Certification failure (missing required functionality)
    \item Costly redesign (discovering timing constraints cannot be met)
    \item Field failures (environmental conditions not anticipated)
\end{itemize}

\begin{keyidea}
Verification can only prove that you built what you specified. It cannot prove that you specified the right thing. Requirements engineering is where you determine the ``right thing.''
\end{keyidea}

\subsection{Requirements in the Development Process}

Requirements serve as the contract between stakeholders and developers. They answer the fundamental question: \emph{What should this system do?} The development process then addresses \emph{how} to achieve those requirements through design, implementation, and verification.

\begin{center}
\begin{tikzpicture}[every node/.style={font=\small}]
    % Boxes
    \node[draw, rectangle, rounded corners, fill=blue!15, minimum width=2.5cm, minimum height=1cm, align=center] (stake) at (0, 0) {Stakeholder\\Needs};
    \node[draw, rectangle, rounded corners, fill=green!15, minimum width=2.5cm, minimum height=1cm, align=center] (req) at (4, 0) {Requirements};
    \node[draw, rectangle, rounded corners, fill=yellow!15, minimum width=2.5cm, minimum height=1cm, align=center] (design) at (8, 0) {Design \&\\Implementation};
    \node[draw, rectangle, rounded corners, fill=red!15, minimum width=2.5cm, minimum height=1cm, align=center] (verify) at (12, 0) {Verification};

    % Arrows
    \draw[-{Stealth}, thick] (stake) -- (req) node[midway, above] {\scriptsize elicit};
    \draw[-{Stealth}, thick] (req) -- (design) node[midway, above] {\scriptsize implement};
    \draw[-{Stealth}, thick] (design) -- (verify) node[midway, above] {\scriptsize verify};

    % Feedback
    \draw[-{Stealth}, dashed, gray] (verify.south) -- ++(0, -0.8) -| (req.south) node[midway, below] {\scriptsize validate};
\end{tikzpicture}
\end{center}

For CPS, requirements must capture aspects from multiple domains:
\begin{itemize}
    \item \textbf{Physical behavior}: How the system interacts with the physical world
    \item \textbf{Timing constraints}: When things must happen
    \item \textbf{Safety properties}: What must never happen
    \item \textbf{Resource limits}: Memory, power, weight, cost constraints
\end{itemize}

\section{Requirements Elicitation}
\index{requirements elicitation}

\emph{Elicitation} is the process of discovering and gathering requirements from stakeholders and other sources. It is often the most challenging phase because stakeholders may not know exactly what they need, may have conflicting needs, or may express needs in terms of solutions rather than problems.

\subsection{Stakeholder Identification}
\index{stakeholder}

The first step is identifying who has requirements. For a quadrotor system, stakeholders might include:

\begin{center}
\begin{tabular}{ll}
\toprule
\textbf{Stakeholder} & \textbf{Example Concerns} \\
\midrule
Pilot/Operator & Responsiveness, ease of control, flight time \\
Safety Officer & Failure modes, emergency procedures, containment \\
Maintenance Technician & Diagnostics, repairability, component access \\
Regulatory Authority & Airworthiness, operational limits, documentation \\
Manufacturer & Cost, manufacturability, warranty \\
General Public & Noise, privacy, safety from falling drones \\
\bottomrule
\end{tabular}
\end{center}

Each stakeholder brings different perspectives and constraints. Missing a stakeholder early often means discovering their requirements late, when changes are expensive.

\subsection{Elicitation Techniques}

Several techniques help extract requirements from stakeholders:

\paragraph{Interviews and Workshops}
Direct conversations with stakeholders are the most common technique. Structured interviews follow a prepared list of questions; unstructured interviews allow free-form discussion. Workshops bring multiple stakeholders together to discuss and negotiate requirements.

\paragraph{Observation and Domain Analysis}
Watching users operate similar systems reveals implicit requirements that users cannot articulate. Domain analysis studies the problem space---physics, regulations, existing solutions---to identify constraints.

\paragraph{Prototyping and Simulation}
Early prototypes help stakeholders understand possibilities and refine their needs. Simulation allows exploring ``what if'' scenarios before committing to requirements.

\paragraph{Scenario-Based Elicitation}
\index{use case}
Use cases describe how users interact with the system to achieve goals. Misuse cases describe how the system might be used incorrectly or maliciously, revealing safety and security requirements.

\begin{example}[Quadrotor Use Case]
\textbf{Use Case}: Automated Landing

\textbf{Primary Actor}: Pilot

\textbf{Preconditions}: Quadrotor is in flight, battery above 10\%

\textbf{Main Success Scenario}:
\begin{enumerate}
    \item Pilot initiates landing command
    \item System confirms safe landing zone below
    \item System descends at controlled rate
    \item System detects ground contact
    \item System disarms motors
\end{enumerate}

\textbf{Extensions}:
\begin{itemize}
    \item 2a. No safe landing zone: System alerts pilot, maintains hover
    \item 3a. Obstacle detected during descent: System pauses, alerts pilot
    \item 4a. Ground contact not detected within timeout: System performs emergency motor cutoff
\end{itemize}
\end{example}

\subsection{Common Elicitation Pitfalls}

Several traps await the unwary requirements engineer:

\paragraph{Solutions Disguised as Requirements}
Stakeholders often state solutions rather than needs: ``The system shall use a Kalman filter'' instead of ``The system shall estimate attitude with error less than 2 degrees.'' The solution may be appropriate, but the requirement should express the need, leaving solution choice to design.

\paragraph{Implicit Assumptions}
Stakeholders assume certain things are ``obvious'' and do not state them. The quadrotor ``obviously'' should not fly into walls, but unless this is stated as a requirement, it may not be tested.

\paragraph{Conflicting Requirements}
Different stakeholders want incompatible things. The pilot wants maximum agility; the safety officer wants conservative limits. These conflicts must be identified and resolved, not ignored.

\paragraph{Vague Requirements}
``The system shall be reliable'' or ``The system shall respond quickly'' are not requirements---they are wishes. Requirements must be specific enough to be testable.

\begin{warningbox}
If you cannot describe a test that would verify a requirement, the requirement is not well-defined. Every requirement should have a corresponding verification method.
\end{warningbox}

\section{Requirements Categories}

Requirements for embedded systems fall into several categories, each requiring different elicitation and verification approaches.

\subsection{Functional Requirements}
\index{functional requirements}

Functional requirements describe what the system must \emph{do}---its behavior and capabilities:

\begin{itemize}
    \item \textbf{Input-output relationships}: ``When the pilot commands roll right, the quadrotor shall bank right.''
    \item \textbf{Operating modes}: ``The system shall support manual, assisted, and autonomous flight modes.''
    \item \textbf{State transitions}: ``The system shall transition from Armed to Flying when takeoff command is received and motors reach minimum thrust.''
    \item \textbf{Data processing}: ``The system shall fuse IMU and barometer data to estimate altitude.''
\end{itemize}

\subsection{Non-Functional Requirements}
\index{non-functional requirements}

Non-functional requirements describe \emph{how well} the system must perform:

\paragraph{Performance Requirements}
Quantitative measures of system behavior:
\begin{itemize}
    \item ``Attitude controller shall achieve 90\% settling time within 200~ms.''
    \item ``Position tracking error shall not exceed 0.5~m in steady state.''
    \item ``System shall process sensor data within 500~$\mu$s of acquisition.''
\end{itemize}

\paragraph{Safety Requirements}
\index{safety requirements}
Constraints that prevent harm:
\begin{itemize}
    \item ``Attitude shall not exceed 45 degrees from vertical during normal flight.''
    \item ``System shall initiate controlled descent if battery falls below 15\%.''
    \item ``Motors shall not spin while system is in disarmed state.''
\end{itemize}

\paragraph{Reliability Requirements}
Availability and failure characteristics:
\begin{itemize}
    \item ``System shall achieve mean time between failures of at least 1000 flight hours.''
    \item ``System shall detect sensor failures within 100~ms.''
    \item ``System shall remain controllable after single motor failure.''
\end{itemize}

\paragraph{Resource Requirements}
Constraints on system resources:
\begin{itemize}
    \item ``Flight controller software shall use no more than 128~KB RAM.''
    \item ``System shall operate for at least 7 minutes on full battery.''
    \item ``Total system weight shall not exceed 35~g.''
\end{itemize}

\paragraph{Environmental Requirements}
Operating conditions:
\begin{itemize}
    \item ``System shall operate in ambient temperatures from 0 to 40$^\circ$C.''
    \item ``System shall withstand wind gusts up to 5~m/s.''
    \item ``System shall tolerate vibration levels specified in MIL-STD-810G.''
\end{itemize}

\subsection{Interface Requirements}

Interface requirements define how the system connects to its environment:

\paragraph{Hardware Interfaces}
\begin{itemize}
    \item ``IMU shall communicate via SPI at 10~MHz clock rate.''
    \item ``Motors shall accept PWM signals with 400~Hz update rate.''
    \item ``Radio receiver shall provide CPPM signal on designated pin.''
\end{itemize}

\paragraph{Software Interfaces}
\begin{itemize}
    \item ``Attitude controller shall accept setpoints as quaternions.''
    \item ``Logging module shall output data in MAVLink format.''
    \item ``Ground station shall communicate via CRTP protocol.''
\end{itemize}

\paragraph{User Interfaces}
\begin{itemize}
    \item ``LED shall indicate system state: solid = ready, blinking = low battery.''
    \item ``System shall emit audio warning when battery reaches 20\%.''
\end{itemize}

\subsection{Regulatory Requirements}

For certified systems, regulations impose additional requirements:

\paragraph{Standards Compliance}
\begin{itemize}
    \item ``Software shall be developed in compliance with DO-178C, Design Assurance Level C.''
    \item ``System shall meet electromagnetic compatibility requirements of EN 55032.''
\end{itemize}

\paragraph{Documentation Requirements}
\begin{itemize}
    \item ``All requirements shall be uniquely identified and traceable to tests.''
    \item ``Design decisions shall be documented with rationale.''
\end{itemize}

\paragraph{Derived Requirements}
When a design decision creates new requirements, these are \emph{derived requirements}. For example, if the design uses a particular sensor with specific calibration needs, this creates derived requirements for calibration procedures. Derived requirements must be traced to their source and verified like original requirements.

\section{Requirements Analysis}

Once requirements are gathered, they must be analyzed for quality before being used as the basis for design.

\subsection{Completeness Analysis}

A requirements set is complete if it covers all necessary aspects of the system. Techniques for checking completeness include:

\paragraph{Checklists}
Standard lists of requirement categories ensure nothing is forgotten. For embedded systems, check for:
\begin{itemize}
    \item All operating modes defined
    \item All state transitions specified
    \item Startup and shutdown behavior
    \item Error handling and recovery
    \item Boundary conditions and edge cases
\end{itemize}

\paragraph{Traceability to Sources}
\index{traceability!requirements}
Every stakeholder need should trace to at least one requirement. If a need has no corresponding requirement, something is missing.

\paragraph{Coverage of Hazards}
Every identified hazard should trace to safety requirements that mitigate it.

\subsection{Consistency Analysis}

Requirements are consistent if they do not contradict each other. Inconsistencies arise from:

\begin{itemize}
    \item \textbf{Direct contradiction}: ``System shall respond within 10~ms'' vs. ``System shall complete all checks before responding'' (if checks take 20~ms)
    \item \textbf{Resource conflicts}: Requirements that together exceed available resources
    \item \textbf{Timing conflicts}: Requirements with incompatible timing constraints
\end{itemize}

Detecting inconsistencies early prevents discovering them during integration when fixes are expensive.

\subsection{Feasibility Analysis}

Some requirements may be technically impossible or prohibitively expensive. Feasibility analysis identifies these early:

\paragraph{Technical Feasibility}
Can the requirement be achieved with available technology? For example, requiring 0.01 degree attitude accuracy may exceed what affordable sensors can achieve.

\paragraph{Resource Feasibility}
Do the requirements fit within resource budgets? Add up all timing requirements to check if the processor can execute everything. Add up all memory requirements to check if they fit in available RAM.

\paragraph{Schedule and Cost Feasibility}
Can the requirements be met within project constraints? Some requirements may be deferrable to later versions.

\subsection{Safety Analysis Connection}

Safety analysis techniques identify hazards and trace them to safety requirements:

\paragraph{FMEA (Failure Mode and Effects Analysis)}
\index{FMEA (Failure Mode and Effects Analysis)}
For each component, identify how it can fail and what effects each failure has. Each significant effect generates a safety requirement to detect, mitigate, or prevent the failure.

\begin{example}[FMEA to Requirement]
\textbf{Component}: Motor 1

\textbf{Failure Mode}: Motor stops spinning

\textbf{Effect}: Loss of thrust, uncontrolled descent

\textbf{Severity}: Critical

\textbf{Derived Safety Requirement}: SR-101: System shall detect motor failure within 50~ms and initiate controlled descent using remaining motors.
\end{example}

\paragraph{FTA (Fault Tree Analysis)}
\index{FTA (Fault Tree Analysis)}
Start from an undesired event (the ``top event'') and work backward to identify combinations of failures that could cause it. Each path through the tree suggests requirements to break the causal chain.

\begin{center}
\begin{tikzpicture}[
    every node/.style={font=\small},
    event/.style={draw, rectangle, minimum width=2cm, minimum height=0.8cm, align=center},
    gate/.style={draw, trapezium, trapezium left angle=70, trapezium right angle=110, minimum width=1cm}
]
    % Top event
    \node[event, fill=red!20] (top) at (0, 0) {Uncontrolled\\Descent};

    % OR gate
    \node[gate] (or1) at (0, -1.5) {OR};

    % Second level
    \node[event] (e1) at (-3, -3) {Total Power\\Loss};
    \node[event] (e2) at (0, -3) {Control\\Failure};
    \node[event] (e3) at (3, -3) {Structural\\Failure};

    % Connections
    \draw (top) -- (or1);
    \draw (or1.south) -- ++(-3, -0.5) -- (e1);
    \draw (or1.south) -- (e2);
    \draw (or1.south) -- ++(3, -0.5) -- (e3);

    % Derived requirements
    \node[right of=e1, xshift=0.5cm, font=\scriptsize\itshape, align=left] {$\rightarrow$ Battery\\monitoring};
    \node[right of=e2, xshift=0.5cm, font=\scriptsize\itshape, align=left] {$\rightarrow$ Redundant\\sensors};
\end{tikzpicture}
\end{center}

\section{Requirements Documentation}

Well-documented requirements are unambiguous, testable, and traceable.

\subsection{Anatomy of a Good Requirement}

Each requirement should be:

\paragraph{Unambiguous}
Only one interpretation is possible. Avoid words like ``appropriate,'' ``sufficient,'' ``fast,'' or ``user-friendly'' without quantification.

\paragraph{Testable}
A clear test can determine whether the requirement is satisfied. ``Attitude error shall be less than 5 degrees'' is testable; ``Attitude control shall be accurate'' is not.

\paragraph{Atomic}
Each requirement expresses one capability. ``System shall track position and log data'' should be two requirements.

\paragraph{Feasible}
The requirement can be implemented with available resources and technology.

\paragraph{Traceable}
The requirement has a unique identifier and links to its source and verification.

\subsection{Requirement Language}

Standards like IEEE~830 recommend specific language patterns:

\begin{itemize}
    \item \textbf{``Shall''} indicates a mandatory requirement: ``The system \emph{shall} maintain attitude within 30 degrees.''
    \item \textbf{``Should''} indicates a goal that is desirable but not mandatory: ``The system \emph{should} achieve settling time under 100~ms.''
    \item \textbf{``May''} indicates an optional feature: ``The system \emph{may} support multiple radio protocols.''
\end{itemize}

\begin{notebox}
Consistent use of ``shall,'' ``should,'' and ``may'' makes requirements easier to prioritize and verify. Mixing these arbitrarily causes confusion about what is actually required.
\end{notebox}

\subsection{Documentation Structure}

Requirements should be organized hierarchically:

\begin{center}
\begin{tikzpicture}[
    every node/.style={font=\small},
    level/.style={draw, rectangle, rounded corners, minimum width=4cm, minimum height=0.8cm, align=center}
]
    \node[level, fill=blue!20] (sys) at (0, 0) {System Requirements\\SYS-001, SYS-002, ...};
    \node[level, fill=green!20] (sub1) at (-3, -1.5) {Attitude Subsystem\\ATT-001, ATT-002, ...};
    \node[level, fill=green!20] (sub2) at (3, -1.5) {Navigation Subsystem\\NAV-001, NAV-002, ...};
    \node[level, fill=yellow!20] (comp1) at (-4.5, -3) {IMU\\IMU-001, ...};
    \node[level, fill=yellow!20] (comp2) at (-1.5, -3) {Controller\\CTL-001, ...};

    \draw (sys) -- (sub1);
    \draw (sys) -- (sub2);
    \draw (sub1) -- (comp1);
    \draw (sub1) -- (comp2);
\end{tikzpicture}
\end{center}

Each requirement should include:

\begin{itemize}
    \item \textbf{Identifier}: Unique ID (e.g., ATT-017)
    \item \textbf{Title}: Brief descriptive name
    \item \textbf{Description}: Full requirement statement using ``shall''
    \item \textbf{Rationale}: Why this requirement exists
    \item \textbf{Source}: Stakeholder need or parent requirement
    \item \textbf{Priority}: Must-have, should-have, nice-to-have
    \item \textbf{Verification method}: Test, analysis, inspection, or demonstration
\end{itemize}

\subsection{From Natural Language to Formal Specification}

Natural language requirements are readable but ambiguous. Formal specifications (like STL, covered in Module~4) are precise but harder to read. The bridge between them is crucial:

\begin{example}[Natural to Formal]
\textbf{Natural Language}: ``The attitude shall remain within safe limits during flight.''

\textbf{Refined Natural Language}: ``During flight, roll and pitch angles shall not exceed 30 degrees, and yaw rate shall not exceed 200 degrees per second.''

\textbf{Formal (STL)}: $\Box_{[0,T]}(|\phi| < 30° \land |\theta| < 30° \land |\dot{\psi}| < 200°/\text{s})$

where $\Box_{[0,T]}$ means ``always during interval $[0,T]$.''
\end{example}

Not all requirements need formal specification. Formalize requirements that:
\begin{itemize}
    \item Are safety-critical
    \item Will be checked by automated tools
    \item Have been ambiguous in the past
    \item Are complex enough that natural language is inadequate
\end{itemize}

\section{Requirements Traceability}

Traceability is the ability to follow requirements through the development process---from source through design, implementation, and testing.

\subsection{Why Traceability Matters}

Traceability serves multiple purposes:

\paragraph{Impact Analysis}
When a requirement changes, traceability shows which design elements, code, and tests are affected.

\paragraph{Completeness Verification}
Every requirement should trace to at least one test. If a requirement has no test, it may not be verified.

\paragraph{Certification Evidence}
Safety standards (DO-178C, ISO~26262) require traceability as evidence of development rigor.

\paragraph{Debugging}
When a test fails, traceability shows which requirement is violated, which helps identify the root cause.

\subsection{Traceability Matrix}

A traceability matrix shows relationships between requirements and other artifacts:

\begin{center}
\begin{tabular}{llll}
\toprule
\textbf{Req ID} & \textbf{Design Element} & \textbf{Code Module} & \textbf{Test Case} \\
\midrule
ATT-001 & Attitude Controller & attitude\_ctrl.c & TC-ATT-001 \\
ATT-002 & Rate Limiter & rate\_limit.c & TC-ATT-002 \\
ATT-003 & Motor Mixer & motor\_mix.c & TC-ATT-003a, TC-ATT-003b \\
NAV-001 & Position Estimator & pos\_est.c & TC-NAV-001 \\
\bottomrule
\end{tabular}
\end{center}

\begin{center}
\begin{tikzpicture}[every node/.style={font=\small}]
    % Requirements
    \node[draw, rectangle, fill=blue!20, minimum width=1.5cm] (r1) at (0, 2) {ATT-001};
    \node[draw, rectangle, fill=blue!20, minimum width=1.5cm] (r2) at (0, 1) {ATT-002};
    \node[draw, rectangle, fill=blue!20, minimum width=1.5cm] (r3) at (0, 0) {ATT-003};

    % Design
    \node[draw, rectangle, fill=green!20, minimum width=1.8cm] (d1) at (3.5, 1.5) {Controller};
    \node[draw, rectangle, fill=green!20, minimum width=1.8cm] (d2) at (3.5, 0) {Mixer};

    % Code
    \node[draw, rectangle, fill=yellow!20, minimum width=2cm] (c1) at (7, 1.5) {attitude\_ctrl.c};
    \node[draw, rectangle, fill=yellow!20, minimum width=2cm] (c2) at (7, 0) {motor\_mix.c};

    % Tests
    \node[draw, rectangle, fill=red!20, minimum width=1.5cm] (t1) at (10.5, 2) {TC-001};
    \node[draw, rectangle, fill=red!20, minimum width=1.5cm] (t2) at (10.5, 1) {TC-002};
    \node[draw, rectangle, fill=red!20, minimum width=1.5cm] (t3) at (10.5, 0) {TC-003};

    % Traces
    \draw[-{Stealth}] (r1) -- (d1);
    \draw[-{Stealth}] (r2) -- (d1);
    \draw[-{Stealth}] (r3) -- (d2);
    \draw[-{Stealth}] (d1) -- (c1);
    \draw[-{Stealth}] (d2) -- (c2);
    \draw[-{Stealth}] (c1) -- (t1);
    \draw[-{Stealth}] (c1) -- (t2);
    \draw[-{Stealth}] (c2) -- (t3);

    % Labels
    \node[above] at (0, 2.5) {\textbf{Requirements}};
    \node[above] at (3.5, 2.5) {\textbf{Design}};
    \node[above] at (7, 2.5) {\textbf{Code}};
    \node[above] at (10.5, 2.5) {\textbf{Tests}};
\end{tikzpicture}
\end{center}

\subsection{Tools for Requirements Management}

Industrial projects use specialized tools for requirements management:

\begin{itemize}
    \item \textbf{IBM DOORS}: Industry standard for large aerospace and automotive projects
    \item \textbf{Polarion}: Web-based ALM tool with requirements management
    \item \textbf{Jama Connect}: Requirements management with strong traceability
    \item \textbf{Simulink Requirements Toolbox}: Integrates requirements with Simulink models
\end{itemize}

For smaller projects, spreadsheets or issue trackers can provide basic traceability, though they lack the rigor of dedicated tools.

\section{Requirements Management}

Requirements are not static---they evolve throughout the project. Managing this evolution is critical.

\subsection{Change Management}

A change control process ensures that changes are deliberate and their impacts understood:

\begin{enumerate}
    \item \textbf{Request}: Stakeholder or engineer proposes a change
    \item \textbf{Impact analysis}: Identify affected design, code, and tests
    \item \textbf{Review}: Change control board evaluates cost/benefit
    \item \textbf{Approval/Rejection}: Decision recorded with rationale
    \item \textbf{Implementation}: If approved, update all affected artifacts
    \item \textbf{Verification}: Confirm change implemented correctly
\end{enumerate}

\begin{notebox}
``Just a small change'' is dangerous thinking. Even small requirement changes can have large impacts on safety-critical systems. Always trace the impact before approving.
\end{notebox}

\subsection{Versioning and Baselines}

Requirements should be version-controlled like code. Key concepts:

\paragraph{Baseline}
A frozen set of requirements at a specific point in time. Development phases often start with a baselined set of requirements.

\paragraph{Version History}
Each requirement should track its revision history, showing what changed and why.

\paragraph{Configuration Management}
Requirements, design, code, and tests should be managed together so their versions correspond.

\subsection{Requirements Reviews}

Formal reviews catch defects before they propagate:

\paragraph{Inspection}
Structured review where reviewers independently examine requirements, then meet to discuss findings. Studies show inspections find 60--90\% of defects.

\paragraph{Review Criteria}
Check that each requirement is:
\begin{itemize}
    \item Correct (accurately represents the need)
    \item Complete (no missing information)
    \item Unambiguous (only one interpretation)
    \item Consistent (no contradictions)
    \item Testable (verification method exists)
    \item Traceable (source identified)
\end{itemize}

\paragraph{Sign-off}
Stakeholders formally approve requirements before design proceeds. This creates accountability and a clear baseline.

\section{CPS-Specific Requirements Challenges}

Cyber-physical systems present unique requirements challenges that go beyond traditional software.

\subsection{Timing Requirements}

CPS must meet timing constraints, which must be captured as requirements:

\begin{itemize}
    \item \textbf{Period}: ``Attitude control shall execute every 2~ms $\pm$ 50~$\mu$s.''
    \item \textbf{Deadline}: ``Sensor processing shall complete within 500~$\mu$s of data arrival.''
    \item \textbf{Jitter}: ``Control loop period variation shall not exceed 100~$\mu$s.''
    \item \textbf{Latency}: ``End-to-end delay from sensor to actuator shall not exceed 3~ms.''
\end{itemize}

These requirements must be verifiable---typically through measurement on the actual hardware.

\subsection{Requirements Under Uncertainty}

CPS operate in uncertain environments. Requirements must account for this:

\paragraph{Robustness Requirements}
Specify performance bounds under disturbances:
\begin{itemize}
    \item ``System shall maintain position within 1~m despite wind gusts up to 5~m/s.''
    \item ``Attitude estimation error shall not exceed 5 degrees despite gyro bias of up to 1~deg/s.''
\end{itemize}

\paragraph{Probabilistic Requirements}
Some properties can only be guaranteed with probability:
\begin{itemize}
    \item ``Probability of successful landing shall exceed 99.9\%.''
    \item ``False alarm rate for obstacle detection shall not exceed 1 per hour.''
\end{itemize}

Verifying probabilistic requirements requires statistical testing.

\subsection{Requirements for Learning-Enabled Components}

As machine learning becomes common in CPS, new challenges arise:

\paragraph{Specification Challenges}
How do you specify what a neural network should do? Traditional input-output requirements may be insufficient for complex perception tasks.

\paragraph{Current Approaches}
\begin{itemize}
    \item \textbf{Operational Design Domain (ODD)}: Specify the conditions under which the system should operate correctly
    \item \textbf{Performance envelopes}: Specify bounds on error rates, response times, etc.
    \item \textbf{Safety monitors}: Specify runtime checks that can detect when the ML component is outside its competence
\end{itemize}

This is an active area of research with no complete solutions yet.

\section{Worked Example: Crazyflie Attitude Control Requirements}

This section demonstrates requirements engineering for the Crazyflie's attitude control subsystem.

\subsection{Stakeholders}

\begin{itemize}
    \item \textbf{Pilot}: Wants responsive, predictable control
    \item \textbf{Course instructor}: Wants safe operation in classroom environment
    \item \textbf{Developer}: Wants clear, testable specifications
\end{itemize}

\subsection{High-Level Requirements}

\begin{center}
\begin{tabular}{p{1.5cm}p{10cm}}
\toprule
\textbf{ID} & \textbf{Requirement} \\
\midrule
SYS-001 & The system shall provide attitude stabilization during flight. \\
SYS-002 & The system shall respond to pilot attitude commands within 200~ms. \\
SYS-003 & The system shall prevent attitudes that could cause loss of control. \\
\bottomrule
\end{tabular}
\end{center}

\subsection{Derived Requirements}

From SYS-001:
\begin{center}
\begin{tabular}{p{1.5cm}p{10cm}}
\toprule
\textbf{ID} & \textbf{Requirement} \\
\midrule
ATT-001 & The attitude controller shall execute at 500~Hz $\pm$ 1\%. \\
ATT-002 & Attitude estimation error shall not exceed 3 degrees RMS. \\
ATT-003 & IMU data shall be acquired within 100~$\mu$s of control loop start. \\
\bottomrule
\end{tabular}
\end{center}

From SYS-003:
\begin{center}
\begin{tabular}{p{1.5cm}p{10cm}}
\toprule
\textbf{ID} & \textbf{Requirement} \\
\midrule
SAF-001 & Roll and pitch angles shall not exceed 45 degrees during normal flight. \\
SAF-002 & Angular rates shall not exceed 300~deg/s on any axis. \\
SAF-003 & If attitude exceeds 60 degrees, system shall initiate emergency stop. \\
\bottomrule
\end{tabular}
\end{center}

\subsection{Formal Specifications}

Key requirements formalized in STL:

\textbf{SAF-001} (Attitude bounds):
$$\Box_{[0,T]}(|\phi| < 45° \land |\theta| < 45°)$$

\textbf{SYS-002} (Response time):
$$\Box_{[0,T]}\left(\text{cmd\_change} \Rightarrow \Diamond_{[0, 200\text{ms}]}(|\phi - \phi_{\text{cmd}}| < 5°)\right)$$

\subsection{Traceability}

\begin{center}
\begin{tabular}{llll}
\toprule
\textbf{Req} & \textbf{Source} & \textbf{Code} & \textbf{Test} \\
\midrule
ATT-001 & SYS-001 & stabilizer.c:42 & TC-ATT-001 \\
ATT-002 & SYS-001 & estimator.c:156 & TC-ATT-002 \\
SAF-001 & SYS-003 & stabilizer.c:89 & TC-SAF-001 \\
SAF-002 & SYS-003 & rate\_ctrl.c:34 & TC-SAF-002 \\
\bottomrule
\end{tabular}
\end{center}

\section{Summary}

Requirements engineering is the foundation of successful CPS development:

\begin{itemize}
    \item \textbf{Elicitation} discovers stakeholder needs through interviews, observation, prototyping, and scenario analysis
    \item \textbf{Categories} include functional, performance, safety, reliability, resource, environmental, interface, and regulatory requirements
    \item \textbf{Analysis} checks completeness, consistency, and feasibility before design begins
    \item \textbf{Documentation} makes requirements unambiguous, testable, and traceable
    \item \textbf{Traceability} connects requirements to design, code, and tests throughout development
    \item \textbf{Management} handles the inevitable changes with controlled processes
    \item \textbf{CPS challenges} include timing constraints, uncertainty, and learning-enabled components
\end{itemize}

\begin{keyidea}
Good requirements do not guarantee a good system, but bad requirements guarantee problems. Invest time in getting requirements right---it pays off throughout development.
\end{keyidea}

Requirements engineering connects directly to the verification techniques in Module~4: the formal specifications developed here become the properties checked by STL-based testing, and the traceability developed here demonstrates test coverage for certification.
\index{requirements engineering|)}


% Appendix: Complete Code Listings
%======================================================================
% APPENDIX: Complete Code Listings
%
% This appendix contains longer code examples that are referenced
% from the main text. Each listing includes context about its purpose
% and cross-references to the relevant sections.
%======================================================================

\chapter{Complete Code Listings}
\label{app:code-listings}
\index{code listings}

This appendix provides complete implementations of algorithms and systems discussed in the main text. Shorter code snippets remain inline for immediate context, while these longer listings are collected here for reference without interrupting the narrative flow.

%======================================================================
\section{Sensor Fusion and State Estimation}
\label{app:code-sensor-fusion}
%======================================================================

%----------------------------------------------------------------------
\subsection{Sensor Calibration}
%----------------------------------------------------------------------

\begin{lstlisting}[language=C, caption=Gyroscope bias calibration implementation, label=lst:gyro-calib]
#define GYRO_CALIBRATION_SAMPLES 1000

typedef struct {
    float bias[3];        // Estimated bias for x, y, z
    bool calibrated;      // Flag indicating calibration complete
} GyroCalibration_t;

GyroCalibration_t gyroCalib = {0};

void CalibrateGyroscope(void) {
    float sum[3] = {0.0f, 0.0f, 0.0f};
    int16_t raw[3];

    printf("Gyro calibration: Keep quadrotor still...\n");

    for (int i = 0; i < GYRO_CALIBRATION_SAMPLES; i++) {
        ReadGyroRaw(raw);
        sum[0] += raw[0];
        sum[1] += raw[1];
        sum[2] += raw[2];
        DelayMs(2);  // 500 Hz sampling
    }

    // Convert to physical units (deg/s or rad/s)
    gyroCalib.bias[0] = (sum[0] / GYRO_CALIBRATION_SAMPLES) * GYRO_SCALE;
    gyroCalib.bias[1] = (sum[1] / GYRO_CALIBRATION_SAMPLES) * GYRO_SCALE;
    gyroCalib.bias[2] = (sum[2] / GYRO_CALIBRATION_SAMPLES) * GYRO_SCALE;
    gyroCalib.calibrated = true;

    printf("Gyro bias: [%.4f, %.4f, %.4f] rad/s\n",
           gyroCalib.bias[0], gyroCalib.bias[1], gyroCalib.bias[2]);
}

void GetCalibratedGyro(float *omega) {
    int16_t raw[3];
    ReadGyroRaw(raw);

    omega[0] = raw[0] * GYRO_SCALE - gyroCalib.bias[0];
    omega[1] = raw[1] * GYRO_SCALE - gyroCalib.bias[1];
    omega[2] = raw[2] * GYRO_SCALE - gyroCalib.bias[2];
}
\end{lstlisting}

\begin{lstlisting}[language=C, caption=Online gyro bias update, label=lst:gyro-online]
#define STATIONARY_THRESHOLD 0.05f  // rad/s
#define BIAS_UPDATE_ALPHA    0.001f // Learning rate

void UpdateGyroBiasOnline(float *omega_raw, float *accel) {
    // Check if stationary (low motion)
    float omega_mag = sqrtf(omega_raw[0]*omega_raw[0] +
                            omega_raw[1]*omega_raw[1] +
                            omega_raw[2]*omega_raw[2]);

    float accel_mag = sqrtf(accel[0]*accel[0] +
                            accel[1]*accel[1] +
                            accel[2]*accel[2]);

    bool stationary = (omega_mag < STATIONARY_THRESHOLD) &&
                      (fabsf(accel_mag - GRAVITY) < 0.5f);

    if (stationary) {
        // Slowly update bias estimate
        for (int i = 0; i < 3; i++) {
            gyroCalib.bias[i] += BIAS_UPDATE_ALPHA *
                (omega_raw[i] - gyroCalib.bias[i]);
        }
    }
}
\end{lstlisting}

\begin{lstlisting}[language=C, caption=Six-position calibration data structure, label=lst:accel-sixpos]
typedef struct {
    float bias[3];         // Bias for x, y, z
    float scale[3];        // Scale factor for x, y, z
    float misalign[3][3];  // Misalignment matrix (optional)
    bool calibrated;
} AccelCalibration_t;

AccelCalibration_t accelCalib = {
    .bias = {0.0f, 0.0f, 0.0f},
    .scale = {1.0f, 1.0f, 1.0f},
    .calibrated = false
};

void ComputeAccelCalibration(float meas[6][3]) {
    // meas[0] = +Z up, meas[1] = -Z up, etc.
    const float g = 9.81f;

    // Z-axis (positions 0 and 1)
    accelCalib.bias[2] = (meas[0][2] + meas[1][2]) / 2.0f;
    accelCalib.scale[2] = (meas[0][2] - meas[1][2]) / (2.0f * g);

    // X-axis (positions 2 and 3)
    accelCalib.bias[0] = (meas[2][0] + meas[3][0]) / 2.0f;
    accelCalib.scale[0] = (meas[2][0] - meas[3][0]) / (2.0f * g);

    // Y-axis (positions 4 and 5)
    accelCalib.bias[1] = (meas[4][1] + meas[5][1]) / 2.0f;
    accelCalib.scale[1] = (meas[4][1] - meas[5][1]) / (2.0f * g);

    accelCalib.calibrated = true;
}

void GetCalibratedAccel(float *accel) {
    int16_t raw[3];
    ReadAccelRaw(raw);

    for (int i = 0; i < 3; i++) {
        accel[i] = (raw[i] - accelCalib.bias[i]) / accelCalib.scale[i];
    }
}
\end{lstlisting}

\begin{lstlisting}[language=C, caption=Simplified level calibration, label=lst:accel-level]
void CalibrateLevelAccel(void) {
    float sum[3] = {0};
    int16_t raw[3];

    printf("Place quadrotor level and still...\n");

    for (int i = 0; i < 1000; i++) {
        ReadAccelRaw(raw);
        sum[0] += raw[0];
        sum[1] += raw[1];
        sum[2] += raw[2];
        DelayMs(2);
    }

    // When level, X and Y should read 0, Z should read +g (or -g)
    accelCalib.bias[0] = sum[0] / 1000.0f;
    accelCalib.bias[1] = sum[1] / 1000.0f;
    // Z bias is harder - we assume scale is correct and compute bias
    // such that calibrated Z = g when level
    accelCalib.bias[2] = sum[2] / 1000.0f - (GRAVITY / ACCEL_SCALE);

    accelCalib.calibrated = true;
}
\end{lstlisting}

\begin{lstlisting}[language=C, caption=Hard iron calibration (sphere fitting), label=lst:mag-calib]
typedef struct {
    float offset[3];     // Hard iron offset
    float scale[3];      // Soft iron scale (simplified)
} MagCalibration_t;

void CalibrateMagnetometer(void) {
    float min[3] = {1e6, 1e6, 1e6};
    float max[3] = {-1e6, -1e6, -1e6};

    printf("Rotate the quadrotor in all directions...\n");

    for (int i = 0; i < 3000; i++) {
        float mag[3];
        ReadMagRaw(mag);

        for (int j = 0; j < 3; j++) {
            if (mag[j] < min[j]) min[j] = mag[j];
            if (mag[j] > max[j]) max[j] = mag[j];
        }
        DelayMs(10);
    }

    // Hard iron offset is the center of the ellipsoid
    for (int i = 0; i < 3; i++) {
        magCalib.offset[i] = (max[i] + min[i]) / 2.0f;
        magCalib.scale[i] = (max[i] - min[i]) / 2.0f;
    }
}
\end{lstlisting}

\begin{lstlisting}[language=C, caption=Calibration parameter storage, label=lst:calib-storage]
typedef struct {
    uint32_t magic;        // Magic number for validation
    uint32_t version;      // Calibration format version
    GyroCalibration_t gyro;
    AccelCalibration_t accel;
    MagCalibration_t mag;
    uint32_t checksum;     // CRC for data integrity
} CalibrationData_t;

#define CALIB_MAGIC 0xCAFE1234

bool LoadCalibration(void) {
    CalibrationData_t data;
    EEPROM_Read(CALIB_ADDRESS, &data, sizeof(data));

    if (data.magic != CALIB_MAGIC) {
        printf("No calibration found\n");
        return false;
    }

    if (ComputeCRC(&data, sizeof(data) - 4) != data.checksum) {
        printf("Calibration data corrupted\n");
        return false;
    }

    gyroCalib = data.gyro;
    accelCalib = data.accel;
    magCalib = data.mag;
    return true;
}

void SaveCalibration(void) {
    CalibrationData_t data;
    data.magic = CALIB_MAGIC;
    data.version = 1;
    data.gyro = gyroCalib;
    data.accel = accelCalib;
    data.mag = magCalib;
    data.checksum = ComputeCRC(&data, sizeof(data) - 4);

    EEPROM_Write(CALIB_ADDRESS, &data, sizeof(data));
}
\end{lstlisting}

\begin{lstlisting}[language=C, caption=Calibration validation, label=lst:calib-validate]
bool ValidateCalibration(void) {
    bool valid = true;

    // Check gyro bias is reasonable (< 5 deg/s)
    for (int i = 0; i < 3; i++) {
        if (fabsf(gyroCalib.bias[i]) > DEG2RAD(5.0f)) {
            printf("Gyro bias[%d] out of range: %.2f\n",
                   i, RAD2DEG(gyroCalib.bias[i]));
            valid = false;
        }
    }

    // Check accel scale is reasonable (0.9 to 1.1)
    for (int i = 0; i < 3; i++) {
        if (accelCalib.scale[i] < 0.9f || accelCalib.scale[i] > 1.1f) {
            printf("Accel scale[%d] out of range: %.3f\n",
                   i, accelCalib.scale[i]);
            valid = false;
        }
    }

    // Check that level reading gives |accel| close to g
    float accel[3];
    GetCalibratedAccel(accel);
    float mag = sqrtf(accel[0]*accel[0] + accel[1]*accel[1] + accel[2]*accel[2]);
    if (fabsf(mag - GRAVITY) > 0.5f) {
        printf("Accel magnitude error: %.2f (expected %.2f)\n", mag, GRAVITY);
        valid = false;
    }

    return valid;
}
\end{lstlisting}

%----------------------------------------------------------------------
\subsection{Attitude Estimation Filters}
%----------------------------------------------------------------------

\begin{lstlisting}[language=C, caption=Mahony filter implementation, label=lst:mahony-filter]
void MahonyUpdate(float gx, float gy, float gz,   // Gyro [rad/s]
                  float ax, float ay, float az,   // Accel [m/s^2]
                  float dt) {
    // Normalize accelerometer
    float norm = sqrt(ax*ax + ay*ay + az*az);
    ax /= norm; ay /= norm; az /= norm;

    // Predicted gravity from quaternion (3rd column of R^T)
    float vx = 2*(q1*q3 - q0*q2);
    float vy = 2*(q0*q1 + q2*q3);
    float vz = q0*q0 - q1*q1 - q2*q2 + q3*q3;

    // Error = cross product
    float ex = ay*vz - az*vy;
    float ey = az*vx - ax*vz;
    float ez = ax*vy - ay*vx;

    // Integral error (for bias correction)
    eInt_x += Ki * ex * dt;
    eInt_y += Ki * ey * dt;
    eInt_z += Ki * ez * dt;

    // Apply correction to gyro
    gx += Kp*ex + eInt_x;
    gy += Kp*ey + eInt_y;
    gz += Kp*ez + eInt_z;

    // Integrate quaternion
    float dq0 = 0.5f * (-q1*gx - q2*gy - q3*gz);
    float dq1 = 0.5f * ( q0*gx + q2*gz - q3*gy);
    float dq2 = 0.5f * ( q0*gy - q1*gz + q3*gx);
    float dq3 = 0.5f * ( q0*gz + q1*gy - q2*gx);
    q0 += dq0 * dt;
    q1 += dq1 * dt;
    q2 += dq2 * dt;
    q3 += dq3 * dt;

    // Renormalize
    norm = sqrt(q0*q0 + q1*q1 + q2*q2 + q3*q3);
    q0 /= norm; q1 /= norm; q2 /= norm; q3 /= norm;
}
\end{lstlisting}

\begin{lstlisting}[language=C, caption=Attitude EKF implementation, label=lst:attitude-ekf]
typedef struct {
    float q[4];      // Quaternion estimate [q0, q1, q2, q3]
    float P[4][4];   // Covariance matrix
    float Q[4][4];   // Process noise covariance
    float R[3][3];   // Measurement noise covariance
} AttitudeEKF;

void ekf_init(AttitudeEKF* ekf) {
    // Initial quaternion: identity (no rotation)
    ekf->q[0] = 1.0f; ekf->q[1] = 0.0f;
    ekf->q[2] = 0.0f; ekf->q[3] = 0.0f;

    // Initial covariance: moderate uncertainty
    memset(ekf->P, 0, sizeof(ekf->P));
    for (int i = 0; i < 4; i++) ekf->P[i][i] = 0.1f;

    // Process noise (tuning parameter)
    memset(ekf->Q, 0, sizeof(ekf->Q));
    for (int i = 0; i < 4; i++) ekf->Q[i][i] = 1e-6f;

    // Measurement noise (tuning parameter)
    memset(ekf->R, 0, sizeof(ekf->R));
    for (int i = 0; i < 3; i++) ekf->R[i][i] = 0.25f;
}

void ekf_predict(AttitudeEKF* ekf, float gx, float gy, float gz, float dt) {
    float* q = ekf->q;

    // Build Omega matrix from gyroscope
    float Omega[4][4] = {
        {    0, -gx, -gy, -gz},
        {   gx,   0,  gz, -gy},
        {   gy, -gz,   0,  gx},
        {   gz,  gy, -gx,   0}
    };

    // F = I + (dt/2) * Omega
    float F[4][4];
    for (int i = 0; i < 4; i++) {
        for (int j = 0; j < 4; j++) {
            F[i][j] = (i == j ? 1.0f : 0.0f) + 0.5f * dt * Omega[i][j];
        }
    }

    // Predict state: q = F * q
    float q_new[4] = {0};
    for (int i = 0; i < 4; i++) {
        for (int j = 0; j < 4; j++) {
            q_new[i] += F[i][j] * q[j];
        }
    }
    memcpy(q, q_new, sizeof(q_new));

    // Normalize quaternion
    float norm = sqrtf(q[0]*q[0] + q[1]*q[1] + q[2]*q[2] + q[3]*q[3]);
    for (int i = 0; i < 4; i++) q[i] /= norm;

    // Predict covariance: P = F * P * F' + Q
    float FP[4][4], FPFT[4][4];
    mat4x4_mult(F, ekf->P, FP);
    mat4x4_mult_transposed(FP, F, FPFT);
    for (int i = 0; i < 4; i++) {
        for (int j = 0; j < 4; j++) {
            ekf->P[i][j] = FPFT[i][j] + ekf->Q[i][j];
        }
    }
}

void ekf_update(AttitudeEKF* ekf, float ax, float ay, float az) {
    float* q = ekf->q;
    float g = 9.81f;

    // Normalize accelerometer measurement
    float a_norm = sqrtf(ax*ax + ay*ay + az*az);
    if (a_norm < 0.1f) return;  // Skip if no valid reading
    ax /= a_norm; ay /= a_norm; az /= a_norm;

    // Predicted measurement: h(q) = R(q)' * [0; 0; 1]
    float h[3] = {
        2.0f * (q[1]*q[3] - q[0]*q[2]),
        2.0f * (q[0]*q[1] + q[2]*q[3]),
        q[0]*q[0] - q[1]*q[1] - q[2]*q[2] + q[3]*q[3]
    };

    // Innovation (measurement residual)
    float y[3] = {ax - h[0], ay - h[1], az - h[2]};

    // Measurement Jacobian H (3x4)
    float H[3][4] = {
        {-2*q[2],  2*q[3], -2*q[0],  2*q[1]},
        { 2*q[1],  2*q[0],  2*q[3],  2*q[2]},
        { 2*q[0], -2*q[1], -2*q[2],  2*q[3]}
    };

    // S = H * P * H' + R  (3x3)
    float HP[3][4], S[3][3];
    mat3x4_mult_4x4(H, ekf->P, HP);
    mat3x4_mult_4x3T(HP, H, S);
    for (int i = 0; i < 3; i++) S[i][i] += ekf->R[i][i];

    // K = P * H' * S^{-1}  (4x3)
    float K[4][3];
    // ... (compute using matrix operations)

    // Update state: q = q + K * y
    for (int i = 0; i < 4; i++) {
        for (int j = 0; j < 3; j++) {
            q[i] += K[i][j] * y[j];
        }
    }

    // Normalize quaternion
    float norm = sqrtf(q[0]*q[0] + q[1]*q[1] + q[2]*q[2] + q[3]*q[3]);
    for (int i = 0; i < 4; i++) q[i] /= norm;

    // Update covariance: P = (I - K*H) * P
    // ... (standard Joseph form for numerical stability)
}
\end{lstlisting}

\begin{lstlisting}[language=C, caption=Multi-sensor EKF structure, label=lst:multisensor-ekf]
void ekf_full_update(EKF* ekf, SensorData* data) {
    // Predict using gyroscope
    ekf_predict(ekf, data->gyro, dt);

    // Update with accelerometer (roll/pitch)
    if (data->accel_valid) {
        ekf_update_accel(ekf, data->accel);
    }

    // Update with magnetometer (yaw)
    if (data->mag_valid && !data->high_throttle) {
        ekf_update_mag(ekf, data->mag);
    }

    // Update with GPS (position/velocity)
    if (data->gps_valid) {
        ekf_update_gps(ekf, data->gps);
    }

    // Update with barometer (altitude)
    if (data->baro_valid) {
        ekf_update_baro(ekf, data->baro);
    }
}
\end{lstlisting}

%----------------------------------------------------------------------
\subsection{Control Implementation}
%----------------------------------------------------------------------

\begin{lstlisting}[language=C, caption=LQR-tuned PID implementation, label=lst:lqr-pid]
// LQR-computed gains (from MATLAB)
#define KP_ROLL  3.87e-3f   // From LQR K[0]
#define KD_ROLL  5.57e-3f   // From LQR K[1]
#define KI_ROLL  0.05e-3f   // Added empirically

typedef struct {
    float integral;
    float prev_error;
} PIDState;

float pid_roll(PIDState* state, float phi_d, float phi, float p, float dt) {
    float error = phi_d - phi;

    // Anti-windup: limit integral
    state->integral += error * dt;
    state->integral = clamp(state->integral, -0.5f, 0.5f);

    // PID output
    float tau = KP_ROLL * error
              + KI_ROLL * state->integral
              + KD_ROLL * (0.0f - p);  // Assume p_d = 0

    return tau;
}
\end{lstlisting}

\begin{lstlisting}[language=C, caption=LQG attitude controller, label=lst:lqg-controller]
void lqg_attitude_control(AttitudeEKF* ekf, float* tau, float dt) {
    // Get sensor data
    float gx, gy, gz, ax, ay, az;
    imu_read(&gx, &gy, &gz, &ax, &ay, &az);

    // Kalman filter: estimate attitude
    ekf_predict(ekf, gx, gy, gz, dt);
    ekf_update(ekf, ax, ay, az);

    // Extract Euler angles from quaternion estimate
    float phi, theta, psi;
    quaternion_to_euler(ekf->q, &phi, &theta, &psi);

    // Extract angular rates (direct from gyro, filtered by EKF)
    float p = gx, q = gy, r = gz;  // Or extract from EKF if estimating bias

    // LQR control law: u = -K * x
    tau[0] = -(K_PHI * phi + K_P * p);       // Roll torque
    tau[1] = -(K_THETA * theta + K_Q * q);   // Pitch torque
    tau[2] = -(K_PSI * psi + K_R * r);       // Yaw torque
}
\end{lstlisting}

%----------------------------------------------------------------------
\subsection{Parameter Identification (MATLAB)}
%----------------------------------------------------------------------

\begin{lstlisting}[language=Matlab, caption=Bifilar pendulum calculation, label=lst:bifilar]
function I = bifilar_inertia(m, b, l, T)
% BIFILAR_INERTIA Calculate moment of inertia from bifilar pendulum
%   m - mass [kg]
%   b - string separation [m]
%   l - string length [m]
%   T - oscillation period [s]

    g = 9.81;  % gravitational acceleration [m/s^2]
    I = (m * g * b^2 * T^2) / (16 * pi^2 * l);
end

% Example measurement for Crazyflie
m = 0.034;      % kg
b = 0.05;       % string separation [m]
l = 0.30;       % string length [m]

% Measured periods for each axis (average of 10 oscillations)
T_roll = 0.48;   % Roll axis oscillation period [s]
T_pitch = 0.48;  % Pitch axis oscillation period [s]
T_yaw = 0.52;    % Yaw axis oscillation period [s]

Ixx = bifilar_inertia(m, b, l, T_roll);
Iyy = bifilar_inertia(m, b, l, T_pitch);
Izz = bifilar_inertia(m, b, l, T_yaw);

fprintf('Ixx = %.2e kg*m^2\n', Ixx);
fprintf('Iyy = %.2e kg*m^2\n', Iyy);
fprintf('Izz = %.2e kg*m^2\n', Izz);
\end{lstlisting}

\begin{lstlisting}[language=Matlab, caption=Dynamic inertia identification, label=lst:dynamic-inertia]
% Load flight data with step commands
load('step_response_data.mat');

% Data contains:
%   t - time vector [s]
%   omega_x - roll rate from gyroscope [rad/s]
%   motor_cmd - motor commands [0-1, normalized]
%   motor_rpm - motor speeds [rad/s] (if available)

% Apply step command and measure angular acceleration
% Use central difference for differentiation
dt = mean(diff(t));
omega_dot = diff(omega_x) / dt;

% Find period where step is applied and response is clean
step_start = find(t > 1.0 & t < 1.5);  % Adjust for your data

% Calculate torque from motor model
% Tau_x = L * k_f * (omega_1^2 - omega_2^2 + omega_3^2 - omega_4^2)
% For pure roll step, simplify:
k_f = 3.5e-8;  % Thrust coefficient [N/(rad/s)^2]
L = 0.046;     % Arm length [m]

% Average angular acceleration during step
alpha_avg = mean(omega_dot(step_start));

% Known applied torque from motor difference
tau_applied = 0.002;  % Calculate from motor commands [N*m]

Ixx_estimated = tau_applied / alpha_avg;
fprintf('Estimated Ixx = %.2e kg*m^2\n', Ixx_estimated);
\end{lstlisting}

\begin{lstlisting}[language=Matlab, caption=Thrust coefficient identification from test stand data, label=lst:thrust-coeff]
% Thrust stand measurements
% Motor RPM measured with optical tachometer
% Thrust measured with load cell

rpm_data = [5000, 7000, 9000, 11000, 13000, 15000];  % RPM
thrust_data = [0.5, 1.0, 1.6, 2.4, 3.3, 4.4];       % grams

% Convert units
omega_data = rpm_data * 2 * pi / 60;  % rad/s
thrust_N = thrust_data / 1000 * 9.81;  % Newtons

% Fit T = k_f * omega^2
omega_sq = omega_data.^2;
k_f = thrust_N / omega_sq;  % This is a least-squares division

% Better: use polyfit for robustness
p = polyfit(omega_sq, thrust_N, 1);
k_f_fit = p(1);

fprintf('k_f = %.2e N/(rad/s)^2\n', k_f_fit);

% Plot fit quality
figure;
omega_plot = linspace(0, max(omega_data)*1.1, 100);
thrust_fit = k_f_fit * omega_plot.^2;

plot(omega_data, thrust_N, 'ko', 'MarkerSize', 10);
hold on;
plot(omega_plot, thrust_fit, 'b-', 'LineWidth', 2);
xlabel('\omega (rad/s)');
ylabel('Thrust (N)');
title('Thrust vs Motor Speed');
legend('Measured', 'Fit: T = k_f \omega^2');
grid on;
\end{lstlisting}

\begin{lstlisting}[language=Matlab, caption=Torque coefficient from reaction torque measurement, label=lst:torque-coeff]
% Mount motor on torque sensor
% Measure reaction torque at various speeds

rpm_data = [5000, 7000, 9000, 11000, 13000, 15000];
torque_mNm = [0.2, 0.4, 0.65, 1.0, 1.4, 1.9];  % milli-Newton-meters

omega_data = rpm_data * 2 * pi / 60;
torque_Nm = torque_mNm / 1000;

% Fit Q = k_tau * omega^2
omega_sq = omega_data.^2;
k_tau_fit = polyfit(omega_sq, torque_Nm, 1);
k_tau = k_tau_fit(1);

fprintf('k_tau = %.2e N*m/(rad/s)^2\n', k_tau);

% Calculate ratio
ratio = k_tau / k_f_fit;
fprintf('k_tau / k_f = %.4f\n', ratio);
\end{lstlisting}

\begin{lstlisting}[language=Matlab, caption=Motor time constant identification, label=lst:motor-timeconstant]
% Record motor speed response to step command
load('motor_step_response.mat');

% Data:
%   t - time [s]
%   omega_cmd - commanded speed [rad/s]
%   omega_meas - measured speed [rad/s] (from ESC telemetry or tachometer)

% Find step start time
step_idx = find(diff(omega_cmd) > 100, 1);
t_step = t(step_idx);

% Shift time to start at step
t_rel = t(step_idx:end) - t_step;
omega_rel = omega_meas(step_idx:end);

% Initial and final values
omega_0 = omega_rel(1);
omega_final = mean(omega_rel(end-10:end));
Delta_omega = omega_final - omega_0;

% First-order response: omega(t) = omega_final * (1 - exp(-t/tau))
% At t = tau: omega = omega_final * (1 - exp(-1)) = 0.632 * omega_final

% Find time to reach 63.2% of final value
target = omega_0 + 0.632 * Delta_omega;
tau_idx = find(omega_rel > target, 1);
tau_motor = t_rel(tau_idx);

fprintf('Motor time constant: tau_m = %.3f s\n', tau_motor);

% Fit first-order model
model = @(tau, t) omega_0 + Delta_omega * (1 - exp(-t/tau));
tau_fit = lsqcurvefit(model, 0.05, t_rel, omega_rel);

fprintf('Fitted time constant: tau_m = %.3f s\n', tau_fit);

% Plot
figure;
plot(t_rel*1000, omega_rel, 'b-', 'LineWidth', 2);
hold on;
plot(t_rel*1000, model(tau_fit, t_rel), 'r--', 'LineWidth', 2);
xlabel('Time (ms)');
ylabel('Motor Speed (rad/s)');
legend('Measured', 'First-order fit');
title(['Motor Step Response (\tau = ' num2str(tau_fit*1000, '%.1f') ' ms)']);
grid on;
\end{lstlisting}

\begin{lstlisting}[language=Matlab, caption=Frequency sweep analysis, label=lst:freq-sweep]
% Frequency sweep flight test
% Command: roll rate reference = A * sin(2*pi*f*t) for various f

frequencies = [0.5, 1, 2, 3, 4, 5, 7, 10, 15, 20];  % Hz
amplitude = 0.3;  % rad/s

% For each frequency, compute gain and phase from measured response
gains = zeros(size(frequencies));
phases = zeros(size(frequencies));

for i = 1:length(frequencies)
    % Load data for this frequency
    filename = sprintf('sweep_%.1fHz.mat', frequencies(i));
    load(filename);

    % t, cmd, response are loaded

    % Use FFT or correlation to find gain and phase
    f = frequencies(i);
    omega = 2*pi*f;

    % Cross-correlation method
    [R, lags] = xcorr(response - mean(response), cmd - mean(cmd), 'coeff');
    [~, idx] = max(R);
    phase_lag = lags(idx) * mean(diff(t));
    phases(i) = -phase_lag * omega * 180/pi;  % degrees

    % Gain from amplitude ratio
    gain = std(response) / std(cmd);
    gains(i) = 20*log10(gain);  % dB
end

% Bode plot
figure;
subplot(2,1,1);
semilogx(frequencies, gains, 'bo-', 'LineWidth', 2);
ylabel('Gain (dB)');
title('Experimental Bode Plot - Roll Rate');
grid on;

subplot(2,1,2);
semilogx(frequencies, phases, 'ro-', 'LineWidth', 2);
ylabel('Phase (deg)');
xlabel('Frequency (Hz)');
grid on;
\end{lstlisting}

\begin{lstlisting}[language=Matlab, caption=Grey-box identification for quadrotor dynamics, label=lst:greybox]
% Define grey-box model structure
% State: [phi, omega_x]' (roll angle, roll rate)
% Input: u = motor command difference
% Parameters: [Ixx, k_f, L, tau_m]

function [A, B, C, D] = roll_model(params, Ts)
    Ixx = params(1);
    k_f = params(2);
    L = params(3);
    tau_m = params(4);

    % Continuous-time model
    % Ixx * omega_dot = L * k_f * u
    % tau_m * omega_motor_dot + omega_motor = u

    % Simplified: assume motor dynamics fast
    % phi_dot = omega_x
    % omega_x_dot = (L * k_f / Ixx) * u

    Ac = [0, 1; 0, 0];
    Bc = [0; L * k_f / Ixx];
    Cc = [1, 0];
    Dc = 0;

    % Discretize
    sys_c = ss(Ac, Bc, Cc, Dc);
    sys_d = c2d(sys_c, Ts, 'zoh');
    [A, B, C, D] = ssdata(sys_d);
end

% Create idgrey model
params_init = [1.4e-5, 3.5e-8, 0.046, 0.05];
Ts = 0.001;  % Sample time

roll_model_grey = idgrey(@roll_model, params_init, 'cd', ...
    'Name', 'Roll Dynamics');

% Set parameter bounds
roll_model_grey.Structure.Parameters(1).Minimum = 1e-6;  % Ixx
roll_model_grey.Structure.Parameters(1).Maximum = 1e-3;
roll_model_grey.Structure.Parameters(2).Minimum = 1e-9;  % k_f
roll_model_grey.Structure.Parameters(2).Maximum = 1e-6;
% ... etc

% Estimate parameters from flight data
load('roll_identification_data.mat');  % Contains u, y, t
data = iddata(y, u, Ts);

roll_model_est = greyest(data, roll_model_grey);

% Display results
fprintf('Identified parameters:\n');
fprintf('  Ixx = %.2e kg*m^2\n', roll_model_est.Structure.Parameters(1).Value);
fprintf('  k_f = %.2e N/(rad/s)^2\n', roll_model_est.Structure.Parameters(2).Value);
fprintf('  L = %.4f m\n', roll_model_est.Structure.Parameters(3).Value);
fprintf('  tau_m = %.4f s\n', roll_model_est.Structure.Parameters(4).Value);
\end{lstlisting}

\begin{lstlisting}[language=Matlab, caption=Model validation with cross-validation, label=lst:model-validate]
% Split data: 70% identification, 30% validation
n = length(t);
n_id = round(0.7 * n);

data_id = iddata(y(1:n_id), u(1:n_id), Ts);
data_val = iddata(y(n_id+1:end), u(n_id+1:end), Ts);

% Identify on training data
model_est = greyest(data_id, roll_model_grey);

% Validate on held-out data
[y_sim, fit, ~] = compare(data_val, model_est);

fprintf('Validation fit: %.1f%%\n', fit);

% Plot validation
figure;
plot(data_val.OutputData, 'b-', 'LineWidth', 1);
hold on;
plot(y_sim.OutputData, 'r--', 'LineWidth', 1);
xlabel('Sample');
ylabel('Roll angle (rad)');
legend('Measured', 'Simulated');
title(['Model Validation (Fit: ' num2str(fit, '%.1f') '%)']);
grid on;
\end{lstlisting}

\begin{lstlisting}[language=Matlab, caption=Parameter reasonableness check, label=lst:param-check]
function valid = check_parameters(params, quadrotor_type)
% CHECK_PARAMETERS Verify identified parameters are physically reasonable

    Ixx = params.Ixx;
    k_f = params.k_f;
    L = params.L;
    tau_m = params.tau_m;
    m = params.mass;

    valid = true;

    % Check moment of inertia (should scale with mass * length^2)
    I_expected = 0.1 * m * L^2;  % Order of magnitude
    if Ixx < I_expected/100 || Ixx > I_expected*100
        warning('Ixx = %.2e seems unreasonable for this size quadrotor', Ixx);
        valid = false;
    end

    % Check thrust coefficient (should produce ~hover thrust at ~50% throttle)
    omega_hover = sqrt(m * 9.81 / (4 * k_f));  % rad/s for hover
    rpm_hover = omega_hover * 60 / (2*pi);
    if rpm_hover < 3000 || rpm_hover > 50000
        warning('k_f = %.2e implies hover at %.0f RPM, seems unreasonable', k_f, rpm_hover);
        valid = false;
    end

    % Check motor time constant
    if tau_m < 0.001 || tau_m > 1.0
        warning('tau_m = %.3f s seems unreasonable', tau_m);
        valid = false;
    end

    % Check arm length
    if L < 0.01 || L > 1.0
        warning('L = %.3f m seems unreasonable', L);
        valid = false;
    end

    if valid
        fprintf('All parameters within reasonable ranges.\n');
    end
end
\end{lstlisting}

%======================================================================
\section{Simulation and Modeling}
\label{app:code-simulation}
%======================================================================

%----------------------------------------------------------------------
\subsection{Simscape Motor-Propeller Model}
%----------------------------------------------------------------------

\begin{lstlisting}[language=Matlab, caption=Creating the motor-propeller Simscape model, label=lst:simscape-model]
% create_motor_propeller_model.m
% Creates a complete motor-propeller Simscape model

% Create new model
model = 'motor_propeller_sim';
new_system(model);
open_system(model);

% Add Simscape Solver Configuration
add_block('nesl_utility/Solver Configuration', ...
    [model '/Solver'], 'Position', [50 50 100 100]);

% Add ESC (custom component)
add_block('nesl_utility/Simulink-PS Converter', ...
    [model '/PWM_to_PS'], 'Position', [150 150 200 180]);

% Add DC Motor from Simscape Electrical
add_block('elec_lib/Rotational Electromechanics/DC Motor', ...
    [model '/Motor'], 'Position', [350 140 420 200]);
set_param([model '/Motor'], ...
    'Ra', '0.5', ...           % Armature resistance (Ohm)
    'La', '1e-4', ...          % Armature inductance (H)
    'Ke', '0.01', ...          % Back-EMF constant (V/(rad/s))
    'J', '1e-6', ...           % Rotor inertia (kg*m^2)
    'B', '1e-7');              % Viscous friction

% Add Propeller (custom component)
add_block('nesl_utility/Simscape Component', ...
    [model '/Propeller'], 'Position', [500 150 560 190]);

% Add Ideal Rotational Motion Sensor
add_block('fl_lib/Mechanical/Rotational Elements/Ideal Rotational Motion Sensor', ...
    [model '/Speed_Sensor'], 'Position', [500 50 560 90]);

% Add Mechanical Rotational Reference
add_block('fl_lib/Mechanical/Rotational Elements/Mechanical Rotational Reference', ...
    [model '/Mech_Ref'], 'Position', [600 200 640 240]);

% Add Electrical Reference
add_block('fl_lib/Electrical/Electrical Elements/Electrical Reference', ...
    [model '/Elec_Ref'], 'Position', [250 250 290 290]);

% Add Controlled Voltage Source
add_block('fl_lib/Electrical/Electrical Sources/Controlled Voltage Source', ...
    [model '/Voltage_Source'], 'Position', [250 140 290 180]);

% Connect the components
add_line(model, 'PWM_to_PS/1', 'Voltage_Source/1');
add_line(model, 'Voltage_Source/RConn1', 'Motor/+');
add_line(model, 'Voltage_Source/RConn2', 'Motor/-');
add_line(model, 'Motor/-', 'Elec_Ref/1');
add_line(model, 'Motor/C', 'Propeller/R');
add_line(model, 'Motor/C', 'Speed_Sensor/R');
add_line(model, 'Propeller/R', 'Mech_Ref/1', 'autorouting', 'on');
add_line(model, 'Speed_Sensor/C', 'Mech_Ref/1');

% Add input port for PWM
add_block('built-in/Inport', [model '/PWM_In'], 'Position', [50 155 80 175]);
add_line(model, 'PWM_In/1', 'PWM_to_PS/1');

% Add output ports
add_block('nesl_utility/PS-Simulink Converter', ...
    [model '/Thrust_PS'], 'Position', [620 155 670 175]);
add_block('nesl_utility/PS-Simulink Converter', ...
    [model '/Speed_PS'], 'Position', [620 55 670 75]);
add_block('built-in/Outport', [model '/Thrust'], 'Position', [720 155 750 175]);
add_block('built-in/Outport', [model '/Speed'], 'Position', [720 55 750 75]);

% Save model
save_system(model);
fprintf('Motor-propeller model created: %s.slx\n', model);
\end{lstlisting}

\begin{lstlisting}[language=Matlab, caption=Test script for motor-propeller model, label=lst:simscape-test]
% test_motor_propeller.m
% Validates the motor-propeller Simscape model

% Load model
model = 'motor_propeller_sim';
load_system(model);

% Create test input: Step from 0 to 80% throttle at t=0.1s
t_sim = 2;  % Simulation time

% Run simulation with step input
simIn = Simulink.SimulationInput(model);
simIn = simIn.setVariable('pwm_input', 0.8);

% Run
out = sim(model, 'StopTime', num2str(t_sim));

% Extract results
t = out.tout;
thrust = out.yout{1}.Values.Data;
speed_rpm = out.yout{2}.Values.Data * 60/(2*pi);

% Plot results
figure('Position', [100 100 800 600]);

subplot(2,1,1);
plot(t, thrust, 'b-', 'LineWidth', 1.5);
xlabel('Time (s)');
ylabel('Thrust (N)');
title('Motor-Propeller Step Response');
grid on;

subplot(2,1,2);
plot(t, speed_rpm, 'r-', 'LineWidth', 1.5);
xlabel('Time (s)');
ylabel('Speed (RPM)');
grid on;

% Calculate steady-state values
ss_thrust = mean(thrust(end-100:end));
ss_speed = mean(speed_rpm(end-100:end));
fprintf('Steady-state thrust: %.3f N\n', ss_thrust);
fprintf('Steady-state speed: %.0f RPM\n', ss_speed);

% Calculate time constant (63% of final value)
idx_63 = find(speed_rpm >= 0.632 * ss_speed, 1);
tau_m = t(idx_63) - 0.1;  % Subtract step time
fprintf('Mechanical time constant: %.3f s\n', tau_m);
\end{lstlisting}

\begin{lstlisting}[language=Matlab, caption=Motor parameter identification for Simscape, label=lst:simscape-param-id]
% motor_param_identification.m
% Fits Simscape motor parameters to measured step response data

% Measured data from actual motor (example values)
measured_data = struct();
measured_data.steady_state_rpm = 15000;  % At 80% throttle
measured_data.time_constant = 0.05;       % seconds
measured_data.steady_state_thrust = 0.015; % N per motor

% Back-calculate parameters
V_motor = 0.8 * 3.7;  % Motor voltage at 80% throttle
omega_ss = measured_data.steady_state_rpm * 2*pi/60;  % rad/s

% For steady state: V = Ke*omega + R*I, and I ~= 0 at no load
Ke_est = V_motor / omega_ss;
fprintf('Estimated Ke: %.5f V/(rad/s)\n', Ke_est);

% Time constant: tau = J / (B + Ke*Kt/R)
% Assuming B << Ke*Kt/R: tau ~= J*R / (Ke*Kt)
% For DC motor, Kt = Ke (in consistent units)
J_est = measured_data.time_constant * Ke_est^2 / 0.5;  % R = 0.5 assumed
fprintf('Estimated J: %.2e kg*m^2\n', J_est);

% Thrust coefficient: T = CT * rho * D^4 * omega^2
D = 0.046;  % 46mm propeller diameter
rho = 1.225;
CT_est = measured_data.steady_state_thrust / (rho * D^4 * omega_ss^2);
fprintf('Estimated CT: %.4f\n', CT_est);
\end{lstlisting}

%----------------------------------------------------------------------
\subsection{Ground Effect Modeling}
%----------------------------------------------------------------------

\begin{lstlisting}[language=Matlab, caption=Propeller with ground effect (Simscape component), label=lst:propeller-ge]
component PropellerWithGroundEffect
% Propeller with ground effect model
% Thrust increases when operating near the ground

parameters
    CT = 0.1;            % Thrust coefficient
    CQ = 0.01;           % Torque coefficient
    D = 0.046;           % Propeller diameter [m]
    rho = 1.225;         % Air density [kg/m^3]
    direction = 1;       % +1 for CCW, -1 for CW
    % Ground effect parameters (exponential model)
    A_ge = 0.4;          % Ground effect amplitude
    B_ge = 3.0;          % Ground effect decay rate
end

inputs
    z = {1, 'm'};        % Height above ground
end

outputs
    T = {0, 'N'};        % Thrust force
end

nodes
    R = foundation.mechanical.rotational.rotational;
end

variables
    w = {0, 'rad/s'};
    tau = {0, 'N*m'};
end

branches
    tau : R.t -> *;
end

equations
    w == R.w;

    % Ground effect multiplier (exponential model)
    let
        z_safe = max(z, 0.01*D);  % Prevent division issues
        z_ratio = z_safe / (D/2); % Height in rotor radii
        sigma = 1 + A_ge * exp(-B_ge * z_ratio);
        T_free = direction * CT * rho * D^4 * w * abs(w);
    in
        T == sigma * T_free;
        tau == CQ * rho * D^5 * w * abs(w);  % Torque unaffected
    end
end
end
\end{lstlisting}

\begin{lstlisting}[language=C, caption=Ground effect compensation in control loop, label=lst:ground-effect-ctrl]
// Ground effect feedforward compensation
float GroundEffectFactor(float height_m, float prop_diameter_m)
{
    const float A = 0.4f;
    const float B = 3.0f;

    float z_ratio = height_m / (prop_diameter_m * 0.5f);

    // For high altitudes, return 1 (no effect)
    if (z_ratio > 5.0f) {
        return 1.0f;
    }

    // Exponential ground effect model
    return 1.0f + A * expf(-B * z_ratio);
}

void AltitudeController(float z_ref, float z_meas, float z_dot_meas)
{
    // PID controller output
    float throttle_pid = PID_Update(&altitude_pid, z_ref - z_meas);

    // Compensate for ground effect
    // Since ground effect increases thrust, we need LESS throttle
    float ge_factor = GroundEffectFactor(z_meas, PROP_DIAMETER);
    float throttle_compensated = throttle_pid / ge_factor;

    SetThrottle(throttle_compensated);
}
\end{lstlisting}

\begin{lstlisting}[language=Matlab, caption=Ground effect parameter identification, label=lst:ground-effect-id]
% Experimental data: thrust measured at various heights
% (motor speed held constant)
heights_m = [0.02, 0.03, 0.05, 0.07, 0.10, 0.15, 0.20, 0.30, 0.50];
thrust_N = [0.52, 0.50, 0.48, 0.465, 0.455, 0.445, 0.44, 0.435, 0.43];

% Free-air thrust (from high-altitude measurement)
T_infinity = 0.43;  % N

% Calculate sigma from measurements
sigma_meas = thrust_N / T_infinity;

% Fit exponential model: sigma = 1 + A*exp(-B*z/R)
R = 0.023;  % Propeller radius [m]
z_ratio = heights_m / R;

% Define model for fitting
sigma_model = @(params, z) 1 + params(1) * exp(-params(2) * z);

% Nonlinear least squares fit
params_init = [0.4, 3];
params_fit = lsqcurvefit(sigma_model, params_init, z_ratio, sigma_meas);

A_fit = params_fit(1);
B_fit = params_fit(2);

fprintf('Fitted parameters:\n');
fprintf('  A = %.3f\n', A_fit);
fprintf('  B = %.3f\n', B_fit);

% Plot results
figure;
z_plot = linspace(0.5, 20, 100);
sigma_fit = sigma_model(params_fit, z_plot);

plot(z_ratio, sigma_meas, 'ko', 'MarkerSize', 8);
hold on;
plot(z_plot, sigma_fit, 'b-', 'LineWidth', 2);
xlabel('Height / Rotor Radius (z/R)');
ylabel('Thrust Ratio \sigma');
title('Ground Effect Characterization');
legend('Measured', 'Exponential fit');
grid on;
\end{lstlisting}

%----------------------------------------------------------------------
\subsection{Hybrid Systems Implementation}
%----------------------------------------------------------------------

\begin{lstlisting}[language=Pascal, caption=Quadrotor landing sequence in Modelica, label=lst:modelica-landing]
model QuadrotorLanding "Hybrid landing sequence"
  parameter Real m = 0.03 "Mass [kg]";
  parameter Real g = 9.81 "Gravity [m/s^2]";
  parameter Real r = 0.023 "Rotor radius [m]";
  parameter Real h_ge = 0.05 "Ground effect height [m]";
  parameter Real k_g = 1000 "Ground stiffness [N/m]";
  parameter Real c_g = 10 "Ground damping [Ns/m]";
  parameter Real Kp = 2.0, Kd = 1.0, Ki = 0.5;

  Real z(start = -1.0) "Altitude (NED, negative = above ground)";
  Real vz(start = 0) "Vertical velocity";
  Real Iz(start = 0) "Integrator state";
  Real T "Thrust";
  Real sigma "Ground effect factor";
  Integer mode(start = 1) "1=Descent, 2=GE, 3=Contact, 4=Settled";

equation
  // Ground effect factor
  sigma = if z > -h_ge and z < 0 then 1/(1 - (r/(4*(-z)))^2) else 1.0;

  // Mode-dependent dynamics
  if mode == 4 then  // Settled
    der(z) = 0;
    der(vz) = 0;
    der(Iz) = 0;
    T = 0;
  elseif mode == 3 then  // Contact
    der(z) = vz;
    der(vz) = g - T/m + (k_g*z - c_g*vz)/m;
    der(Iz) = 0;  // Integrator frozen
    T = m*g * 0.3;  // Reduced thrust during contact
  else  // Descent or GroundEffect
    der(z) = vz;
    der(vz) = g - T*sigma/m;
    der(Iz) = 0 - z;  // z_target = 0 (ground level)
    T = m*g + Kp*(0-z) + Kd*((-0.1)-vz) + Ki*Iz;
  end if;

  // Mode transitions
  when z > -h_ge and mode == 1 then
    mode := 2;  // Enter ground effect
  end when;

  when z >= 0 and mode == 2 then
    mode := 3;  // Contact
    reinit(Iz, 0);  // Reset integrator
  end when;

  when abs(z) < 0.001 and abs(vz) < 0.01 and mode == 3 then
    mode := 4;  // Settled
  end when;

end QuadrotorLanding;
\end{lstlisting}

\begin{lstlisting}[caption=Stateflow chart for Crazyflie flight modes, label=lst:stateflow-flight]
%% State: Disarmed (default)
State: Disarmed
  Entry: motors_enabled = false;
         integrators_reset();
  During: T = 0; tau = [0;0;0];

Transition: Disarmed -> Armed
  Guard: [arm_switch && throttle < 0.1 && battery_voltage > 3.0]
  Action: arm_time = current_time;

%% State: Armed
State: Armed
  Entry: motors_enabled = true;
  During: T = T_idle;  % Motors spinning at idle
          tau = attitude_control(phi_cmd, theta_cmd, psi_cmd);

Transition: Armed -> Disarmed
  Guard: [!arm_switch]

Transition: Armed -> TakingOff
  Guard: [throttle > 0.5 && (current_time - arm_time) > 1.0]
  Action: z_target = z_current - 1.0;  % 1m above current

%% State: TakingOff
State: TakingOff
  Entry: takeoff_start_z = z_current;
  During: [T, tau] = position_control(x,y,z, z_target);

Transition: TakingOff -> Hovering
  Guard: [abs(z - z_target) < 0.1 && abs(vz) < 0.1]

%% State: Hovering
State: Hovering
  During: [T, tau] = position_control(x,y,z, position_setpoint);

Transition: Hovering -> Landing
  Guard: [land_command || battery_voltage < 3.3]
  Action: descent_rate = -0.3;  % 0.3 m/s descent

%% State: Landing
State: Landing
  During: z_target = z_target + descent_rate * dt;
          [T, tau] = position_control(x,y,z, [x_land, y_land, z_target]);

Transition: Landing -> Disarmed
  Guard: [z > -0.05 && abs(vz) < 0.05]  % Near ground, slow
  Action: integrators_reset();

%% Emergency transitions (highest priority)
Transition: * -> Emergency
  Guard: [!signal_ok || battery_voltage < 2.9 || abs(phi) > 60 || abs(theta) > 60]
  Priority: 1

State: Emergency
  Entry: emergency_start_time = current_time;
  During: T = 0.5 * m * g;  % Gentle descent
          tau = attitude_control(0, 0, psi_current);  % Level attitude

Transition: Emergency -> Disarmed
  Guard: [z > -0.1 || (current_time - emergency_start_time) > 10]
\end{lstlisting}

%======================================================================
\section{Real-Time and Embedded Systems}
\label{app:code-rtos}
%======================================================================

%----------------------------------------------------------------------
\subsection{Timer and PWM Configuration}
%----------------------------------------------------------------------

\begin{lstlisting}[language=C, caption=PWM configuration for quadrotor motors, label=lst:pwm-motors]
// Configure TIM1 Channel 1-4 for PWM output (motor ESCs)
void PWM_Init(void)
{
    // Enable clocks
    RCC->APB2ENR |= RCC_APB2ENR_TIM1EN;
    RCC->AHB1ENR |= RCC_AHB1ENR_GPIOAEN;

    // Configure PA8-11 as alternate function (TIM1 CH1-4)
    GPIOA->MODER |= (2 << 16) | (2 << 18) | (2 << 20) | (2 << 22);
    GPIOA->AFR[1] |= (1 << 0) | (1 << 4) | (1 << 8) | (1 << 12);

    // Timer configuration for 400 Hz PWM (standard ESC frequency)
    // 168 MHz / 168 = 1 MHz, period of 2500 gives 400 Hz
    TIM1->PSC = 167;
    TIM1->ARR = 2499;

    // Configure channels 1-4 for PWM mode 1
    TIM1->CCMR1 = (6 << 4) | (6 << 12);  // PWM mode 1, CH1 & CH2
    TIM1->CCMR2 = (6 << 4) | (6 << 12);  // PWM mode 1, CH3 & CH4

    // Enable outputs
    TIM1->CCER = TIM_CCER_CC1E | TIM_CCER_CC2E |
                 TIM_CCER_CC3E | TIM_CCER_CC4E;
    TIM1->BDTR |= TIM_BDTR_MOE;  // Main output enable (TIM1 specific)

    // Initialize to minimum throttle (1000 us pulse)
    TIM1->CCR1 = 1000;
    TIM1->CCR2 = 1000;
    TIM1->CCR3 = 1000;
    TIM1->CCR4 = 1000;

    // Start timer
    TIM1->CR1 |= TIM_CR1_CEN;
}

// Set motor output (1000-2000 us pulse width)
void PWM_SetMotor(uint8_t channel, uint16_t pulseWidth_us)
{
    // Clamp to valid range
    if (pulseWidth_us < 1000) pulseWidth_us = 1000;
    if (pulseWidth_us > 2000) pulseWidth_us = 2000;

    // Set compare register (1 MHz timer = 1 us per tick)
    switch (channel) {
        case 1: TIM1->CCR1 = pulseWidth_us; break;
        case 2: TIM1->CCR2 = pulseWidth_us; break;
        case 3: TIM1->CCR3 = pulseWidth_us; break;
        case 4: TIM1->CCR4 = pulseWidth_us; break;
    }
}
\end{lstlisting}

%----------------------------------------------------------------------
\subsection{DMA Configuration}
%----------------------------------------------------------------------

\begin{lstlisting}[language=C, caption=DMA configuration for SPI IMU read, label=lst:dma-spi]
// Buffer for IMU data (14 bytes: accel XYZ, temp, gyro XYZ)
uint8_t imuDmaBuffer[14];
uint8_t imuTxBuffer[14] = {0x80 | 0x3B, 0, 0, 0, 0, 0, 0,
                           0, 0, 0, 0, 0, 0, 0};  // Read from 0x3B

void DMA_SPI_Init(void)
{
    // Enable DMA2 clock (SPI1 uses DMA2)
    RCC->AHB1ENR |= RCC_AHB1ENR_DMA2EN;

    // Configure DMA2 Stream 0 for SPI1 RX (Channel 3)
    DMA2_Stream0->CR = 0;  // Disable stream first
    while (DMA2_Stream0->CR & DMA_SxCR_EN);  // Wait for disable

    DMA2_Stream0->PAR = (uint32_t)&SPI1->DR;  // Peripheral address
    DMA2_Stream0->M0AR = (uint32_t)imuDmaBuffer;  // Memory address
    DMA2_Stream0->NDTR = 14;  // Number of transfers

    DMA2_Stream0->CR = (3 << 25) |  // Channel 3
                       DMA_SxCR_MINC |  // Memory increment
                       DMA_SxCR_TCIE;   // Transfer complete interrupt

    // Configure DMA2 Stream 3 for SPI1 TX (Channel 3)
    DMA2_Stream3->CR = 0;
    while (DMA2_Stream3->CR & DMA_SxCR_EN);

    DMA2_Stream3->PAR = (uint32_t)&SPI1->DR;
    DMA2_Stream3->M0AR = (uint32_t)imuTxBuffer;
    DMA2_Stream3->NDTR = 14;

    DMA2_Stream3->CR = (3 << 25) |    // Channel 3
                       DMA_SxCR_MINC |  // Memory increment
                       (1 << 6);        // Memory to peripheral

    // Enable DMA requests in SPI
    SPI1->CR2 |= SPI_CR2_RXDMAEN | SPI_CR2_TXDMAEN;

    // Enable DMA interrupt
    NVIC_SetPriority(DMA2_Stream0_IRQn, 5);
    NVIC_EnableIRQ(DMA2_Stream0_IRQn);
}

void IMU_StartDmaRead(void)
{
    // Assert chip select
    GPIOB->BSRR = GPIO_BSRR_BR0;

    // Reset transfer counts
    DMA2_Stream0->NDTR = 14;
    DMA2_Stream3->NDTR = 14;

    // Enable DMA streams
    DMA2_Stream0->CR |= DMA_SxCR_EN;
    DMA2_Stream3->CR |= DMA_SxCR_EN;
}

void DMA2_Stream0_IRQHandler(void)
{
    if (DMA2->LISR & DMA_LISR_TCIF0) {
        DMA2->LIFCR = DMA_LIFCR_CTCIF0;  // Clear flag

        // Deassert chip select
        GPIOB->BSRR = GPIO_BSRR_BS0;

        // Signal processing task
        BaseType_t woken = pdFALSE;
        xSemaphoreGiveFromISR(xImuDataReady, &woken);
        portYIELD_FROM_ISR(woken);
    }
}
\end{lstlisting}

%----------------------------------------------------------------------
\subsection{Battery Management}
%----------------------------------------------------------------------

\begin{lstlisting}[language=C, caption=Battery state management, label=lst:battery-state]
typedef enum {
    BATTERY_NORMAL,     // Normal operation
    BATTERY_LOW,        // Low warning - notify user
    BATTERY_CRITICAL,   // Critical - initiate landing
    BATTERY_CUTOFF      // Cutoff - disable motors
} BatteryState_t;

typedef struct {
    BatteryState_t state;
    uint32_t state_entry_time;
    uint8_t warning_count;
} BatteryStateMachine_t;

void Battery_UpdateState(BatteryStateMachine_t *sm, float voltage)
{
    float cell_voltage = voltage / BATTERY_CELLS;
    BatteryState_t new_state = sm->state;

    // Hysteresis to prevent rapid state changes
    const float HYSTERESIS = 0.05f;

    switch (sm->state) {
        case BATTERY_NORMAL:
            if (cell_voltage < CELL_LOW_VOLTAGE) {
                new_state = BATTERY_LOW;
            }
            break;

        case BATTERY_LOW:
            if (cell_voltage > CELL_LOW_VOLTAGE + HYSTERESIS) {
                new_state = BATTERY_NORMAL;
            } else if (cell_voltage < CELL_CRITICAL_VOLTAGE) {
                new_state = BATTERY_CRITICAL;
            }
            break;

        case BATTERY_CRITICAL:
            if (cell_voltage > CELL_CRITICAL_VOLTAGE + HYSTERESIS) {
                new_state = BATTERY_LOW;
            } else if (cell_voltage < CELL_CUTOFF_VOLTAGE) {
                new_state = BATTERY_CUTOFF;
            }
            break;

        case BATTERY_CUTOFF:
            // No recovery from cutoff - require power cycle
            break;
    }

    if (new_state != sm->state) {
        sm->state = new_state;
        sm->state_entry_time = HAL_GetTick();
        Battery_HandleStateChange(new_state);
    }
}

void Battery_HandleStateChange(BatteryState_t state)
{
    switch (state) {
        case BATTERY_LOW:
            LED_SetPattern(LED_SLOW_BLINK);
            TelemetryLog("Battery low warning");
            break;

        case BATTERY_CRITICAL:
            LED_SetPattern(LED_FAST_BLINK);
            TelemetryLog("Battery critical - auto-landing");
            FlightMode_SetAutoLand();
            break;

        case BATTERY_CUTOFF:
            LED_SetPattern(LED_SOLID);
            TelemetryLog("Battery cutoff - motors disabled");
            Motors_DisableAll();
            break;

        default:
            break;
    }
}
\end{lstlisting}

%----------------------------------------------------------------------
\subsection{micro-ROS Integration}
%----------------------------------------------------------------------

\begin{lstlisting}[language=C, caption=micro-ROS with FreeRTOS, label=lst:microros-freertos]
#include <rcl/rcl.h>
#include <rclc/rclc.h>
#include <rclc/executor.h>
#include <geometry_msgs/msg/pose_stamped.h>

// micro-ROS entities
rcl_allocator_t allocator;
rclc_support_t support;
rcl_node_t node;
rcl_publisher_t position_publisher;
rclc_executor_t executor;

geometry_msgs__msg__PoseStamped position_msg;

// Timer callback for publishing position
void position_timer_callback(rcl_timer_t *timer, int64_t last_call_time) {
    (void)last_call_time;

    // Get current position from flight controller
    position_msg.pose.position.x = GetPositionX();
    position_msg.pose.position.y = GetPositionY();
    position_msg.pose.position.z = GetPositionZ();

    // Update timestamp
    int64_t time_ns = rmw_uros_epoch_nanos();
    position_msg.header.stamp.sec = time_ns / 1000000000;
    position_msg.header.stamp.nanosec = time_ns % 1000000000;

    rcl_publish(&position_publisher, &position_msg, NULL);
}

// FreeRTOS task for micro-ROS
void MicroRosTask(void *argument) {
    // Initialize micro-ROS
    allocator = rcl_get_default_allocator();

    // Wait for agent connection
    while (rmw_uros_ping_agent(100, 1) != RMW_RET_OK) {
        vTaskDelay(pdMS_TO_TICKS(1000));
    }

    // Create support and node
    rclc_support_init(&support, 0, NULL, &allocator);
    rclc_node_init_default(&node, "flight_controller", "", &support);

    // Create publisher
    rclc_publisher_init_default(
        &position_publisher,
        &node,
        ROSIDL_GET_MSG_TYPE_SUPPORT(geometry_msgs, msg, PoseStamped),
        "/drone/position"
    );

    // Create timer (50 Hz)
    rcl_timer_t timer;
    rclc_timer_init_default(&timer, &support, RCL_MS_TO_NS(20),
                            position_timer_callback);

    // Create executor
    rclc_executor_init(&executor, &support.context, 1, &allocator);
    rclc_executor_add_timer(&executor, &timer);

    // Main loop
    for (;;) {
        rclc_executor_spin_some(&executor, RCL_MS_TO_NS(10));
        vTaskDelay(pdMS_TO_TICKS(10));
    }

    // Cleanup (never reached in normal operation)
    rcl_publisher_fini(&position_publisher, &node);
    rcl_node_fini(&node);
    rclc_support_fini(&support);
}

// Main initialization
void main(void) {
    // ... hardware initialization ...

    // Create micro-ROS task (lower priority than control loops)
    xTaskCreate(MicroRosTask, "MicroROS", 4096, NULL, 3, NULL);

    // Create control tasks (higher priority)
    xTaskCreate(AttitudeControlTask, "AttCtrl", 512, NULL, 6, NULL);

    vTaskStartScheduler();
}
\end{lstlisting}

%======================================================================
\section{Testing and Continuous Integration}
\label{app:code-testing}
%======================================================================

%----------------------------------------------------------------------
\subsection{Unit Testing with Unity}
%----------------------------------------------------------------------

\begin{lstlisting}[language=C, caption=Unit tests for PID controller, label=lst:test-pid]
// test_pid.c

#include "unity.h"
#include "pid.h"

static PID_t pid;

void setUp(void) {
    // Reset PID before each test
    PID_Init(&pid, 1.0f, 0.1f, 0.01f);  // Kp=1, Ki=0.1, Kd=0.01
    PID_SetLimits(&pid, -100.0f, 100.0f);
}

void tearDown(void) {}

void test_pid_proportional_only(void) {
    // With Ki=0, Kd=0, output should be Kp * error
    PID_Init(&pid, 2.0f, 0.0f, 0.0f);
    float output = PID_Update(&pid, 10.0f, 0.01f);  // error=10, dt=0.01s
    TEST_ASSERT_FLOAT_WITHIN(0.001f, 20.0f, output);  // 2.0 * 10 = 20
}

void test_pid_integral_accumulates(void) {
    PID_Init(&pid, 0.0f, 1.0f, 0.0f);  // Ki only
    // Apply constant error over multiple steps
    for (int i = 0; i < 10; i++) {
        PID_Update(&pid, 5.0f, 0.1f);  // error=5, dt=0.1s
    }
    float output = PID_Update(&pid, 5.0f, 0.1f);
    // Integral should be 5 * 0.1 * 11 = 5.5 (cumulative)
    TEST_ASSERT_FLOAT_WITHIN(0.1f, 5.5f, output);
}

void test_pid_derivative_responds_to_change(void) {
    PID_Init(&pid, 0.0f, 0.0f, 1.0f);  // Kd only
    PID_Update(&pid, 0.0f, 0.01f);      // First call, error=0
    float output = PID_Update(&pid, 10.0f, 0.01f);  // error jumps to 10
    // Derivative = (10 - 0) / 0.01 = 1000 (but limited)
    TEST_ASSERT_FLOAT_WITHIN(1.0f, 100.0f, output);  // Limited to 100
}

void test_pid_output_saturation(void) {
    PID_Init(&pid, 100.0f, 0.0f, 0.0f);  // High gain
    PID_SetLimits(&pid, -50.0f, 50.0f);
    float output = PID_Update(&pid, 10.0f, 0.01f);  // Would be 1000
    TEST_ASSERT_FLOAT_WITHIN(0.001f, 50.0f, output);  // Saturated at 50
}

void test_pid_anti_windup(void) {
    PID_Init(&pid, 1.0f, 10.0f, 0.0f);
    PID_SetLimits(&pid, -50.0f, 50.0f);
    // Saturate the output for many steps
    for (int i = 0; i < 100; i++) {
        PID_Update(&pid, 100.0f, 0.01f);  // Large error, saturates
    }
    // Now reverse error - should respond quickly if anti-windup works
    float output = PID_Update(&pid, -10.0f, 0.01f);
    TEST_ASSERT_TRUE(output < 0.0f);  // Should go negative
}

int main(void) {
    UNITY_BEGIN();
    RUN_TEST(test_pid_proportional_only);
    RUN_TEST(test_pid_integral_accumulates);
    RUN_TEST(test_pid_derivative_responds_to_change);
    RUN_TEST(test_pid_output_saturation);
    RUN_TEST(test_pid_anti_windup);
    return UNITY_END();
}
\end{lstlisting}

\begin{lstlisting}[language=C, caption=Unit tests for sensor filtering, label=lst:test-filters]
// test_filters.c

#include "unity.h"
#include "filters.h"
#include <math.h>

void setUp(void) {}
void tearDown(void) {}

void test_lowpass_filter_initialization(void) {
    LowpassFilter_t lpf;
    Lowpass_Init(&lpf, 10.0f, 0.002f);  // 10 Hz cutoff, 500 Hz sample
    TEST_ASSERT_FLOAT_WITHIN(1e-6f, 0.0f, lpf.output);
}

void test_lowpass_filter_steady_state(void) {
    LowpassFilter_t lpf;
    Lowpass_Init(&lpf, 10.0f, 0.002f);
    // Apply constant input - output should converge to input
    for (int i = 0; i < 1000; i++) {
        Lowpass_Update(&lpf, 5.0f);
    }
    TEST_ASSERT_FLOAT_WITHIN(0.01f, 5.0f, lpf.output);
}

void test_lowpass_filter_attenuates_high_freq(void) {
    LowpassFilter_t lpf;
    float cutoff = 10.0f;
    float dt = 0.002f;  // 500 Hz
    Lowpass_Init(&lpf, cutoff, dt);
    // Apply 100 Hz sine wave (well above cutoff)
    float amplitude = 0.0f;
    for (int i = 0; i < 1000; i++) {
        float t = i * dt;
        float input = sinf(2.0f * M_PI * 100.0f * t);
        Lowpass_Update(&lpf, input);
        if (i > 500) {
            if (fabsf(lpf.output) > amplitude) {
                amplitude = fabsf(lpf.output);
            }
        }
    }
    // 100 Hz is 10x above cutoff, expect ~10x attenuation
    TEST_ASSERT_TRUE(amplitude < 0.2f);
}

void test_complementary_filter_combines_sources(void) {
    ComplementaryFilter_t cf;
    CompFilter_Init(&cf, 0.98f);  // alpha = 0.98
    float angle_gyro = 0.0f;
    float angle_accel = 0.0f;
    for (int i = 0; i < 1000; i++) {
        angle_gyro += 0.01f;  // Gyro drifts 0.01 per step
        CompFilter_Update(&cf, angle_gyro, angle_accel, 0.002f);
    }
    // With alpha=0.98, gyro dominates short-term but accel corrects drift
    TEST_ASSERT_TRUE(cf.angle < 5.0f);  // Not fully drifted
    TEST_ASSERT_TRUE(cf.angle > -1.0f); // Not negative
}

int main(void) {
    UNITY_BEGIN();
    RUN_TEST(test_lowpass_filter_initialization);
    RUN_TEST(test_lowpass_filter_steady_state);
    RUN_TEST(test_lowpass_filter_attenuates_high_freq);
    RUN_TEST(test_complementary_filter_combines_sources);
    return UNITY_END();
}
\end{lstlisting}

%----------------------------------------------------------------------
\subsection{CI/CD Configuration}
%----------------------------------------------------------------------

\begin{lstlisting}[language=yaml, caption=Flight Controller CI workflow, label=lst:ci-yaml]
name: Flight Controller CI

on:
  push:
    branches: [main, develop]
  pull_request:
    branches: [main]

jobs:
  build-and-test:
    runs-on: ubuntu-latest

    steps:
    - name: Checkout code
      uses: actions/checkout@v4

    - name: Install ARM toolchain
      run: |
        sudo apt-get update
        sudo apt-get install -y gcc-arm-none-eabi

    - name: Build for target (ARM)
      run: |
        mkdir build-arm && cd build-arm
        cmake -DCMAKE_TOOLCHAIN_FILE=../cmake/arm-none-eabi.cmake ..
        make -j$(nproc)

    - name: Build for host (unit tests)
      run: |
        mkdir build-host && cd build-host
        cmake -DBUILD_TESTS=ON ..
        make -j$(nproc)

    - name: Run unit tests
      run: |
        cd build-host
        ctest --output-on-failure

    - name: Upload test results
      uses: actions/upload-artifact@v4
      if: always()
      with:
        name: test-results
        path: build-host/test-results.xml

  static-analysis:
    runs-on: ubuntu-latest
    steps:
    - name: Checkout code
      uses: actions/checkout@v4

    - name: Run cppcheck
      run: |
        sudo apt-get install -y cppcheck
        cppcheck --enable=warning,style --error-exitcode=1 src/

    - name: Check formatting
      run: |
        sudo apt-get install -y clang-format
        find src/ -name '*.c' -o -name '*.h' | xargs clang-format --dry-run -Werror

  coverage:
    runs-on: ubuntu-latest
    steps:
    - name: Checkout code
      uses: actions/checkout@v4

    - name: Build with coverage
      run: |
        mkdir build-cov && cd build-cov
        cmake -DBUILD_TESTS=ON -DENABLE_COVERAGE=ON ..
        make -j$(nproc)

    - name: Run tests and collect coverage
      run: |
        cd build-cov
        ctest
        gcovr --xml coverage.xml --xml-pretty

    - name: Upload coverage to Codecov
      uses: codecov/codecov-action@v3
      with:
        file: build-cov/coverage.xml

    - name: Check coverage threshold
      run: |
        cd build-cov
        gcovr --fail-under-line 80

  simulation-test:
    runs-on: ubuntu-latest
    steps:
    - name: Checkout code
      uses: actions/checkout@v4

    - name: Setup MATLAB
      uses: matlab-actions/setup-matlab@v1

    - name: Run MIL tests
      uses: matlab-actions/run-command@v1
      with:
        command: |
          cd simulation
          results = runtests('tests/')
          assertSuccess(results)
\end{lstlisting}


% ===== Bibliography =====
\cleardoublepage
\addcontentsline{toc}{chapter}{Bibliography}
\printbibliography[title={Bibliography}]

% ===== Index =====
\cleardoublepage
\addcontentsline{toc}{chapter}{Index}
\printindex

\end{document}
